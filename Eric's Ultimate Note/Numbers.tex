\documentclass{report}
%%%%%%%%%%%%%% preamble.tex %%%%%%%%%%%%%%
\usepackage[T1]{fontenc}
\usepackage{etoolbox}
% Page Setup
\usepackage[letterpaper, tmargin=2cm, rmargin=0.5in, lmargin=0.5in, bmargin=80pt, footskip=.2in]{geometry}
\usepackage{adjustbox}
\usepackage{graphicx}
\usepackage{tikz}
\usepackage{mathrsfs}
\usepackage{mdframed}

% Create a new toggle
\newtoggle{firstsection}

% Redefine the \chapter command to reset the toggle for each new chapter
\let\oldchapter\chapter
\renewcommand{\chapter}{\toggletrue{firstsection}\oldchapter}

% Redefine the \section command to check the toggle
\let\oldsection\section
\renewcommand{\section}{
    \iftoggle{firstsection}
    {\togglefalse{firstsection}} % If it's the first section, just switch off the toggle for next sections
    {\clearpage} % If it's not the first section, start a new page
    \oldsection
}

% Abstract Design

\usepackage{lipsum}

\renewenvironment{abstract}
 {% Start of environment
  \quotation
  \small
  \noindent
  \rule{\linewidth}{.5pt} % Draw the rule to match the linewidth
  \par\smallskip
  {\centering\bfseries\abstractname\par}\medskip
 }
 {% End of environment
  \par\noindent
  \rule{\linewidth}{.5pt} % Ensure the closing rule also matches
  \endquotation
 }

% Mathematics
\usepackage{amsmath,amsfonts,amsthm,amssymb,mathtools}
\usepackage{xfrac}
\usepackage[makeroom]{cancel}
\usepackage{enumitem}
\usepackage{nameref}
\usepackage{multicol,array}
\usepackage{tikz-cd}
\usepackage{array}
\usepackage{multirow}% http://ctan.org/pkg/multirow
\usepackage{graphicx}

% Colors
\usepackage[dvipsnames]{xcolor}
\definecolor{myg}{RGB}{56, 140, 70}
\definecolor{myb}{RGB}{45, 111, 177}
\definecolor{myr}{RGB}{199, 68, 64}
% Define more colors here...
\definecolor{olive}{HTML}{6B8E23}
\definecolor{orange}{HTML}{CC5500}
\definecolor{brown}{HTML}{8B4513}
% Hyperlinks
\usepackage{bookmark}
\usepackage[colorlinks=true,linkcolor=blue,urlcolor=blue,citecolor=blue,anchorcolor=blue]{hyperref}
\usepackage{xcolor}
\hypersetup{
    colorlinks,
    linkcolor={red!50!black},
    citecolor={blue!50!black},
    urlcolor={blue!80!black}
}

% Text-related
\usepackage{blindtext}
\usepackage{fontsize}
\changefontsize[14]{14}
\setlength{\parindent}{0pt}
\linespread{1.2}

% Theorems and Definitions
\usepackage{amsthm}
\renewcommand\qedsymbol{$\blacksquare$}

% Define a new theorem style
\newtheoremstyle{mytheoremstyle}% name
  {}% Space above
  {}% Space below
  {}% Body font
  {}% Indent amount
  {\bfseries}% Theorem head font
  {.}% Punctuation after theorem head
  {.5em}% Space after theorem head
  {}% Theorem head spec (can be left empty, meaning ‘normal’)

% Apply the new theorem style to theorem-like environments
\theoremstyle{mytheoremstyle}

\newtheorem{theorem}{Theorem}[section]  
\newtheorem{definition}[theorem]{Definition} 
\newtheorem{lemma}[theorem]{Lemma}  
\newtheorem{corollary}[theorem]{Corollary}
\newtheorem{axiom}[theorem]{Axiom}
\newtheorem{example}[theorem]{Example}
\newtheorem{equiv_def}[theorem]{Equivalent Definition}

% tcolorbox Setup
\usepackage[most,many,breakable]{tcolorbox}
\tcbuselibrary{theorems}

% Define custom tcolorbox environments here...

%================================
% EXAMPLE BOX
%================================
% After you have defined the style and other theorem environments
\definecolor{myexamplebg}{RGB}{245, 245, 245} % Very light grey for background
\definecolor{myexamplefr}{RGB}{120, 120, 120} % Medium grey for frame
\definecolor{myexampleti}{RGB}{60, 60, 60}    % Darker grey for title

\newtcbtheorem[]{Example}{Example}{
    colback=myexamplebg,
    breakable,
    colframe=myexamplefr,
    coltitle=myexampleti,
    boxrule=1pt,
    sharp corners,
    detach title,
    before upper=\tcbtitle\par\vspace{-20pt}, % Reduced the space after the title
    fonttitle=\bfseries,
    description font=\mdseries,
    separator sign none,
    description delimiters={}{}, % No delimiters around the title
}{ex}
%================================
% Solution BOX
%================================
\makeatletter
\newtcolorbox{solution}{enhanced,
	breakable,
	colback=white,
	colframe=myg!80!black,
	attach boxed title to top left={yshift*=-\tcboxedtitleheight},
	title=Solution,
	boxed title size=title,
	boxed title style={%
			sharp corners,
			rounded corners=northwest,
			colback=tcbcolframe,
			boxrule=0pt,
		},
	underlay boxed title={%
			\path[fill=tcbcolframe] (title.south west)--(title.south east)
			to[out=0, in=180] ([xshift=5mm]title.east)--
			(title.center-|frame.east)
			[rounded corners=\kvtcb@arc] |-
			(frame.north) -| cycle;
		},
}
\makeatother

% %================================
% % Question BOX
% %================================
\makeatletter
\newtcbtheorem{question}{Question}{enhanced,
	breakable,
	colback=white,
	colframe=myb!80!black,
	attach boxed title to top left={yshift*=-\tcboxedtitleheight},
	fonttitle=\bfseries,
	title={#2},
	boxed title size=title,
	boxed title style={%
			sharp corners,
			rounded corners=northwest,
			colback=tcbcolframe,
			boxrule=0pt,
		},
	underlay boxed title={%
			\path[fill=tcbcolframe] (title.south west)--(title.south east)
			to[out=0, in=180] ([xshift=5mm]title.east)--
			(title.center-|frame.east)
			[rounded corners=\kvtcb@arc] |-
			(frame.north) -| cycle;
		},
	#1
}{question}
\makeatother

%%%%%%%%%%%%%%%%%%%%%%%%%%%%%%%%%%%%%%%%%%%
% TABLE OF CONTENTS
%%%%%%%%%%%%%%%%%%%%%%%%%%%%%%%%%%%%%%%%%%%


\usepackage{tikz}
\definecolor{doc}{RGB}{0,60,110}
\usepackage{titletoc}
\contentsmargin{0cm}
\titlecontents{chapter}[14pc]
{\addvspace{30pt}%
	\begin{tikzpicture}[remember picture, overlay]%
		\draw[fill=doc!60,draw=doc!60] (-7,-.1) rectangle (-0.9,.5);%
		\pgftext[left,x=-5.5cm,y=0.2cm]{\color{white}\Large\sc\bfseries Chapter\ \thecontentslabel};%
	\end{tikzpicture}\color{doc!60}\large\sc\bfseries}%
{}
{}
{\;\titlerule\;\large\sc\bfseries Page \thecontentspage
	\begin{tikzpicture}[remember picture, overlay]
		\draw[fill=doc!60,draw=doc!60] (2pt,0) rectangle (4,0.1pt);
	\end{tikzpicture}}%
\titlecontents{section}[3.7pc]
{\addvspace{2pt}}
{\contentslabel[\thecontentslabel]{3pc}}
{}
{\hfill\small \thecontentspage}
[]
\titlecontents*{subsection}[3.7pc]
{\addvspace{-1pt}\small}
{}
{}
{\ --- \small\thecontentspage}
[ \textbullet\ ][]

\makeatletter
\renewcommand{\tableofcontents}{
	\chapter*{%
	  \vspace*{-20\p@}%
	  \begin{tikzpicture}[remember picture, overlay]%
		  \pgftext[right,x=15cm,y=0.2cm]{\color{doc!60}\Huge\sc\bfseries \contentsname};%
		  \draw[fill=doc!60,draw=doc!60] (13,-.75) rectangle (20,1);%
		  \clip (13,-.75) rectangle (20,1);
		  \pgftext[right,x=15cm,y=0.2cm]{\color{white}\Huge\sc\bfseries \contentsname};%
	  \end{tikzpicture}}%
	\@starttoc{toc}}
\makeatother

\newcommand{\liff}{\llap{$\iff$}}
\newcommand{\rap}[1]{\rrap{\text{ (#1)}}}
\newcommand{\red}[1]{\textcolor{red}{#1}}
\newcommand{\blue}[1]{\textcolor{blue}{#1}}
\newcommand{\vi}[1]{\textcolor{violet}{#1}}
\newcommand{\olive}[1]{\textcolor{olive}{#1}}
\newcommand{\teal}[1]{\textcolor{teal}{#1}}
\newcommand{\brown}[1]{\textcolor{brown}{#1}}
\newcommand{\orange}[1]{\textcolor{orange}{#1}}
\newcommand{\tCaC}{\text{ \CaC }}
\newcommand{\CaC}{\red{CaC} }
\newcommand{\As}[1]{Assume \red{#1}}
\newcommand{\vdone}{\vi{\text{ (done) }}}
\newcommand{\bdone}{\blue{\text{ (done) }}}
\newcommand{\tdone}{\teal{\text{ (done) }}}
\newcommand{\odone}{\olive{\text{ (done) }}}
\newcommand{\bodone}{\brown{\text{ (done) }}}
\newcommand{\ordone}{\orange{\text{ (done) }}}
\newcommand{\ld}{\lambda}
\newcommand{\vecta}[1]{\textbf{#1}}
\newcommand{\set}[1]{\left\{ #1 \right\}}
\newcommand{\bset}[1]{\Big\{ #1 \Big\}}
\newcommand{\inR}{\in\R}
\newcommand{\inn}{\in\N}
\newcommand{\inz}{\in\Z}
\newcommand{\inr}{\in\R}
\newcommand{\inc}{\in\C}
\newcommand{\inq}{\in\Q}
\newcommand{\norm}[1]{\| #1 \|}
\newcommand{\bnorm}[1]{\Big\| #1 \Big\|}
\newcommand{\gen}[1]{\langle #1 \rangle}
\newcommand{\abso}[1]{\left|#1\right|}
\newcommand{\myref}[2]{\hyperref[#2]{#1\ \ref*{#2}}}
\newcommand{\customref}[2]{\hyperref[#1]{#2}}
\newcommand{\power}[1]{\mathcal{P}(#1)}
\newcommand{\dcup}{\mathbin{\dot{\cup}}}
\newcommand{\diam}[1]{\text{diam}\, #1}
\newcommand{\at}{\Big|}
\newcommand{\quotient}{\diagup}
\let\originalphi\phi % Store the original \phi in \originalphi
\renewcommand{\phi}{\varphi} % Redefine \phi to \varphi
\newcommand{\pfi}{\originalphi} % Define \pfi to display the original \phi
\newcommand{\diota}{\dot{\iota}}
\newcommand{\Log}{\operatorname{Log}}
\newcommand{\id}{\text{\textbf{id}}}
\usepackage{amsmath}

\makeatletter
\NewDocumentCommand{\extp}{e{^}}{%
  \mathop{\mathpalette\extp@{#1}}\nolimits
}
\NewDocumentCommand{\extp@}{mm}{%
  \bigwedge\nolimits\IfValueT{#2}{^{\extp@@{#1}#2}}%
  \IfValueT{#1}{\kern-2\scriptspace\nonscript\kern2\scriptspace}%
}
\newcommand{\extp@@}[1]{%
  \mkern
    \ifx#1\displaystyle-1.8\else
    \ifx#1\textstyle-1\else
    \ifx#1\scriptstyle-1\else
    -0.5\fi\fi\fi
  \thinmuskip
}
\makeatletter
\usepackage{pifont}
\makeatletter
\newcommand\Pimathsymbol[3][\mathord]{%
  #1{\@Pimathsymbol{#2}{#3}}}
\def\@Pimathsymbol#1#2{\mathchoice
  {\@Pim@thsymbol{#1}{#2}\tf@size}
  {\@Pim@thsymbol{#1}{#2}\tf@size}
  {\@Pim@thsymbol{#1}{#2}\sf@size}
  {\@Pim@thsymbol{#1}{#2}\ssf@size}}
\def\@Pim@thsymbol#1#2#3{%
  \mbox{\fontsize{#3}{#3}\Pisymbol{#1}{#2}}}
\makeatother
% the next two lines are needed to avoid LaTeX substituting upright from another font
\input{utxmia.fd}
\DeclareFontShape{U}{txmia}{m}{n}{<->ssub * txmia/m/it}{}
% you may also want
\DeclareFontShape{U}{txmia}{bx}{n}{<->ssub * txmia/bx/it}{}
% just in case
%\DeclareFontShape{U}{txmia}{l}{n}{<->ssub * txmia/l/it}{}
%\DeclareFontShape{U}{txmia}{b}{n}{<->ssub * txmia/b/it}{}
% plus info from Alan Munn at https://tex.stackexchange.com/questions/290165/how-do-i-get-a-nicer-lambda?noredirect=1#comment702120_290165
\newcommand{\pilambdaup}{\Pimathsymbol[\mathord]{txmia}{21}}
\renewcommand{\lambda}{\pilambdaup}
\renewcommand{\tilde}{\widetilde}
\DeclareMathOperator*{\esssup}{ess\,sup}
\newcommand{\bluecheck}{}%
\DeclareRobustCommand{\bluecheck}{%
  \tikz\fill[scale=0.4, color=blue]
  (0,.35) -- (.25,0) -- (1,.7) -- (.25,.15) -- cycle;%
}


\usepackage{tikz}
\newcommand*{\DashedArrow}[1][]{\mathbin{\tikz [baseline=-0.25ex,-latex, dashed,#1] \draw [#1] (0pt,0.5ex) -- (1.3em,0.5ex);}}

\newcommand{\C}{\mathbb{C}}	
\newcommand{\F}{\mathbb{F}}
\newcommand{\N}{\mathbb{N}}
\newcommand{\Q}{\mathbb{Q}}
\newcommand{\R}{\mathbb{R}}
\newcommand{\Z}{\mathbb{Z}}



\title{Theory of Numbers}
\author{Eric Liu}
\date{}
\begin{document}
\maketitle
\newpage % or \cleardoublepage
% \pdfbookmark[<level>]{<title>}{<dest>}
\pdfbookmark[section]{\contentsname}{toc}

\tableofcontents
\pagebreak
\chapter{Groups}
\section{Isomorphism theorems}
Let $M$ be a set equipped with a binary operation $M\times M \rightarrow M$. We say $M$ is a \textbf{monoid} if the binary operation is associative and there exists a two-sided identity $e \in M$. 
\begin{example}
Defining $(x,y)\mapsto y$, we see that the operation is associative and every element is a left identity, but no element is a right identity unless $\abso{M}=1$. This is an example why identity must be two-sided. 
\end{example}
Because the identity of a monoid is defined to be two-sided, clearly it must be unique.  Suppose every element of monoid $M$ has a left inverse. Fix $x \in M$. Let $x^{-1}\in M$ be a left inverse of $x$. To see that  $x^{-1}$ is also a right inverse of $x$, let  $(x^{-1})^{-1}\in M$ be a left inverse of $x$ and use  
\begin{align*}
  (x^{-1})^{-1}=(x^{-1})^{-1}e=(x^{-1})^{-1}(x^{-1}x)= ((x^{-1})^{-1}x^{-1})x= ex=x
\end{align*}
to deduce
\begin{align*}
xx^{-1}=(x^{-1})^{-1}x^{-1}= e
\end{align*}
In other words, if we require every element of a monoid $M$ to has a left inverse, then immediately every left inverse upgrades to a right inverse. In such case, we call $M$ a  \textbf{group}. Notice that inverses of elements of a group are clearly unique. \\

Unlike the category of monoids, the category of groups behaves much better. Given two groups $G,H$ and a function  $\phi : G\rightarrow H$, if $\phi$ respects the binary operation, then $\phi$ also respects the identity:
\begin{align*}
e_H = (\phi (x)^{-1})\phi (x) = (\phi(x)^{-1}) \phi(x e_G) =  (\phi (x)^{-1} \phi (x)) \phi (e_G)=\phi (e_G)
\end{align*}
which implies that $\phi$ must also respect inverse. In such case, we call $\phi$ a \textbf{group homomorphism}. Given a subset $H \subseteq G$ closed under the binary operation, if $H$ forms a group itself, then since the set inclusion $H \hookrightarrow G$ forms a group homomorphism, we have $e_H=e_G$, and thus $x^{-1}$ in $H,G$ are the same element. \\

 
In this note, by a \textbf{subgroup} $H$ of $G$, we mean an injective group homomorphism $H \hookrightarrow G $. Clearly, a subset of $G$ forms a subgroup if and only it is closed under both the binary operation and inverse. Note that one of the key basic property of subgroup $H \subseteq G$ is that if $g \not \in H$, then $hg \not \in H$, since otherwise $g=h^{-1}hg \in H$. \\







Let $S$ be a subset of $G$. The group of \textbf{words} in $S$ is clearly the smallest subgroup of $G$ containing $S$. We say this subgroup is \textbf{generated} by $S$. If $G$ is generated by a single element, we say $G$ is \textbf{cyclic}. Let $x \in G$. The \textbf{order} of $G$ is the cardinality of $G$, and the order of  $x$ is the cardinality of the cyclic subgroup $\langle x\rangle \subseteq G$, or equivalently the infimum of the set of natural numbers $n$ that makes $x^n=e$. Clearly, finite cyclic groups of order $n$ are all isomorphic to  $\Z_n$.  \\


Let $G$ be a group and $H$ a subgroup of $G$. The \textbf{right cosets} $Hx$ are defined by $Hx\triangleq \set{hx \in G: h \in H}$. Clearly, when we define an equivalence relation in $G$ by setting: 
\begin{align*}
x\sim  y \overset{\triangle}{\iff } xy^{-1} \in H
\end{align*}
the equivalence class $[x]$ coincides with the right coset $Hx$. Note that if we partition $G$ using \textbf{left cosets}, the equivalence relation being $x\sim  y \iff  x^{-1}y\in H$, then the two partitions need not to be identical. 
\begin{example}
Let $H\triangleq \set{e,(1,2)}\subseteq S_3$. The right cosets are 
\begin{align*}
H(2,3)=\set{(2,3),(1,2,3)}\quad \text{ and }\quad  H(1,3)=\set{(1,3),(1,3,2)}
\end{align*}
while the left cosets being
\begin{align*}
(2,3)H= \set{(2,3),(1,3,2)} \quad \text{ and }\quad (1,3)H= \set{(1,3),(1,2,3)}
\end{align*}
\qed
\end{example}
However, as one may verify, we have a well-defined bijection $xH\mapsto Hx^{-1}$ between the sets of left cosets and right cosets of $H$. Therefore, we may define the \textbf{index} $\abso{G:H}$ of $H$ in  $G$ to be the cardinality of the collection of left cosets of $H$, without falling into the discussion of left and right. Moreover, by axiom of choice, there exists a set $T\subseteq G$ such that $\abso{T\cap xH}=1$ for all $x \in G$. Such $T$ clearly makes the set map  $T \times H \rightarrow G$ defined by: 
\begin{align*}
  (t,h)\mapsto th
\end{align*}
a bijection. This proves the \textbf{Lagrange's theorem}: 
\begin{align*}
\abso{G}= \abso{G:H} \cdot \abso{H}
\end{align*}
Consider a group $G$ of prime order. If $x \neq e\in G$, then clearly the cyclic subgroup $\langle x\rangle $ must be $G$ by Lagrange's theorem.  \\



Because the inverse of an injective group homomorphism forms a group homomorphism, we know the set $\operatorname{Aut}(G)$ of automorphisms of $G$ forms a group. We say $\pfi \in \operatorname{Aut}(G)$ is an \textbf{inner automorphism} if  $\pfi $ takes the form $x\mapsto  gxg^{-1}$ for some fixed $g \in G$. We say two elements  $x,y \in G$ are \textbf{conjugated} if there exists some inner automorphism that maps $x$ to $y$. Clearly conjugacy forms a equivalence relation. We call its classes \textbf{conjugacy classes}. 










\begin{example}
  Consider $G\triangleq \operatorname{GL}_2(\R)$  and consider: 
\begin{align*}
H\triangleq \set{ \begin{pmatrix}
    1 & n \\
    0 & 1
\end{pmatrix} \in \operatorname{GL}_2(\R) : n \inz}  \quad \text{ and }\quad g\triangleq \begin{pmatrix} 
 2 & 0 \\
 0 & 1
\end{pmatrix} \in \operatorname{GL}_2(\R)
\end{align*}
Note that $gHg^{-1}\subset H $. In other words, inner automorphisms can maps a subgroup $H$ into a subgroup strictly contained by $H$ if  $G$ is infinite. 
\end{example}


\begin{equiv_def}
\textbf{(Normal subgroups)} Let $G$ be a group and $N$ a subgroup. We say $N$ is a \textbf{normal subgroup} of $G$ if any of the followings hold true: 
\begin{enumerate}[label=(\roman*)]
  \item $\pfi  (N) \subseteq N$  for all $\pfi  \in \operatorname{Inn}(G)$
  \item $\pfi  (N)=N$ for all $\pfi  \in \operatorname{Inn}(G)$ 
  \item $xN=Nx$ for all  $x \in G$.  
  \item The set of all left cosets of $N$ equals the set of all right cosets of $N$. 
  \item $N$ is a union of conjugacy classes. 
  \item For all $n \in N$ and $x \in G$, their \textbf{commutator} $nxn^{-1}x^{-1} \in G$ lies in $N$.  
  \item For all $x,y \in G$, we have $xy\in N \iff  yx \in N$. 
\end{enumerate}
\end{equiv_def}
\begin{proof}
  (i)$\implies $(ii): Let $\pfi  \in \operatorname{Inn}(G)$. By premise, $\pfi  (N)\subseteq N$ and $\pfi ^{-1}(N)\subseteq N$. Applying  $\pfi $ to both side of $\pfi ^{-1}(N)\subseteq N$, we have $\pfi (N)\subseteq N \subseteq \pfi  (N)$, as desired. \\

(ii)$\implies $(iii): Consider the automorphisms:  
\begin{align*}
 \pfi_{L,x}(g)= xg\quad \text{ and }\quad \pfi _{L,x^{-1}}(g)=x^{-1}g \quad \text{ and }\quad \pfi _{R,x}(g)= gx
\end{align*}
 Because $\pfi _{L,x^{-1}}\circ \pfi _{R,x} \in \operatorname{Inn}(G)$, by premise we have:
\begin{align*}
xN= \pfi _{L,x}(N)= \pfi _{L,x}\circ \pfi _{L,x^{-1}}\circ \pfi _{R,x}(N)= \pfi _{R,x}(N)=Nx 
\end{align*}

(iii)$\implies $(iv) is clear. (iv)$\implies $(iii): Let $x\in G$. By premise, there exists some $y\in G$ that makes $xN=Ny$. Let $x=ny$. The proof then follows from noting 
\begin{align*}
xN=Ny=N(n^{-1}x)=Nx
\end{align*}
(iii)$\implies $(v): Let $n \in N$ and $x \in G$. We are required to show $xnx^{-1} \in N$. Because $xN=NX$, we know  $xn=\tilde{n}x$ for some $\tilde{n}\in N$. This implies 
\begin{align*}
xnx^{-1}= \tilde{n}xx^{-1}=\tilde{n}\in N  
\end{align*}
(v)$\implies $(vi): Fix $n\in N$ and $x\in G$. By premise, $xn^{-1}x^{-1} \in N$. Therefore, $n(xn^{-1}x^{-1})\in N$, as desired.\\


(vi)$\implies $(vii): Let $xy \in N$. To see $yx$ also belong to $N$, observe: 
\begin{align*}
 (xy)^{-1}(yx) =(xy)^{-1}x^{-1}xyx=[xy,x] \in N
\end{align*}
(viii)$\implies $(i): Let $n \in N$ and $x\in G$. Because $(nx)x^{-1}=n \in N$, by premise we have $x^{-1}nx \in N$, as desired.
\end{proof}



From the point of view of inner automorphism, we see that it is well-defined whether an element $g\in G$ \textbf{normalize} a subset $S\subseteq G$, independent of left and right.  Because of the independence, For each subset $S\subseteq G$, we see that the set of elements $g\in G$ that normalize $S$ forms a group, called the \textbf{normalizer} of $S$. Note that if $g$ normalize  $S$, then $gS = Sg$.\\

As we shall show, normal subgroup are what we may perform quotient on in the category of groups. Let $G$ be a group and $N\subseteq  G$ a subgroup. We say a group homomorphism $G \rightarrow   G\quotient N$ that vanishes on $N$ satisfies the \textbf{universal property of quotient group $G\quotient N$} if 
\begin{enumerate}[label=(\roman*)]
  \item it vanishes on $N$. \textbf{(Group condition)}
  \item for all group homomorphism $f:G\rightarrow H$ that vanishes on $N$ there exist a unique group homomorphism $\tilde{f} :G\quotient N\rightarrow H$ that makes the diagram: 
% https://q.uiver.app/#q=WzAsMyxbMiwwLCJHXFxxdW90aWVudCBOICJdLFswLDAsIkciXSxbMiwyLCJMIl0sWzAsMiwiZyJdLFsxLDIsImYiLDJdLFsxLDAsIiIsMCx7InN0eWxlIjp7ImhlYWQiOnsibmFtZSI6ImVwaSJ9fX1dXQ==
\[\begin{tikzcd}
	G && {G\quotient N } \\
	\\
	&& H 
	\arrow[two heads, from=1-1, to=1-3]
	\arrow["f"', from=1-1, to=3-3]
	\arrow["\tilde{f} ", from=1-3, to=3-3]
\end{tikzcd}\]
commute. \textbf{(Universality)}
\end{enumerate}




\begin{theorem}
\textbf{(The first isomorphism theorem for groups)} Let $N$ be a subgroup of  $G$ and $j:G \rightarrow G\quotient N$ satisfies the universal property. Then $j$ is surjective with kernel  $N$.\end{theorem}
\begin{proof}

\end{proof}
Because the kernel of a group homomorphism is clearly normal, if $N$ is not normal, then there can not be a pair $G\rightarrow G\quotient N$ that satisfies the universal property.  If any things, this is the "reason" why normal subgroups are what meant to be quotiented in the category of group. 
\begin{corollary}
\textbf{(The first isomorphism theorem)} 
\end{corollary}
\begin{proof}

\end{proof}


\begin{example}
$G\triangleq S_3$. $S\triangleq  \langle (1,2)\rangle $ and $H\triangleq \langle (2,3)\rangle $. $SH$ doesn't form a group. $(2,3)(1,2)\not \in SH$. 
\end{example}
Second isomorphism theorem. 

Third isomorphism theorem. 

Correspondence theorem.


Because $ \phi\circ \pfi _g \circ \phi^{-1} = \pfi _{\phi(g)}$, we know $\operatorname{Inn}(G)$ forms a normal subgroup of $\operatorname{Aut}(G)$. \\
\section{Group action}
Let $G$ be a group and $X$ a set. If we say  $G$ \textbf{acts} on $X$ we are defining a function $G \times X\rightarrow X$ such that 
\begin{enumerate}[label=(\roman*)]
  \item $e\cdot x=x$ for all $x\in X$. 
  \item $(gh)\cdot x= g \cdot (h \cdot x)$ for all $g,h \in G$. 
\end{enumerate}
Because groups admit inverses, a $G$-action is in fact a group homomorphism $G \rightarrow \operatorname{Sym}(X)$. The trivial action then correspond to the trivial group homomorphism.  An action is \textbf{faithful} if it is injective. \\


Show that $Z(G)\subseteq \operatorname{Ker} \theta$ if and only if $\theta$ is faithful.  \\


An action is \textbf{free} if $g\cdot x=x$ for a $x\in X$ implies $g=e$. Note that the isomorphism $\operatorname{Sym}(X)\rightarrow \operatorname{Sym}(X)$ is always injective but never free unless $\abso{X}\leq 2$. The action is \textbf{transitive} if for any $x,y \in X$, there always exists some $g \in G$ such that $y= g \cdot x$. An action is \textbf{regular} if it is both free and transitive.    \\

\end{document}
