\documentclass{report}
%%%%%%%%%%%%%% macros.tex %%%%%%%%%%%%%%
% Place your custom macros here, if any.

%%%%%%%%%%%%%% letterfonts.tex %%%%%%%%%%%%%%
% Place your font setup here, if any.

%%%%%%%%%%%%%% preamble.tex %%%%%%%%%%%%%%
\usepackage[T1]{fontenc}
\usepackage{lmodern}
\usepackage{etoolbox}
\usepackage{pdfpages}
\usepackage{transparent}
\usepackage[utf8]{inputenc}
\usepackage[english]{babel}

% Page Setup
\usepackage[tmargin=2cm, rmargin=0.5in, lmargin=0.5in, bmargin=80pt, footskip=.2in]{geometry}

% Mathematics
\usepackage{amsmath,amsfonts,amsthm,amssymb,mathtools}
\usepackage{xfrac}
\usepackage[makeroom]{cancel}
\usepackage{enumitem}
\usepackage{nameref}
\usepackage{multicol,array}
\usepackage{tikz-cd}
\usepackage[ruled,vlined,linesnumbered]{algorithm2e}

% Colors
\usepackage[dvipsnames]{xcolor}
\definecolor{myg}{RGB}{56, 140, 70}
\definecolor{myb}{RGB}{45, 111, 177}
\definecolor{myr}{RGB}{199, 68, 64}
% Define more colors here...

% Hyperlinks
\usepackage{bookmark}
\usepackage{hyperref}
\hypersetup{
    pdftitle={Assignment},
    colorlinks=true, linkcolor=doc!90,
    bookmarksnumbered=true,
    bookmarksopen=true
}

% Figures and Graphics
\usepackage{import}
\usepackage{svg}
\newcommand{\incfig}[1]{%
    \def\svgwidth{\columnwidth}
    \import{./figures/}{#1.pdf_tex}
}

% Text-related
\usepackage{blindtext}
\usepackage{fontsize}
\changefontsize[14]{14}
\setlength{\parindent}{0pt}

% Theorems and Definitions
\usepackage{amsthm}
\renewcommand\qedsymbol{$\blacksquare$}

% Define a new theorem style
\newtheoremstyle{mytheoremstyle}% name
  {}% Space above
  {}% Space below
  {\sffamily}% Body font
  {}% Indent amount
  {\bfseries}% Theorem head font
  {.}% Punctuation after theorem head
  {.5em}% Space after theorem head
  {}% Theorem head spec (can be left empty, meaning ‘normal’)

% Apply the new theorem style to theorem-like environments
\theoremstyle{mytheoremstyle}
\newtheorem{theorem}{Theorem}[section]
\newtheorem{definition}{Definition}[section]
\newtheorem{corollary}{Corollary}[section]
\newtheorem{lemma}{Lemma}[section]
\newtheorem{axiom}{Axiom}[section]

% tcolorbox Setup
\usepackage[most,many,breakable]{tcolorbox}

% Define custom tcolorbox environments here...

%================================
% EXAMPLE BOX
%================================
\newtcbtheorem[definition]{Example}{Example}
{%
    colback = myexamplebg,
    breakable,
    colframe = myexamplefr,
    coltitle = myexampleti,
    boxrule = 1pt,
    sharp corners,
    detach title,
    before upper=\tcbtitle\par\smallskip,
    fonttitle = \bfseries,
    description font = \mdseries,
    separator sign none,
    description delimiters parenthesis,
}
{ex}

%================================
% Solution BOX
%================================
\makeatletter
\newtcolorbox{solution}{enhanced,
	breakable,
	colback=white,
	colframe=myg!80!black,
	attach boxed title to top left={yshift*=-\tcboxedtitleheight},
	title=Solution,
	boxed title size=title,
	boxed title style={%
			sharp corners,
			rounded corners=northwest,
			colback=tcbcolframe,
			boxrule=0pt,
		},
	underlay boxed title={%
			\path[fill=tcbcolframe] (title.south west)--(title.south east)
			to[out=0, in=180] ([xshift=5mm]title.east)--
			(title.center-|frame.east)
			[rounded corners=\kvtcb@arc] |-
			(frame.north) -| cycle;
		},
}
\makeatother

%================================
% Question BOX
%================================
\makeatletter
\newtcbtheorem{question}{Question}{enhanced,
	breakable,
	colback=white,
	colframe=myb!80!black,
	attach boxed title to top left={yshift*=-\tcboxedtitleheight},
	fonttitle=\bfseries,
	title={#2},
	boxed title size=title,
	boxed title style={%
			sharp corners,
			rounded corners=northwest,
			colback=tcbcolframe,
			boxrule=0pt,
		},
	underlay boxed title={%
			\path[fill=tcbcolframe] (title.south west)--(title.south east)
			to[out=0, in=180] ([xshift=5mm]title.east)--
			(title.center-|frame.east)
			[rounded corners=\kvtcb@arc] |-
			(frame.north) -| cycle;
		},
	#1
}{def}
\makeatother
\makeatletter
\newtcbtheorem{qstion}{Question}{enhanced,
    breakable,
    colback=white,
    colframe=mygr,
    attach boxed title to top left={yshift*=-\tcboxedtitleheight},
    fonttitle=\bfseries,
    title={#2},
    boxed title size=title,
    boxed title style={%
        sharp corners,
        rounded corners=northwest,
        colback=tcbcolframe,
        boxrule=0pt,
    },
    underlay boxed title={%
        \path[fill=tcbcolframe] (title.south west)--(title.south east)
        to[out=0, in=180] ([xshift=5mm]title.east)--
        (title.center-|frame.east)
        [rounded corners=\kvtcb@arc] |-
        (frame.north) -| cycle;
    },
    #1
}{def}
\makeatother

%%%%%%%%%%%%%%%%%%%%%%%%%%%%%%%%%%%%%%%%%%%
% TABLE OF CONTENTS
%%%%%%%%%%%%%%%%%%%%%%%%%%%%%%%%%%%%%%%%%%%
\usepackage{tikz}
\definecolor{doc}{RGB}{0,60,110}
\usepackage{titletoc}
\contentsmargin{0cm}
\titlecontents{chapter}[14pc]
{\addvspace{30pt}%
	\begin{tikzpicture}[remember picture, overlay]%
		\draw[fill=doc!60,draw=doc!60] (-7,-.1) rectangle (-0.9,.5);%
		\pgftext[left,x=-4.5cm,y=0.2cm]{\color{white}\Large\sc\bfseries Chapter\ \thecontentslabel};%
	\end{tikzpicture}\color{doc!60}\large\sc\bfseries}%
{}
{}
{\;\titlerule\;\large\sc\bfseries Page \thecontentspage
	\begin{tikzpicture}[remember picture, overlay]
		\draw[fill=doc!60,draw=doc!60] (2pt,0) rectangle (4,0.1pt);
	\end{tikzpicture}}%
\titlecontents{section}[3.7pc]
{\addvspace{2pt}}
{\contentslabel[\thecontentslabel]{2pc}}
{}
{\hfill\small \thecontentspage}
[]
\titlecontents*{subsection}[3.7pc]
{\addvspace{-1pt}\small}
{}
{}
{\ --- \small\thecontentspage}
[ \textbullet\ ][]

\makeatletter
\renewcommand{\tableofcontents}{
	\chapter*{%
	  \vspace*{-20\p@}%
	  \begin{tikzpicture}[remember picture, overlay]%
		  \pgftext[right,x=15cm,y=0.2cm]{\color{doc!60}\Huge\sc\bfseries \contentsname};%
		  \draw[fill=doc!60,draw=doc!60] (13,-.75) rectangle (20,1);%
		  \clip (13,-.75) rectangle (20,1);
		  \pgftext[right,x=15cm,y=0.2cm]{\color{white}\Huge\sc\bfseries \contentsname};%
	  \end{tikzpicture}}%
	\@starttoc{toc}}
\makeatother

\newcommand{\liff}{\llap{$\iff$}}
\newcommand{\rap}[1]{\rrap{\text{ (#1)}}}
\newcommand{\red}[1]{\textcolor{red}{#1}}
\newcommand{\blue}[1]{\textcolor{blue}{#1}}
\newcommand{\vi}[1]{\textcolor{violet}{#1}}
\newcommand{\teal}[1]{\textcolor{teal}{#1}}
\newcommand{\tCaC}{\text{ \CaC }}
\newcommand{\CaC}{\red{CaC} }
\newcommand{\As}[1]{Assume \red{#1}}
\newcommand{\vdone}{\vi{\text{ (done) }}}
\newcommand{\bdone}{\blue{\text{ (done) }}}
\newcommand{\tdone}{\teal{\text{ (done) }}}
\newcommand{\set}[1]{\{ #1 \}}
\newcommand{\inS}{\in S}
\newcommand{\inF}{\in\F}
\newcommand{\inE}{\in E}
\newcommand{\inA}{\in A}
\newcommand{\inB}{\in B}
\newcommand{\inC}{\in C}
\newcommand{\inU}{\in U}

\newcommand{\C}{\mathbb{C}}	
\renewcommand{\H}{\mathbb{H}}
\newcommand{\F}{\mathbb{F}}
\newcommand{\N}{\mathbb{N}}
\newcommand{\Q}{\mathbb{Q}}
\newcommand{\R}{\mathbb{R}}
\newcommand{\Z}{\mathbb{Z}}
\renewcommand{\P}{\mathbb{P}}
\renewcommand{\S}{\mathbb{S}}
\newcommand{\A}{\mathbb{A}}
\newcommand{\RP}{\R P}


\title{Theory of Numbers}
\author{Eric Liu}
\date{}
\begin{document}
\maketitle
\newpage % or \cleardoublepage
% \pdfbookmark[<level>]{<title>}{<dest>}
\pdfbookmark[section]{\contentsname}{toc}

\tableofcontents
\pagebreak
\chapter{Groups}
\section{Subgroups}
Let $G$ be a group, a \textbf{subgroup} of $G$ is a group $H$ together with an injective group homomorphism $H \longhookrightarrow G$. Clearly, if $H \subseteq G$ satisfies:
\begin{enumerate}[label=(\roman*)]
  \item $e \in H$ 
  \item $xy \in H$ for all $x,y \in H$ 
  \item $x^{-1} \in H$ for all $x \in H$ 
\end{enumerate}
then the set inclusion makes $H$ a subgroup of $G$. The easiest spotted subgroups of a group $G$ are perhaps the  \textbf{cyclic subgroups}:  
\begin{align*}
\langle x\rangle \triangleq \set{x^{n}\in G: n\inz}
\end{align*}
namely, the smallest subgroup of $G$ containing $x$. Note that $G$ is said to be \textbf{cyclic} if $G= \langle x\rangle $ for some $x \in G$.  Let $G$ be a group, and $H$ a subgroup of  $G$. The \textbf{right cosets} $Hx$ are defined by $Hx\triangleq \set{hx \in G: h \in H}$. Clearly, when we define an equivalence relation in $G$ by setting: 
\begin{align*}
x\sim  y \overset{\triangle}{\iff } xy^{-1} \in H
\end{align*}
the equivalence class $[x]$ coincides with the right coset $Hx$. Note that if we partition $G$ using \textbf{left cosets}, the equivalence relation being $x\sim  y \iff  x^{-1}y\in H$, then the two partitions need not to be identical. 
\begin{example}
Let $H\triangleq \set{e,(1,2)}\subseteq S_3$. The right cosets are 
\begin{align*}
H(2,3)=\set{(2,3),(1,2,3)}\quad \text{ and }\quad  H(1,3)=\set{(1,3),(1,3,2)}
\end{align*}
while the left cosets being
\begin{align*}
(2,3)H= \set{(2,3),(1,3,2)} \quad \text{ and }\quad (1,3)H= \set{(1,3),(1,2,3)}
\end{align*}
\qed
\end{example}
However, as one may verify, we have a well-defined bijection $xH\mapsto Hx^{-1}$ between the sets of left cosets and right cosets of $H$. Therefore, we may define the \textbf{index} $\abso{G:H}$ of $H$ in  $G$ to be the cardinality of the collection of left cosets of $H$, without falling into the discussion of left and right. Moreover, by axiom of choice, there exists a set $T\subseteq G$ such that $\abso{T\cap xH}=1$ for all $x \in G$. Such $T$ clearly makes the set map  $T \times H \rightarrow G$ defined by: 
\begin{align*}
  (t,h)\mapsto th
\end{align*}
a bijection. This proves the \textbf{Lagrange's theorem}: 
\begin{align*}
\abso{G}= \abso{G:H} \cdot \abso{H}
\end{align*}
\begin{theorem}
\textbf{(Structure theorems of finite groups)} Let $G$ be a group and $x \in G$. The \textbf{order} of $G$ and  $x$ are respectively the cardinality $G$ and  $\langle x\rangle $. We denote them by $\abso{G},\operatorname{ord}(G)$, and $\operatorname{ord}(x)$. We have the followings: 
\begin{enumerate}[label=(\roman*)]
  \item If the order of $x$ is finite, then it is the smallest natural number $n$ that makes $x^n=e$. 
  \item  If $G$ is finite, then $\operatorname{ord}(x)$ divides $\abso{G}$. 
  \item If $G$ is finite cyclic $\langle x\rangle $, then for all  
  \item If $\abso{G}=p$, then it is cyclic. 
\end{enumerate}




\end{theorem}
\begin{proof}

\end{proof}





Consider a group $G$ of prime order. If $x \neq e\in G$, then clearly the cyclic subgroup $\langle x\rangle $ must be $G$ by Lagrange's theorem.  



\begin{equiv_def}
\textbf{(Normal subgroups)} Let $G$ be a group and $N$ a subgroup. We say $N$ is a \textbf{normal subgroup} of $G$ if any of the followings hold true: 
\begin{enumerate}[label=(\roman*)]
  \item $xNx^{-1}\subseteq N$ for all $x\in G$.   
  \item $xN x^{-1}=N$ for all $x \in G$ 
\end{enumerate}
\end{equiv_def}
\begin{proof}

\end{proof}



\section{Group homomorphisms}
Let $G$ be a group. There are essentially two ways to embed $G$ into $\operatorname{Aut}(G)$: 
\begin{align*}
x \mapsto  \left( y \mapsto  xyx^{-1} \right)\quad \text{ and }\quad x\mapsto \left( y \mapsto x^{-1}yx  \right)
\end{align*}



For all $x \in G$, we say the image of $x$ under the homomorphism 
\begin{align*}
z \mapsto y^{-1}zy
\end{align*}
is the \textbf{conjugate} of $x$ by $y$. 
\section{Normal subgroups}





\end{document}
