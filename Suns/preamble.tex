%%%%%%%%%%%%%% preamble.tex %%%%%%%%%%%%%%
\usepackage[T1]{fontenc}
\usepackage{etoolbox}
% Page Setup
\usepackage[letterpaper, tmargin=2cm, rmargin=0.5in, lmargin=0.5in, bmargin=80pt, footskip=.2in]{geometry}
\usepackage{adjustbox}
\usepackage{graphicx}
\usepackage{tikz}
\usepackage{mathrsfs}
\usepackage{mdframed}
\usepackage[new]{old-arrows}

% Create a new toggle
\newtoggle{firstsection}

% Redefine the \chapter command to reset the toggle for each new chapter
\let\oldchapter\chapter
\renewcommand{\chapter}{\toggletrue{firstsection}\oldchapter}

% Redefine the \section command to check the toggle
\let\oldsection\section
\renewcommand{\section}{
    \iftoggle{firstsection}
    {\togglefalse{firstsection}} % If it's the first section, just switch off the toggle for next sections
    {\clearpage} % If it's not the first section, start a new page
    \oldsection
}

% Abstract Design

\usepackage{lipsum}

\renewenvironment{abstract}
 {% Start of environment
  \quotation
  \small
  \noindent
  \rule{\linewidth}{.5pt} % Draw the rule to match the linewidth
  \par\smallskip
  {\centering\bfseries\abstractname\par}\medskip
 }
 {% End of environment
  \par\noindent
  \rule{\linewidth}{.5pt} % Ensure the closing rule also matches
  \endquotation
 }

% Mathematics
\usepackage{amsmath,amsfonts,amsthm,amssymb,mathtools}
\usepackage{xfrac}
\usepackage[makeroom]{cancel}
\usepackage{enumitem}
\usepackage{nameref}
\usepackage{multicol,array}
\usepackage{tikz-cd}
\usepackage{array}
\usepackage{multirow}% http://ctan.org/pkg/multirow
\usepackage{graphicx}

% Colors
\usepackage[dvipsnames]{xcolor}
\definecolor{myg}{RGB}{56, 140, 70}
\definecolor{myb}{RGB}{45, 111, 177}
\definecolor{myr}{RGB}{199, 68, 64}
% Define more colors here...
\definecolor{olive}{HTML}{6B8E23}
\definecolor{orange}{HTML}{CC5500}
\definecolor{brown}{HTML}{8B4513}
% Hyperlinks
\usepackage{bookmark}
\usepackage[colorlinks=true,linkcolor=blue,urlcolor=blue,citecolor=blue,anchorcolor=blue]{hyperref}
\usepackage{xcolor}
\hypersetup{
    colorlinks,
    linkcolor={red!50!black},
    citecolor={blue!50!black},
    urlcolor={blue!80!black}
}

% Text-related
\usepackage{blindtext}
\usepackage{fontsize}
\changefontsize[14]{14}
\setlength{\parindent}{0pt}
\linespread{1.2}

% Theorems and Definitions
\usepackage{amsthm}
\renewcommand\qedsymbol{$\blacksquare$}

% Define a new theorem style
\newtheoremstyle{mytheoremstyle}% name
  {}% Space above
  {}% Space below
  {}% Body font
  {}% Indent amount
  {\bfseries}% Theorem head font
  {.}% Punctuation after theorem head
  {.5em}% Space after theorem head
  {}% Theorem head spec (can be left empty, meaning ‘normal’)

% Apply the new theorem style to theorem-like environments
\theoremstyle{mytheoremstyle}

\newtheorem{theorem}{Theorem}[section]  
\newtheorem{definition}[theorem]{Definition} 
\newtheorem{lemma}[theorem]{Lemma}  
\newtheorem{corollary}[theorem]{Corollary}
\newtheorem{axiom}[theorem]{Axiom}
\newtheorem{proposition}[theorem]{Proposition}
\newtheorem{equiv_def}[theorem]{Equivalent Definition}
\newtheorem{example}[theorem]{Example}

% tcolorbox Setup
\usepackage[most,many,breakable]{tcolorbox}

% Define custom tcolorbox environments here...

%================================
% EXAMPLE BOX
%================================
% After you have defined the style and other theorem environments
% \definecolor{myexamplebg}{RGB}{245, 245, 245} % Very light grey for background
% \definecolor{myexamplefr}{RGB}{120, 120, 120} % Medium grey for frame
% \definecolor{myexampleti}{RGB}{60, 60, 60}    % Darker grey for title
%
% \newtcbtheorem[]{Example}{Example}{
%     colback=myexamplebg,
%     breakable,
%     colframe=myexamplefr,
%     coltitle=myexampleti,
%     boxrule=1pt,
%     sharp corners,
%     detach title,
%     before upper=\tcbtitle\par\vspace{-20pt}, % Reduced the space after the title
%     fonttitle=\bfseries,
%     description font=\mdseries,
%     separator sign none,
%     description delimiters={}{}, % No delimiters around the title
% }{ex}
%================================
% Solution BOX
%================================
\makeatletter
\newtcolorbox{solution}{enhanced,
	breakable,
	colback=white,
	colframe=myg!80!black,
	attach boxed title to top left={yshift*=-\tcboxedtitleheight},
	title=Solution,
	boxed title size=title,
	boxed title style={%
			sharp corners,
			rounded corners=northwest,
			colback=tcbcolframe,
			boxrule=0pt,
		},
	underlay boxed title={%
			\path[fill=tcbcolframe] (title.south west)--(title.south east)
			to[out=0, in=180] ([xshift=5mm]title.east)--
			(title.center-|frame.east)
			[rounded corners=\kvtcb@arc] |-
			(frame.north) -| cycle;
		},
}
\makeatother

% %================================
% % Question BOX
% %================================
\makeatletter
\newtcbtheorem{question}{Question}{enhanced,
	breakable,
	colback=white,
	colframe=myb!80!black,
	attach boxed title to top left={yshift*=-\tcboxedtitleheight},
	fonttitle=\bfseries,
	title={#2},
	boxed title size=title,
	boxed title style={%
			sharp corners,
			rounded corners=northwest,
			colback=tcbcolframe,
			boxrule=0pt,
		},
	underlay boxed title={%
			\path[fill=tcbcolframe] (title.south west)--(title.south east)
			to[out=0, in=180] ([xshift=5mm]title.east)--
			(title.center-|frame.east)
			[rounded corners=\kvtcb@arc] |-
			(frame.north) -| cycle;
		},
	#1
}{def}
\makeatother
\makeatletter
\newtcbtheorem{qstion}{Question}{enhanced,
    breakable,
    colback=white,
    colframe=mygr,
    attach boxed title to top left={yshift*=-\tcboxedtitleheight},
    fonttitle=\bfseries,
    title={#2},
    boxed title size=title,
    boxed title style={%
        sharp corners,
        rounded corners=northwest,
        colback=tcbcolframe,
        boxrule=0pt,
    },
    underlay boxed title={%
        \path[fill=tcbcolframe] (title.south west)--(title.south east)
        to[out=0, in=180] ([xshift=5mm]title.east)--
        (title.center-|frame.east)
        [rounded corners=\kvtcb@arc] |-
        (frame.north) -| cycle;
    },
    #1
}{def}
\makeatother

% COMMUTATIVE DIAGRAM
%%%%%%%%%%%%%%%%%%%%%%%%%%%%%%%%%%%%%%%%%%%


\usepackage{tikz}

\usetikzlibrary{arrows} % Needed for custom arrow tips
\usetikzlibrary{arrows.meta} % Needed for custom arrow tips

\tikzcdset{
  doubletail/.style={
    arrows={tail reversed-tail}
  }
}
\tikzstyle{double_arrow} = [thick,<->,>=stealth]

\tikzset{
  broken double arrow/.style={
    thick,                    % line width
    dashed,                   % make it broken
    <->,                      % arrowheads at both ends
    >=Stealth[length=2pt,width=1pt] % thinner arrowheads           
  }
}

% TABLE OF CONTENTS
%%%%%%%%%%%%%%%%%%%%%%%%%%%%%%%%%%%%%%%%%%
\definecolor{doc}{RGB}{0,60,110}
\usepackage{titletoc}
\contentsmargin{0cm}
\titlecontents{chapter}[14pc]
{\addvspace{30pt}%
	\begin{tikzpicture}[remember picture, overlay]%
		\draw[fill=doc!60,draw=doc!60] (-7,-.1) rectangle (-0.9,.5);%
		\pgftext[left,x=-5.5cm,y=0.2cm]{\color{white}\Large\sc\bfseries Chapter\ \thecontentslabel};%
	\end{tikzpicture}\color{doc!60}\large\sc\bfseries}%
{}
{}
{\;\titlerule\;\large\sc\bfseries Page \thecontentspage
	\begin{tikzpicture}[remember picture, overlay]
		\draw[fill=doc!60,draw=doc!60] (2pt,0) rectangle (4,0.1pt);
	\end{tikzpicture}}%
\titlecontents{section}[3.7pc]
{\addvspace{2pt}}
{\contentslabel[\thecontentslabel]{3pc}}
{}
{\hfill\small \thecontentspage}
[]
\titlecontents*{subsection}[3.7pc]
{\addvspace{-1pt}\small}
{}
{}
{\ --- \small\thecontentspage}
[ \textbullet\ ][]

\makeatletter
\renewcommand{\tableofcontents}{
	\chapter*{%
	  \vspace*{-20\p@}%
	  \begin{tikzpicture}[remember picture, overlay]%
		  \pgftext[right,x=15cm,y=0.2cm]{\color{doc!60}\Huge\sc\bfseries \contentsname};%
		  \draw[fill=doc!60,draw=doc!60] (13,-.75) rectangle (20,1);%
		  \clip (13,-.75) rectangle (20,1);
		  \pgftext[right,x=15cm,y=0.2cm]{\color{white}\Huge\sc\bfseries \contentsname};%
	  \end{tikzpicture}}%
	\@starttoc{toc}}
\makeatother
