\documentclass{report}
%%%%%%%%%%%%%% macros.tex %%%%%%%%%%%%%%
% Place your custom macros here, if any.

%%%%%%%%%%%%%% letterfonts.tex %%%%%%%%%%%%%%
% Place your font setup here, if any.

%%%%%%%%%%%%%% preamble.tex %%%%%%%%%%%%%%
\usepackage[T1]{fontenc}
\usepackage{lmodern}
\usepackage{etoolbox}
\usepackage{pdfpages}
\usepackage{transparent}
\usepackage[utf8]{inputenc}
\usepackage[english]{babel}

% Page Setup
\usepackage[tmargin=2cm, rmargin=0.5in, lmargin=0.5in, bmargin=80pt, footskip=.2in]{geometry}

% Mathematics
\usepackage{amsmath,amsfonts,amsthm,amssymb,mathtools}
\usepackage{xfrac}
\usepackage[makeroom]{cancel}
\usepackage{enumitem}
\usepackage{nameref}
\usepackage{multicol,array}
\usepackage{tikz-cd}
\usepackage[ruled,vlined,linesnumbered]{algorithm2e}

% Colors
\usepackage[dvipsnames]{xcolor}
\definecolor{myg}{RGB}{56, 140, 70}
\definecolor{myb}{RGB}{45, 111, 177}
\definecolor{myr}{RGB}{199, 68, 64}
% Define more colors here...

% Hyperlinks
\usepackage{bookmark}
\usepackage{hyperref}
\hypersetup{
    pdftitle={Assignment},
    colorlinks=true, linkcolor=doc!90,
    bookmarksnumbered=true,
    bookmarksopen=true
}

% Figures and Graphics
\usepackage{import}
\usepackage{svg}
\newcommand{\incfig}[1]{%
    \def\svgwidth{\columnwidth}
    \import{./figures/}{#1.pdf_tex}
}

% Text-related
\usepackage{blindtext}
\usepackage{fontsize}
\changefontsize[14]{14}
\setlength{\parindent}{0pt}

% Theorems and Definitions
\usepackage{amsthm}
\renewcommand\qedsymbol{$\blacksquare$}

% Define a new theorem style
\newtheoremstyle{mytheoremstyle}% name
  {}% Space above
  {}% Space below
  {\sffamily}% Body font
  {}% Indent amount
  {\bfseries}% Theorem head font
  {.}% Punctuation after theorem head
  {.5em}% Space after theorem head
  {}% Theorem head spec (can be left empty, meaning ‘normal’)

% Apply the new theorem style to theorem-like environments
\theoremstyle{mytheoremstyle}
\newtheorem{theorem}{Theorem}[section]
\newtheorem{definition}{Definition}[section]
\newtheorem{corollary}{Corollary}[section]
\newtheorem{lemma}{Lemma}[section]
\newtheorem{axiom}{Axiom}[section]

% tcolorbox Setup
\usepackage[most,many,breakable]{tcolorbox}

% Define custom tcolorbox environments here...

%================================
% EXAMPLE BOX
%================================
\newtcbtheorem[definition]{Example}{Example}
{%
    colback = myexamplebg,
    breakable,
    colframe = myexamplefr,
    coltitle = myexampleti,
    boxrule = 1pt,
    sharp corners,
    detach title,
    before upper=\tcbtitle\par\smallskip,
    fonttitle = \bfseries,
    description font = \mdseries,
    separator sign none,
    description delimiters parenthesis,
}
{ex}

%================================
% Solution BOX
%================================
\makeatletter
\newtcolorbox{solution}{enhanced,
	breakable,
	colback=white,
	colframe=myg!80!black,
	attach boxed title to top left={yshift*=-\tcboxedtitleheight},
	title=Solution,
	boxed title size=title,
	boxed title style={%
			sharp corners,
			rounded corners=northwest,
			colback=tcbcolframe,
			boxrule=0pt,
		},
	underlay boxed title={%
			\path[fill=tcbcolframe] (title.south west)--(title.south east)
			to[out=0, in=180] ([xshift=5mm]title.east)--
			(title.center-|frame.east)
			[rounded corners=\kvtcb@arc] |-
			(frame.north) -| cycle;
		},
}
\makeatother

%================================
% Question BOX
%================================
\makeatletter
\newtcbtheorem{question}{Question}{enhanced,
	breakable,
	colback=white,
	colframe=myb!80!black,
	attach boxed title to top left={yshift*=-\tcboxedtitleheight},
	fonttitle=\bfseries,
	title={#2},
	boxed title size=title,
	boxed title style={%
			sharp corners,
			rounded corners=northwest,
			colback=tcbcolframe,
			boxrule=0pt,
		},
	underlay boxed title={%
			\path[fill=tcbcolframe] (title.south west)--(title.south east)
			to[out=0, in=180] ([xshift=5mm]title.east)--
			(title.center-|frame.east)
			[rounded corners=\kvtcb@arc] |-
			(frame.north) -| cycle;
		},
	#1
}{def}
\makeatother
\makeatletter
\newtcbtheorem{qstion}{Question}{enhanced,
    breakable,
    colback=white,
    colframe=mygr,
    attach boxed title to top left={yshift*=-\tcboxedtitleheight},
    fonttitle=\bfseries,
    title={#2},
    boxed title size=title,
    boxed title style={%
        sharp corners,
        rounded corners=northwest,
        colback=tcbcolframe,
        boxrule=0pt,
    },
    underlay boxed title={%
        \path[fill=tcbcolframe] (title.south west)--(title.south east)
        to[out=0, in=180] ([xshift=5mm]title.east)--
        (title.center-|frame.east)
        [rounded corners=\kvtcb@arc] |-
        (frame.north) -| cycle;
    },
    #1
}{def}
\makeatother

%%%%%%%%%%%%%%%%%%%%%%%%%%%%%%%%%%%%%%%%%%%
% TABLE OF CONTENTS
%%%%%%%%%%%%%%%%%%%%%%%%%%%%%%%%%%%%%%%%%%%
\usepackage{tikz}
\definecolor{doc}{RGB}{0,60,110}
\usepackage{titletoc}
\contentsmargin{0cm}
\titlecontents{chapter}[14pc]
{\addvspace{30pt}%
	\begin{tikzpicture}[remember picture, overlay]%
		\draw[fill=doc!60,draw=doc!60] (-7,-.1) rectangle (-0.9,.5);%
		\pgftext[left,x=-4.5cm,y=0.2cm]{\color{white}\Large\sc\bfseries Chapter\ \thecontentslabel};%
	\end{tikzpicture}\color{doc!60}\large\sc\bfseries}%
{}
{}
{\;\titlerule\;\large\sc\bfseries Page \thecontentspage
	\begin{tikzpicture}[remember picture, overlay]
		\draw[fill=doc!60,draw=doc!60] (2pt,0) rectangle (4,0.1pt);
	\end{tikzpicture}}%
\titlecontents{section}[3.7pc]
{\addvspace{2pt}}
{\contentslabel[\thecontentslabel]{2pc}}
{}
{\hfill\small \thecontentspage}
[]
\titlecontents*{subsection}[3.7pc]
{\addvspace{-1pt}\small}
{}
{}
{\ --- \small\thecontentspage}
[ \textbullet\ ][]

\makeatletter
\renewcommand{\tableofcontents}{
	\chapter*{%
	  \vspace*{-20\p@}%
	  \begin{tikzpicture}[remember picture, overlay]%
		  \pgftext[right,x=15cm,y=0.2cm]{\color{doc!60}\Huge\sc\bfseries \contentsname};%
		  \draw[fill=doc!60,draw=doc!60] (13,-.75) rectangle (20,1);%
		  \clip (13,-.75) rectangle (20,1);
		  \pgftext[right,x=15cm,y=0.2cm]{\color{white}\Huge\sc\bfseries \contentsname};%
	  \end{tikzpicture}}%
	\@starttoc{toc}}
\makeatother

\newcommand{\liff}{\llap{$\iff$}}
\newcommand{\rap}[1]{\rrap{\text{ (#1)}}}
\newcommand{\red}[1]{\textcolor{red}{#1}}
\newcommand{\blue}[1]{\textcolor{blue}{#1}}
\newcommand{\vi}[1]{\textcolor{violet}{#1}}
\newcommand{\teal}[1]{\textcolor{teal}{#1}}
\newcommand{\tCaC}{\text{ \CaC }}
\newcommand{\CaC}{\red{CaC} }
\newcommand{\As}[1]{Assume \red{#1}}
\newcommand{\vdone}{\vi{\text{ (done) }}}
\newcommand{\bdone}{\blue{\text{ (done) }}}
\newcommand{\tdone}{\teal{\text{ (done) }}}
\newcommand{\set}[1]{\{ #1 \}}
\newcommand{\inS}{\in S}
\newcommand{\inF}{\in\F}
\newcommand{\inE}{\in E}
\newcommand{\inA}{\in A}
\newcommand{\inB}{\in B}
\newcommand{\inC}{\in C}
\newcommand{\inU}{\in U}

\newcommand{\C}{\mathbb{C}}	
\renewcommand{\H}{\mathbb{H}}
\newcommand{\F}{\mathbb{F}}
\newcommand{\N}{\mathbb{N}}
\newcommand{\Q}{\mathbb{Q}}
\newcommand{\R}{\mathbb{R}}
\newcommand{\Z}{\mathbb{Z}}
\renewcommand{\P}{\mathbb{P}}
\renewcommand{\S}{\mathbb{S}}
\newcommand{\A}{\mathbb{A}}
\newcommand{\RP}{\R P}

%================================
% Abstract 
%================================

\usepackage{lipsum}

\renewenvironment{abstract}
 {% Start of environment
  \quotation
  \small
  \noindent
  \rule{\linewidth}{.5pt} % Draw the rule to match the linewidth
  \par\smallskip
  {\centering\bfseries\abstractname\par}\medskip
 }
 {% End of environment
  \par\noindent
  \rule{\linewidth}{.5pt} % Ensure the closing rule also matches
  \endquotation
 }


%================================
% Question BOX
%================================

\newcounter{qnum} % The counter name for Question box is qnum

\makeatletter
\newtcbtheorem[use counter=qnum]{question}{Question}{enhanced,
	breakable,
	colback=white,
	colframe=myb!80!black,
	attach boxed title to top left={yshift*=-\tcboxedtitleheight},
	fonttitle=\bfseries,
	title={#2},
	boxed title size=title,
	boxed title style={%
			sharp corners,
			rounded corners=northwest,
			colback=tcbcolframe,
			boxrule=0pt,
		},
	underlay boxed title={%
			\path[fill=tcbcolframe] (title.south west)--(title.south east)
			to[out=0, in=180] ([xshift=5mm]title.east)--
			(title.center-|frame.east)
			[rounded corners=\kvtcb@arc] |-
			(frame.north) -| cycle;
		},
	#1
}{def}
\makeatother
\makeatletter
\newtcbtheorem{qstion}{Question}{enhanced,
    breakable,
    colback=white,
    colframe=mygr,
    attach boxed title to top left={yshift*=-\tcboxedtitleheight},
    fonttitle=\bfseries,
    title={#2},
    boxed title size=title,
    boxed title style={%
        sharp corners,
        rounded corners=northwest,
        colback=tcbcolframe,
        boxrule=0pt,
    },
    underlay boxed title={%
        \path[fill=tcbcolframe] (title.south west)--(title.south east)
        to[out=0, in=180] ([xshift=5mm]title.east)--
        (title.center-|frame.east)
        [rounded corners=\kvtcb@arc] |-
        (frame.north) -| cycle;
    },
    #1
}{def}
\makeatother


% COMMUTATIVE DIAGRAM
%%%%%%%%%%%%%%%%%%%%%%%%%%%%%%%%%%%%%%%%%%%


\usepackage{tikz}

\usetikzlibrary{arrows} % Needed for custom arrow tips
\usetikzlibrary{arrows.meta} % Needed for custom arrow tips

\newcommand*{\DashedArrow}[1][]{\mathbin{\tikz [baseline=-0.25ex,-latex, dashed,#1] \draw [#1] (0pt,0.5ex) -- (1.3em,0.5ex);}}


\tikzcdset{
  doubletail/.style={
    arrows={tail reversed-tail}
  }
}
\tikzstyle{double_arrow} = [thick,<->,>=stealth]

\tikzset{
  broken double arrow/.style={
    thick,                    % line width
    dashed,                   % make it broken
    <->,                      % arrowheads at both ends
    >=Stealth[length=2pt,width=1pt] % thinner arrowheads           
  }
}




% TABLE OF CONTENTS
%%%%%%%%%%%%%%%%%%%%%%%%%%%%%%%%%%%%%%%%%%
\definecolor{doc}{RGB}{0,60,110}
\usepackage{titletoc}
\contentsmargin{0cm}
\titlecontents{chapter}[14pc]
{\addvspace{30pt}%
	\begin{tikzpicture}[remember picture, overlay]%
		\draw[fill=doc!60,draw=doc!60] (-7,-.1) rectangle (-0.9,.5);%
		\pgftext[left,x=-5.5cm,y=0.2cm]{\color{white}\Large\sc\bfseries Chapter\ \thecontentslabel};%
	\end{tikzpicture}\color{doc!60}\large\sc\bfseries}%
{}
{}
{\;\titlerule\;\large\sc\bfseries Page \thecontentspage
	\begin{tikzpicture}[remember picture, overlay]
		\draw[fill=doc!60,draw=doc!60] (2pt,0) rectangle (4,0.1pt);
	\end{tikzpicture}}%
\titlecontents{section}[3.7pc]
{\addvspace{2pt}}
{\contentslabel[\thecontentslabel]{3pc}}
{}
{\hfill\small \thecontentspage}
[]
\titlecontents*{subsection}[3.7pc]
{\addvspace{-1pt}\small}
{}
{}
{\ --- \small\thecontentspage}
[ \textbullet\ ][]

\makeatletter
\renewcommand{\tableofcontents}{
	\chapter*{%
	  \vspace*{-20\p@}%
	  \begin{tikzpicture}[remember picture, overlay]%
		  \pgftext[right,x=15cm,y=0.2cm]{\color{doc!60}\Huge\sc\bfseries \contentsname};%
		  \draw[fill=doc!60,draw=doc!60] (13,-.75) rectangle (20,1);%
		  \clip (13,-.75) rectangle (20,1);
		  \pgftext[right,x=15cm,y=0.2cm]{\color{white}\Huge\sc\bfseries \contentsname};%
	  \end{tikzpicture}}%
	\@starttoc{toc}}
\makeatother

\title{Suns}
\author{Eric Liu}
\date{}
\begin{document}
\maketitle
\newpage % or \cleardoublepage
% \pdfbookmark[<level>]{<title>}{<dest>}
\pdfbookmark[section]{\contentsname}{toc}

\tableofcontents
\pagebreak
\chapter{Groups}

\section{Definition of Groups}
Let $M$ be a set equipped with a binary operation $M\times M \rightarrow M$. We say $M$ is a \textbf{monoid} if the binary operation is associative and there exists a two-sided identity $e \in M$. 
\begin{Example}{Identity of  monoids is required to be two-sided}{}
Defining $(x,y)\mapsto y$, we see that the operation is associative and every element is a left identity, but no element is a right identity unless $\abso{M}=1$. This is an example why identity must be two-sided.  
\end{Example}

Because the identity of a monoid is defined to be two-sided, clearly it must be unique.  Suppose every element of monoid $M$ has a left inverse. Fix $x \in M$. Let $x^{-1}\in M$ be a left inverse of $x$. To see that  $x^{-1}$ is also a right inverse of $x$, let  $(x^{-1})^{-1}\in M$ be a left inverse of $x^{-1}$ and use  
\begin{align*}
  (x^{-1})^{-1}=(x^{-1})^{-1}e=(x^{-1})^{-1}(x^{-1}x)= ((x^{-1})^{-1}x^{-1})x= ex=x
\end{align*}
to deduce
\begin{align*}
xx^{-1}=(x^{-1})^{-1}x^{-1}= e
\end{align*}
In other words, if we require every element of a monoid $M$ to has a left inverse, then immediately every left inverse upgrades to a right inverse. In such case, we call $M$ a  \textbf{group}. Notice that inverses of elements of a group are clearly unique. 
\begin{theorem}
\textbf{(Group criteria)} If a binary operation $G \times G \rightarrow G$ is associative, has a left identity, and always has a left inverse, then $G$ forms a group. 
\end{theorem}
\begin{proof}
If  $y \in G$ is  \textbf{idempotent}, then it must be identity, since $y=(y^{-1}y)y=y^{-1}y=e$. Because of such, we see a left inverse is also a right inverse, since $(xx^{-1})(xx^{-1})=xe x^{-1}=xx^{-1}$. This then shows that the left identity is also a right identity, since $xe=x(x^{-1}x)=x$. 
\end{proof}
\begin{theorem}
\label{THwc}
\textbf{(Group criteria for finite set)} Let $\abso{G}\inn$. If the binary operation $G \times G \rightarrow G$ is associative and both cancellation laws holds: 
\begin{align*}
au=aw \implies u=w \quad \text{ and }\quad ua=wa \implies  u=w
\end{align*}
then $G$ forms a group. 
\end{theorem}
\begin{proof}
Because the set is finite, for all $a$, we may attach it with an natural number $n(a)$ such that  $a^{n(a)+1}=a$. Clearly, 
\begin{align*}
aa^{n(a)}b= a^{n(a)+1}b=ab=ab^{n(b)+1}=ab^{n(b)}b
\end{align*}
This then by cancellation laws implies $a^{n(a)}=b^{n(b)}$, which can be easily checked to be the identity.  
\end{proof}
\begin{theorem}
\textbf{(Subgroup criteria for finite subset)} Let $G$ be a group and $S \subseteq G$ be finite. If $S$ is closed under the binary operation, then $S$ forms a group. 
\end{theorem}
\begin{proof}
 Because $S$ is finite, for all $a \in S$, there exists $n(a)\inn$ such that 
\begin{align*}
a^{n(a)}a=a
\end{align*}
Multiplying both side with $a^{-1}$, we see that $a^{n(a)}=e$ and $a^{-1}=a^{n(a)} \in S$. 
\end{proof}
\begin{Example}{Euler's totient function}{}
By \myref{theorem}{THwc}, we see that the set of nonzero integer relatively prime to $n$ and modulo $n$ forms a group under multiplication modulo $n$, called the \textbf{multiplicative group of integer modulo $n$}, or equivalently the unit group of the ring $\Z_n$. This immediately shows that 
\begin{align*}
a^{\pfi  (a)} \equiv 1 \pmod{n}
\end{align*}
for all $a\inn$ coprime with  $n$, where the \textbf{totient function} $\pfi (a)$ is the number of natural numbers $\leq a$ and coprime with $a$. We now have \textbf{Fermat's little theorem} as a special case.  
\end{Example}
Unlike the category of monoids, the category of groups behaves much better. Given two groups $G,H$ and a function  $\phi : G\rightarrow H$, if $\phi$ respects the binary operation, then $\phi$ also respects the identity:
\begin{align*}
e_H = (\phi (x)^{-1})\phi (x) = (\phi(x)^{-1}) \phi(x e_G) =  (\phi (x)^{-1} \phi (x)) \phi (e_G)=\phi (e_G)
\end{align*}
which implies that $\phi$ must also respect inverse. In such case, we call $\phi$ a \textbf{group homomorphism}. In this note, by a \textbf{subgroup} $H$ of $G$, we mean an injective group homomorphism $H \hookrightarrow G $. Clearly, a subset of $G$ forms a subgroup if and only if it is closed under both the binary operation and inverse. Note that one of the key basic property of subgroup $H \leq  G$ is that if $g \not \in H$, then $hg \not \in H$, since otherwise $g=h^{-1}hg \in H$. \\


Let $S$ be a subset of $G$. The group of \textbf{words} in $S$:
\begin{align*}
\set{s_1^{\epsilon_1}\cdots s_n^{\epsilon _n}\in G: n\inn \cup  \set{0}\text{ and }s_i \in S\text{ and }\epsilon_i = \pm 1}
\end{align*}
is clearly the smallest subgroup of $G$ containing $S$. We say this subgroup is \textbf{generated} by $S$. If $G$ is generated by a single element, we say $G$ is \textbf{cyclic}. Let $x \in G$. The \textbf{order} of $G$ is the cardinality of $G$, and the order of  $x$ is the cardinality of the cyclic subgroup $\langle x\rangle \subseteq G$, or equivalently the infimum of the set of natural numbers $n$ that makes $x^n=e$. \\

Let $G$ be a group and $H$ a subgroup of $G$. The \textbf{right cosets} $Hx$ are defined by $Hx\triangleq \set{hx \in G: h \in H}$. Clearly, when we define an equivalence relation in $G$ by setting: 
\begin{align*}
x\sim  y \overset{\triangle}{\iff } xy^{-1} \in H
\end{align*}
the equivalence class $[x]$ coincides with the right coset $Hx$. Note that if we partition $G$ using \textbf{left cosets}, the equivalence relation being $x\sim  y \iff  x^{-1}y\in H$, then the two partitions need not to be identical. 
\begin{Example}{An non-normal subgroup}{} 
Let $H\triangleq \set{e,(1,2)}\subseteq S_3$. The right cosets are 
\begin{align*}
H(2,3)=\set{(2,3),(1,2,3)}\quad \text{ and }\quad  H(1,3)=\set{(1,3),(1,3,2)}
\end{align*}
while the left cosets being
\begin{align*}
(2,3)H= \set{(2,3),(1,3,2)} \quad \text{ and }\quad (1,3)H= \set{(1,3),(1,2,3)}
\end{align*}
\end{Example}
However, as one may verify, we have a well-defined bijection $xH\mapsto Hx^{-1}$ between the sets of left cosets and right cosets of $H$. Therefore, we may define the \textbf{index} $[G:H]$ of $H$ in  $G$ to be the cardinality of the collection of left cosets of $H$, without falling into the discussion of left and right. Moreover, let $K$ be a subgroup of  $H$, by axiom of choice, clearly we have: 
\begin{align*}
[G:K]= [G:H] \cdot [H:K]
\end{align*}
which gives \textbf{Lagrange's theorem}  
\begin{align*}
o(G)= [G:H] \cdot o(H)
\end{align*}
as a corollary.  \label{THLt}
\begin{Example}{Isomorphic subgroups can have different indices}{}
Note that every nontrivial subgroups of $\Z$ are isomorphic, yet they are of distinct index. However, subgroups $H\leq G$ isomorphic through an automorphism $\phi \in \operatorname{Aut}(G)$ must have the same index, since $xH\mapsto \phi (x)\phi (H)$ forms a well-defined bijection. 
\end{Example}
\begin{theorem}
\label{THof}
\textbf{(Order formula)} Let $H, K \leq G$. Then  
\begin{align*}
\abso{HK}\cdot o (H \cap K)=o(H) \cdot o(K)
\end{align*}
and 
\begin{align*}
[G:H \cap K]\leq [G:H]\cdot [G:K]
\end{align*}
If moreover that $H,K$ both have finite index, then 
\begin{align*}
\operatorname{lcm} \left([G:H],[G,K] \right) \leq [G: H \cap K]
\end{align*}
\end{theorem}
\begin{proof}
The first formula follows from checking that the natural map from the left coset space $K \quotient H \cap K$ to the left coset space $HK \quotient H$ forms a well-defined bijection. The second formula follows from checking that the natural map from the left coset space  $G \quotient H \cap K$ to the product $G \quotient H \times G \quotient K$ of left coset spaces forms a well-defined injection. The third formula follows from
\begin{align*}
[G:H\cap K]= [G:H] \cdot [H : H \cap K]= [G:K]\cdot [K:H \cap K]
\end{align*}
\end{proof}
\section{Group Action}
Let $G$ be a group and $X$ a set. If we say  $G$ \textbf{acts on $X$ from left}, we are defining a function $G \times X\rightarrow X$ such that 
\begin{enumerate}[label=(\roman*)]
  \item $e\cdot x=x$ for all $x\in X$. 
  \item $(gh)\cdot x= g \cdot (h \cdot x)$ for all $g,h \in G$. 
\end{enumerate}
Note that there is a difference between left action and right action, as $gh$ means $g \circ h$ in left action and means $h \circ g$ in right action. Because groups admit inverses, a $G$-action is in fact a group homomorphism $G \rightarrow \operatorname{Bij}(X)$.\\

Let $x \in X$. We call the set $\Gamma \triangleq \set{g\cdot x \in X: g\in G }$ the \textbf{orbit} of $x$. Clearly the set $\operatorname{Stab}(x)$ of all elements of $G$ that fixes $x$ forms a group, called the \textbf{stabilizer subgroup}. Consider the action left, and let $G \quotient \operatorname{Stab}(x)$ denote the left coset space. The fact that the obvious mapping between $G\quotient \operatorname{Stab}(x)$ and $\Gamma $ forms a bijection is called the \textbf{orbit-stabilizer theorem}, which relates the index of $\operatorname{Stab}(x)$ and the length of $\Gamma $:   \label{THost}
\begin{align*}
[G:\operatorname{Stab}(x)]= \abso{\Gamma }
\end{align*}
The two most important group action are \textbf{left multiplication} and \textbf{left conjugation}:   
\begin{align*}
g \cdot A \triangleq \set{ga \in G: a \in A}\quad \text{ and }\quad g \cdot A \triangleq \set{gag^{-1} \in G : a\in A} 
\end{align*}
on the power set of $G$. Clearly, orbit of a subgroup under left multiplication is its left coset space.
\begin{theorem}
\textbf{(Cauchy's theorem for finite group)}  Let $G$ be a finite group and $p$ a prime number that divides  $o(G)$. Then the number of elements of order divided by $p$ is a positive multiple of  $p$.  
\end{theorem}
\begin{proof}
The set $X$ of $p$-tuples  $(x_1,\dots ,x_p)$ that satisfies $x_1\cdots x_p=e$ clearly has cardinality $o(G)^{p-1}$. Consider the group action $C_p \rightarrow \operatorname{Bij}(X)$ defined by 
\begin{align*}
g \cdot (x_1,\dots ,x_p) \triangleq (x_p,x_1,\dots ,x_{p-1}),\quad \text{ where }C_p = \langle g\rangle 
\end{align*}
Notice that $x^p=e$ if and only if  $(x,\dots ,x) \in X$. Therefore the number of cardinality $1$ orbit equals to number of solution to $x^p=e$. By \customref{THost}{orbit-stabilizer theorem}, an orbit in $X$ either has cardinality $p$ or  $1$. Therefore, we may write
\begin{align*}
p\mid  o(G)^{p-1}=m+kp
\end{align*}
with $m$ the number of cardinality $1$ orbits and $k$ the number of cardinality  $p$ orbits. Clearly we have $p\mid m$, as desired.
\end{proof}
\section{Normal Subgroups}
Because the inverse of an injective group homomorphism forms a group homomorphism, we know  $\operatorname{Aut}(G)$  forms a group. We say $\pfi \in \operatorname{Aut}(G)$ is an \textbf{inner automorphism} if  $\pfi $ takes the form $x\mapsto  gxg^{-1}$ for some fixed $g \in G$. We say two elements  $x,y \in G$ are \textbf{conjugated} if there exists some inner automorphism that maps $x$ to $y$. Clearly conjugacy forms an equivalence relation. We call its classes \textbf{conjugacy classes}. \\


From the point of view of inner automorphism, we see that it is well-defined whether an element $g\in G$ \textbf{normalize} a subset $S\subseteq G$:
\begin{align*}
g Sg^{-1} = S
\end{align*}
independent of left and right.  Because of the independence, for each subset $S\subseteq G$, we see that the set of elements $g\in G$ that normalize $S$ forms a group, called the \textbf{normalizer} of $S$, in fact the stabilizer subgroup $\operatorname{Stab}(S)$ under the conjugacy action. 
\begin{Example}{Conjugation can send subgroups to proper subgroup}{} 
  Consider $G\triangleq \operatorname{GL}_2(\R)$  and consider: 
\begin{align*}
H\triangleq \set{ \begin{pmatrix}
    1 & n \\
    0 & 1
\end{pmatrix} \in \operatorname{GL}_2(\R) : n \inz}  \quad \text{ and }\quad g\triangleq \begin{pmatrix} 
 2 & 0 \\
 0 & 1
\end{pmatrix} \in \operatorname{GL}_2(\R)
\end{align*}
Note that $gHg^{-1} < H $. \end{Example}
Given $x,y \in G$, the notation $[x,y]\in G$ is called the \textbf{commutator of  $x$ and $y$}. In this note, we take the convention:  
\begin{align*}
[x,y]\triangleq xyx^{-1}y^{-1}
\end{align*}
The other convention is $[x,y]=x^{-1}y^{-1}xy$, and the differences lies in sides. In our convention, we see that $[x,y]\in H$ if and only if $Hxy=Hyx$, while the other convention leads us to $[x,y] \in H \iff  xyH=yxH$. However, because $[x,y]$ in our convention is just $[x^{-1},y^{-1}]$ in the other convention, if $H,K\leq G$, then the set $[H,K]$ is defined the same using either. In general, $[H,K]$ doesn't form a group. In fact, we clearly have $[H,K]=[K,H]^{-1}$. 
\begin{equiv_def}
\label{EDns}
\textbf{(Normal subgroups)} Let $N \leq G$. We say $N$ is a \textbf{normal subgroup} of $G$ if any of the followings hold true: 
\begin{enumerate}[label=(\roman*)]
  \item $\pfi  (N) \subseteq N$  for all $\pfi  \in \operatorname{Inn}(G)$
  \item $\pfi  (N)=N$ for all $\pfi  \in \operatorname{Inn}(G)$ 
  \item $xN=Nx$ for all  $x \in G$.  
  \item The set of all left cosets of $N$ equals the set of all right cosets of $N$. 
  \item $N$ is a union of conjugacy classes. 
  \item $[N,G]\subseteq N$.   
  \item For all $x,y \in G$, we have $xy\in N \iff  yx \in N$. 
\end{enumerate}
\end{equiv_def}
\begin{proof}
  (i)$\implies $(ii): Let $\pfi  \in \operatorname{Inn}(G)$. By premise, $\pfi  (N)\subseteq N$ and $\pfi ^{-1}(N)\subseteq N$. Applying  $\pfi $ to both side of $\pfi ^{-1}(N)\subseteq N$, we have $\pfi (N)\subseteq N \subseteq \pfi  (N)$, as desired. \\

(ii)$\implies $(iii): Consider the automorphisms:  
\begin{align*}
 \pfi_{L,x}(g)= xg\quad \text{ and }\quad \pfi _{L,x^{-1}}(g)=x^{-1}g \quad \text{ and }\quad \pfi _{R,x}(g)= gx
\end{align*}
 Because $\pfi _{L,x^{-1}}\circ \pfi _{R,x} \in \operatorname{Inn}(G)$, by premise we have:
\begin{align*}
xN= \pfi _{L,x}(N)= \pfi _{L,x}\circ \pfi _{L,x^{-1}}\circ \pfi _{R,x}(N)= \pfi _{R,x}(N)=Nx 
\end{align*}

(iii)$\implies $(iv) is clear. (iv)$\implies $(iii): Let $x\in G$. By premise, there exists some $y\in G$ that makes $xN=Ny$. Let $x=ny$. The proof then follows from noting 
\begin{align*}
xN=Ny=N(n^{-1}x)=Nx
\end{align*}
(iii)$\implies $(v): Let $n \in N$ and $x \in G$. We are required to show $xnx^{-1} \in N$. Because $xN=Nx$, we know  $xn=\tilde{n}x$ for some $\tilde{n}\in N$. This implies 
\begin{align*}
xnx^{-1}= \tilde{n}xx^{-1}=\tilde{n}\in N  
\end{align*}
(v)$\implies $(vi): Fix $n\in N$ and $x\in G$. By premise, $xn^{-1}x^{-1} \in N$. Therefore, $n(xn^{-1}x^{-1})\in N$, as desired.\\


(vi)$\implies $(vii): Let $xy \in N$. To see $yx$ also belong to $N$, observe: 
\begin{align*}
 (xy)^{-1}(yx) =(xy)^{-1}x^{-1}xyx=[xy,x] \in N
\end{align*}
(viii)$\implies $(i): Let $n \in N$ and $x\in G$. Because $(nx)x^{-1}=n \in N$, by premise we have $x^{-1}nx \in N$, as desired.
\end{proof}
Notably, since given the "conjugate by left" $\phi_g \in \operatorname{Inn}(G)$ and $\pfi  \in \operatorname{Aut}(G)$, we have $\pfi  \circ \phi_g \circ \pfi ^{-1}=\phi_{\pfi (g)}$, we see that $\operatorname{Inn}(G)\trianglelefteq \operatorname{Aut}(G)$. We call  $\operatorname{Aut}(G)\quotient \operatorname{Inn}(G)$ the \textbf{outer automorphism group} of $G$. 
\begin{Example}{Dedekind and Hamiltonian group}{}
A group is said to be \textbf{Dedekind} if all of its subgroups are normal. Clearly every abelian group is Dedekind. Non-abelian Dedekind groups are called \textbf{Hamiltonian}. The simplest Hamiltonian group is the \textbf{quaternion group} $Q_8$, which is the group of the quaternions under multiplication: 
\begin{align*}
Q_8 \triangleq \set{1,i,j,k,-1,-i,-j,-k}
\end{align*}
Note that $Q_8$ is Dedekind because every nontrivial element is of order $4$. Clearly, $Q_8$ has center $\set{\pm 1}$. Because $Z(Q_8)$ has index $4$, we know  $Q_8 \quotient Z(Q_8)$ is abelian. This then by  \customref{THtit}{correspondence theorem} implies  $Q_8^{(1)} \leq Z(Q_8)$. Because $Q_8$ is non-abelian, we now see  $Q_8^{(1)}=Z(Q_8)= \set{\pm 1}$. Note that $Q_8$ clearly has the conjugacy classes 
\begin{align*}
\set{1}\cup  \set{-1} \cup  \set{\pm i} \cup  \set{\pm j}\cup  \set{\pm k}
\end{align*}
Interestingly, $Q_8$ is a group that contains a smallest non-trivial subgroup. Every non-trivial subgroup of  $Q_8$ must contains the group  $\set{\pm 1}$. 
\end{Example}
\section{Isomorphism Theorems}
 Let $N\trianglelefteq   G$. We say a group homomorphism $ \pi :G \rightarrow   G\quotient N$ satisfies the \textbf{universal property of quotient group $G\quotient N$} if 
\begin{enumerate}[label=(\roman*)]
  \item $\pi $ vanishes on $N$. \textbf{(Group condition)}
  \item For all group homomorphism $f:G\rightarrow H$ that vanishes on $N$ there exist a unique group homomorphism $\tilde{f} :G\quotient N\rightarrow H$ that makes the diagram: 
% https://q.uiver.app/#q=WzAsMyxbMiwwLCJHXFxxdW90aWVudCBOICJdLFswLDAsIkciXSxbMiwyLCJMIl0sWzAsMiwiZyJdLFsxLDIsImYiLDJdLFsxLDAsIiIsMCx7InN0eWxlIjp7ImhlYWQiOnsibmFtZSI6ImVwaSJ9fX1dXQ==
\[\begin{tikzcd}
	G && {G\quotient N } \\
	\\
	&& H 
	\arrow["\pi",two heads, from=1-1, to=1-3]
	\arrow["f"', from=1-1, to=3-3]
	\arrow["\tilde{f} ", dashed,from=1-3, to=3-3]
\end{tikzcd}\]
commute. \textbf{(Universality)}
\end{enumerate}
\begin{theorem}
\label{THfit}
\textbf{(First isomorphism theorem for groups)} The group homomorphism $\pi : G \rightarrow G \quotient N$ is always surjective with kernel $N$. Let $f :G \rightarrow H$ be a group homomorphism. Then $\operatorname{ker}f$ is normal in $G$, and the induced homomorphism $\tilde{f}: G \quotient \operatorname{ker}f \rightarrow H$ is injective.   
\end{theorem}
\begin{proof}
They are consequences of the construction. 
\end{proof}
\begin{theorem}
\label{THtit}
\textbf{(Third isomorphism theorem and correspondence theorem for groups)} Let $N \trianglelefteq G$. The canonical projection $\pi : G \rightarrow G\quotient N$ gives rise to a bijection between the set of subgroups of $G$ that contains $N$ and the set of subgroups of $G \quotient N$. The bijection is moreover a bijection between the set of normal subgroups of $G$ that contains $N$ and the set of normal subgroups of $G \quotient N$. The bijection also maps normalizer of subgroups $\leq G$ that contains $N$ to the normalizer of the image of the subgroup.\\


In fact, given $K \trianglelefteq G$ that contains $N$, if we identify $K \quotient N$ as a subgroup of $G \quotient N$ the natural way: 
% https://q.uiver.app/#q=WzAsMyxbMCwwLCJLIl0sWzIsMCwiXFxmcmFje0t9e059Il0sWzIsMiwiXFxmcmFje0d9e059Il0sWzAsMl0sWzEsMiwiIiwyLHsic3R5bGUiOnsidGFpbCI6eyJuYW1lIjoiaG9vayIsInNpZGUiOiJ0b3AifX19XSxbMCwxLCJcXHBpIiwwLHsic3R5bGUiOnsiaGVhZCI6eyJuYW1lIjoiZXBpIn19fV1d
\[\begin{tikzcd}
	K && {\frac{K}{N}} \\
	\\
	&& {\frac{G}{N}}
	\arrow["\pi", two heads, from=1-1, to=1-3]
	\arrow[from=1-1, to=3-3]
	\arrow[hook, from=1-3, to=3-3]
\end{tikzcd}\]
then $\frac{K}{N}\trianglelefteq  \frac{G}{N}$ is the normal subgroup that corresponds to $K\trianglelefteq G$, and we have a natural isomorphism  $\frac{G}{K} \cong  \frac{\frac{G}{N}}{\frac{K}{N}}$. 
\end{theorem}
\begin{proof}
Routine. 
\end{proof}
\begin{theorem}
\label{THsitg}
\textbf{(Second isomorphism theorem)} Let $H\leq G$. If $K\leq N_G(H)$, then their product: 
\begin{align*}
HK\triangleq  \set{hk \in G: h \in H\text{ and } k \in K}
\end{align*}
forms a group (in fact, the subgroup generated by $H\cup K$) and is defined independent of left and right. Moreover, $H\trianglelefteq HK$ with $hkH=Hk$, and $H\cap K \trianglelefteq K$ with 
\begin{align*}
HK \quotient H \cong  K \quotient H \cap K \quad \text{ via } \quad  kH  \longleftrightarrow k (H \cap K) 
\end{align*}
\end{theorem}
\begin{proof}
To see $HK \subseteq KH$, simply observe $hk=k(k^{-1}hk)$. The converse inclusion is proved similarly. The fact that $HK$ forms a group now follows. The rest are clear. 
\end{proof}
\begin{Example}{Product of two subgroups}{} 
In general, product of two subgroups needs not to form a group. For example, consider the product of the subgroup $H$ generated by $(1,2) \in S_3$ and the subgroup $K$ generated by  $(2,3) \in S_3$. Since $(2,3)(1,2)\not \in HK$, we see $HK$ isn't a group.  \\

On the other hand, given two normal subgroups $N,M \trianglelefteq G$. By the \customref{EDns}{preserved-by-conjugations definition of normal subgroups}, clearly both $NM$ and  $N \cap M$ are normal in $G$.  
\end{Example}
\section{Free Group and Presentation}
\begin{equiv_def}
\textbf{(Core of a subgroup)} Let $H\leq G$. The largest subgroup of $H$ normal in  $G$ exists, called the \textbf{core} of $H$.  Let $\phi : G \rightarrow \operatorname{Bij}(G \quotient H)$ be the left multiplicative action on the left coset space of $H$. It is exactly:  
\begin{align*}
\operatorname{ker} \phi = \bigcap_{g \in G} H^g, \quad \text{ where }H^g \triangleq gHg^{-1}
\end{align*}
\end{equiv_def}
\begin{proof}
Routine. 
\end{proof}
As we will see, consideration of core is in fact useful in theory of finite group. 
\begin{theorem}
\textbf{(Properties of core)} Let $G$ be a finite group. Then 
\begin{enumerate}[label=(\roman*)]
  \item $[G:\operatorname{Core}(H)]$ divides $ \left([G:H] \right)!$, for all $H\leq G$ 
  \item Any proper subgroup of $G$ of smallest possible index is normal. 
\end{enumerate}
\end{theorem}
\begin{proof}
  (i) is a consequence of \customref{THfit}{first isomorphism theorem}, since we have an injective group homomorphism $G\quotient \operatorname{Core}(H)\longhookrightarrow \operatorname{Bij}(G \quotient H)$. (ii) follows from (i), since we would get $\operatorname{Core}(H)=H$.   
\end{proof}
\begin{equiv_def}
\textbf{(Normal closure)} Let $S \subseteq G$. Then 
\begin{align*}
\langle \set{s^g \in G: s \in S,g \in G}\rangle  = \bigcap_{S \subseteq N \trianglelefteq G} N
\end{align*}
is the smallest normal subgroup of $G$ that contains $S$, called the \textbf{normal closure} of $S$ in  $G$.  
\end{equiv_def}
\begin{proof}
The latter expression is clearly the smallest normal subgroup of $G$ that contains $S$.  $\leq $ part is clear. The only part we need to prove is $\geq $, which requires us to prove the former expression is normal, which is a consequence of computing 
\begin{align*}
h(g_1s_1g_1^{-1})^{\epsilon_1} \cdots (g_ns_ng_n^{-1})^{\epsilon_n} h^{-1}= (x_1s_1x_1^{-1})^{\epsilon _1} \cdots (x_ns_nx_n^{-1})^{\epsilon _n}
\end{align*}
where $x_i\triangleq hg_1$ if $\epsilon _i=1$ and $hg_i^{-1}$ if $\epsilon _i=-1$. 
\end{proof}
Let $S$ be a set. By the \textbf{free group generated by $S$}, we mean a group $F_S$ together with an injective function  $\diota:S \hookrightarrow F_S $ such that for all group $G$ and function  $f:S\rightarrow G$, there exists a unique group homomorphism $\tilde{f}: F_S \rightarrow G$ that makes the diagram:   
% https://q.uiver.app/#q=WzAsMyxbMCwwLCJTIl0sWzIsMCwiRl9TIl0sWzIsMiwiRyJdLFswLDEsIlxcaW90YSIsMCx7InN0eWxlIjp7InRhaWwiOnsibmFtZSI6Imhvb2siLCJzaWRlIjoidG9wIn19fV0sWzAsMiwiZiIsMl0sWzEsMiwiXFx0aWxkZXtmfSIsMCx7InN0eWxlIjp7ImJvZHkiOnsibmFtZSI6ImRhc2hlZCJ9fX1dXQ==
\[\begin{tikzcd}
	S && {F_S} \\
	\\
	&& G
	\arrow["\iota", hook, from=1-1, to=1-3]
	\arrow["f"', from=1-1, to=3-3]
	\arrow["{\tilde{f}}", dashed, from=1-3, to=3-3]
\end{tikzcd} \label{THp}\]
commutes. Given a set of \textbf{relators} $R \subseteq F_S$, by \textbf{presentation} $\langle S\mid R\rangle $, we mean the group $F_S \quotient \operatorname{ncl}_{F_S}(R)$. Since kernel is normal, such group clearly satisfies the universal property that for all group $G$ and function  $f:S \rightarrow G$ such that the kernel of induced group homomorphism $\tilde{f}:F_S \rightarrow G$ contains $R$, there exists a unique group homomorphism $\widehat{f}: \langle S \mid  R\rangle \rightarrow G$ that makes the diagram
% https://q.uiver.app/#q=WzAsMyxbMCwwLCJTIl0sWzIsMCwiXFxsYW5nbGUgUyBcXG1pZCBSXFxyYW5nbGUiXSxbMiwyLCJHIl0sWzAsMSwiXFxwaSBcXGNpcmMgXFxpb3RhIl0sWzAsMiwiZiIsMl0sWzEsMiwiXFxoYXR7Zn0iLDAseyJzdHlsZSI6eyJib2R5Ijp7Im5hbWUiOiJkYXNoZWQifX19XV0=
\[\begin{tikzcd}
	S && {\langle S \mid R\rangle} \\
	\\
	&& G
	\arrow["{\pi \circ \iota}", from=1-1, to=1-3]
	\arrow["f"', from=1-1, to=3-3]
	\arrow["{\widehat{f}}", dashed, from=1-3, to=3-3]
\end{tikzcd}\]
\begin{Example}{Dihedral group}{}
The \textbf{Dihedral group} $D_n$ is defined by 
\begin{align*}
D_n \triangleq \langle r,s \mid r^n =s^2=e, srs^{-1}=r^{-1}\rangle 
\end{align*} 
Clearly, every element can be written as $r^as^b$ with  $0\leq a \leq n-1$ and $b \in \set{0,1}$. This implies that $D_n$ has at most $2n$ number of elements. To see that $o(D_n)=2n$, one first consider the group homomorphism $D_n \rightarrow \operatorname{GL}_2(\C)$ induced by 
\begin{align*}
r \mapsto \begin{pmatrix} 
  \xi & 0 \\
  0 & \xi^{-1}
\end{pmatrix}\quad \text{ and }\quad s \mapsto \begin{pmatrix} 
  0 & 1 \\
  1 & 0
\end{pmatrix}
\end{align*}
where $\xi\triangleq \exp ( 2\pi i \quotient n)$. Because 
\begin{align*}
r^a= \begin{pmatrix} 
  \xi^a & 0 \\
  0 & \xi^{-a}
\end{pmatrix} \quad \text{ and }\quad  r^as = \begin{pmatrix} 
  0 & \xi^a \\
  \xi^{-a} & 0
\end{pmatrix}
\end{align*}
We now see that indeed $o(D_n)=2n$. Moreover, because from the relators, we have the formula 
\begin{align*}
  (r^as^b)(r^cs^d)= r^{a+(-1)^bc}s^{b+d}
\end{align*}
which by direct computation implies $D_n' = \langle r^{2}\rangle $. Suppose $r^as^b \in Z(D_{n})$ with $0 \leq a \leq n-1$ and $b \in \set{0,1}$. Then form the formula, we must have  
\begin{align}
\label{EQa-1b}
a+(-1)^b c\equiv c+(-1)^d a \pmod{n},\quad \text{ for all }c,d\inz 
\end{align}
Because $d$ can be arbitrary, this implies 
\begin{align*}
2a \equiv 0 \pmod{n}
\end{align*}
Clearly we have $D_1 \cong  C_2$ and $D_2 \cong  C_2 \times C_2$. Therefore, we from now on suppose $n\geq 3$. \\

Let $n$ be odd.  Then clearly  $a=0$. Since  $rs\neq sr$ in general, we see that we also must have $b=0$. We have shown that $D_n$ is centerless for odd $n\geq 3$. \\

Let $n$ be even. Then we must have $a \in \set{0,\frac{n}{2}}$. If $a=0$, then again because $rs \neq sr$ in general, we must have $b=0$. If  $a= \frac{n}{2}$, then from the \myref{equation}{EQa-1b}, regardless of $d$, we see 
\begin{align*}
  (-1)^b c \equiv c \pmod{n},\quad \text{ for all }c \inz
\end{align*}
which can only be true if $b=0$. We have shown that $D_n$ has center  $\set{e,r^{\frac{n}{2}}}$ for even $n\geq 3$. Note that again by direct computation, one can check that the pattern of conjugacy classes differ according to parity of $n$. Regardless of parity, for fixed $k\inz$, the set $\set{r^{\pm k} }$ form a conjugacy class. If $n$ is odd, then the rest  $\set{r^ts:1\leq t \leq n-1}$ forms a class. If $n$ is even, then the rest forms two classes: 
 \begin{align*}
\set{r^rs \in D_{n}: 1\leq t \leq n-1\text{ is even }}\cup  \set{r^ts \in D_{n}: 1 \leq t \leq n-1\text{ is odd }}
\end{align*}
\end{Example}
\section{Center and Commutator}
Let $S\subseteq G$. Clearly $N_G(S)$ is the largest subgroup in which $S$ is preserved by inner automorphism. Consider its \textbf{centralizer} $C_G(S)\triangleq \set{g \in G: gs=sg\text{ for all }s \in S}$. To see that $C_G(S)\trianglelefteq N_G(S)$, one simply observe that $C_G(S)$ is the kernel of the conjugacy action $N_G(S) \longrightarrow  \operatorname{Bij}(S)$, where $\operatorname{Bij}(S)$ can be replaced by $\operatorname{Aut}(S)$ if $S$ forms a subgroup  $G$. In such case, \customref{THfit}{first isomorphism theorem} gives us an injection 
\begin{align*}
N_G(S) \quotient C_G(S) \longhookrightarrow  \operatorname{Aut}(S)
\end{align*}
We call the centralizer of the whole group $Z(G)\triangleq C_G(G)$ \textbf{center}.  
\begin{equiv_def}
\label{EDpoc}
\textbf{(Property of center)} Let $S\subseteq G$. Then 
\begin{enumerate}[label=(\roman*)]
  \item $S \subseteq Z(G) \iff  C_G(S)=G$. \textbf{(Equivalent condition to lies in center)}
  \item  $o(G) = o(Z(G))+ \sum [G:C_G (x)]$  \textbf{(Class equation)}\label{THce} 
  \item Regardless of $G$, we always have a natural surjective group homomorphism  $G \longtwoheadrightarrow \operatorname{Inn}(G)$. Such group homomorphism is injective (and thus an isomorphism) if and only if $G$ is centerless. 
\end{enumerate}
\end{equiv_def}
\begin{proof}
Routine. Class equation is a consequence of \customref{THost}{orbit-stabilizer theorem}. 
\end{proof}
Let $N \trianglelefteq G$. Because \customref{EDns}{$xy \in N$ if and only if $yx \in N$}, regardless of notation convention for commutator, we see that 
\begin{align*}
[g,h] \in N \iff gN,hN \in G \quotient N\text{ commutes }
\end{align*}
Therefore, the factor group  $G \quotient N$ is abelian if and only if $[G,G]\subseteq N$. In this note, we use $G^{(1)}$ to denote the  \textbf{commutator subgroup} of $G$, the subgroup generated by  $[G,G]$. From our observation, clearly $G^{(1)}$ is the smallest normal subgroup that makes the quotient abelian. In fact, any subgroup $H$ containing $G^{(1)}$  is normal, since if $ghg^{-1}h^{-1} \in H$, then we clearly have $ghg^{-1} \in H$. We call $G^{\operatorname{ab}}\triangleq G \quotient G^{(1)}$ the \textbf{abelianization} of $G$. 
\begin{Example}{$\operatorname{GL}_2(\R)^{(1)}=\operatorname{SL}_2(\R)$}{}
Clearly we have $ \operatorname{GL}_2(\R)^{(1)} \leq \operatorname{SL}_2(\R)$. The opposite relation requires computation. Compute  
\begin{align*}
\begin{pmatrix} 
  1 & x \\
  0 & 1 
\end{pmatrix} = \left[\begin{pmatrix} 
  2 & 0 \\
  0 & 1 
\end{pmatrix}, \begin{pmatrix} 
  1 & x \\
  0 & 1 
\end{pmatrix}\right],\quad \text{ for all }x\inr
\end{align*}
Compute that 
\begin{align*}
\begin{pmatrix} 
  x & 0 \\
  0 & x^{-1}
\end{pmatrix} = \left[ \begin{pmatrix} 
  x & 0 \\
  0 & 1 
\end{pmatrix}, \begin{pmatrix} 
  0 & 1 \\
  1 & 0 
\end{pmatrix} \right],\quad \text{ for all }x\inr^{\times}
\end{align*}
We now see that for $a\neq 0$, we have 
\begin{align*}
\begin{pmatrix} 
  a & b \\
  c & d 
\end{pmatrix}= \begin{pmatrix} 
  1 & 0 \\
  \frac{c}{a} & 1 
\end{pmatrix} \begin{pmatrix} 
  1 & ab \\
  0 & 1 
\end{pmatrix} \begin{pmatrix} 
  a & 0\\
  0 & \frac{1}{a}
\end{pmatrix} \in \operatorname{GL}_2(\R) ^{(1)} 
\end{align*}
Compute that 
\begin{align*}
\begin{pmatrix} 
  0 & 1\\
  -1 & 0 
\end{pmatrix} = \left[\begin{pmatrix} 
  1 & 2 \\
  0 & 1 
\end{pmatrix}, \begin{pmatrix} 
  -1 & 0 \\
  1 & 2
\end{pmatrix} \right] 
\end{align*}
We now see that for $a=0$, we have 
\begin{align*}
\begin{pmatrix} 
  0 & b \\
  c & d
\end{pmatrix} = \begin{pmatrix} 
  0 & 1 \\
  -1 & 0
\end{pmatrix} \begin{pmatrix} 
  1 & - \frac{d}{b}\\
  0 & 1
\end{pmatrix} \begin{pmatrix} 
  \frac{1}{b} & 0 \\
  0 & b
\end{pmatrix} \in   \operatorname{GL}_2(\R) ^{(1)} 
\end{align*}
\end{Example}
\begin{theorem}
\textbf{("Symmetry" between commutator subgroup and center)}  Let $G$ be a group and $N \trianglelefteq G$. We have:
\begin{enumerate}[label=(\roman*)]
  \item  If $N \cap G^{(1)} =1$, then $N \leq Z(G)$.  \label{THpcs} 
  \item  Let $C\subseteq G$ be the set of commutators.  If $[G:Z(G)]\triangleq n$, then
\begin{align*}
 \abso{C}\leq n^2 \quad \text{ and }\quad o\left(G^{(1)} \right) \leq n^{2n^3}
\end{align*}
 \item  Let $G\triangleq \langle g_1, \dots , g_m\rangle $ be finitely generated. If $o(G^{(1)})\triangleq n$, then 
\begin{align*}
[G:Z(G)] \leq n^m
\end{align*}
\end{enumerate}
The last two are called \textbf{Schur's upper bound}. 
\end{theorem}
\begin{proof}
(i): Fix $n \in N$ and $g\in G$. Because $N$ is normal, we know  $gng^{-1}n^{-1} \in N$. It then follows from $N \cap G^{(1)}=1$ that $gng^{-1}n^{-1}=e$, which implies $gn=ng$.  \\

(ii): To prove $\abso{C}\leq n^2$, we show that 
\begin{align*}
\vi{\begin{cases}
  aZ(G)=cZ(G) \\
  bZ(G)=dZ(G)
\end{cases}\implies [a,b]=[c,d]}
\end{align*}
The premise gives us $ac^{-1}\in Z(G)$ and $bd^{-1}\in Z(G)$. We then can compute 
\begin{align*}
aba^{-1}b^{-1}= aba^{-1}(cc^{-1})b^{-1}=cbc^{-1}b^{-1}= cbc^{-1}b^{-1}(dd^{-1})= cdc^{-1}d^{-1} \vdone
\end{align*}
Before proving the second part, we first prove a necessary lemma: 
\begin{align*}
\olive{[a,b]^{n+1}=[a,b^2]\cdot [bab^{-1},b]^{n-1},\quad \text{ for all }a,b \in G }  
\end{align*}
The premise $[G:Z(G)]=n$ implies $g^n \in Z(G)$ for all $g \in G$. Therefore, we may compute 
\begin{align*}
  [a,b]^{n+1}&= aba^{-1}[a,b]^n b^{-1}= ab^2a^{-1} b^{-1}[a,b]^{n-1}b^{-1}= [a,b^2]b[a,b]^{n-1}b^{-1} \\
  &= [a,b^2] \left(b [a,b]b^{-1} \right)^{n-1}= [a,b^2] \cdot [bab^{-1},b]^{n-1} \odone
\end{align*}



Recall that $C= \set{c^{-1} \in G: c\in C}$. Since $\abso{C}\leq n^2$, to prove $o \left(G^{(1)} \right)\leq n^{2n^3}$, we only have to show that 
\begin{center}
   \begin{minipage}{0.9\linewidth}  
     \blue{For all $g \in G^{(1)}$, when written in the form $g\triangleq c_1\cdots  c_m$, where $c_i \in C$, we may require $m \leq n^3$}.  
   \end{minipage}
\end{center}
Fix $g$. Let  $g\triangleq c_1 \cdots c_m$ with smallest $m$. We prove such by showing that each  $c_i$ can occurs at most $n$ times in the expression  $c_1 \cdots c_m$. Assume some $c_j$ occurs  $> n$ times in the expression. Because in general, we have 
\begin{align*}
z[x,y]= [zxz^{-1},zyz^{-1}]z^{-1}
\end{align*}
in groups, we may pull the $c_j$  to the most left in the expression. Then our lemma \olive{$[a,b]^{n+1}=[a,b^2]\cdot [bab^{-1},b]^{n-1}$} gives a contradiction, to the minimality of $m$. \bdone    \\

(iii): Clearly we have 
\begin{align*}
Z(G)= \bigcap_{i=1}^m C_G(g_i) 
\end{align*}
\customref{THost}{Orbit-stabilizer theorem} says that 
\begin{align*}
[G:C_G(g_i)]= \abso{\Gamma }
\end{align*}
where $\Gamma $ is the orbit of $g_i$ under the conjugacy action $G \longrightarrow \operatorname{Inn}(G)$. Since $xg_ix^{-1}=[x,g_i]g_i$, for all $x \in G$, we see $\abso{\Gamma }\leq n$. The proof then follows from \customref{THof}{order formula}: 
\begin{align*}
[G:Z(G)]=[G: \bigcap_{i=1}^m C_G(g_i)] \leq \prod_{i=1}^m [G: C_G(g_i)] \leq  n^m
\end{align*}
\end{proof}
\begin{equiv_def}
\label{EDAg}
\textbf{(Abelian group)} A group $G$ if \textbf{abelian} if any of the followings hold true: 
\begin{enumerate}[label=(\roman*)]
  \item  $Z(G)=G$
  \item $G^{(1)}=1$ 
  \item $g\mapsto g^{-1}$ forms an automorphism.  
  \item $\operatorname{Inn}(G)$ is trivial. 
  \item $G\quotient Z(G)$ is cyclic.  
\end{enumerate}
\end{equiv_def}
\begin{proof}
Routine. 
\end{proof}
\begin{Example}{Criteria for abelian group}{Cfa}
The cyclic criteria of $G \quotient Z(G)$ can not be weaken to that $G\quotient Z(G)$ being abelian. For example, consider $D_4$. Because  $o(Z(D_4))=2$, we know $D_4 \quotient Z(D_4)$ is abelian, but $D_4$ isn't. 
\end{Example}
\begin{corollary}
\textbf{(Finite group with order $\geq 3$ has a nontrivial automorphism)} If $o(G)\geq 3$, then $\operatorname{Aut}(G)$ is nontrivial. 
\end{corollary}
\begin{proof}
  If \customref{EDAg}{$G$ is non-abelian, then  $\operatorname{Inn}(G)$ is non-trivial}. If $G$ is abelian, then we have the inversion automorphism. Such inversion automorphism is non-trivial unless all nontrivial elements of $G$ has order $2$. In such case, $G$ forms a finite-dimensional $\F_2$-vector space, which allow us to easily find an inversion automorphism. 
\end{proof}
\begin{theorem}
\label{THcafg}
\textbf{(Abelian criteria for finite group)} Let $G$ be a finite group and $\phi \in \operatorname{Aut}(G)$ be an automorphism that satisfies 
\begin{align*}
\abso{\set{g \in G: \phi (g)=g^{-1}}} > \frac{3}{4}\cdot  o(G)
\end{align*}
Then $G$ is abelian.  
\end{theorem}
\begin{proof}
Denote $S \triangleq \set{x\in G: \phi (x)=x^{-1}}$. Because of consideration of order, we only have to prove $S \subseteq Z(G)$.  Fix $x \in S$. We prove that $C_G(x)=G$. Now, since 
\begin{align*}
y^{-1}x^{-1}= \phi (xy)=\phi(x)\phi (y)=x^{-1}y^{-1},\quad \text{ for all }y \in S \cap x^{-1}S
\end{align*}
we see that $S \cap x^{-1}S \subseteq C_G(x)$. The proof then follows from computing  
\begin{align*}
\abso{S \cap x^{-1}S}&= \abso{S} + \abso{x^{-1}S}- \abso{S \cup  x^{-1}S} \\
&\geq \frac{3}{2} \cdot o (G)+ 2 \epsilon  - o(G) > \frac{1}{2} \cdot o (G)
\end{align*}
and consideration of the order.
\end{proof}
\begin{Example}{Lower bound in \customref{THcafg}{our abelian criteria for finite group} is necessary}{}
Consider 
\begin{align*}
D_4  \triangleq \langle r,s \mid  r^4=s^2=e,srs^{-1}=r^{-1}\rangle 
\end{align*}
There are exactly three quarters of elements of order dividing $2$, i.e.,  $\set{e,r^2,rs,r^2s,r^3s}$. The identity automorphism send them to their inverse, but the group is not abelian. 
\end{Example}
\section{Characteristic Subgroups}

\begin{equiv_def}
\textbf{(Characteristic subgroup)} Let $G$ be a group. We say $K \leq  G$ is a \textbf{characteristic subgroup} and write $K\operatorname{char}G$ if any of the followings holds true: 
\begin{enumerate}[label=(\roman*)]
  \item $\phi (K)\leq K$ for all $\phi \in \operatorname{Aut}(G)$
  \item $\phi (K)=K$ for all $\phi \in \operatorname{Aut}(G)$.  
\end{enumerate}
\end{equiv_def}
\begin{proof}
(i)$\implies $(ii) follows from noting $\phi^{-1}(K)\leq K \leq \phi^{-1}(K)$. (ii)$\implies $(i) is clear. 
\end{proof}
\begin{theorem}
\label{THus}
\textbf{(Basic properties of characteristic subgroups)} Let $G$ be a group. Then: 
\begin{enumerate}[label=(\roman*)]
\item If there exists a unique subgroup $H \leq G$ of a fixed index, then $H$ is is characteristic. \textbf{(unique subgroup of fixed index is characteristic)} 
\item If $K\operatorname{char}H \trianglelefteq G$, then $K \trianglelefteq G$. \textbf{(characteristic subgroup is transitive)}
\item If $K\operatorname{char}H \operatorname{char}G$, then $K\operatorname{char}G$. 
\end{enumerate}   
\end{theorem}
\begin{proof}
(i): To show that $H$ is characteristic, we are required to prove $[G:H]=[G:\phi (H)]$ for all $\phi \in \operatorname{Aut}(H)$, which follows from checking that  $xH \mapsto \phi (x)\phi (H)$ forms a well-defined bijection between the left cosets spaces of $H$ and $\phi (H)$. \\

(ii) and (iii): Because $H \trianglelefteq G$, every inner automorphism can be restricted $\operatorname{Aut}(H)$. (ii) then follows. The proof for (iii) is the same, in which every automorphism of $G$ can be restricted automorphism of $H$. 
\end{proof}
\begin{Example}{A normal subgroup that isn't characteristic}{}
Consider the additive group of $\Q$.  $\Z \leq \Q$ is then normal but not characteristic, since $x \mapsto \frac{1}{2}$ is an automorphism that doesn't preserve $\Z$.   
\end{Example}
Note that even though normal subgroups need not be preserved by automorphisms, the property of being a normal subgroup is: Given $N \trianglelefteq G$ and $\phi\in \operatorname{Aut}(G)$, we have $\phi(N)\trianglelefteq G$. \\

A subgroup $H\leq G$ is said to be \textbf{strictly characteristic} if it is preserved by all surjective endomorphism. If $H$ is moreover preserved by all endomorphism, then we say it is \textbf{fully characteristic}. Clearly, centers are all strictly characteristic, and commutator subgroups are all fully characteristic. 
\begin{Example}{A center that isn't fully characteristic}{}
Consider $S_3\times C_2$. \customref{THzSn}{This group has center $1 \times C_2$}. The function:
\begin{align*}
(\pi  ,0)\mapsto \left(e ,0 \right)  \quad \text{ and }\quad (\pi ,1)\mapsto \left( (1,2),0 \right),\quad \text{ for all }\pi  \in S_3
\end{align*}
is clearly an endomorphism that doesn't preserve the center. 
\end{Example}
Notably, given a normal subgroup $N \trianglelefteq G$ and an endomorphism $f\in \operatorname{End}(G)$, in general we can't naturally induce an endomorphism on $G \quotient N$. If we were to require a subgroup to always allow us to induce endomorphism on its factor group, then we would need it to be fully characteristic. However, when we only want to induce automorphism from an automorphism $f \in \operatorname{Aut}(G)$, a characteristic subgroup clearly suffices. In fact, given a characteristic subgroup $K\operatorname{char}G$, this gives us a group homomorphism $\operatorname{Aut}(G) \longrightarrow  \operatorname{Aut}(G \quotient K)$. 
\section{Semi-Direct Product}
Let $N,H$ be two groups and  $\phi: H \rightarrow \operatorname{Aut}(N)$ be a group homomorphism. Clearly, when we define a binary operation on  $N\times H$ by 
\begin{align*}
  (n_1,h_1)\cdot (n_2,h_2)\triangleq (n_1 \phi_{h_1}(n_2),h_1h_2)
\end{align*}
We have the \textbf{external semidirect product group} $N \rtimes_\phi H$ in which the inverse of $(n,h)$ is $(\phi_{h^{-1}}(n^{-1}),h^{-1})$. Remarkably, the automorphism $\phi_h \in \operatorname{Aut}(N)$ is always the restriction of the inner automorphism $n \mapsto hnh^{-1}$ in the parent group $N\rtimes_{\phi}H$. In particular, given a group $G$, indeed, every automorphism $\psi \in \operatorname{Aut}(G)$ of $G$ is a restriction of the inner automorphism
\begin{align*}
(g,\phi) \mapsto (e,\psi)\cdot (g,\phi) \cdot (e,\psi)^{-1}
\end{align*}
of the \textbf{holomorph group} $\operatorname{Hol}(G)\triangleq G \rtimes_{\id }\operatorname{Aut}(G)$.  
\begin{theorem}
\textbf{(Universal property of semidirect product)} Let $N,H$ be two groups and  $\phi : H \rightarrow \operatorname{Aut}(N)$ be a group homomorphism. If group homomorphisms $f:N\rightarrow G,g:H\rightarrow G$ satisfy
\begin{align*}
f\left(\phi_h(n) \right) =g(h)f(n)g(h^{-1}),\quad \text{ for all }n\in N,h\in H
\end{align*}
then there exists a unique $k:N\rtimes_{\phi}H \rightarrow G$ that makes the diagram: 
% https://q.uiver.app/#q=WzAsNCxbMCwyLCJOIl0sWzIsMCwiSCJdLFsyLDIsIkciXSxbMCwwLCJOXFxydGltZXNfe1xccGhpfUgiXSxbMCwyLCJmIiwyXSxbMSwyLCJnIl0sWzMsMiwiayIsMCx7InN0eWxlIjp7ImJvZHkiOnsibmFtZSI6ImRhc2hlZCJ9fX1dLFsxLDMsIiIsMSx7InN0eWxlIjp7InRhaWwiOnsibmFtZSI6Imhvb2siLCJzaWRlIjoiYm90dG9tIn19fV0sWzAsMywiIiwxLHsic3R5bGUiOnsidGFpbCI6eyJuYW1lIjoiaG9vayIsInNpZGUiOiJib3R0b20ifX19XV0=
\[\begin{tikzcd}
	{N\rtimes_{\phi}H} && H \\
	\\
	N && G
	\arrow["k", dashed, from=1-1, to=3-3]
	\arrow[hook', from=1-3, to=1-1]
	\arrow["g", from=1-3, to=3-3]
	\arrow[hook', from=3-1, to=1-1]
	\arrow["f"', from=3-1, to=3-3]
\end{tikzcd}\]
commutes. 
\end{theorem}
\begin{proof}
It is routine to check that $k(n,h)\triangleq f(n)g(h)$ suffices. The uniqueness follows from noting $(n,e)\cdot (e,h)=(n,h)$ for all $n \in N$ and $h \in H$.  
\end{proof}
It is worth mentioning here that direct product is a special case of semidirect product. If $\phi : H \rightarrow \operatorname{Aut}(N)$ is trivial, then $N \rtimes_{\phi}H\cong  N \times H$. Also, it is not true that every two distinct group homomorphism $\psi,\phi : H \rightarrow \operatorname{Aut}(N)$ must induce distinct semidirect product. For example, given $f \in  \operatorname{Aut}(H)$, we have 
\begin{align}
\label{EQfAutH}
N \rtimes_{\phi \circ f}  H \cong  N \rtimes_{\phi } H \quad \text{ via }\quad (n,f^{-1}(h)) \mapsto (n,h)
\end{align}
\begin{theorem}
\label{THposd}
\textbf{(Presentation of semidirect product)} Let $N\triangleq \langle X\mid R\rangle ,H \triangleq \langle Y\mid S\rangle $ and $\phi : H \rightarrow \operatorname{Aut}(N)$ be a group homomorphism. We have 
\begin{align*}
N \rtimes_{\phi} H = \langle X\cup  Y \mid R,S,yxy^{-1}=w_y(x)\text{ for all }x \in X,y \in Y\rangle 
\end{align*}
where $w_y(x)$ is a fixed word in $X$ that stands for $\phi_y(x) \in N$.  
\end{theorem}
\begin{proof}
  See \href{https://books.google.com.tw/books/about/Presentations_of_Groups.html?id=p0okzgEACAAJ&redir_esc=y}{Chapter 10, Section 1, Corollary 1 of the book Presentations of Groups (2nd ed.) by D.L. Johnson}. The proof is not difficult, but without his approach, can be rather lengthy and tedious.    
\end{proof}

\begin{Example}{Classification of semidirect product of $C_8 \rtimes C_2$}{}
Write $\operatorname{Aut}(C_{8})\triangleq \set{1,3,5,7}$, so there are four distinct action $C_2 \longrightarrow \operatorname{Aut}(C_8)$ giving us possibly four distinct semidirect product of $C_8 \rtimes C_2:$
\begin{align*}
\langle x,y \mid  x^8=y^2=e,yxy^{-1}=x^n\rangle ,\quad n \in \set{1,3,5,-1}
\end{align*}
Because every word clearly can be written as $x^ay^b$ with  $0 \leq a \leq 7$ and $b \in \set{0,1}$, regardless of $n$, and because as a priori, we are already aware that  $C_8 \rtimes C_2$ has order $16$, we know that indeed the elements $x^ay^b$ are nontrivial unless  $a=b=0$. To see that the four groups are non-isomorphic, one can first use the formula:
\begin{align*}
  (x^ay)^2=x^{a(n+1)}
\end{align*}
to count the number of elements of order $2$, and therefore confirm that the only possible isomorphism is between $n=1$ and  $n=5$, which is also impossible, since  $n=5$ is clearly non-abelian.  
\end{Example}

\begin{equiv_def}
\label{EDrip}
\textbf{(Recognition theorem for inner semidirect product)} Let $N\trianglelefteq G$ and $H \leq G$. The followings are equivalent 
\begin{enumerate}[label=(\roman*)]
  \item $G=NH$ and  $N\cap H=1$. 
  \item Every $g$ can be uniquely written as $g=nh$.  
  \item The composition of $\pi  : G \twoheadrightarrow G\quotient N$ and $\diota : H \hookrightarrow G $ forms an isomorphism $H \rightarrow  G \quotient N$. 
  \item There exists a homomorphism $r:G \rightarrow  H$, called the \textbf{retraction}, that is identity on $H$ and has kernel $N$. Such retraction then give us a right split sequence 
\begin{align*}
1 \longrightarrow N \longrightarrow G \overset{r}{\longrightarrow }H \longrightarrow 1
\end{align*}
since $r \circ \diota = \id _H$. 
\end{enumerate}
\end{equiv_def}
\begin{proof}
  (i)$\implies $(ii) is clear. (ii)$\implies $(iii): Let $h \in N$. To prove that $H \longrightarrow G \quotient N$ is injective, we are required to show $h=e$. Because $h \in N$, we know $h=eh=he$ implies that  $h=e$. Surjectivity is clear. \\

  (iii)$\implies $(iv): Clearly $r\triangleq (\pi  \circ \diota   )^{-1}\circ \pi$ suffices. \\

(iv)$\implies $(i): $N \cap H=1$ is clear. To see that $G=NH$, just observe that  $g=gr(g^{-1})r(g)$, where $gr(g^{-1})\in N$, since $r(gr(g^{-1}))=r(g)r(r(g^{-1}))=r(g)r(g^{-1})=e$. 
\end{proof}
Suppose $N\trianglelefteq  G$ and $H \leq G$ satisfies the conditions in the \customref{EDrip}{recognition theorem for inner semidirect product}. Defining $\phi : H \rightarrow \operatorname{Aut}(N)$ by $\phi_h(n)\triangleq hnh^{-1}$, we see that the natural map $N \rtimes_{\phi} H \rightarrow G$ indeed forms a well-defined group isomorphism. Because of such, when the short exact sequence 
\begin{align*}
1 \longrightarrow N \longrightarrow G \longrightarrow G \quotient N \longrightarrow 1
\end{align*}
right splits, we know that 
\begin{align*}
G = N \rtimes_{\phi} (G \quotient N)
\end{align*}

\begin{Example}{Inner semidirect product}{}
Clearly we have a right split sequence:  
\begin{align*}
1 \longrightarrow A_n \longrightarrow S_n \overset{\operatorname{sgn}}{\longrightarrow } C_2 \longrightarrow 1
\end{align*} 
Since the sequence right splits via $g \mapsto (1,2)$, where $C_2 \triangleq  \langle g\rangle $, \customref{EDrip}{recognition theorem for inner semi-direct product implies}: 
\begin{align*}
A_n \rtimes_{\phi} C_2 \cong  S_n  ,\quad \text{ with }\phi_g (\sigma) \triangleq  (1,2)\sigma (1,2)^{-1}
\end{align*}
where the semi-direct product can be considered internal if we view  $C_2 \triangleq \langle (1,2)\rangle \leq S_n$. \\

Let $\F$ be a field, and view $\F^{\times}$ as a subgroup of $\operatorname{GL}_n(\F)$ by sending $c\mapsto \operatorname{diag}(c,1,\dots ,1)$. We see that we have a right split sequence: 
\begin{align*}
  1 \longrightarrow \operatorname{SL}_n(\F) \longrightarrow \operatorname{GL}_n(\F) \overset{\operatorname{det}}{\longrightarrow} \F^{\times} \longrightarrow 1
\end{align*}
Therefore, \customref{EDrip}{recognition theorem for inner semi-direct product implies}: 
\begin{align*}
\operatorname{SL}_n(\F) \rtimes_{\phi} \F^{\times} \cong  \operatorname{GL}_n(\F)  ,\quad \text{ with }\phi_c(A) \triangleq D(c) A  D(c^{-1}) \text{ and } D(c) \triangleq \operatorname{diag}(c,1,\dots,1)
\end{align*}
where the semi-direct product can be considered internal if we view  $\F^{\times}\triangleq \set{\operatorname{diag}(c,\dots ,1)} \leq \operatorname{GL}_n(\F)$.
\end{Example}
Note that if $H$ is also normal in $G$, then the action  $\phi_h(n)=hnh^{-1}$ become trivial, since we would have $[H,N]\subseteq H \cap N= 1$. This agrees with the \customref{EDidpfg}{recognition theorem for direct product}. 
\begin{equiv_def}
\label{EDidpfg}
  \textbf{(Recognition theorem for direct product)} Let $N_1,\dots ,N_k$ be normal subgroups of $G$. We say $G$ is an \textbf{internal direct  products of $N_i$} if any of the followings hold true: 
\begin{enumerate}[label=(\roman*)]
  \item The natural map $N_1 \times \cdots \times N_k \rightarrow G$ forms a group isomorphism. 
  \item $N_1\cdots N_k=G$ and $N_i \cap \prod_{j\neq i}N_j=1$ for all $i$.  
  \item $N_1\cdots N_k=G$ and $N_i \cap  \prod_{j<i}N_j=1$ for all $i$. 
\end{enumerate}
\end{equiv_def}
\begin{proof}
  (i)$\implies $(ii): Clearly we have $N_1 \cdots N_k = G$. Let $n_2 \cdots n_k \in N_1$. Because $n_2 \cdots n_k$ is both the image of $(n_2\cdots n_k,e,\dots ,e)$ and $(e,n_2,\dots ,n_k)$, by injectivity of the natural map, we know $n_2=\cdots =n_k =e$.  \\

  (ii)$\implies $(iii) is clear. It remains to show (iii)$\implies $(i). The proof relies on induction on $k$. We first prove the base case $k=2$. Because $\customref{EDns}{[N_1,N_2]\leq N_1 \cap N_2}=1$, we know the natural map forms homomorphism. Surjectivity is clear. For injectivity, if $n_1n_2=e$, then since  $n_1= n_2^{-1}\in N_2$, we know $n_1=n_2=e$. \\

We now prove the inductive case. By the base case, the natural map: 
\begin{align*}
  (N_1 \cdots N_k) \times N_{k+1}\longrightarrow N_1\cdots N_k N_{k+1}=G
\end{align*}
forms a group isomorphism. By inductive hypothesis, the natural map: 
\begin{align*}
N_1 \times \cdots \times N_k \longrightarrow N_1\cdots N_k
\end{align*}
also forms a group isomorphism. Composing the two together, we get the desired isomorphism. 
\end{proof}
It should be noted that we didn't define internal direct products for infinite index, since the original statement can not be naively generalized to the infinite case. The ill behavior can be seen from multiple aspect. For example,  we have 
\begin{align*}
\prod_{i \in I} N_i \trianglelefteq \prod_{i \in I} G_i\quad \text{ and }\quad  \frac{\prod G_i}{\prod N_i} \cong  \prod_{i \in I} \frac{G_i}{N_i}\quad \text{ and }\quad  Z\left(\prod_{i\in I} G_i\right)= \prod_{i\in I} Z(G_i)
\end{align*}
even if the index set $I$ is infinite, but we only have  
\begin{align*}
\left(\prod_{i \in I} G_i \right)^{(1)} \leq \prod_{i\in I} (G_i^{(1)})
\end{align*}
where the equality holds if $I$ is finite. \label{THcgi} 
\begin{equiv_def}
\label{EDdpfg}
\textbf{(Recognition theorem for direct product for finite group)} Let $G$ be  a finite group, and let  $N_1 ,\dots ,N_k \trianglelefteq G$ satisfy $G=N_1 \cdots N_k$ and $o(G)=o(N_1)\cdots o(N_k)$. Then $G$ is the internal direct product of $N_i$. 
\end{equiv_def}
\begin{proof}
The proof is done by induction on $k$. The base case $k=1$ is trivial. For the inductive case, one first use \customref{THof}{the order formula} to compute 
\begin{align*}
o(N_1)\cdots o (N_k)o(N_1 \cap (N_2 \cdots N_k))=o(G)o(N_1 \cap (N_2 \cdots N_k)) = o(N_1)o(N_2 \cdots N_k)
\end{align*}
which gives   
\begin{align*}
o(N_1 \cap  (N_2 \cdots N_k)) =  \frac{o(N_2 \cdots N_k)}{o(N_2)\cdots o(N_k)} \leq 1
\end{align*}
which by \customref{EDidpfg}{recognition theorem} implies $G$ is the internal direct product  $N_1 \times (N_2 \cdots N_k)$. The rest then follows from the inductive hypothesis. 
\end{proof}
\begin{Example}{The requirement in definitions of internal direct products for groups}{}
Let $G\triangleq C_4 \times C_2$. Clearly the direct product of $\langle (1,0)\rangle $ and $\langle (2,0)\rangle $ is isomorphic to $G$, but they do not form an internal direct product of  $G$. It is because of such, we must require  $N_1 \times \cdots \times N_k$ not only isomorphic to $G$, but moreover the natural way in \customref{EDidpfg}{definition of internal direct products for groups}.
\end{Example}
\begin{Example}{The requirement in definitions of internal direct products for groups}{}
Let $G\triangleq \Z \times \Z$. Clearly $N_1 \triangleq \Z \times 1,N_2 \triangleq 1 \times \Z,N_3 \triangleq \set{(x,x)\in G: x \inz}$ satisfies 
\begin{align*}
G=N_1N_2 N_3 \quad \text{ and }\quad N_1 \cap N_2 = N_1 \cap  N_3 = N_2 \cap  N_3 =1 
\end{align*}
Yet, later we will see that \customref{THfgag}{we can never have the isomorphism $\Z\times \Z\cong  \Z \times \Z \times \Z$}. This is why we must require $N_i \cap \prod_{j\neq i}N_j=1$ in the definition internal direct product.  
\end{Example}
\begin{Example}{Subgroups are not products of intersections}{}
In general, we don't have 
\begin{align*}
H \leq G_1 \times \cdots \times G_k \implies H = (G_1 \cap  H) \times \cdots \times (G_k \cap H)
\end{align*}
even in the category of abelian group. For example, consider $H \triangleq \set{(x,x)\in \Z^2 : x \inz}$. 
\end{Example}


\section{Structure Theorem for Finitely Generated Abelian Groups}

\begin{theorem}
\label{THcob}
\textbf{(Change of basis for finitely generated abelian group)} Let $k\inn$, $G= \langle x_1,\dots ,x_k\rangle $ be abelian and $c_1,\dots ,c_k \inn$ satisfies $\operatorname{gcd}(c_1,\dots ,c_k)=1$. Then there exists $y_1,\dots ,y_k \in G$ such that $G=\langle y_1,\dots ,y_k\rangle $ and $y_1=c_1x_1+\cdots + c_kx_k$. 
\end{theorem}
\begin{proof}
Such is proved via induction on $s\triangleq c_1 + \cdots + c_k$. The base case $s=k$ is clear. We now prove the inductive case. Because $\operatorname{gcd}(c_1,\dots ,c_k)=1$, by changing the order if necessary, we may write $c_1  > c_2 $. Now, because $\operatorname{gcd}(c_1-c_2,c_2,\dots,c_k)=1$ and $G= \langle x_1,x_2-x_1,x_3,\dots ,x_k\rangle $, we see by inductive hypothesis that there exists $y_1,\dots ,y_k$ such that $G=\langle y_1,\dots ,y_k\rangle $ and 
\begin{align*}
y_1&= (c_1 -c_2)x_1 + c_2(x_2+x_1) + \cdots + c_k x_k \\
&=c_1 x_1 + \cdots + c_kx_k
\end{align*}
as desired. 
\end{proof}
\begin{theorem}
\label{THfgag}
\textbf{(Structure theorem for finitely generated abelian group)} Let $G$ be a finitely generated abelian group. Then we may write: 
\begin{align*}
G  \cong  C_{n_1} \times \cdots \times C_{n_s} \times \Z^{r}
\end{align*}
for $n_i \geq 2$. Moreover, we know that 
\begin{enumerate}[label=(\roman*)]
  \item Such $r$ is unique, called the \textbf{rank} of $G$.  
  \item Under the assumption that $n_i$ each is a power of some prime, the expression exists and is unique, called the \textbf{primary decomposition form}.  
  \item Under the assumption that $n_i \mid n_{i+1}$ for all $i$, the expression exists and is unique, called the  \textbf{invariant factor form}. 
\end{enumerate}
\end{theorem}
\begin{proof}
We first show the \vi{existence} via induction on the number of generators. The base case is clear. Suppose that the existence holds true for any abelian group that has a generating set of cardinality $k-1$, and suppose that $G$ has a generating set $\set{x_1,\dots ,x_k}$ where $x_1$ has order smaller than any elements of any generating sets of cardinality $k$. \\

By inductive hypothesis, we only have to show that $G$ is an internal direct product of $\langle x_1\rangle $ and $\langle x_2,\dots ,x_k\rangle $. Assume not for a contradiction. Then there exists $m_1 \neq 0$ such that $m_1x_1+m_2x_2+ \cdots + m_kx_k=0$. By possibly changing sign of some of the $x_i$ and exchanging the order, we may suppose  $m_1< o(x_1)$ and $m_1,\dots ,m_t\inn$ and $m_{t+1}=\cdots = m_k=0$.\\

Let $c_i \triangleq \frac{m_i}{\operatorname{gcd}(m_1,\dots ,m_t)}$ for all $i \in \set{1,\dots ,t}$. By \myref{theorem}{THcob}, we know there exists $y_1,\dots ,y_t \in G$ such that $\langle y_1,\dots ,y_t\rangle = \langle x_1,\dots ,x_t\rangle $ with $y_1= c_1 x_1+ \cdots + c_t x_t$. Clearly, we have $G= \langle y_1,\dots ,y_t,x_{t+1},\dots ,x_k\rangle $. Compute 
\begin{align*}
\operatorname{gcd}(m_1, \dots, m_t) y_1= m_1 x_1 + \cdots + m_t x_t=0
\end{align*}
We now see $o(y_1) \leq m_1 < o(x_1)$, a contradiction to the choice that $x_1$ has the smallest order. $\vdone$ \\

(i): Fix an expression of $G$, and let $p$ be a prime that satisfies  $p \nmid n_i$ for all $i$. The rest then follows from checking that $G \quotient pG \cong  C_p^r$. \\ 


(ii): The existence follows from noting that 
\begin{align*}
\operatorname{gcd}(m,n)=1 \implies    C_{mn}\cong  C_m \times C_n
\end{align*}
The uniqueness follows from noting that given a primary decomposition form whose $p$-part has the form  $C_{p^{n_1}} \times \cdots \times C_{p^{n_k}}$, then the \textbf{torsion $p$-subgroup} $G_{T_p}$ of $G$ 
\begin{align*}
G_{T_p} \triangleq \set{x \in G: o(x)=p^n\text{ for some }n\geq 0}
\end{align*}
is 
\begin{align*}
G_{T_p} \cong  C_{p^{n_1}}\times \cdots \times C_{p^{n_k}}
\end{align*}
and that 
\begin{align*}
  \frac{p^{d-1}C_{p^{n_i}}}{p^{d}C_{p^{n_i}}}\cong \begin{cases}
    C_p & \text{ if }d-1 <n_i \\
    1& \text{ if }d-1 \geq n_i
  \end{cases}
\end{align*}
(iii): Both the existence and uniqueness of invariant factor form follows form that of primary decomposition form: Just consider that $C_{n_s}$ can only be $C_{p_1}^{d_1}\times \cdots \times C_{p_m}^{d_m}$, where $p_i$ non-repeatedly running through all the occurring prime with $d_i$ being the highest exponential.  
\end{proof}
\customref{THfgag}{Structure theorem for finitely generated abelian group} in fact also gives a structure theorem for automorphism group of finite abelian group. See \href{https://arxiv.org/pdf/math/0605185}{this paper}. A particular case is that 
\begin{align*}
\operatorname{Aut}(C_p^n)\cong \operatorname{GL}_n(\F_p) 
\end{align*}
\begin{corollary}
\label{COfao}
\textbf{(Finite abelian group has subgroups of all possible order)} Let $G$ be a finite abelian group with $m \mid  o(G)$. Then there exists subgroup of $G$ of order  $m$.  
\end{corollary}
\begin{proof}
The proof follows from the \customref{THfgag}{primary decomposition form} and the fact that for all $e<d\inn \cup  \set{0}$,  
\begin{align*}
\set{0,p^{d-e},\dots ,(p^e -1)p^{d-e}}\subseteq C_{p^{d}}
\end{align*}
is a subgroup of $C_{p^d}$ of order $p^e$. 
\end{proof}
\begin{theorem}
\textbf{(Subgroup of finitely generated abelian group)} Let $G$ be a finitely generated abelian group that has rank $r$ whose $p$-part is  
\begin{align*}
G_{T_p} \cong  C_{p^{n_1}}\times \cdots \times C_{p^{n_k}},\quad \text{ with }n_1 \geq \cdots \geq  n_k
\end{align*}
Then for any subgroup $H \leq G$, if we write 
\begin{align*}
H_{T_p}\cong  C_{p^{d_1}}\times \cdots \times C_{p^{d_s}},\quad \text{ with }d_1 \geq \cdots \geq  d_s
\end{align*}
then the rank of $H$ is  $\leq r$, and we have $s \leq k$ and $n_i \geq d_i$ for all $i$. 
\end{theorem}
\begin{proof}
Clearly we have $H_{T_p}\leq G_{T_p}$. Consider $\Omega_1 (H_{T_p})\triangleq  \set{h \in H_{T_p}: h^p=e}$. Clearly, we have 
\begin{align*}
C_p^s \cong  \Omega_1(H_{T_p}) \leq \Omega_1 (G_{T_p}) \cong  C_p^k
\end{align*}
Therefore, by counting order, we see that indeed $s\leq k$. The rest is 
\begin{align*}
C_{p}^{\operatorname{card}\set{\geq d_i}}\cong  \Omega_1 (p^{d_i-1}H_{T_p}) \leq  \Omega_1 (p^{d_i-1}G_{T_p}) \cong  C_p^{\operatorname{card}\set{\geq d_i}}
\end{align*}
We now prove that $H$ has rank  $\leq r$. Let $q$ be a prime not in the decomposition. Since 
\begin{align*}
\Z^s \cong  qH \leq qG \cong  \Z^r
\end{align*}
We have an injective group homomorphism $f:\Z^s\rightarrow \Z^r$. Denote the group homomorphism by a $r$-by-$s$ matrix $A\in M_{r \times s}(\Z)$. The injectivity of $f$ then means $Av=0 \implies  v=0$ for all $v\in \Z^s$. Let $w \in \Q^s$ and $m \gg 0$ be a natural number large enough so that $mw \in \Z^s$. We then see 
\begin{align*}
Aw=0 \implies Amw=mAw=0 \implies  mw=0 \implies  w=0 
\end{align*}
In other words, the matrix $A \in M_{r \times s}(\Q)$ has rank $s$, which is only possible given that  $s \leq r$. 
\end{proof}
\section{Sylow theorems}
\label{SECSt}
In this section, we prove \customref{THfSt}{Sylow theorems} using combinatorics. Note that \customref{THfSt}{first Sylow theorem} also shows that indeed as one might expect: \customref{COepc}{Every $p$-subgroup of a finite group is a subgroup of some Sylow  $p$-subgroup}. 
\begin{theorem}
\label{THCf}
\textbf{(Combinatorial facts)} Let $p$ be prime. Then: 
\begin{enumerate}[label=(\roman*)]
  \item Given $m\geq r \inn \cup \set{0}$ with $t\inn$ coprime with $p$, the natural number $\binom{p^mt}{p^r}$ has $p$-part  $p^{m-r}$.  
  \item We have 
\begin{align*}
\binom{m}{n} \equiv \prod_{i=0}^k \binom{m_i}{n_i}\pmod{p}
\end{align*}
when we write $m=m_kp^k + \cdots + m_0p^0$ and $n=n_kp^k + \cdots + n_0p^0$.  This is called \textbf{Lucas modulo binomial formula}. 
\end{enumerate}
\end{theorem}
\begin{proof}
  (i) follows from noting that 
\begin{align*}
\binom{p^mt}{p^r}= \frac{p^mt (p^mt-1)\cdots (p^mt-(p^r-1))}{p^r (p^r -1)\cdots (p^r - (p^r-1))}
\end{align*}
and that for all $i \in \set{1,\dots ,p^r-1}$, the three natural number $\set{i,p^mt-i,p^r-i}$ share the same $p$-part.\\

(ii): Let $M$ be a set of  $m$ elements. Partition $M$ into $m_i$ cycles of length  $p^i$, i.e., 
 \begin{align*}
M \triangleq \bigcup_{i=0}^k \bigcup_{j=1}^{m_i} \Gamma _{i,j},\quad \text{ where }\abso{\Gamma _{i,j}}=p^i 
\end{align*}
Because of such, we see that $M$ can be acted on by the group 
\begin{align*}
G\triangleq \prod_{i=0}^k \overbrace{C_{p^i} \times \cdots \times C_{p^i}}^{m_i}
\end{align*}
Clearly, $G$ also acts on the set $X$ of the set of subsets of  $M$ that has  $n$ elements.  Because $G$ is a $p$-group, \customref{THost}{orbit-stabilizer theorem} tell us that 
\begin{align*}
\binom{m}{n}= \abso{X} \equiv \abso{\operatorname{Fix}(G)}\pmod{p}
\end{align*}
where $\operatorname{Fix}(G)$ is the set of elements of $X$ fixed by all $g \in G$. Because of uniqueness of representation in base $p$, we know that elements of $\operatorname{Fix}(G)$ are exactly those subsets of $X$ that contains $n_i$ cycles of length $p^i$. The proof now follows from directly computing that $\abso{\operatorname{Fix}(G)}= \prod_{i=0}^k \binom{m_i}{n_i}$. 
\end{proof}
\begin{theorem}
\label{THfSt}
\textbf{(First and Third Sylow theorem, Wielandt's proof)} Let $G$ be a finite group of order $p^mt$ with $\operatorname{gcd}(p,t)=1$. Let $1\leq r\leq m$. Then the number $n_p$ of $p$-subgroup with order $p^r$ satisfies 
\begin{align*}
n_p \equiv 1 \pmod{p}
\end{align*}
\end{theorem}
\begin{proof}
Let $X$ be the set of subset of  $G$ with cardinality $p^r$. Our goal is to find all elements of $X$ that forms a group. Clearly we may define a left $G$-action on  $X$ be setting 
 \begin{align*}
g \cdot \set{x_1,\dots ,x_{p^r}} \triangleq \set{gx_1,\dots ,gx_{p^r}}
\end{align*}
Let $\Gamma $ be an orbit. If $\Gamma $ contains a group, then we see that $\Gamma $ is the left coset space of that group,  containing exactly one group and satisfying $\abso{\Gamma }=p^{m-r}t$. If $\Gamma $ doesn't contain any group, there still exists some $S \in \Gamma $ such that $e \in S$, and clearly we will have $\operatorname{Stab}(S)\subseteq S$. Because $S$ isn't a group, we see  $p^r=\abso{S}>o(\operatorname{Stab}(S))$, which by \customref{THost}{orbit-stabilizer theorem} implies that $\abso{\Gamma }= [G: \operatorname{Stab}(S)]=p^{m-r+c}t$ for some $c\geq 1$. \\

In summary, by counting orbit, we have shown that: 
\begin{align*}
\binom{p^mt}{p^r} = \abso{X} = n_pp^{m-r}t + l p^{m-r+1}t ,\quad \text{ for some }l \inn
\end{align*}
Let $ut \equiv 1 \pmod{p}$. \customref{THCf}{Recalling that $\binom{p^mt}{p^r}$ has  $p$-power  $p^{m-r}$}, it remains to show 
\begin{align*}
u \cdot \frac{\binom{p^mt}{p^r}}{p^{m-r}}\equiv 1 \pmod{p}
\end{align*}
which follows from noting: 
\begin{align*}
u \cdot \frac{\binom{p^mt}{p^r}}{p^{m-r}}= ut \cdot \binom{p^mt-1}{p^r-1}  \equiv\binom{p^{m}t-1}{p^{r}-1}\equiv 1 \pmod{p}
\end{align*}
where the last equality follows from \customref{THCf}{Lucas modulo binomial formula} and the observation that 
\begin{align*}
p^mt-1 = t_kp^{m+k} + \cdots + (t_0-1)p^m + (p-1) p^{m-1} + \cdots + (p-1)p^0
\end{align*}
where $t=t_kp^k + \cdots + t_0 p^0$ with $t_0>0$.  
\end{proof}
\begin{corollary}
\label{COepc}
\textbf{(Every $p$-subgroup is contained by some Sylow $p$-subgroup)} Let $G$ be a finite group and $H\leq G$ a $p$-group. Then $H$ must be contained by some Sylow $p$-subgroup of $G$. 
\end{corollary}
\begin{proof}
Consider the conjugacy action $H \longrightarrow \operatorname{Bij}\left(\operatorname{Syl}_p(G) \right)$. \customref{THfSt}{First Sylow theorem} and \customref{THost}{orbit-stabilizer theorem} shows that there must be a singleton orbit. Let that singleton be $P$.\\

We claim $H\leq P$. Because $\set{P}$ is a singleton orbit of the conjugacy action, we know $H \leq N_G(P)$. Then by \customref{THsitg}{second isomorphism theorem}, we see that $HP$ is a group such that  $HP \quotient P\cong  H \quotient  H \cap P$. This implies that $HP$ is a  $p$-group. The fact $HP$ contains $P$ forces $P=HP$, which implies $H \leq P$. 
\end{proof}
Before \customref{THsSt}{proving Second Sylow theorem}, we need a lemma for actions of $p$-groups. 
\begin{lemma}
\label{THcl}
\textbf{(Counting lemma for $p$-group)} Let $G$ be a  $p$-group acting on a finite set $X$. Then 
\begin{align*}
\abso{X} \equiv \abso{\operatorname{Fix}(G)} \pmod{p}
\end{align*}
where $\operatorname{Fix}(G) \triangleq \set{x \in X: gx=x\text{ for all }g\in G}$ is the set of points fixed by all $g  \in G$.  
\end{lemma}
\begin{proof}
This is a consequence of \customref{THost}{orbit-stabilizer theorem}. 
\end{proof}
\begin{theorem}
\label{THsSt}
\textbf{(Second Sylow theorem)} Sylow $p$-subgroups are conjugated to each other.    
\end{theorem}
\begin{proof}
Let $H$ and $P$ be two Sylow $p$-subgroups of $G$, and let $H$ acts on left coset space of  $P$ by left multiplication. Because $P$ is Sylow, by \customref{THcl}{counting lemma for $p$-group}, we know the number of fixed points $gP$ is nonzero. Let $gP$ be a fixed point. We then see that, as desired, $g^{-1}hg  \in P$ for all $h \in H$, since  $hgP=gP$. 
\end{proof}
Even without \customref{THtSt}{third Sylow theorem}, \customref{THsSt}{second Sylow theorem} already gives some interesting applications.   
\begin{corollary}
\label{COnoS}
\textbf{(Normalizers of Sylow subgroups Don't satisfy normalizer condition)} Let $G$ be a finite group and  $P \in \operatorname{Syl}_p(G)$. Then 
\begin{align*}
N_G(P)=N_G(N_G(P))
\end{align*}
\end{corollary}
\begin{proof}
Let $x \in N_G(N_G(P))$. We are required to show $x \in N_G(P)$. By definition, we have 
\begin{align*}
xPx^{-1}\leq xN_G(P)x^{-1} =N_G(P)
\end{align*}
In other words, both $P$ and  $xPx^{-1}$ are Sylow $p$-subgroup of  $N_G(P)$. Therefore, by \customref{THsSt}{second Sylow theorem}, there exists some $y \in N_G(P)$ such that 
 \begin{align*}
xPx^{-1}=yPy^{-1}=P
\end{align*}
This then implies $x \in N_G(P)$. 
\end{proof}
\begin{corollary}
\textbf{(Restriction of conjugacy classes onto normalizer of Sylow subgroups, on element of center of Sylow subgroups)} Let $G$ be a finite group, $P \in \operatorname{Syl}_p(G)$, and $a,b \in Z(P)$. If  $a,b$ are conjugate in  $G$, then they are conjugate in  $N_G(P)$. 
\end{corollary}
\begin{proof}
Write $b=xax^{-1}$ with $x \in G$. By definition, we have 
\begin{align*}
  (x^{-1}gx)a (x^{-1}gx)^{-1}= x^{-1}gbg^{-1}x= x^{-1}bx = a,\quad \text{ for all }g \in P
\end{align*}
In other words, $x^{-1}Px \leq C_G(a)$. Then since both  $P$ and  $x^{-1}Px$ are Sylow $p$-subgroup of $C_G(a)$, by \customref{THsSt}{second Sylow theorem}, we know there exists $y \in C_G(a)$ such that $P=y^{-1}x^{-1}Pxy$. We now see that $xy \in N_G(P)$ and $(xy)a(xy)^{-1}=b$, as desired.      
\end{proof}
\customref{THsSt}{Second Sylow theorem} moreover stated that given $n_p>1$, the conjugacy action $G \longrightarrow  \operatorname{Bij}(\operatorname{Syl}_p(G))\cong  S_{n_p}$ is nontrivial, and thus injective when $G$ is simple. This is a trick particularly useful to classify finite simple group. 

\begin{theorem}
\label{THtSt}
\textbf{(Remaining part of third  Sylow theorem)} Let $G$ be a finite group, and let $n_p$ be the number of Sylow  $p$-subgroup of  $G$. For all Sylow  $p$-subgroup $P$ of  $G$, we have 
 \begin{align*}
n_p =  [G:N_G(P)]
\end{align*}
\end{theorem}
\begin{proof}
This is a consequence of \customref{THsSt}{second Sylow theorem} and \customref{THost}{orbit stabilizer theorem}, where we note that when $G$ acts on  $\operatorname{Syl}_p(G)$ by conjugation we have $\operatorname{Stab}(P)=N_G(P)$. 
\end{proof}
\section{Nilpotency and Solvability}
A \textbf{normal series} of a group $G$ is a finite chain of subgroups: 
\begin{align*}
1 \trianglelefteq \cdots \trianglelefteq  G_{n-1}    \trianglelefteq G_n \trianglelefteq G_{n+1} \trianglelefteq \cdots \trianglelefteq  G
\end{align*}
where each $G_n$ is normal only $G_{n+1}$, but not necessarily $G$. A \textbf{composition series} is a maximal normal series. 
\begin{Example}{Subgroups in normal series need not all be normal}{}
Consider 
\begin{align*}
 D_4 \triangleq \langle r,s \mid r^4=s^2=e,srs^{-1}=r^{-1} \rangle 
\end{align*}
We have normal series  
\begin{align*}
\langle s\rangle  \trianglelefteq \langle s,r^2\rangle  \trianglelefteq D_4
\end{align*}
where normality follows from index $2$. Clearly $\langle s\rangle $ is not normal in $D_4$.   
\end{Example}
\begin{equiv_def}
\textbf{(Solvable groups)} We say a group $G$ is \textbf{solvable} if any of the followings holds true: 
\begin{enumerate}[label=(\roman*)]
  \item $G$ admits a finite normal series whose factor  groups are all abelian. 
  \item The \textbf{derived series}
\begin{align*}
\cdots \trianglelefteq  G^{(2)} \trianglelefteq G^{(1)}\trianglelefteq G^{(0)} \triangleq  G,\quad \text{ where }G^{(k+1)}\triangleq \langle [G^{(k)},G^{(k)}]\rangle 
\end{align*}
reach to $1$. 
\end{enumerate}
\end{equiv_def}
\begin{proof}
Routine. Note that if $1=G_n \trianglelefteq \cdots \trianglelefteq G_0 =G$ is a normal series whose factor groups are all abelian, then clearly we have $G^{(k)}\leq G_k$. 
\end{proof}
Clearly, solvable groups are closed under \textbf{group extension}. Given a short exact sequence of groups 
\begin{align*}
1 \longrightarrow A \longrightarrow G \longrightarrow B \longrightarrow 1
\end{align*}
If $A$ and $B$ are  both solvable, then $G$ is solvable. Conversely, let $G$ be solvable and $H\leq G$. Then clearly   
\begin{align*}
1= G^{(n)}\cap  H \trianglelefteq   \cdots \trianglelefteq  G^{(1)} \cap H \trianglelefteq  H 
\end{align*}
is normal series whose factor groups are all abelian. Because of such, we say solvable groups are \textbf{subgroup-closed}. Let $N\trianglelefteq G$ and $\pi : G \rightarrow G \quotient N$ be the natural projection. Since $\pi  \left(G^{(k)}\right)$ is the derived series of $G \quotient N$, we see that solvable groups are also \textbf{quotient-closed}. Since \customref{THcgi}{taking direct product and taking commutator subgroup commutes}, we moreover have 
\begin{align*}
\left(\prod_{i=1}^nG_i \right)^{(k)}  =\prod_{i=1}^n G_i^{(k)}
\end{align*}
Therefore, solvable groups are also \textbf{finite direct product-closed}. 
\begin{equiv_def}
\textbf{(Finite solvable groups)} A finite group $G$ is solvable if and only if it admits a composition series whose factor group are all cyclic of prime order. 
\end{equiv_def}
\begin{proof}
Routine. 
\end{proof}

\begin{equiv_def}
\textbf{(Nilpotent groups)} We call a group $G$ \textbf{nilpotent} if $G$ admits a  \textbf{central series}, a normal series: 
\begin{align*}
1=G_0 \trianglelefteq \cdots \trianglelefteq G_n =G 
\end{align*}
such that 
\begin{align*}
[G,G_{k}]\leq G_{k-1},\quad \text{ for all }k \in \set{1,\dots ,n}
\end{align*}
Clearly, all $G_k$ are normal in  $G$. Given a series of normal subgroups, we then clearly see that it is central if and only if 
\begin{align*}
 \frac{G_{k}}{G_{k-1}} \leq Z\left(\frac{G}{G_{k-1}} \right),\quad \text{ for all }k\in \set{1,\dots ,n}
\end{align*}
A group is nilpotent if and only if its \textbf{lower central series}: 
\begin{align*}
\cdots \trianglelefteq G_{(2)} \trianglelefteq  G_{(1)}\trianglelefteq G_{(0)}\triangleq  G,\quad \text{ where }G_{(k)}\triangleq  [G,G_{(k-1)}]\text{ for all }k\inn
\end{align*}
reach to $1$. A group is  nilpotent if and only if its \textbf{upper central series}: 
\begin{align*}
  1\triangleq Z_{(0)}\trianglelefteq  Z_{(1)} \trianglelefteq  Z_{(2)} \trianglelefteq  \cdots, \quad \text{ where } \frac{Z_{(k+1)}}{Z_{(k)}} \triangleq Z\left(\frac{G}{Z_{(k)}} \right)\text{ for all }k\inn
\end{align*}
reach to $G$. The central series $1=G_0 \triangleleft \cdots \triangleleft G_n =G$ is said to have \textbf{length} $n$. The \textbf{nilpotency class} of a nilpotent group is the smallest length of its central series. If a group is nilpotent, then both its lower and upper central series has length of its nilpotent class.  
\end{equiv_def}
\begin{proof}
Clearly both lower and upper central series are central by definition. Suppose a central series 
\begin{align*}
1=G_0 \trianglelefteq \cdots \trianglelefteq G_n =G 
\end{align*}
exists. To see the lower central series reach to $1$ with smallest length, use induction to show  $G_{(k)}\leq G_{n-k}$ for all $k$. (thus the name \textbf{fastest descending series}). To see the upper central series reach to $G$ with smallest length, use induction to show $G_k \leq Z_{(k)}$ for all $k$. (thus the name \textbf{fastest ascending series}) 
\end{proof}
Clearly, every nilpotent group is solvable, but solvable group need not be nilpotent.  
\begin{Example}{A solvable group that isn't nilpotent}{}
Consider $S_3$. Because $S_3$ is the extension of  $C_3$ and  $C_2$: 
 \begin{align*}
1 \longrightarrow A_3 \longrightarrow S_3 \longrightarrow C_2 \longrightarrow 1
\end{align*}
We know $S_3$ is solvable. However, since $S_3$ clearly has $3$ Sylow $2$-subgroups and \customref{EDfng}{finite nilpotent group has normal Sylow subgroups}, we know $S_3$ isn't nilpotent.  
\end{Example}

Again, one can check that central series are closed under taking intersection with a subgroup, under taking finite direct product and under surjective group homomorphism, so nilpotency is also subgroup-closed, quotient-closed and finite-direct-product-closed. 
\begin{theorem}
\label{THsnnc}
\textbf{(Proper subgroups of nilpotent groups satisfy normalizer condition)} If $G$ is nilpotent, then any $H<G$ satisfies normalizer condition.  
\end{theorem}
\begin{proof}
Note that if $H$ doesn't contain $Z(G)$, then the elements of $Z(G)$ that lies outside $H$ complete the proof, so we only have to consider the case $Z(G)\leq H$.\\

This is proved by induction on nilpotency class $n$ of $G$. The base case $n=1$ is clear. The inductive case follows from \customref{THtit}{third isomorphism theorem for groups} and the observation $G \quotient Z(G)$ has the nilpotent class one smaller than that of $G$.
\end{proof}
\begin{theorem}
\label{THpp}
\textbf{(Properties of finite $p$-groups)} Let $P$ be a finite $p$-group. Then:  
\begin{enumerate}[label=(\roman*)]
  \item $P$ has nontrivial center. 
  \item $P$ is nilpotent.  
  \item Groups of order $p^2$ is either $C_p\times C_p$ or $C_{p^2}$.  
  \item Groups of order $p^3$, if not abelian, must satisfies $o(Z(G))=p$ and $Z(G)=G^{(1)}$.  
  \item Nontrivial normal subgroup $1 < N \trianglelefteq P$ satisfies $N \cap Z(P)>1$.  
\end{enumerate}
\end{theorem}
\begin{proof}
  (i) is a consequence of \customref{THce}{class equation}. Both (ii) and (iii) follows from (i). If the center of a group of order $p^2$ has order $p$, then \customref{EDAg}{a contradiction about abelian occurs}. Let $G$ be a group of order $p^3$. The fact $o(Z(G))=p$ also follows from (i) and \customref{EDAg}{same abelian contradiction}. The fact $G^{(1)}=Z(G)$ follows from \customref{EDAg}{definition of abelian group} and the fact that (ii) implies $G \quotient Z(G)$ is abelian. We have shown (iv).\\

Lastly, we prove (v). Let $P$ acts on $N$ by conjugation. By \customref{THcl}{our counting lemma for $p$-group}, we have 
\begin{align*}
0 \equiv o(N) \equiv \abso{\set{n \in N: gng^{-1}=n\text{ for all } g \in P}} \pmod{p} 
\end{align*}
Noting that the latter set $=N \cap Z(P)$, we now see $N \cap Z(P)>1$. 
\end{proof}
\begin{Example}{Infinite $p$-group}{}
The \textbf{Prüfer group} $\Z(p^{\infty})$ is defined by 
\begin{align*}
\Z(p^{\infty})\triangleq \set{\exp (2\pi i m \quotient p^n)\inc: m \inz\text{ and }n\inn }
\end{align*}
It is an infinite group whose every nontrivial element has order $p^n$ for some  $n\inn$.    
\end{Example}

\begin{equiv_def}
\label{EDfng}
\textbf{(Finite nilpotent group)} Let $G$ be a finite group. The followings are equivalent: 
\begin{enumerate}[label=(\roman*)]
  \item $G$ is nilpotent.  
  \item Proper subgroups of $G$ satisfies normalizer condition.
  \item Sylow subgroups of $G$ are all normal. 
  \item  $G$ is the internal direct product of its Sylow subgroups. 
\end{enumerate}
\end{equiv_def}
\begin{proof}
(i)$\implies $(ii): \customref{THsnnc}{This is true even if $G$ is infinite}.\\


(ii)$\implies $(iii): If $G$ is a  $p$-group, then the proof is trivial. Let $G$ not be a $p$-group and let  $P \in \operatorname{Syl}_p(G)$. To see $P$ is normal, just observe that since \customref{COnoS}{normalizers of Sylow subgroups don't satisfy normalizer condition}, the normalizer of $P$ must be  $G$. \\

(iii)$\implies $(iv): This follows from \customref{EDdpfg}{the definition of finite internal direct product}. \\

(iv)$\implies $(i): This follows from the fact that \customref{THpp}{$p$-groups is nilpotent} and that nilpotency is closed under taking finite direct product. 
\end{proof}
\section{Old Numbers}
\begin{abstract}
This section proves some elementary number theory that are in essence group theory. 
\end{abstract}
\begin{theorem}
\textbf{(Group property of totient function)} Let $\pfi $ be the Euler totient function. Then 
\begin{align*}
n \mid \pfi (a^n-1),\quad \text{ for all }a,n \inn
\end{align*}
\end{theorem}
\begin{proof}
This is a consequence of the fact that $a \in \Z_{a^n-1}^{\times}$ and  that $a$ has order $n$ in the group.  
\end{proof}
Before the first application, we first show that:  
\begin{theorem}
\label{THzpc}
\textbf{($\Z_p^{\times}$ is cyclic)} Let $p$ be prime. Then the unit group $\Z_p^{\times}$ is cyclic. 
\end{theorem}
\begin{proof}
We are required to show the existence of elements of order $p-1$. Let $l$ be the least common multiple of orders of all elements, so $x^l-1\in \Z_p[x]$ has $p-1$ distinct roots in  $\Z_p$. This implies  $l\geq p-1$. Because the group is abelian, we now see from \customref{THfgag}{its primary decomposition form} that indeed it is cyclic.  
\end{proof}
\begin{theorem}
\label{THWt}
\textbf{(Group theoretic side of Wilson's theorem)} Let $p$ be prime. Then 
\begin{align*}
(p-1)! \equiv -1\pmod{p}
\end{align*}
\end{theorem}
\begin{proof}
Consider the symmetric group $S_p$. The proof follows from $n_p \equiv 1\pmod{p}$ and directly counting that $n_p=(p-2)!$. This is easy to count, since the Sylow $p$-group of  $S_p$ are those generated by a single  $p$-cycle.   
\end{proof}
\begin{theorem}
\textbf{(Wolstenholme's theorem)} Let $p$ be an odd prime, and write 
\begin{align*}
H(p-1)\triangleq 1+ \frac{1}{2} + \cdots + \frac{1}{p-1}    \inq
\end{align*}
Then
\begin{align*}
H(p-1) \equiv 0\pmod{p^2}
\end{align*}
\end{theorem}
\begin{proof}
Because $p$ is odd, we can group the terms of $H(p-1)$ by pairs: 
\begin{align*}
H(p-1)= \left( 1+ \frac{1}{p-1} \right) + \left( \frac{1}{2} + \frac{1}{p-2} \right) + \cdots + \left( \frac{1}{\frac{p-1}{2}} + \frac{1}{p - \frac{p-1}{2}}\right)   
\end{align*}
Reduce each pair to a common denominator: 
\begin{align*}
H(p-1)= \frac{p}{p-1}+ \frac{p}{2(p-2)}+ \cdots + \frac{p}{\frac{p-1}{2}\cdot \left(p- \frac{p-1}{2} \right)}
\end{align*}
We then can write 
\begin{align}
\label{EQHp1}
H(p-1)\triangleq  p \cdot \frac{A}{(p-1)!}
\end{align}
where 
\begin{align*}
A\triangleq  \frac{(p-1)!}{p-1} + \frac{(p-1)!}{2(p-2)} + \cdots + \frac{(p-1)!}{\frac{p-1}{2}\cdot (p- \frac{p-1}{2})}
\end{align*}
Since $p\nmid  (p-1)!$, \myref{equation}{EQHp1} reduce the proof into proving $p \mid  A$. We first show that 
\begin{center}
   \begin{minipage}{0.9\linewidth}  
\vi{$\frac{(p-1)!}{a (p-a)} \equiv a^{-2}\pmod{p}$, for $a \in \set{1,\dots ,p-1}$, where the power $a^{-2}$ occurs in the cyclic group $\Z_p^\times\cong  C_{p-1}$.   }
   \end{minipage}
\end{center}
Denote $x \triangleq \frac{(p-1)!}{a(p-a)}\inz$, so by \customref{THWt}{Wilson's theorem}, we have 
\begin{align*}
a(p-a)x=(p-1)! \equiv -1\pmod{p}
\end{align*}
which gives us 
\begin{align*}
a^2x \equiv 1 \pmod{p} \vdone
\end{align*}
We may now write 
\begin{align*}
A \equiv 1^{-2} + \cdots + \left(\frac{p-1}{2}  \right)^{-2} \pmod{p}
\end{align*}
Because $p$ is odd, we only have to prove $2A \equiv 0\pmod{p}$: 
\begin{align*}
  2A&\equiv 1^{-2} + \cdots + \left(\frac{p-1}{2}  \right)^{-2} + 1^{-2} + \cdots + \left(\frac{p-1}{2}  \right)^{-2}  \\
  &\equiv   1^{-2} + \cdots + \left(\frac{p-1}{2}  \right)^{-2}  + (-1)^{-2} + \cdots + \left(-\frac{p-1}{2} \right)^{-2} \\
  &\equiv 1^{-2} + \cdots + \left(\frac{p-1}{2} \right)^{-2} + \left(\frac{p+1}{2} \right)^{-2} + \cdots + \left(p-1 \right)^{-2} 
\end{align*}
Since the inversion $x\mapsto x^{-1}$ forms a bijection in $\Z_p^{\times}$. Therefore, we have 
\begin{align*}
  2A &\equiv 1^2 + \cdots + (p-1)^2  \\
  &= \frac{p(p-1)(2p-1)}{6} \equiv 0 \pmod{p}
\end{align*}
\end{proof}
\section{Symmetric Groups}
\begin{abstract}
This section develop some common sense about symmetric groups.  
\end{abstract}
Given the \textbf{symmetric group} $S_n$, to define parity, the fastest way is to realize it as the  \textbf{group of permutation matrix}, and then call those that have determinant $1$  \textbf{even}, while those that have determinant  $-1$  \textbf{odd}. Clearly we have the \textbf{alternating group}
\begin{align*}
A_n \triangleq \set{g \in S_n: \operatorname{det}(g)=1}
\end{align*}
To see that  $[S_n:A_n]=2$ for all $n\geq 2$, just consider that for any $g\in S_n - A_n$, we have a bijection between $A_n$ and  $S_n- A_n$ defined by  
\begin{align*}
a \mapsto  ag
\end{align*}
By direct observation, we see that every permutation has a fixed \textbf{cycle type}.     Because  
\begin{align*}
\sigma \circ (a_1,\dots ,a_k) \circ \sigma^{-1} = \left(\sigma (a_1),\dots ,\sigma (a_k)\right)
\end{align*}
we see that the conjugacy classes of $S_n$ coincides with cycle type classes. \\

At this point, it is worth mentioning that, given $g \in S_n$, even though we are definitely aware of what its centralizer is, in the sense that given any $h \in S_n$, we can immediately tell whether $h \in C_{S_n}(g)$, but to actually describe $C_{S_n}(g)$ may be annoying. See this \href{https://math.stackexchange.com/questions/208790/centralizer-of-a-given-element-in-s-n}{MSE post}. 
\begin{Example}{Conjugacy classes of $A_n$ are either half a cycle type or all.}{}
Recall that conjugacy classes is just the orbit of conjugacy action, and \customref{THost}{orbit-stabilizer theorem} links the orbit $\Gamma $ of an element $x$ with its stabilizer group by
\begin{align*}
\abso{\Gamma } = [G:\operatorname{Stab}(x)]
\end{align*}
Now, since we clearly have 
\begin{align*}
\operatorname{Stab}_{A_n}(x)= A_n \cap \operatorname{Stab}_{S_n}(x)
\end{align*}
if $\operatorname{Stab}_{S_n}(x)\leq A_n$, then the conjugacy class of $x$ in $A_n$ is the half of its conjugacy class in $S_n$. On the other hand, if $\operatorname{Stab}_{S_n}(x)\not \leq A_n$, then since $A_n\cdot \operatorname{Stab}_{S_n}(x)=S_n$, by \customref{THof}{order formula}, we know $[A_n:\operatorname{Stab}_{A_n}(x)]=[S_n:\operatorname{Stab}_{S_n}(x)]$, which implies that the conjugacy class of $x$ in $A_n$ is the same as  in  $S_n$. 
\end{Example}
\begin{theorem}
\label{THzSn}
\textbf{($S_n$ is centerless for $n\geq 3$)} Let $n \geq 3$. Then $Z(S_n)=1$.
\end{theorem}
\begin{proof}
Let $\tau \neq e\in S_n$. Because $\tau$ is nontrivial, we know $\tau (i)=j$ for some $i \neq j$. Let $\sigma (i)=i$ and $\sigma (j) \neq j$. We now see $\sigma \circ \tau \circ \sigma^{-1}\neq \tau$, as desired. 
\end{proof}
\begin{theorem}
\label{THgos}
\textbf{(Generators of $S_n$)} Let $n\geq 2$. Then $S_n$ can be generated by any of the followings: 
\begin{enumerate}[label=(\roman*)]
  \item transpositions.
  \item  $\set{(1,k) \in S_n : 2 \leq k \leq n}$. 
  \item  $\set{(1,2),(1,\dots ,n)}$.
\end{enumerate}
\end{theorem}
\begin{proof}
(i) follows from: 
\begin{align*}
  (1,\dots ,k)= (2,3) (3,4) \cdots (k-2,k-1)(k-1,k)(1,k)
\end{align*}
(ii) then follows from: 
\begin{align*}
  (i,j)= (1,i)(1,j)(1,i)^{-1}
\end{align*}
To prove (iii), we then only have to prove that (ii) can be generated by (iii), which is done by two inductions, with one of them being 
\begin{align*}
  (k,k+1)(1,k)(k,k+1)^{-1}= (1,k+1)
\end{align*}
while the other being 
\begin{align*}
g(i,i+1)g^{-1} =(i+1,i+2),\quad \text{ for all }i\leq n-2
\end{align*}
where we denote $g\triangleq (1,\dots ,n)$.
\end{proof}
\begin{theorem}
\label{TH3c}
\textbf{($A_n$ are generated by  $3$-cycles for  $n\geq 3$)} Let $n\geq 3$. Then $A_n$ is generated by $3$-cycles. 
\end{theorem}
\begin{proof}
Such is proved via induction on $n$. The base case is clear. We now prove the inductive case. Let $\sigma \in A_{n}$. By inductive hypothesis, if $\sigma $ doesn't move $n$, then  $\sigma$ can be generated by $3$-cycles. If  $\sigma$ move $n$, then because inverse of a $3$-cycle is a $3$-cycle, and because there exists a  $3$-cycle  $\tau$ such that $\tau \circ \sigma$ fix $n$, we see  $\sigma$ can also be generated by $3$-cycles.  
\end{proof}
\begin{theorem}
\label{THcsS}
\textbf{($S_n^{(1)}=A_n$ for $n\geq 3$)} Let $n\geq 3$. Then $S_n^{(1)}=A_n$. 
\end{theorem}
\begin{proof}
$S^{(1)}_n \leq A_n$ follows from computation. Because \customref{TH3c}{$A_n$ is generated by $3$-cycles}, to show $S_n^{(1)}=A_n$, we only have to show $S_n^{(1)}$ contains all  $3$-cycles, which follows from computing 
\begin{align*}
  (a,b,c)=[(a,b),(a,c)]
\end{align*}
\end{proof}
We are now ready to prove that \customref{THsa}{$A_n$ is simple for  $n\geq 5$}.  
\begin{theorem}
\label{THsa}
\textbf{(Simplicity of $A_n$)} $A_3$ is a simple group. $A_4$ is not a simple group, since it has the characteristic $2$-Sylow subgroup: 
\begin{align}
\label{EQsa}
\set{e, (1,2)(3,4),(1,3)(2,4),(1,4)(2,3)}
\end{align}
Let $n\geq 5$. Then $A_n$ is simple. 
\end{theorem}
\begin{proof}
Simplicity of $A_3$ follows from  $o(A_3)=3$. We now prove (ii): There are three ways to partition $\set{1,2,3,4}$ into $2$ disjoint subsets, each of cardinality $2$, i.e., 
\begin{align*}
\Pi_1 \triangleq \set{\set{1,2},\set{3,4}}\quad \text{ and } \quad \Pi_2 \triangleq \set{\set{1,3},\set{2,4}}\quad \text{ and }\quad \Pi_3 \triangleq \set{\set{1,4},\set{2,3}}
\end{align*}
Clearly, $S_4$ acts on $\set{\Pi_1,\Pi_2,\Pi_3}$, that is, $S_4 \longrightarrow \operatorname{Sym}(\set{\Pi_1,\Pi_2,\Pi_3})\cong S_3$. Direct computation now shows that we have a surjective group homomorphism $A_4 \longtwoheadrightarrow A_3 $ with kernel is \myref{set}{EQsa}. \\


(iii): We first prove  \vi{$A_5$ is simple}. Because $A_5\trianglelefteq S_5$ is normal, we know $A_5$ is the union of the classes of cycle types 
\begin{align*}
  (1,2,3)\quad \text{ and }\quad (1,2)(3,4) \quad \text{ and }\quad (1,2,3,4,5)
\end{align*}
Direct computation shows that the first cycle type class has $20$ elements, that the second cycle type class has  $15$ elements, and that the third cycle type class has  $24$ elements. It now follows from \customref{THLt}{Lagrange's theorem} and from \customref{EDns}{normal subgroups are unions of conjugacy classes} that $A_5$ has no proper nontrivial subgroups. $\vdone$\\

Let $N \trianglelefteq A_n$, with $\tau \neq e \in N$ and $n>5$. We are required to show $N=A_n$.  Because $n\geq 5$, clearly for any pair of $3$-cycles $\set{(a_1,a_2,a_3),(b_1,b_2,b_3)}$, there exists some $\sigma \in A_n$ such that $(b_1,b_2,b_3)= \sigma \circ  (a_1,a_2,a_3)\circ  \sigma ^{-1}$. In other words, the class of $3$-cycles forms a conjugacy class of  $A_n$. This together with normality $N\trianglelefteq A_n$ and the fact that \customref{TH3c}{$A_n$ is generated by set of $3$-cycles} reduce the problem into proving $N$ contains a $3$-cycle.  \\

Consider $A_5$ as a subgroup of $A_n$ that fixes a particular set of $n-5$ numbers. Then clearly we have $N \cap A_5 \trianglelefteq A_5$. Because  $A_5$ is simple and contains a  $3$-cycle, we now only have to show  $ N \cap A_5>1$, that is, to show $N$ contains an element that fixes at least $n-5$ elements. \\

Let   $(a_1,a_2,a_3)$ be a $3$-cycle such that $\set{\tau (a_1),\tau (a_2),\tau (a_3)}, \set{a_1,a_2,a_3}$ overlap. We now see $\tau \circ (a_1,a_2,a_3)\circ \tau^{-1} \circ  (a_1,a_2,a_3)^{-1} \in N $ is an element that fixes at least $n-5$ elements.  
\end{proof}
\begin{theorem}
\label{THnSn}
\textbf{($A_n$ is the only proper nontrivial normal subgroup of $S_n$ for $n\geq 5$)} Let $n\geq 5$. Then $A_n$ is the only proper nontrivial subgroup of $S_n$.  
\end{theorem}
\begin{proof}
Let $N$ be a proper nontrivial subgroup of $S_n$. We are required to show $N=A_n$. Because $[S_n:A_n]=2$, we only have to show $A_n \leq N$. This boils down to showing $N\cap A_n>1$, since \customref{THsa}{$A_n$ is simple}. Assume for a contradiction that $N\cap A_n=1$. The contradiction $N=1$ then follows from \customref{THcsS}{$A_n = S_n^{(1)}$}  and \customref{THzSn}{$Z(S_n)=1$}, since \customref{THpcs}{normal subgroup that disjoint with the commutator subgroup must be contained by the center}.  
\end{proof}
A group is said to be \textbf{complete} if the group has no outer automorphism and centerless. In other words, the natural map $G \longrightarrow \operatorname{Inn}(G)$ forms an isomorphism between $G$ and  $\operatorname{Aut}(G)$.
\begin{theorem}
\textbf{($S_n$ is complete for $n\neq 2$ or $6$)} Let $n\geq 3$. Then $S_n$ is complete for  $n\neq 6$.  
\end{theorem}
\begin{proof}
Clearly, automorphism group $\operatorname{Aut}(G)$ in general acts on conjugacy classes of $G$. That is, we have a group homomorphism  $\operatorname{Aut}(G)\longrightarrow \operatorname{Bij}(\operatorname{cl}(G))$.\\

Let $\alpha  \in \operatorname{Aut}(S_n)$. We first show that \vi{$\alpha  \in \operatorname{Stab}(\operatorname{cl}((1,2)))\implies  \alpha \in \operatorname{Inn}(S_n)$}. Because \customref{THgos}{$S_n$ is generated by $\set{(1,k) \in S_n: 2 \leq k \leq n}$}, we only have to show the existence of some $\sigma \in S_n$ such that 
\begin{align*}
\alpha ((1,k))= \sigma (1,k) \sigma^{-1}= (\sigma (1),\sigma (k)),\quad \text{ for all }k\geq 2
\end{align*}
Because of the premise, we already know that $\alpha  ((1,k))$ is a $2$-cycle. Our first task is to show that the intersection of the $2$-cycles  $\set{\alpha ((1,k)): k\geq 2}$ is nonempty. Write 
\begin{align*}
\alpha ((1,2))\triangleq (a,b_2) 
\end{align*}
Because $(1,2)(1,3)$ is a $3$-cycle, we know  $\alpha ((1,2))\alpha  ((1,3))=\alpha ((1,2)(1,3))$ is of order $3$, which is only possible if the $2$-cycle $\alpha ((1,3))$ share exactly one element with the $2$-cycle  $(a,b_2)$. WLOG, let that element be $a$, and write 
\begin{align*}
\alpha  ((1,3))=(a,b_3),\quad \text{ where }b_3 \not \in \set{a,b_2}
\end{align*}
Applying the same logic to $\alpha ((1,4))$, we see that the $2$-cycle $\alpha ((1,4))$ must share exactly one element with $(a,b_2)$ and must share exactly one element with $(a,b_3)$. Therefore, we know $\alpha ((1,4))$ either is of the form 
\begin{align*}
  (a,b_4),\quad \text{ with }b_4 \not \in \set{a,b_2,b_3}
\end{align*}
or is of the form $(b_2,b_3)$. To see that the latter is impossible, simply compute the order of $(1,2)(1,3)(1,4)$ and $(a,b_2)(a,b_3)(b_2,b_3)$. Now, we apply the same logic to $\alpha ((1,5))$, and this time we can conclude that  $\alpha ((1,5))$ must be of the form $(a,b_5)$ with $b_5 \not\in \set{a,b_2,b_3,b_4}$. Repeating the process, we then see 
\begin{align*}
\sigma (1)\triangleq a \quad \text{ and }\quad \sigma (k)\triangleq b_k
\end{align*}
suffices. $\vdone$ \\

Ir remains to show every $\alpha  \in \operatorname{Aut}(S_n)$ fix the class of transposition. Since $\alpha $ must maps a transposition to an element of order $2$. We only have to prove 
\begin{center}
   \begin{minipage}{0.9\linewidth}  
     \blue{For $n \neq 6$, the class of transposition has unique cardinality, which is $\binom{n}{2}$,  among the conjugacy classes whose elements are of order $2$}.
   \end{minipage}
\end{center}
Clearly, the conjugacy classes whose elements has order $2$ are the classes of product of disjoint $2$-cycles. Assume for a contradiction that there really is such a class that has the same cardinality of the class of transposition, says, this class is the class of $k$ product of disjoint  $2$-cycle. We then would have 
\begin{align*}
\binom{n}{2}= \frac{1}{k!} \binom{n}{2} \binom{n-2}{2} \cdots \binom{n-2(k-1)}{2}
\end{align*}
Noticing the RHS is telescoping, we compute  
\begin{align}
\label{EQsn2}
\binom{n-2}{2k-2}=  \frac{k!(2^{k-1})}{(2k-2)!}
\end{align}
Because the RHS now is not an integer for $k\geq 4$, we now have $k \in \set{1,2,3}$. Direct computation now shows that for \myref{equation}{EQsn2} to holds, we must have $k=3$ and  $n=6$.  $\bdone$
\end{proof}
We say a group action $G \longrightarrow  \operatorname{Bij}(X)$ is \textbf{transitive} if for each $x,y \in X$, there exists some $g \in G$ such that $g\cdot x=y$. We say a subset $S\subseteq S_n$ is \textbf{transitive} if for each $i,j \in \set{1,\dots ,n}$ there exists some $s \in S$ such that $s (i)=j$  
\begin{theorem}
\textbf{($S_6$ is incomplete)}  $S_6$ is not complete. 
\end{theorem}
\begin{proof}
  \customref{SECSt}{Sylow theorems} shows that $S_5$ has $6$ Sylow  $5$-subgroups, and the conjugacy action $\pfi : S_5 \rightarrow \operatorname{Bij}(\operatorname{Syl}_5(S_5))\cong S_6$ is transitive. In particular, the $6$ Sylow  $5$-subgroups are the cyclic subgroups generated by the $6$ distinct  $5$-cycles. Because \customref{THnSn}{$A_5$ is the only proper nontrivial normal subgroup of $S_5$}, we know $\operatorname{ker}(\pfi )\in \set{1,A_5,S_5}$. Clearly it isn't  $S_5$. To see it isn't  $A_5$, just note that if it is, then its images would have all have order $2$, which just isn't the case, as one can observe that 
\begin{align*}
  (1,2,3)^2 \langle (2,3,4,5,6)\rangle   (1,2,3)^{-2} \neq \langle (2,3,4,5,6)\rangle 
\end{align*}
Therefore, the only possibility is that $\pfi $ is injective. Denote $H \triangleq \pfi  (S_5)\leq S_6$. The fact that $\pfi $ is transitive now means that $H\leq S_6$ is transitive. \\

Now, consider the left multiplication action $\sigma : S_6 \rightarrow \operatorname{Bij}(S_6 \quotient H)$ on the left coset spaces of $H$. Because \customref{THnSn}{$A_6$ is the only proper nontrivial normal subgroup of $S_6$} and $\operatorname{Core}(H)\leq H$, we know $\sigma $ must be injective. In other words, $\sigma$ is an automorphism. \\

Note that all elements of $\sigma (H)\leq S_6$ fix a point, i.e., $H \in S_6 \quotient H$ itself, while $H\leq S_6$, since being transitive, fix no points. If $\sigma$ is inner, this is clearly impossible, thus the proof.  
\end{proof}
\section{Commonsense in Finite Group Theory}
\begin{abstract}
This section develop some commonsense about finite groups. 
\end{abstract}
\begin{theorem}
\label{COzqp}
\textbf{(Classification of semidirect product of $C_q \rtimes C_p$ with $p\mid  q-1$)} Let $p,q$ be two primes. If $p \nmid q-1$, then the semidirect product $C_q \rtimes C_p$ must be a direct product. If $p \mid  q-1$, then the semidirect product $C_q \rtimes C_p$ has exactly two non-isomorphic meanings. 
\end{theorem}
\begin{proof}
Clearly, we have 
\begin{align*}
\operatorname{Aut}(C_q) \cong    \customref{THzpc}{\Z_q ^{\times}\cong C_{q-1}} 
\end{align*}
Let $x$ and $y$ each be the generators of  $C_p$ and $C_{q-1}$. Let $x\mapsto y^c$ defines a group homomorphism. Because $x^p=e$, we must have $q-1 \mid  pc$. Therefore, if $p\nmid q-1$, the only action  $\phi\in \operatorname{Hom}(C_p,\operatorname{Aut}(C_{q}))$ is trivial.  \\


Suppose $p \mid  q-1$. By the \customref{THp}{universal property of presentation}, we now see that the possible group homomorphism $\operatorname{Hom}(C_p,C_{q-1})$ are exactly  $x \mapsto  y^{\frac{q-1}{p}d}$ for $d\inz$. Let $\phi_d$ and $\phi:\Z_p\rightarrow \Z_{q-1}$ denote $x\mapsto y^{\frac{q-1}{p}d}$ and $x \mapsto y^{\frac{q-1}{p}}$ with $d\neq 0$. Clearly we have 
\begin{align*}
\phi_d = \phi \circ \psi 
\end{align*}
where $\psi\in \operatorname{Aut}(C_p)$ is defined by $\psi (x)\triangleq x^d$. Because \customref{EQfAutH}{two actions differs by a automorphism in front induces the same semidirect product}, we have shown that there can be at most two distinct semidirect product $C_q \rtimes C_p$.  \\

To see that they are distinct, we claim that the nontrivial one is not abelian, which follows from noting that  
\begin{align*}
  (n,e)\cdot (e,h) = (n,h)\quad  \text{ and }\quad (e,h)\cdot (n,e)= (\phi_h (n),h) 
\end{align*}
in general. 
\end{proof}
We now have enough tools to state and prove our result. Note that \customref{THus}{normal Sylow subgroups are clearly characteristic}, so we will use the word "characteristic Sylow subgroup" instead.  
\begin{theorem}
\label{COaSo}
\textbf{(Analysis of finite group of fixed prime structure)} Let $p,q,r$ be three distinct prime. Then, 
\begin{enumerate}[label=(\roman*)]
  \item Groups of order $pq$, where $p<q$, is always a semidirect product $C_q \rtimes C_p$, \customref{COzqp}{and we know there are at most $2$ of them}, depending on whether $p \mid  q-1$. 
  \item Groups of order $p^2q$ has a characteristic Sylow subgroup.   
  \item Groups of order $p^3q$ has a characteristic Sylow subgroup, given that $(p,q)\neq (2,3)$. 
  \item Groups of order $pqr$ has a characteristic $r$-Sylow subgroup and a normal subgroup of order $qr$, where $p<q<r$. If $q \nmid r-1$, then groups of order $pqr$ moreover has a characteristic $q$-Sylow subgroup. 
  \item Simple groups of order $p^am$, where $a,m \inn$ satisfies $p\nmid m>1$, must satisfies $o(G)\mid  n_p!$.   
\end{enumerate}
\end{theorem}
\begin{proof}
(i): Clearly $n_q=1$. Let $P\in \operatorname{Syl}_p(G)$. Clearly $P\cap Q=1$. This by \customref{THof}{order formula} implies $PQ=G$. The rest then follows from  \customref{EDrip}{recognition theorem for inner semidirect product}. \\ 


(ii): If $p>q$, then clearly  $n_p$ can only be  $1$. If  $p<q$, then we must have $n_q \in \set{1,p^2}$. If $n_q=1$, then we are done. If not, then from \customref{THfSt}{$n_q \equiv 1 \pmod{q}$}, we see $q=p+1$, which can only happens if $q=3$ and  $p=2$. We claim that in such case, i.e., $o(G)=12$, then either $n_3=1$ or  $G \cong  A_4$, which contains a characteristic $2$-Sylow subgroup. \\

If $n_3=4$, then for any $P \in \operatorname{Syl}_3(G)$, we have $N_G(P)=P$, since \customref{THtSt}{$4=n_3=[G:N_G(P)]$}.  Clearly, the conjugacy action $G \longrightarrow   \operatorname{Bij}(\operatorname{Syl}_3(G))$ has kernel contained by $N_G(P)=P$. This by non-normality of $P$ and $o(P)=3$ implies the conjugacy action  $G \longhookrightarrow \operatorname{Bij}(\operatorname{Syl}_3(G))\cong  S_4$ is injective. Note that $G$ has  $8$ elements of order  $3$. Since  $A_4$ is the only subgroup of  $S_4$ with order $12$ and  $8$ elements of order  $3$, we now see  $G\cong  A_4$.  \footnote{One may argue that the original assertion that groups of order $p^2q$ all have a normal Sylow subgroup holds true, but we don't know whether it is possible $n_3=4$. Such is possible, since  $A_4$ really makes $n_3=4$.} The proof then follows from recalling that \customref{THsa}{$A_4$ has a characteristic  $2$-Sylow subgroup}. 


(iii): Suppose $G$ has no characteristic Sylow subgroup. We are required to show  $(p,q)=(2,3)$, which will follows from showing $q=p+1$. \\

 Now, since $G$ has no characteristic Sylow subgroup, we know  $n_p=q$, which by $n_p \equiv 1 \pmod{p}$ implies $p<q$, which further implies $n_q\in \set{p^2,p^3}$. Counting now give us $n_q=p^2$, which by $n_q \equiv 1 \pmod{p}$ and $p<q$ implies  $q=p+1$.\\



(iv): We first show $n_r=1$. By counting, we know $1 \in \set{n_p,n_q,n_r}$. If $n_p$ and  $n_q$ are both  $>1$, then we are done. We now show that $n_p=1 \implies n_r=1$, and the proof for $n_q=1 \implies n_r=1$ is similar.  \\

Denote $P$ the characteristic Sylow $p$-subgroup of $G$. Applying (i) on $\frac{G}{P}$, we see the existence of $H\trianglelefteq  G$ with order $pr$ containing $P$. Applying (i) on $H$, we see the existence of $K\operatorname{char}H$ with $o(K)=r$. It then follows from  \customref{THus}{transitive property of characteristic  subgroup} that $K$ is the characteristic  $r$-Sylow subgroup of  $G$ we are looking for.  \\

To see $G$ has a normal subgroup of order  $qr$, just apply  (i) on $G \quotient K$. Denote that normal subgroup by $T \trianglelefteq G$. If $q \nmid  r-1$, then again by (i), there exists $L\operatorname{char}T$ with $o(L)=r$. Because of the \customref{THus}{transitive property of characteristic  subgroup}, we now see that $L$ is the characteristic  $r$-Sylow subgroup of  $G$ we are looking for.\\

(v): Let $P \in \operatorname{Syl}_p(G)$. Non-normality of $P$ together with \customref{THsSt}{second Sylow theorem} implies that the  conjugacy action $G \longrightarrow \operatorname{Bij}(\operatorname{Syl}_p(G))\cong  S_{n_p}$ is nontrivial. Simplicity of $G$ then forces the action to be injective, as desired.
\end{proof}
\begin{theorem} \textbf{(Analysis of groups of fixed order)} We have: 
\begin{enumerate}[label=(\roman*)]
  \item There are exactly four groups of order $30$. Precisely, they are the four possible semidirect product of $C_{15}\rtimes C_2$ : 
\begin{align*}
\langle x,y\mid  x^{15}=y^2=e,yxy^{-1}=x^n\rangle  \quad \text{ where }n \in \set{1,4,11,-1}
\end{align*} 
\item The $11$-sylow subgroup of groups of order $231=3 \cdot 7 \cdot 11$ lies in the centers.   
\item The $7$-sylow subgroup of groups of order  $385=5\cdot 7 \cdot 11$ lies in the centers.
  \item No group of order $132=2^2\cdot 3 \cdot 11$ is simple. 
  \item Groups of order $108=2^2\cdot 3^3$ either has a normal subgroup of order $9$, or has a  normal subgroup of order $27$.  
  \item Simple groups of order $60$ must be $A_5$.  
\end{enumerate}
\end{theorem}
\begin{proof}
  (i): Since $30=2\cdot 3\cdot 5$, \customref{COaSo}{we know $G$ has a normal subgroup $N$ of order $15$}. This allows us to write $G \cong  N \rtimes C_2$. Since \customref{COaSo}{the only group of order $15$ is $C_{15}$}, we now see $G$ must be of the form: 
\begin{align*}
G\cong C_{15} \rtimes C_2
\end{align*}
To classify the possible semidirect product, one first compute $\operatorname{Aut}(C_{15})\cong C_4\times C_2$, \footnote{This can proved by noting $\operatorname{Aut}(C_{15})\cong  \Z_{15}^{\times}\cong  (\Z_5 \times \Z_3)^{\times} \cong  \Z_5^{\times}\times \Z_3 ^{\times}\cong C_4 \times C_2$. Of course, one can also just brute force the computation.} which show us that $\operatorname{Hom}(C_2,\operatorname{Aut}(C_{15}))$ has only $4$ elements, which can be easily checked to be $y \mapsto (x \mapsto x^n)$, where $C_2= \langle y\rangle ,C_{15}=\langle x\rangle $ and $n \in \set{1,4,11,14}$. To see that these four possibly distinct groups are really pairwise distinct, one can use \customref{THposd}{their presentation} to easily compute that the number of elements of order $2$ they contain are indeed pairwise different.  \\


(ii): Let $R$ be the characteristic  $11$-sylow subgroup of $G$.  \customref{THpoc}{We are required to prove $C_G(R)=G$}. Because $C_G(R)$ is exactly the kernel of the conjugacy action $G = N_G(R)\longrightarrow \operatorname{Aut}(R)$, by \customref{THfit}{first isomorphism theorem}, we have an injection $G \quotient C_G(R) \longhookrightarrow \customref{THzpc}{\operatorname{Aut}(R) \cong C_{10}}$. The proof then follows from counting order. \\


(iii): The proof is similar to that of (ii). \\   

(iv):  If $G$ is simple, then $n_2\geq 3,n_3\geq 4$, and $n_{11}\geq 12$, which is impossible by counting.  \\

(v): Let $n_3=4$. We are required to prove the existence of a normal subgroup of order  $9$. Let $P\in \operatorname{Syl}_3(G)$. Consider the left multiplicative action $G \longrightarrow \operatorname{Bij}(G\quotient P)\cong S_4$ on the left cosets space $G \quotient P$. The proof then follows from comparing orders and the observation that the kernel must lies in $P$.   \\

(vi): Because  \customref{THnSn}{the only proper nontrivial normal subgroup of $S_5$ is $A_5$} and because $o(G)=60$, to show $G \cong  A_5$, we only have to establish an injective group homomorphism $G\longhookrightarrow S_5$. \\

Let $P\in \operatorname{Syl}_2(G)$. By simplicity of $G$, we always have an injective group homomorphism $G \longhookrightarrow  \operatorname{Bij}(G \quotient N_G(P))\cong S_{[G:N_G(P)]}$, the left multiplicative action on the left coset space of $N_G(P)$. Since  \customref{THtSt}{$[G:N_G(P)]=n_2$}, it remains to show $n_2=5$. \\

Because $G$ is simple and $60\nmid 4!$, from $G \quotient \operatorname{Core}(H)\longhookrightarrow  \operatorname{Bij}(G \quotient H)$, where $G \quotient H$ is the left coset space of $H$, we know that $G$ can not have proper subgroup with index $<5$. This with  \customref{THtSt}{$[G:N_G(P)]=n_2$} in particular implies $n_2\geq 5$. Because $G$ is simple, we now have $n_2 \in \set{5,15}$. \\


Assume $n_2=15$ for a contradiction. Because $n_5=6$, by counting, we see that there must be a pair of distinct  $2$-Sylow subgroup  $Q,R\in \operatorname{Bij}_2(G)$ such that $o(Q\cap R)=2$. Let $M \triangleq N_G(Q\cap R)$. Since  $Q$ and $R$, of order $4$, is abelian, we know  $Q,R \leq  M$, which implies $4 \mid  o(M)$, that is, $o(M)\in \set{4,12,20,60}$. Simplicity of  $G$ forces $o(M)\neq 60$, $Q\neq R$ forces $o(M)\neq 4$, and the fact that $G$ has no proper subgroup of index $<5$ forces  $o(M)\neq 20$. Therefore, we must have $o(M)=12$. By simplicity of $G$, we now have an injective group homomorphism  $G \longhookrightarrow \operatorname{Bij}(G \quotient M)\cong  S_5$, the left multiplicative action on the left coset space of $M$, as desired.
\end{proof}

\section{Simplicity of $\operatorname{PSL}_n(\F)$} 
\begin{abstract}
This section shows that for $n\geq 3$, $\operatorname{PSL}_n(\F)$ is simple, and for $n=2$, $\operatorname{PSL}_n(\F)$ is simple if $\F$ has $\geq 4$ elements.  
\end{abstract}
An action $\phi : G \longrightarrow \operatorname{Bij}(X)$ is called \textbf{transitive} if for all $x,y \in X$, there exist some $g$ that takes $x$ to  $y$. We say $\phi$ is \textbf{doubly transitive} if the naturally induced action on $G \longrightarrow \operatorname{Bij}(X^2 - D)$, where $D \triangleq \set{(x,x)\in X^2 : x \in X}$ is the diagonal,  is transitive. 
\begin{theorem}
\label{THpdta}
\textbf{(Property of doubly transitive action)} Let $\phi : G \rightarrow  \operatorname{Bij}(X)$ be a doubly transitive action and $\abso{X}\geq 2$. Then 
\begin{enumerate}[label=(\roman*)]
  \item $\operatorname{Stab}(x) < G$ is maximal for all $x \in X$. 
  \item For all $N \trianglelefteq G$, the action  $\phi |_N:N \rightarrow  \operatorname{Bij}(X)$ is either trivial or transitive.  
\end{enumerate}
\end{theorem}
\begin{proof}
(i): Denote $H \triangleq \operatorname{Stab}(x)$. We first show that: 
\begin{align*}
\vi{G= H \cup  HgH,\quad \text{ for all }g \in G-H,\text{ where }HgH\triangleq \set{hg\tilde{h}\in G: h,\tilde{h}\in H  }}
\end{align*}
Fix $g \in G-H$. Let $\tilde{g}\in G-H$. We are required to show $\tilde{g} \in HgH$. Because $g, \tilde{g}\not \in H$, doubly transitivity of $\phi$ implies the existence of some element in $G$ that takes  $(x,gx)$ to $(x,\tilde{g}x)$. Such element is clearly in $H$. One then can check $\tilde{g} \in hgH$, where $h$ is the element that takes  $gx$ to  $\tilde{g}x$. \vdone\\

$H<G$ is clear. To see that $H$ is maximal, just observe that if a subgroup $K$ of $G$ properly contains  $H$, then the subgroup contains  $H \cup HgH=G$, where $g \in K-H$. \\

(ii): Suppose the action $N \longrightarrow \operatorname{Bij}(X)$ is nontrivial. We are required to show it is transitive. Fix $x \neq  y \in X$. Because $N$ is nontrivial, we know there exists some  $z \in X$ and $n \in N$ such that $nz \neq z$. Doubly transitivity of $\phi$ then implies the existence of some $g \in G$ such that $g(z,nz)=(x,y)$. The proof then follows from checking that $gng^{-1}$ takes $x$ to  $y$. 
\end{proof}

\begin{theorem}
\label{THIc}
\textbf{(Iwasawa criterion)} Let $\phi : G \rightarrow \operatorname{Bij}(X)$ be a doubly transitive action. If 
\begin{enumerate}[label=(\roman*)]
  \item $G$ is \textbf{perfect}, i.e., $G^{(1)}=G$. 
  \item There exists some $x \in X$ whose stabilizer subgroup has an abelian normal subgroup $U$ such that $\bigcup_{g \in G}gUg^{-1}$  generates $G$. 
\end{enumerate}
then $G \quotient \operatorname{ker}\phi$ is simple. 
\end{theorem}
\begin{proof}
Denote $K\triangleq \operatorname{ker}\phi$ and $H \triangleq \operatorname{Stab}(x)$. Let $N$ be a normal subgroup of $G$ that contains  $K$. We are required to prove $N=K\text{ or }G$. \customref{THpdta}{Maximality of $H$} splits the proof into two cases $NH=H$ or $NH=G$. We will prove that the case $NH=H$ leads to $N=K$, and the case  $NH=G$ leas to  $N=G$. \\

Case ($NH=H$): In such case, clearly $N \leq H$, and therefore \customref{THpdta}{$N$ acts trivially on $X$}, which implies $N \leq K$, as desired. \\

Case ($NH=G$): In such case, normality $U \trianglelefteq H$ implies $NU \trianglelefteq G$. This then implies that $gUg^{-1} \leq g(NU)g^{-1} = NU$ for all $g \in G$. Therefore by premise, $NU=G$. \customref{THsit}{Second isomorphism theorem} then shows that $G \quotient N \cong  NU \quotient N \cong  U \quotient  N \cap U$ is abelian, which by perfectness of $G$ implies that  $N=G$.    
\end{proof}
Let $\F$ be a field. The \textbf{general linear group} $\operatorname{GL}_n(\F)$ is the group of $n$-by-$n$  $\F$-valued matrices that has nonzero determinant. Clearly, $\operatorname{GL}_n(\F)$ acts naturally on the affine space $\A^n(\F)$, and moreover on the projective space $\P^{n-1}(\F)$. 
\begin{theorem}
\label{THkgl}
\textbf{(Kernel of general linear group acting on projective space)} The kernel of the group action $\operatorname{GL}_n(\F) \longrightarrow \operatorname{Bij}(\P^{n-1})$ is exactly the group of scalar diagonal:
\begin{align*}
\set{cI \in \operatorname{GL}_n(\F): c \in\F^{\times}}
\end{align*}
coinciding with its center $Z(\operatorname{GL}_n(\F))$. 
\end{theorem}
\begin{proof}
We first show that the group of the scalar diagonal $=$ the kernel. Clearly the group of scalar transformations lies in the kernel. To see the converse inclusion holds, let $A$ maps 
 \begin{align*}
v \mapsto  \ld  v\quad \text{ and }\quad w \mapsto \mu w
\end{align*}
and observes that if $\ld  \neq \mu$, then $A$ doesn't fix $[v+w]$. \\

It remains to show the center $=$ the group of scalar diagonal. Again, clearly the group of scalar diagonal lies in the center. Let $A \in Z\left(\operatorname{GL}_n(\F) \right)$. To prove that $A$ is scalar diagonal, one simply consider $E_{i,j}\triangleq I_n + e_{i,j}$, where $e_{i,j}\in M_n(\F)$ is the \textbf{matrix unit}. 
\end{proof}
Because \customref{THkgl}{the group action $\operatorname{GL}_n(\F)\rightarrow \operatorname{Bij}(\P^{n-1})$ has such kernel}, we define the \textbf{projective general linear group} $\operatorname{PGL}_n(\F)$ to be the quotient group
\begin{align*}
\operatorname{PGL}_n(\F)\triangleq \operatorname{GL}_n(\F) \quotient \set{cI_n \in \operatorname{GL}_n(\F): c \in \F^{\times}}  
\end{align*}
The \textbf{special linear group} $\operatorname{SL}_n(\F) \leq  \operatorname{GL}_n(\F)$ is the subgroup whose elements has determinant $1$, and \customref{THkgl}{the kernel of its action on $\P^{n-1}$ is clearly $\set{cI_n \in \operatorname{SL}_n(\F): c^n =1}$}, so we define the \textbf{projective special linear group} $\operatorname{PSL}_n(\F)$ to be 
\begin{align*}
\operatorname{PSL}_n(\F) \triangleq \operatorname{SL}_n(\F) \quotient \set{c I_n \in \operatorname{SL}_n(\F): c\in\F^{\times}\text{ is a $n$-th root of unity.}}
\end{align*}
\begin{theorem}
\label{THpslg}
\textbf{(Basic properties of special linear group)} Let $\F$ be an arbitrary field and  $n\geq 2$. Then,   
\begin{enumerate}[label=(\roman*)]
  \item $\operatorname{SL}_n(\F)$ acts doubly transitive on $\P^{n-1}$.  
  \item $\operatorname{SL}_n(\F)$ has center $\set{cI_n \in \operatorname{SL}_n(\F):c\in\F^{\times}\text{ is a $n$-th root of unity.}}$, thus equal to the kernel of its action on $\P^{n-1}$.
\end{enumerate}
\end{theorem}
\begin{proof}
  (i): Let $\left([v_1],[v_2] \right),\left([w_1],[w_2] \right) \in (\P^{n-1})^2$ be two pairs off the diagonal. We are required to show the existence of some $\widehat{L} \in \operatorname{SL}_n(\F)$ that send $[v_1]$ to  $[w_1]$ and  $[w_1]$ to  $[w_2]$. The proof then follows from extending them to two bases  $\set{v_1,\dots ,v_n}, \set{w_1,\dots ,w_n}$ for $\F^{n}$ and setting $\widehat{L}(v_i)\triangleq c_iw_i$ with 
\begin{align*}
\operatorname{det}(PQ^{-1})\cdot \left(\prod_{i=1}^n c_i    \right)= 1
\end{align*}
where $Q,P\in \operatorname{GL}_n(\F)$ are respectively the matrix whose $i$-th column is $v_i,w_i$.  (ii): \customref{THkgl}{The proof that computes $Z(\operatorname{GL}_n(\F))$} applies here, since $E_{i,j} \in \operatorname{SL}_n(\F)$. 
\end{proof}
To apply \customref{THIc}{Iwasawa criterion} on the projective special linear groups $\operatorname{PSL}_n(\F)$, it remains to show 
\begin{enumerate}[label=(\Roman*)]
  \item $\operatorname{SL}_n(\F)$ is perfect. 
  \item There exists some $x \in \P^{n-1}$ whose stabilizer subgroup $H \leq \operatorname{SL}_n(\F)$ has an abelian normal subgroup $U$ such that $\bigcup_{g \in G}gUg^{-1}$ generates $G$.  
\end{enumerate}
The reasons why simplicity of  $\operatorname{SL}_2(\F)$ requires $\F$ to have $\geq 4$ elements lies in the fact (I) may fails to be true if $\F$ has $\leq 3$ elements.  
\begin{Example}{$\operatorname{SL}_2(\F_2)$ and $\operatorname{SL}_2(\F_3)$ are not perfect}{}
By counting, we know 
\begin{align*}
o\left(\operatorname{GL}_n(\F_p) \right)=  (p^n-1)(p^n - p)\cdots  (p^n-p^{n-1})
\end{align*}
Since determinant function is a  surjective group homomorphism from $\operatorname{GL}_n(\F_p)$ to $\F_p^{\times}$  with kernel $\operatorname{SL}_n(\F_p)$, we know 
\begin{align*}
o \left(\operatorname{SL}_n(\F_p) \right) = \frac{1}{p-1} \cdot  (p^n-1)(p^n - p)\cdots  (p^n-p^{n-1})
\end{align*}
In particular, $\operatorname{SL}_2(\F_2)$ and $ \operatorname{SL}_2(\F_3)$ has order $6$ and $24$. Now, clearly $\operatorname{SL}_2(\F_2)$ has a faithful action on:   
  \begin{align*}
  \set{\begin{pmatrix} 
  1 \\
  0
\end{pmatrix} , \begin{pmatrix} 
0 \\
1
\end{pmatrix}, \begin{pmatrix} 
1 \\
1
\end{pmatrix}}
  \end{align*}
So by comparing order, we know $\operatorname{SL}_2(\F_2)\cong  S_3$. Therefore non-perfectness of $\operatorname{SL}_2(\F_2)$ then follows from \customref{THcsS}{$S_3^{(1)}=A_3$}. In fact, since $\operatorname{PSL}_2(\F_2) \cong  \operatorname{SL}_2(\F_2)$, we have shown that $\operatorname{PSL}_2(\F_2)$ is non-simple. \\ 

Now, recall that \customref{THpslg}{$\operatorname{PSL}_2(\F_3)$ acts faithfully on $\P^{1}(\F_3)$}, which has $\frac{3^2-1}{3-1}=4$ points, so we have an injective group homomorphism $\operatorname{PSL}_2(\F_3)\longhookrightarrow S_4$. Computing that $o \left(\operatorname{PSL}_2(\F_3) \right)=12$, we see that \href{https://groupprops.subwiki.org/wiki/Subgroup_structure_of_symmetric_group:S4}{since the only index $2$ subgroup of $S_4$ is  $A_4$}, we have the isomorphism $\operatorname{PSL}_2(\F_3) \cong  A_4$, \customref{THsa}{which is non-simple}. 
\end{Example}
Regardless of whether $n>2$,  the second condition is proved using the same $H$ and  $U$. We pick $x \triangleq [1:0 : \cdots :0 ]$, and therefore 
\begin{align*}
H= \set{\begin{pmatrix} 
    a & * \\
    \textbf{0} & M 
\end{pmatrix} : a \in \F^{\times}\text{ and }M\in \operatorname{GL}_{n-1}(\F)}
\end{align*}
and since: 
\begin{align*}
\begin{pmatrix} 
  a & * \\
  \textbf{0} & M
\end{pmatrix} \begin{pmatrix} 
  b & * \\
  \textbf{0} & J
\end{pmatrix} = \begin{pmatrix} 
  ab & * \\
  \textbf{0} & MJ
\end{pmatrix}
\end{align*}
We know 
\begin{align*}
U \triangleq \set{\begin{pmatrix} 
    1 & * \\
    \textbf{0} & I_{n-1}
\end{pmatrix}}\text{ is normal in }H.
\end{align*}
$U$ is abelian since 
\begin{align*}
\begin{pmatrix} 
    1 & \textbf{v} \\
    \textbf{0} & I_{n-1}
\end{pmatrix} \begin{pmatrix} 
    1 & \textbf{w} \\
    \textbf{0} & I_{n-1}
\end{pmatrix} = \begin{pmatrix} 
    1 & \textbf{v}+ \textbf{w} \\
    \textbf{0} & I_{n-1}
\end{pmatrix}
\end{align*}
Interestingly, both  (I) and (II) requires boring computations within $\operatorname{SL}_n(\F)$. 
\begin{theorem}
\label{THpsl2s}
\textbf{(One dimensional projective special linear group is simple over field that has $\geq 4$ elements)} If $\F$ has  $\geq 4$ elements, then $\operatorname{PSL}_2(\F)$ is simple.  
\end{theorem}
\begin{proof}
(II): Since 
\begin{align*}
 \begin{pmatrix} 
  0 & -1 \\
  1 & 0
 \end{pmatrix} \begin{pmatrix} 
  1 & \ld  \\
  0 & 1
 \end{pmatrix} \begin{pmatrix} 
  0 & -1 \\
  1 & 0
 \end{pmatrix}^{-1}= \begin{pmatrix} 
  1 & 0 \\
  -\ld  & 1 
 \end{pmatrix}
\end{align*}
We know the group of the lower triangular matrices:
\begin{align*}
\set{\begin{pmatrix} 
    1 & 0 \\
    * & 1 
\end{pmatrix}}
\end{align*}
is a conjugate of $U$. We prove that $\operatorname{SL}_2(\F)$ is generated by $U$ and the lower triangular matrices. To see such, just observe that if any of $b,c$ is nonzero, then by transposing the matrix if necessary, we have  
\begin{align*}
\begin{pmatrix} 
  a & b \\
  c & d
\end{pmatrix}= \begin{pmatrix} 
  1 & 0 \\
  (d-1)b^{-1} & 1
\end{pmatrix} \begin{pmatrix} 
  1 & b \\
  0 & 1 
\end{pmatrix} \begin{pmatrix} 
  1 & 0 \\
  (a-1)b^{-1} & 1 
\end{pmatrix},\quad \text{ where }b \neq 0
\end{align*}
and if both of them are zero, then $d=a^{-1}$ and we have 
\begin{align*}
\begin{pmatrix} 
  a & 0 \\
  0 & a^{-1}
\end{pmatrix}= \begin{pmatrix} 
  1 & 0 \\
  (1-a)a^{-1} & 1 
\end{pmatrix} \begin{pmatrix} 
  1 & 1 \\
  0 & 1
\end{pmatrix} \begin{pmatrix} 
  1 & 0 \\
  a-1 & 1 
\end{pmatrix} \begin{pmatrix} 
  1 & -a^{-1} \\
  0 & 1 
\end{pmatrix}
\end{align*}

(I): Compute 
\begin{align*}
\left[\begin{pmatrix} 
    a & 0\\
    0 & a^{-1}
\end{pmatrix}, \begin{pmatrix} 
    1 & b \\
    0 & 1 
\end{pmatrix} \right]= \begin{pmatrix} 
    1 & b(a^2-1) \\
    0 & 1 
\end{pmatrix} 
\end{align*}
Because $\F$ has  $\geq 4$ elements, we know that there exists some $a \in \F^{\times}$ such that $a^2 \neq 1$ (Consider $x^2-1 \in \F[x]$). Fix such $a$. We then see that  $U \leq \operatorname{SL}_2(\F)^{(1)}$ by letting $b$ run through  $\F$. Normality of $\operatorname{SL}_2(\F)^{(1)}$ and (II) then implies $\operatorname{SL}_2(\F)^{(1)}=\operatorname{SL}_2(\F)$
\end{proof}

\begin{theorem}
\textbf{($\geq 2$ dimensional projective special linear groups are all simple)} Let $n\geq  3$. Then we have: 
\begin{enumerate}[label=(\roman*)] 
  \item $I_n + \ld  e_{ij}$ is conjugate to $I_n+ e_{12}$ in $\operatorname{SL}_n(\F)$, for all $\ld \in \F^{\times}$ and $i\neq j$. 
  \item $\set{I_n + \ld e_{ij}:\ld \in\F\text{ and }i\neq j}$ generates $\operatorname{SL}_n(\F)$. 
\end{enumerate}
Since $I_n+ e_{12}\in U$, together they implies (II), and using them we may prove (I).   
\end{theorem}
\begin{proof}
(i): We are required to find some $P \in \operatorname{SL}_n(\F)$ such that $P(I_n+ \ld e_{ij})P^{-1}=I_n + e_{12}$. Note that $I_n+ \ld e_{ij}$ maps 
\begin{align*}
e_j\mapsto e_j+ \ld  e_i\quad\text{ and }e_k\mapsto e_k\text{ for all }k\neq j
\end{align*}
The construction of $P$ then follows from letting its first column to be  $\ld e_i$, its second column to be $e_j$, and the rest to be other basis vector unchanged by $I_n+ \ld e_{ij}$, where the third column is multiplied by $\ld ^{-1}$. (We used the fact $n\geq 3$ here) \\

(ii): This is proved via induction. We have proved the base case in our proof for \customref{THpsl2s}{simplicity of $\operatorname{PSL}_2(\F)$}. We now prove the inductive case. Let $A \in \operatorname{SL}_n(\F)$. We are required to show that 
\begin{align*}
PAQ = \begin{pmatrix} 
  1 & \textbf{0}\\
  \textbf{0} & \tilde{A} 
\end{pmatrix},\quad \text{ for some }P,Q\text{ generated by $I_n+ \ld e_{ij}$ }
\end{align*}
and the rest would follows the inductive hypothesis. The construction of $P,Q$ is obvious once one observe the effect of  multiplying $I_n+ \ld e_{ij}$ on a matrix. \\

(I):  By (i), (ii) and normality of commutator subgroup, we only have to prove  $I_n + e_{12} \in \operatorname{SL}_n(\F)^{(1)}$, which follows from computing 
\begin{align*}
I_n+ e_{12} = \left[ I_n + e_{13} , I_n + e_{32} \right]
\end{align*}
\end{proof}
\section{selection of advanced topics}
\begin{theorem}
\textbf{(Burnside's $p^aq^b$ theorem)} If $o(G)=p^aq^b$ for some primes  $p,q$, then  $G$ is solvable. 
\end{theorem}
\begin{proof}
\href{https://en.wikipedia.org/wiki/Burnside%27s_theorem}{Wikipedia has a detailed proof}. 
\end{proof}
\begin{theorem}
\textbf{(Schur-Zassenhaus theorem)} Let $G$ be a finite group and $N\trianglelefteq G$ be Hall. Then the short exact sequence 
\begin{align*}
1 \longrightarrow N \longrightarrow G \longrightarrow G \quotient N \longrightarrow 1
\end{align*}
right splits, and thus $G$ must be a semidirect product of $N\rtimes (G\quotient N)$.  
\end{theorem}
\begin{proof}
\href{https://en.wikipedia.org/wiki/Schur–Zassenhaus_theorem}{Wikipedia has a proof sketch that relies on group cohomology}
\end{proof}
\begin{theorem}
\textbf{(Thompson's fixed-point-free theorem)} Let $G$ be a finite group that admits a fixed-point-free automorphism of prime order. Then $G$ is nilpotent. 
\end{theorem}
\begin{proof}
\href{https://www.jstor.org/stable/90107?seq=1}{This is Thompson's proof}.
\end{proof}
\begin{theorem}
\textbf{(Nielsen-Schreier Theorem)} The subgroup of a free group is isomorphic to some free group. 
\end{theorem}
\begin{proof}
\href{https://en.wikipedia.org/wiki/Nielsen–Schreier_theorem}{Wikipedia has a proof based on Algebraic Topology}. 
\end{proof}
\begin{theorem}
\textbf{(Schreier conjecture, now a theorem)} Outer automorphism group of finite simple groups are solvable. 
\end{theorem}
\begin{proof}
\href{https://mathoverflow.net/questions/308496/schreier-conjecture-without-a-simple-proof-and-sporadic-simple-groups}{See this MO post}. 
\end{proof}
\begin{theorem}
\textbf{(Feit-Thompson theorem)} Finite groups of odd order are solvable. 
\end{theorem}
\begin{proof}
\href{https://en.wikipedia.org/wiki/Feit–Thompson_theorem}{Wikipedia has a proof sketch}. 
\end{proof}
\begin{theorem}
\textbf{(Frobenius theorem about equation in group)} Let $G$ be a finite group, and let  $n \mid  o(G)$. The number of solution to the equation
\begin{align*}
g^n = e 
\end{align*}
is a positive multiple of $n$. 
\end{theorem}
\begin{proof}
  \href{https://www.jstor.org/stable/2324902?seq=2}{Isaacs and Robinson gives a relatively simple proof of this}. 
\end{proof}
\begin{theorem}
\textbf{(Frobenius conjecture)} Let $G$ be a finite group, and let $n \mid  o (G)$. If the number of solution to the equation $g^n=e$ is exactly  $n$, then these  $n$ elements form a normal subgroup of  $G$. 
\end{theorem}
\begin{proof}
  \href{https://www.ams.org/journals/bull/1991-25-02/S0273-0979-1991-16084-2/S0273-0979-1991-16084-2.pdf}{Iiyori and Yamaki show that this is a corollary of classification of finite simple groups}. 
\end{proof}
\section{recreational exercises}
\begin{question}{Groups as unions of proper subgroups}{}
Show that a group can not be written as the union of two proper subgroup. 
\end{question}
\begin{proof}
  Let $G=H \cup K$. We are required to prove $G \in \set{H,K}$. We prove such by showing that one is contained by the other.  Assume not for a contradiction. Letting $h \in H-K$ and $k \in K-H$, we see that $hk \in G - H \cup K$, a contradiction. For more on this kind theorems, see the survey article \href{http://www.jstor.org/stable/40391117}{\emph{Groups as unions of proper subgroups}, by Bhargava, M. (2009)}
\end{proof}
\begin{question}{$n-1,n,n+1$-abelian implies abelian}{}
Let $n\inz$ satisfies 
\begin{align*}
  (ab)^n =a^nb^n\quad \text{ and }\quad (ab)^{n-1}=a^{n-1}b^{n-1}\quad \text{ and }\quad (ab)^{n+1}=a^{n+1}b^{n+1}
\end{align*}
for all $a,b \in G$. Show that $G$ is abelian.  
\end{question}
\begin{proof}
\begin{align*}
&a^nb^n ab=(ab)^n (ab)=(ab)^{n+1}=a^{n+1}b^{n+1} \\
\implies & b^na=ab^n \\
\implies  & (a^{n-1}b^{n-1})ba= a^nb^n= (a^{n-1}b^{n-1})ab \\
\implies& ba=ab
\end{align*}
\end{proof}
\begin{question}{finite $3$-abelian whose order is not divisible by  $3$ is abelian}{}
Let $G$ be a finite group whose order is not divisible by $3$. Suppose 
\begin{align*}
  (ab)^3 = a^3b^3,\quad \text{ for all }a,b \in G
\end{align*}
Show that $G$ is abelian. 
\end{question}
\begin{proof}
Because $3\nmid  o(G)$, we know if $g^3=e$, then  $g=e$. In other words, the endomorphism  $x\mapsto x^3$ is an automorphism. Because of such, we only have to prove $a^3b^3=b^3a^3$ for all  $a,b  \in G$. This then follows from 
\begin{align*}
&(ab)^3= a^3b^3,\quad \text{ for all }a,b \in G \\
\implies & baba=a^2b^2,\quad \text{ for all }a,b \in G \\
\implies &(ba)^2= a^2b^2,\quad \text{ for all }a,b \in G \\
\implies & (ab)^4= [(ab)^2]^2= [ b^2a^2]^2= a^4b^4,\quad \text{ for all }a,b \in G \\
\implies &(ab)^4 = a(ba)^3 b =ab^3a^3 b,\quad \text{ for all }a,b \in G \\
\implies & a^3b^3=b^3a^3 ,\quad \text{ for all }a,b \in G 
\end{align*}
\end{proof}
\begin{question}{}{}
Let $a,b \in G$ satisfies 
\begin{align*}
o(a)=5\quad \text{ and }\quad  aba^{-1}=b^2
\end{align*}
Find $o(b)$. 
\end{question}
\begin{proof}
\begin{align*}
&aba^{-1}=b^2  \\
\implies & a^2ba^{-2}=ab^2a^{-1}= (aba^{-1})^2= b^4 \\
\implies & b=a^5a^{-5}=a^4b^2a^{-4}= (a^2ba^{-2})^2=b^8
\end{align*}
Therefore $o(b)=7$. 
\end{proof}
\begin{question}{\customref{THfgag}{Structure theorem for finitely generated abelian group}}{}
Let $G$ be an abelian group, and $x,y \in G$ has order $m,n$. Show that  $G$ has an element of order $\operatorname{lcm}(m,n)$. 
\end{question}
\begin{proof}
The proof follows from noting that the fact that in general, given $(a_1,\dots ,a_n) \in G_1 \times \cdots \times G_n$, we have 
\begin{align*}
o((a_1,\dots ,a_n))= \operatorname{lcm}(o(a_1),\dots , o(a_n))
\end{align*}
and the fact that since $x,y \in \langle x\rangle + \langle y\rangle$, in the \customref{THfgag}{primary decomposition form} of $\langle x\rangle +\langle y\rangle $, there must be a component of high enough power. 
\end{proof}
\begin{question}{\customref{THfgag}{Structure theorem for finitely generated abelian group}}{}
Let $G$ be an abelian groups that has subgroups of order $m$ and $n$. Show that is has a subgroup of order $\operatorname{lcm}(m,n)$. 
\end{question}
\begin{proof}
Let the subgroups be $M,N$. The proof then follows from noting that  $\operatorname{lcm}(m,n)\mid  o (MN)$ and the fact that  \customref{COfao}{for every divisor $d$ of the order of a finite abelian group  $H$, there always exists a subgroup  $\leq H$ of order $d$}. 
\end{proof}
\begin{question}{\customref{THfgag}{Structure theorem for finitely generated abelian group}}{}
Let  $G$ be a finite abelian group in which the number of solution of the equation  $x^n=e$ is $\leq n$ for all  $n\inn$. Show that $G$ is cyclic. 
\end{question}
\begin{proof}
Write $G$ in its  \customref{THfgag}{primary decomposition form}. We are required to show that its $p$-torsion subgroup has at most one component. This follows from noting that for $d,e\geq 1$, the group $C_{p^d}\times C_{p^e}$ has $p^2$ number of solution to the equation $x^n=e$.  
\end{proof}
\begin{question}{}{}
Let $a \in G$. Show that the equation 
\begin{align*}
x^2ax=a^{-1}
\end{align*}
is solvable for $x \in G$ if and only if $a$ is the cube of some element.  
\end{question}
\begin{proof}
Suppose $x^2ax=a^{-1}$ for some $x \in G$. One can show that $a= (xa)^3$. If $a=y^3$, then  $x\triangleq y^{-2}$ is a solution. 
\end{proof}
\chapter{Commutative Algebra}
\section{Isomorphism Theorems for Rings and Modules} 
The precise meaning of the term \textbf{ring} varies across different books. In this note, ring multiplication is always associative, commutative, and has an identity. The additive and multiplicative identity are denoted by $0$ and $1$. Let $A$ be a ring, clearly we have 
 \begin{align*}
x \cdot 0 =0 ,\quad \text{ for all }x \in A
\end{align*}
Consequently, $1 \neq 0$ unless the ring contain only one element. In such case, we call the ring  \textbf{zero ring}. A \textbf{ring homomorphisms} $f: A\rightarrow B$ is a function that respect $+$,  $\times$, and $1$. Clearly, $f$ must also respect $0$, $-1$, and units, $f(A)$ forms a subring of $B$, and if $f$ is injective, then the inverse $f^{-1}:f(A)\rightarrow A$ forms a ring homomorphism. \\

Let $A$ be a ring. An  \textbf{ideal} $\mathfrak{a}\subseteq A$ the is kernel of some ring homomorphism, or equivalently, an additive subgroup  such that $xy \in \mathfrak{a}$ for all $x \in \mathfrak{a}$ and $y \in A$.  We say a ring homomorphism $ \pi :A \twoheadrightarrow   A\quotient \mathfrak{a}$ satisfies the \textbf{universal property of quotient ring $A\quotient \mathfrak{a}$} if 
\begin{enumerate}[label=(\roman*)]
  \item $\pi $ vanishes on $\mathfrak{a}$. \textbf{(Ring condition)}
  \item For all ring homomorphism $f:A\rightarrow B$ that vanishes on $\mathfrak{a}$ there exist a unique ring homomorphism $\tilde{f} :A\quotient \mathfrak{a}\rightarrow B$ that makes the diagram: 
% https://q.uiver.app/#q=WzAsMyxbMiwwLCJHXFxxdW90aWVudCBOICJdLFswLDAsIkciXSxbMiwyLCJMIl0sWzAsMiwiZyJdLFsxLDIsImYiLDJdLFsxLDAsIiIsMCx7InN0eWxlIjp7ImhlYWQiOnsibmFtZSI6ImVwaSJ9fX1dXQ==
\[\begin{tikzcd}
	A && {A\quotient \mathfrak{a} } \\
	\\
	&& B 
	\arrow["\pi",two heads, from=1-1, to=1-3]
	\arrow["f"', from=1-1, to=3-3]
	\arrow["\tilde{f} ", dashed,from=1-3, to=3-3]
\end{tikzcd}\]
commute. \textbf{(Universality)}
\end{enumerate}

\begin{theorem}
\textbf{(First isomorphism theorem for rings)} Let $f:A \rightarrow B$ be a ring homomorphism. Then the induced map $\tilde{f} : A \quotient \operatorname{ker}f \rightarrow B$ is injective.  
\end{theorem}
\begin{proof}
Routine. 
\end{proof}



\begin{theorem}
\textbf{(Second isomorphism theorem for rings)} Let $B\subseteq A$ be rings, and $\mathfrak{a}\subseteq A$ an ideal. Then
\begin{enumerate}[label=(\roman*)]
  \item $B+ \mathfrak{a}\triangleq \set{b+a\in A: b \in B,a \in \mathfrak{a}}$ forms a subring of $A$.  
  \item $B\cap \mathfrak{a}$ forms an ideal in $B$.  
  \item $(B+\mathfrak{a})\quotient \mathfrak{a}\cong B\quotient (B \cap \mathfrak{a})$ as rings.
\end{enumerate}
\end{theorem}
\begin{proof}
Routine. 
\end{proof}
One should always be careful with the word \textbf{unit} and the word \textbf{zero-divisor}.\footnote{Note that $0$ is defined to be a zero-divisor. To justify this convention, note that if we allow  $0$ to be a non-zero-divisor, then the total ring of fraction is always the zero ring.}
\begin{theorem}
\label{THunz}
\textbf{(Unit can not be zero-divisors in nonzero ring)} Let $A$ be a ring and $u \in A$ be a unit. If $A=\Z_1$, then $u$ is vacuously a zero-divisor. If $A$ is nonzero, then  $u$ isn't a zero-divisor. 
\end{theorem}
\begin{proof}
Unroll the definitions. 
\end{proof}
Given rings $A \subseteq B$, it is possible that $a \in A$ forms a non-zero-divisor (or unit) in $A$ but not in  $B$. Under our initial requirement that rings are commutative unital, for a nonzero ring $A$ to be an \textbf{integral domain}, we only need all nonzero elements to be non-zero-divisors, and for $A$ to be a  \textbf{field}, we only need all nonzero elements to be units.
\begin{equiv_def}
\label{edF}
\textbf{(Field)} Let $A$ be a nonzero ring. The followings are equivalent:
 \begin{enumerate}[label=(\roman*)]
  \item $A$ is a field.  
  \item $A$ have only two ideals, i.e.,  $\set{0}$ and $A$. 
  \item Every ring homomorphism of $A$ into a nonzero ring  $B$ is injective. 
\end{enumerate}
\end{equiv_def}
\begin{proof}
Unroll the definitions. 
\end{proof}
We use the term \textbf{proper} to describe strict set inclusion. A \textbf{maximal ideal} is  a proper ideal contained by no other proper ideals. A \textbf{prime ideal} is a proper ideal $\mathfrak{p}$ such that the product of two element lies in $\mathfrak{p}$ implies one of then lies in $\mathfrak{p}$. 
\begin{equiv_def}
\label{edP}
\textbf{(Prime ideals)} Let $\mathfrak{p}\subseteq A$ be an ideal. The followings are equivalent:
\begin{enumerate}[label=(\roman*)]
  \item $\mathfrak{p}$ is prime. 
  \item $A\quotient \mathfrak{p}$ forms an integral domain.  
\end{enumerate}
\end{equiv_def}
\begin{proof}
Unroll the definitions. 
\end{proof}
\begin{equiv_def}
\textbf{(Integral domain)} Let $A$ be a ring. The followings are equivalent: 
 \begin{enumerate}[label=(\roman*)]
  \item $A$ is an integral domain.  
  \item The zero ideal of $A$ is prime. 
\end{enumerate}
\end{equiv_def}
\begin{proof}
Unroll the definitions.
\end{proof}
\begin{theorem}
\label{Tit}\label{ctfr}
\textbf{(Third isomorphism theorem and correspondence theorem for rings)} Let $A$ be a ring, $\mathfrak{a}\subseteq \mathfrak{b} \subseteq A$ be two ideals, $S$ the collection of ideals in $A$ that contains  $\mathfrak{a}$, and $\pi :A \twoheadrightarrow A\quotient \mathfrak{a}$ the quotient map. We denote $\mathfrak{b}\quotient \mathfrak{a}\triangleq \pi  (\mathfrak{a})$. Then: 
\begin{enumerate}[label=(\roman*)]
  \item $\mathfrak{b}\quotient \mathfrak{a}\subseteq A\quotient \mathfrak{a}$ forms an ideal.
  \item $(A\quotient \mathfrak{a})\quotient (\mathfrak{b}\quotient \mathfrak{a})\cong A\quotient \mathfrak{b}$ as rings.
  \item The map $\mathfrak{c}\in S \mapsto \mathfrak{c}\quotient \mathfrak{a}$ forms a bijection between $S$ and the collection of ideals in  $A\quotient \mathfrak{a}$, a bijection between $S \cap \operatorname{Spec}A$ and $\operatorname{Spec}(A\quotient \mathfrak{a})$, and a bijection between $S\cap \operatorname{Max}A$ and $\operatorname{Max}(A\quotient \mathfrak{a})$. 
\end{enumerate}
\end{theorem}
\begin{proof}
Unroll the definition. 
\end{proof}
\section{Domains}
Bezout domain, GCD domain, PID, UFD
\chapter{Algebraic Geometry}
\section{Spectrum}

\end{document}

