\documentclass{report}
%%%%%%%%%%%%%% macros.tex %%%%%%%%%%%%%%
% Place your custom macros here, if any.

%%%%%%%%%%%%%% letterfonts.tex %%%%%%%%%%%%%%
% Place your font setup here, if any.

%%%%%%%%%%%%%% preamble.tex %%%%%%%%%%%%%%
\usepackage[T1]{fontenc}
\usepackage{lmodern}
\usepackage{etoolbox}
\usepackage{pdfpages}
\usepackage{transparent}
\usepackage[utf8]{inputenc}
\usepackage[english]{babel}

% Page Setup
\usepackage[tmargin=2cm, rmargin=0.5in, lmargin=0.5in, bmargin=80pt, footskip=.2in]{geometry}

% Mathematics
\usepackage{amsmath,amsfonts,amsthm,amssymb,mathtools}
\usepackage{xfrac}
\usepackage[makeroom]{cancel}
\usepackage{enumitem}
\usepackage{nameref}
\usepackage{multicol,array}
\usepackage{tikz-cd}
\usepackage[ruled,vlined,linesnumbered]{algorithm2e}

% Colors
\usepackage[dvipsnames]{xcolor}
\definecolor{myg}{RGB}{56, 140, 70}
\definecolor{myb}{RGB}{45, 111, 177}
\definecolor{myr}{RGB}{199, 68, 64}
% Define more colors here...

% Hyperlinks
\usepackage{bookmark}
\usepackage{hyperref}
\hypersetup{
    pdftitle={Assignment},
    colorlinks=true, linkcolor=doc!90,
    bookmarksnumbered=true,
    bookmarksopen=true
}

% Figures and Graphics
\usepackage{import}
\usepackage{svg}
\newcommand{\incfig}[1]{%
    \def\svgwidth{\columnwidth}
    \import{./figures/}{#1.pdf_tex}
}

% Text-related
\usepackage{blindtext}
\usepackage{fontsize}
\changefontsize[14]{14}
\setlength{\parindent}{0pt}

% Theorems and Definitions
\usepackage{amsthm}
\renewcommand\qedsymbol{$\blacksquare$}

% Define a new theorem style
\newtheoremstyle{mytheoremstyle}% name
  {}% Space above
  {}% Space below
  {\sffamily}% Body font
  {}% Indent amount
  {\bfseries}% Theorem head font
  {.}% Punctuation after theorem head
  {.5em}% Space after theorem head
  {}% Theorem head spec (can be left empty, meaning ‘normal’)

% Apply the new theorem style to theorem-like environments
\theoremstyle{mytheoremstyle}
\newtheorem{theorem}{Theorem}[section]
\newtheorem{definition}{Definition}[section]
\newtheorem{corollary}{Corollary}[section]
\newtheorem{lemma}{Lemma}[section]
\newtheorem{axiom}{Axiom}[section]

% tcolorbox Setup
\usepackage[most,many,breakable]{tcolorbox}

% Define custom tcolorbox environments here...

%================================
% EXAMPLE BOX
%================================
\newtcbtheorem[definition]{Example}{Example}
{%
    colback = myexamplebg,
    breakable,
    colframe = myexamplefr,
    coltitle = myexampleti,
    boxrule = 1pt,
    sharp corners,
    detach title,
    before upper=\tcbtitle\par\smallskip,
    fonttitle = \bfseries,
    description font = \mdseries,
    separator sign none,
    description delimiters parenthesis,
}
{ex}

%================================
% Solution BOX
%================================
\makeatletter
\newtcolorbox{solution}{enhanced,
	breakable,
	colback=white,
	colframe=myg!80!black,
	attach boxed title to top left={yshift*=-\tcboxedtitleheight},
	title=Solution,
	boxed title size=title,
	boxed title style={%
			sharp corners,
			rounded corners=northwest,
			colback=tcbcolframe,
			boxrule=0pt,
		},
	underlay boxed title={%
			\path[fill=tcbcolframe] (title.south west)--(title.south east)
			to[out=0, in=180] ([xshift=5mm]title.east)--
			(title.center-|frame.east)
			[rounded corners=\kvtcb@arc] |-
			(frame.north) -| cycle;
		},
}
\makeatother

%================================
% Question BOX
%================================
\makeatletter
\newtcbtheorem{question}{Question}{enhanced,
	breakable,
	colback=white,
	colframe=myb!80!black,
	attach boxed title to top left={yshift*=-\tcboxedtitleheight},
	fonttitle=\bfseries,
	title={#2},
	boxed title size=title,
	boxed title style={%
			sharp corners,
			rounded corners=northwest,
			colback=tcbcolframe,
			boxrule=0pt,
		},
	underlay boxed title={%
			\path[fill=tcbcolframe] (title.south west)--(title.south east)
			to[out=0, in=180] ([xshift=5mm]title.east)--
			(title.center-|frame.east)
			[rounded corners=\kvtcb@arc] |-
			(frame.north) -| cycle;
		},
	#1
}{def}
\makeatother
\makeatletter
\newtcbtheorem{qstion}{Question}{enhanced,
    breakable,
    colback=white,
    colframe=mygr,
    attach boxed title to top left={yshift*=-\tcboxedtitleheight},
    fonttitle=\bfseries,
    title={#2},
    boxed title size=title,
    boxed title style={%
        sharp corners,
        rounded corners=northwest,
        colback=tcbcolframe,
        boxrule=0pt,
    },
    underlay boxed title={%
        \path[fill=tcbcolframe] (title.south west)--(title.south east)
        to[out=0, in=180] ([xshift=5mm]title.east)--
        (title.center-|frame.east)
        [rounded corners=\kvtcb@arc] |-
        (frame.north) -| cycle;
    },
    #1
}{def}
\makeatother

%%%%%%%%%%%%%%%%%%%%%%%%%%%%%%%%%%%%%%%%%%%
% TABLE OF CONTENTS
%%%%%%%%%%%%%%%%%%%%%%%%%%%%%%%%%%%%%%%%%%%
\usepackage{tikz}
\definecolor{doc}{RGB}{0,60,110}
\usepackage{titletoc}
\contentsmargin{0cm}
\titlecontents{chapter}[14pc]
{\addvspace{30pt}%
	\begin{tikzpicture}[remember picture, overlay]%
		\draw[fill=doc!60,draw=doc!60] (-7,-.1) rectangle (-0.9,.5);%
		\pgftext[left,x=-4.5cm,y=0.2cm]{\color{white}\Large\sc\bfseries Chapter\ \thecontentslabel};%
	\end{tikzpicture}\color{doc!60}\large\sc\bfseries}%
{}
{}
{\;\titlerule\;\large\sc\bfseries Page \thecontentspage
	\begin{tikzpicture}[remember picture, overlay]
		\draw[fill=doc!60,draw=doc!60] (2pt,0) rectangle (4,0.1pt);
	\end{tikzpicture}}%
\titlecontents{section}[3.7pc]
{\addvspace{2pt}}
{\contentslabel[\thecontentslabel]{2pc}}
{}
{\hfill\small \thecontentspage}
[]
\titlecontents*{subsection}[3.7pc]
{\addvspace{-1pt}\small}
{}
{}
{\ --- \small\thecontentspage}
[ \textbullet\ ][]

\makeatletter
\renewcommand{\tableofcontents}{
	\chapter*{%
	  \vspace*{-20\p@}%
	  \begin{tikzpicture}[remember picture, overlay]%
		  \pgftext[right,x=15cm,y=0.2cm]{\color{doc!60}\Huge\sc\bfseries \contentsname};%
		  \draw[fill=doc!60,draw=doc!60] (13,-.75) rectangle (20,1);%
		  \clip (13,-.75) rectangle (20,1);
		  \pgftext[right,x=15cm,y=0.2cm]{\color{white}\Huge\sc\bfseries \contentsname};%
	  \end{tikzpicture}}%
	\@starttoc{toc}}
\makeatother

\newcommand{\liff}{\llap{$\iff$}}
\newcommand{\rap}[1]{\rrap{\text{ (#1)}}}
\newcommand{\red}[1]{\textcolor{red}{#1}}
\newcommand{\blue}[1]{\textcolor{blue}{#1}}
\newcommand{\vi}[1]{\textcolor{violet}{#1}}
\newcommand{\teal}[1]{\textcolor{teal}{#1}}
\newcommand{\tCaC}{\text{ \CaC }}
\newcommand{\CaC}{\red{CaC} }
\newcommand{\As}[1]{Assume \red{#1}}
\newcommand{\vdone}{\vi{\text{ (done) }}}
\newcommand{\bdone}{\blue{\text{ (done) }}}
\newcommand{\tdone}{\teal{\text{ (done) }}}
\newcommand{\set}[1]{\{ #1 \}}
\newcommand{\inS}{\in S}
\newcommand{\inF}{\in\F}
\newcommand{\inE}{\in E}
\newcommand{\inA}{\in A}
\newcommand{\inB}{\in B}
\newcommand{\inC}{\in C}
\newcommand{\inU}{\in U}

\newcommand{\C}{\mathbb{C}}	
\renewcommand{\H}{\mathbb{H}}
\newcommand{\F}{\mathbb{F}}
\newcommand{\N}{\mathbb{N}}
\newcommand{\Q}{\mathbb{Q}}
\newcommand{\R}{\mathbb{R}}
\newcommand{\Z}{\mathbb{Z}}
\renewcommand{\P}{\mathbb{P}}
\renewcommand{\S}{\mathbb{S}}
\newcommand{\A}{\mathbb{A}}
\newcommand{\RP}{\R P}


\title{Suns}
\author{Eric Liu}
\date{}
\begin{document}
\maketitle
\newpage % or \cleardoublepage
% \pdfbookmark[<level>]{<title>}{<dest>}
\pdfbookmark[section]{\contentsname}{toc}

\tableofcontents
\pagebreak
\chapter{Groups}

\section{Group action}
Let $M$ be a set equipped with a binary operation $M\times M \rightarrow M$. We say $M$ is a \textbf{monoid} if the binary operation is associative and there exists a two-sided identity $e \in M$. 
\begin{example}
Defining $(x,y)\mapsto y$, we see that the operation is associative and every element is a left identity, but no element is a right identity unless $\abso{M}=1$. This is an example why identity must be two-sided. 
\end{example}
Because the identity of a monoid is defined to be two-sided, clearly it must be unique.  Suppose every element of monoid $M$ has a left inverse. Fix $x \in M$. Let $x^{-1}\in M$ be a left inverse of $x$. To see that  $x^{-1}$ is also a right inverse of $x$, let  $(x^{-1})^{-1}\in M$ be a left inverse of $x$ and use  
\begin{align*}
  (x^{-1})^{-1}=(x^{-1})^{-1}e=(x^{-1})^{-1}(x^{-1}x)= ((x^{-1})^{-1}x^{-1})x= ex=x
\end{align*}
to deduce
\begin{align*}
xx^{-1}=(x^{-1})^{-1}x^{-1}= e
\end{align*}
In other words, if we require every element of a monoid $M$ to has a left inverse, then immediately every left inverse upgrades to a right inverse. In such case, we call $M$ a  \textbf{group}. Notice that inverses of elements of a group are clearly unique. \\


Unlike the category of monoids, the category of groups behaves much better. Given two groups $G,H$ and a function  $\phi : G\rightarrow H$, if $\phi$ respects the binary operation, then $\phi$ also respects the identity:
\begin{align*}
e_H = (\phi (x)^{-1})\phi (x) = (\phi(x)^{-1}) \phi(x e_G) =  (\phi (x)^{-1} \phi (x)) \phi (e_G)=\phi (e_G)
\end{align*}
which implies that $\phi$ must also respect inverse. In such case, we call $\phi$ a \textbf{group homomorphism}. Given a subset $H \subseteq G$ closed under the binary operation, if $H$ forms a group itself, then since the set inclusion $H \hookrightarrow G$ forms a group homomorphism, we have $e_H=e_G$, and thus $x^{-1}$ in $H,G$ are the same element. \\
 
In this note, by a \textbf{subgroup} $H$ of $G$, we mean an injective group homomorphism $H \hookrightarrow G $. Clearly, a subset of $G$ forms a subgroup if and only it is closed under both the binary operation and inverse. Note that one of the key basic property of subgroup $H \subseteq G$ is that if $g \not \in H$, then $hg \not \in H$, since otherwise $g=h^{-1}hg \in H$. \\







Let $S$ be a subset of $G$. The group of \textbf{words} in $S$ is clearly the smallest subgroup of $G$ containing $S$. We say this subgroup is \textbf{generated} by $S$. If $G$ is generated by a single element, we say $G$ is \textbf{cyclic}. Let $x \in G$. The \textbf{order} of $G$ is the cardinality of $G$, and the order of  $x$ is the cardinality of the cyclic subgroup $\langle x\rangle \subseteq G$, or equivalently the infimum of the set of natural numbers $n$ that makes $x^n=e$. Clearly, finite cyclic groups of order $n$ are all isomorphic to  $\Z_n$.  \\


Let $G$ be a group and $H$ a subgroup of $G$. The \textbf{right cosets} $Hx$ are defined by $Hx\triangleq \set{hx \in G: h \in H}$. Clearly, when we define an equivalence relation in $G$ by setting: 
\begin{align*}
x\sim  y \overset{\triangle}{\iff } xy^{-1} \in H
\end{align*}
the equivalence class $[x]$ coincides with the right coset $Hx$. Note that if we partition $G$ using \textbf{left cosets}, the equivalence relation being $x\sim  y \iff  x^{-1}y\in H$, then the two partitions need not to be identical. 
\begin{example}
Let $H\triangleq \set{e,(1,2)}\subseteq S_3$. The right cosets are 
\begin{align*}
H(2,3)=\set{(2,3),(1,2,3)}\quad \text{ and }\quad  H(1,3)=\set{(1,3),(1,3,2)}
\end{align*}
while the left cosets being
\begin{align*}
(2,3)H= \set{(2,3),(1,3,2)} \quad \text{ and }\quad (1,3)H= \set{(1,3),(1,2,3)}
\end{align*}
\qed
\end{example}
However, as one may verify, we have a well-defined bijection $xH\mapsto Hx^{-1}$ between the sets of left cosets and right cosets of $H$. Therefore, we may define the \textbf{index} $\abso{G:H}$ of $H$ in  $G$ to be the cardinality of the collection of left cosets of $H$, without falling into the discussion of left and right. Moreover, by axiom of choice, there exists a set $T\subseteq G$ such that $\abso{T\cap xH}=1$ for all $x \in G$. Such $T$ clearly makes the set map  $T \times H \rightarrow G$ defined by: 
\begin{align*}
  (t,h)\mapsto th
\end{align*}
a bijection. This proves the \textbf{Lagrange's theorem}: 
\begin{align*}
\abso{G}= \abso{G:H} \cdot \abso{H}
\end{align*}
Consider a group $G$ of prime order. If $x \neq e\in G$, then clearly the cyclic subgroup $\langle x\rangle $ must be $G$ by Lagrange's theorem.  \\


Let $G$ be a group and $X$ a set. If we say  $G$ \textbf{acts on $X$ from left} we are defining a function $G \times X\rightarrow X$ such that 
\begin{enumerate}[label=(\roman*)]
  \item $e\cdot x=x$ for all $x\in X$. 
  \item $(gh)\cdot x= g \cdot (h \cdot x)$ for all $g,h \in G$. 
\end{enumerate}
Note that there is a difference between left action and right action, as $gh$ means $g \circ h$ in left action and means $h \circ g$ in right action. \\

Because groups admit inverses, a $G$-action is in fact a group homomorphism $G \rightarrow \operatorname{Sym}(X)$. The trivial action then correspond to the trivial group homomorphism.  An action is \textbf{faithful} if it is injective. \\


Show that $Z(G)\subseteq \operatorname{Ker} \theta$ if and only if $\theta$ is faithful.  \\


An action is \textbf{free} if $g\cdot x=x$ for a $x\in X$ implies $g=e$. Note that the isomorphism $\operatorname{Sym}(X)\rightarrow \operatorname{Sym}(X)$ is always injective but never free unless $\abso{X}\leq 2$. The action is \textbf{transitive} if for any $x,y \in X$, there always exists some $g \in G$ such that $y= g \cdot x$. An action is \textbf{regular} if it is both free and transitive.\\

Let $x \in X$. We call the set $G\cdot x\triangleq \set{g\cdot x \in X: g\in G }$ the \textbf{orbit} of $x$. Clearly the set $G_x$ of all elements of $G$ that fixes $x$ forms a group, called the \textbf{stabilizer subgroup} of $G$ with respect to $x$. Consider the action left. The fact that the obvious mapping between the set of left cosets of stabilizer subgroups of $G$ with respect to $x$ to the orbit of  $x$: 
\begin{align*}
\set{g (G_x) \subseteq G : g \in G} \longleftrightarrow G \cdot x
\end{align*}
forms a bijection is called the \textbf{orbit-stabilizer theorem}. \\

\begin{theorem}
\textbf{(Cauchy's theorem for finite group)}  Let $G$ be a finite group whose order is divided by some prime $p$. Then the number of solutions to the equation $x^p =e$ is a positive multiple of  $p$.  
\end{theorem}
\begin{proof}
The set $X$ of $p$-tuples  $(x_1,\dots ,x_p)$ that satisfies $x_1\cdots x_p=e$ clearly has cardinality $\abso{G}^{p-1}$. \\

Consider the group action $\Z_p \rightarrow \operatorname{Sym}(X)$ defined by 
\begin{align*}
g \cdot (x_1,\dots ,x_p) \triangleq (x_p,x_1,\dots ,x_{p-1}) 
\end{align*}
Then by orbit-stabilizer theorem and Lagrange theorem, an orbit in $X$ either has cardinality $p$ or  $1$. \\

\begin{align*}
p| \abso{G}^{p-1}=m+kp
\end{align*}
with  $m$ the number of cardinality $1$ orbits and $k$ the number of cardinality  $p$ orbits.\\

This implies $p|m$, as desired.



Notice that $x^p=e$ if and only if  $(x,\dots ,x) \in X$. Therefore the number of cardinality $1$ orbit equals to number of solution to $x^p=e$.\\

\end{proof}


\section{Normalizer and centralizer}
Because the inverse of an injective group homomorphism forms a group homomorphism, we know the set $\operatorname{Aut}(G)$ of automorphisms of $G$ forms a group. We say $\pfi \in \operatorname{Aut}(G)$ is an \textbf{inner automorphism} if  $\pfi $ takes the form $x\mapsto  gxg^{-1}$ for some fixed $g \in G$. We say two elements  $x,y \in G$ are \textbf{conjugated} if there exists some inner automorphism that maps $x$ to $y$. Clearly conjugacy forms a equivalence relation. We call its classes \textbf{conjugacy classes}. 


\begin{equiv_def}
\textbf{(Normalize)}  
\end{equiv_def}

From the point of view of inner automorphism, we see that it is well-defined whether an element $g\in G$ \textbf{normalize} a subset $S\subseteq G$:
\begin{align*}
\set{gsg^{-1} \in G: s \in S} = S
\end{align*}
independent of left and right.  Because of the independence, For each subset $S\subseteq G$, we see that the set of elements $g\in G$ that normalize $S$ forms a group, called the \textbf{normalizer} of $S$. Note that if $g$ normalize  $S$, then $gS = Sg$.   \\







\begin{example}
  Consider $G\triangleq \operatorname{GL}_2(\R)$  and consider: 
\begin{align*}
H\triangleq \set{ \begin{pmatrix}
    1 & n \\
    0 & 1
\end{pmatrix} \in \operatorname{GL}_2(\R) : n \inz}  \quad \text{ and }\quad g\triangleq \begin{pmatrix} 
 2 & 0 \\
 0 & 1
\end{pmatrix} \in \operatorname{GL}_2(\R)
\end{align*}
Note that $gHg^{-1}\subset H $. In other words, inner automorphisms can maps a subgroup $H$ into a subgroup strictly contained by $H$ if  $G$ is infinite. 
\end{example}


\begin{equiv_def}
\textbf{(Normal subgroups)} Let $G$ be a group and $N$ a subgroup. We say $N$ is a \textbf{normal subgroup} of $G$ if any of the followings hold true: 
\begin{enumerate}[label=(\roman*)]
  \item $\pfi  (N) \subseteq N$  for all $\pfi  \in \operatorname{Inn}(G)$
  \item $\pfi  (N)=N$ for all $\pfi  \in \operatorname{Inn}(G)$ 
  \item $xN=Nx$ for all  $x \in G$.  
  \item The set of all left cosets of $N$ equals the set of all right cosets of $N$. 
  \item $N$ is a union of conjugacy classes. 
  \item For all $n \in N$ and $x \in G$, their \textbf{commutator} $nxn^{-1}x^{-1} \in G$ lies in $N$.  
  \item For all $x,y \in G$, we have $xy\in N \iff  yx \in N$. 
\end{enumerate}
\end{equiv_def}
\begin{proof}
  (i)$\implies $(ii): Let $\pfi  \in \operatorname{Inn}(G)$. By premise, $\pfi  (N)\subseteq N$ and $\pfi ^{-1}(N)\subseteq N$. Applying  $\pfi $ to both side of $\pfi ^{-1}(N)\subseteq N$, we have $\pfi (N)\subseteq N \subseteq \pfi  (N)$, as desired. \\

(ii)$\implies $(iii): Consider the automorphisms:  
\begin{align*}
 \pfi_{L,x}(g)= xg\quad \text{ and }\quad \pfi _{L,x^{-1}}(g)=x^{-1}g \quad \text{ and }\quad \pfi _{R,x}(g)= gx
\end{align*}
 Because $\pfi _{L,x^{-1}}\circ \pfi _{R,x} \in \operatorname{Inn}(G)$, by premise we have:
\begin{align*}
xN= \pfi _{L,x}(N)= \pfi _{L,x}\circ \pfi _{L,x^{-1}}\circ \pfi _{R,x}(N)= \pfi _{R,x}(N)=Nx 
\end{align*}

(iii)$\implies $(iv) is clear. (iv)$\implies $(iii): Let $x\in G$. By premise, there exists some $y\in G$ that makes $xN=Ny$. Let $x=ny$. The proof then follows from noting 
\begin{align*}
xN=Ny=N(n^{-1}x)=Nx
\end{align*}
(iii)$\implies $(v): Let $n \in N$ and $x \in G$. We are required to show $xnx^{-1} \in N$. Because $xN=NX$, we know  $xn=\tilde{n}x$ for some $\tilde{n}\in N$. This implies 
\begin{align*}
xnx^{-1}= \tilde{n}xx^{-1}=\tilde{n}\in N  
\end{align*}
(v)$\implies $(vi): Fix $n\in N$ and $x\in G$. By premise, $xn^{-1}x^{-1} \in N$. Therefore, $n(xn^{-1}x^{-1})\in N$, as desired.\\


(vi)$\implies $(vii): Let $xy \in N$. To see $yx$ also belong to $N$, observe: 
\begin{align*}
 (xy)^{-1}(yx) =(xy)^{-1}x^{-1}xyx=[xy,x] \in N
\end{align*}
(viii)$\implies $(i): Let $n \in N$ and $x\in G$. Because $(nx)x^{-1}=n \in N$, by premise we have $x^{-1}nx \in N$, as desired.
\end{proof}
\begin{equiv_def}
\textbf{(Normal closure)} Let $G$ be a group and $S\subseteq G$. The \textbf{normal closure} $\operatorname{ncl}_G(S)$ of $S$ in  $G$  refer to any one of the followings: 
\begin{enumerate}[label=(\roman*)]
  \item The smallest normal subgroup of $G$ containing $S$, which we know exists as the intersection of all normal subgroups of $G$ containing $S$.  
  \item The subgroup of $G$ generated by 
\begin{align*}
 \bigcup_{\pfi \in \operatorname{Inn}(G)} \set{\pfi (x) \in G:x \in S} 
\end{align*}   
\end{enumerate}
\end{equiv_def}
\begin{proof}
We are required to prove the subgroup of $G$ from (ii) is normal. Clearly, it is the set:
\begin{align*}
\set{g_1^{-1}x_1^{\epsilon _1}g_1 \cdots g_n^{-1}x_n^{\epsilon_n}g_n \in G: n\geq 0,x_i \in S,\epsilon_i =\pm 1, g_i \in G }
\end{align*}
Fix $g \in G$. The proof then follows from noting
\begin{align*}
g^{-1}\left(g_1^{-1}x_1^{\epsilon _1}g_1 \cdots g_n^{-1}x_n^{\epsilon_n}g_n\right)g= \left( \left(g_1g\right)^{-1}x_1 ^{\epsilon _1} \left(g _1g  \right) \right) \cdots   \left( \left(g _ng\right)^{-1}x_n ^{\epsilon _n} \left(g_ng  \right) \right) 
\end{align*}
\end{proof}

We denote the \textbf{centralizer} $C_G(S)\triangleq \set{g \in G: gsg^{-1}=s\text{ for all }s \in S}$. We call the centralizer of the whole group $Z(G)\triangleq C_G(G)$ \textbf{center}. Clearly $Z(G)$ forms an abelian subgroup of $G$, and every element of the center form a single conjugacy classes.     \\

For finite group $G$, we have the \textbf{class equation}
\begin{align*}
 \abso{G} = \abso{Z(G)}+ \sum \abso{G: C_G(x)}
\end{align*}
where $x$ runs through conjugacy classes outside of  $Z(G)$. \\





Clearly $C_G(S)\subseteq N_G(S)$. 

\section{Isomorphism theorems}
 Let $G$ be a group and $N\subseteq  G$ a normal subgroup. We say a group homomorphism $ \pi :G \rightarrow   G\quotient N$ satisfies the \textbf{universal property of quotient group $G\quotient N$} if 
\begin{enumerate}[label=(\roman*)]
  \item it vanishes on $N$. \textbf{(Group condition)}
  \item for all group homomorphism $f:G\rightarrow H$ that vanishes on $N$ there exist a unique group homomorphism $\tilde{f} :G\quotient N\rightarrow H$ that makes the diagram: 
% https://q.uiver.app/#q=WzAsMyxbMiwwLCJHXFxxdW90aWVudCBOICJdLFswLDAsIkciXSxbMiwyLCJMIl0sWzAsMiwiZyJdLFsxLDIsImYiLDJdLFsxLDAsIiIsMCx7InN0eWxlIjp7ImhlYWQiOnsibmFtZSI6ImVwaSJ9fX1dXQ==
\[\begin{tikzcd}
	G && {G\quotient N } \\
	\\
	&& H 
	\arrow["\pi",two heads, from=1-1, to=1-3]
	\arrow["f"', from=1-1, to=3-3]
	\arrow["\tilde{f} ", from=1-3, to=3-3]
\end{tikzcd}\]
commute. \textbf{(Universality)}
\end{enumerate}
\begin{theorem}
\textbf{(The first isomorphism theorem for groups)} The group homomorphism $\pi : G \rightarrow G \quotient N$ is always surjective with kernel $N$. Let $f :G \rightarrow H$ be a group homomorphism. Then $\operatorname{ker}f$ is normal in $G$, and the induced homomorphism $\tilde{f}: G \quotient \operatorname{ker}f \rightarrow H$ is injective.   
\end{theorem}
\begin{proof}
The first part is an immediate consequence of construction of $G \quotient N$. However, it should be noted that such construction can be avoided. The fact that $\operatorname{ker}(\pi )=N$ can be proved by considering the permutation representation $G \rightarrow \operatorname{Sym}(\Omega)$, where $\Omega$ is the set of the cosets of $N$, and the fact that  $\pi $ is surjective is a consequence of $\tilde{\pi }=\id_{G\quotient N}$. \\

We clearly have $\operatorname{ker}f \trianglelefteq G$. The fact that $\tilde{f}:G \quotient \operatorname{ker}f \rightarrow H$ is injective follows from $\pi : G \rightarrow G \quotient \operatorname{ker}f$ being surjective with kernel $\operatorname{ker}f$. 
\end{proof}



Because the kernel of a group homomorphism is clearly normal, if $N$ is not normal, then there can not be a pair $G\rightarrow G\quotient N$ that satisfies the universal property.  If any things, this is the "reason" why normal subgroups are what meant to be quotiented in the category of group.  \\

Given $x,y \in G$, we often write  
\begin{align*}
[x,y]\triangleq xyx^{-1}y^{-1}\quad \text{ or }\quad [x,y]\triangleq x^{-1}y^{-1}xy
\end{align*}
and call $[x,y]$ the \textbf{commutator} of $x$ and  $y$. Independent of differences of the definition, we have $[x,y]\in N$ if and only if $xyN=yxN$. Again, independent of the definition, the \textbf{commutator subgroup} $[G,G]$ of $G$ is the subgroup generated by the commutators. 

\begin{theorem}
\textbf{()} 
\begin{align*}
G\quotient N\text{ is abelian }\iff [G,G] \subseteq N
\end{align*}
\end{theorem}
\begin{proof} 
  ($\implies$): 
\begin{align*}
  (xyx^{-1}y^{-1})N= xN \cdot yN \cdot (x^{-1})N \cdot (y^{-1} )N = N 
\end{align*}
$(\impliedby)$: 
\end{proof}


\begin{example}
$G\triangleq S_3$. $S\triangleq  \langle (1,2)\rangle $ and $H\triangleq \langle (2,3)\rangle $. $SH$ doesn't form a group. $(2,3)(1,2)\not \in SH$. 
\end{example}
\begin{theorem}
\textbf{(Second isomorphism theorem)} Let $H\leq G$. If $K$ is a subgroup of normalizer of $H$, then their product: 
\begin{align*}
HK\triangleq  \set{hk \in G: h \in H\text{ and } k \in K}
\end{align*}
forms a group and is defined independent of left and right. Moreover, $H\trianglelefteq HK$ with $hkH=kH$, and $H\cap K \trianglelefteq K$ with 
\begin{align*}
HK \quotient H \cong  K \quotient H \cap K \quad \text{ via } \quad  kH  \longleftrightarrow k (H \cap K) 
\end{align*}
\end{theorem}
\begin{proof}

\end{proof}
Third isomorphism theorem. 

Correspondence theorem.


Because $ \phi\circ \pfi _g \circ \phi^{-1} = \pfi _{\phi(g)}$, we know $\operatorname{Inn}(G)$ forms a normal subgroup of $\operatorname{Aut}(G)$. \\

\section{Sylow theorems}
Let $o(G)\triangleq p^mq$ with $\operatorname{gcd}(p,q)=1$, and let $n\leq m$. Because 
\begin{align*}
\binom{p^mq}{p^m}= \frac{p^mq(p^mq-1)\cdots (p^mq-p^m+1)}{p^m(p^m-1)\cdots 1}
\end{align*}
and clearly 
\begin{align*}
p^k | p^mq- i \iff p^k |i\iff p^k | p^m-i,\quad \text{ for all $i$ and  $k$ }  \divides
\end{align*}

Let $\mathcal{S}$ be the  set of subsets of $G$ with cardinality  $p^n$. Clearly $\abso{\mathcal{S}}=\binom{o(G)}{p^n}$ and we may define a left $G$-action on $\mathcal{S}$ by
\begin{align*}
g\cdot \set{h_1,\dots ,h_{p^n}} \triangleq \set{gh_1,\dots ,gh_{p^n}}
\end{align*}
we 
\section{Exercises}
\begin{question}{}{}
Show that 
\begin{enumerate}[label=(\roman*)]
  \item If $H \quotient Z(H)$ is cyclic, then $H$ is abelian.  
  \item If $H$ is of order  $p^2$, then  $H$ is abelian.   
\end{enumerate}
From now on, suppose $G$ is non-abelian with order  $p^3$. 
\begin{enumerate}[label=(\roman*), start=3]
  \item $\abso{Z(G)}=p$. 
  \item $Z(G)=[G,G]$.  
\end{enumerate}
\end{question}
\begin{proof}
Let $a,b\in H$ and $H \quotient Z(H)= \langle hZ\rangle $. Write $a=h^nz_1$ and $b=h^mz_2$. Because  $z_1,z_2 \in Z(H)$, we may compute: 
\begin{align*}
ab=h^nz_1h^mz_2=h^{n+m}z_1z_2=ba
\end{align*}
as desired. \\

Let $\abso{H}=p^2$. Because $H$ is a  $p$-group, we know $Z(H)$ is nontrivial, therefore either $\abso{Z(H)}=p$ or $\abso{Z(H)}=p^2$. To see the former is impossible, just observe that if so, then $\abso{H\quotient Z(H)}=p$, which implies $H\quotient Z(H)$ is cyclic, which by part (i) implies $Z(H)=H$. \\

Because $G$ is non-abelian, we know  $\abso{Z(G)}\neq p^3$. Because $G$ is a  $p$-group, we know  $\abso{Z(G)}\neq 1$. Therefore, either $\abso{Z(G)}=p$ or $\abso{Z(G)}=p^2$. Part (i) tell us that $\abso{Z(G)}\neq p^2$, otherwise $G$ is abelian, a contradiction. We have shown  $\abso{Z(G)}=p$, as desired. \\

We now prove $Z(G)=[G,G]$. Because $\abso{Z(G)}=p$, by part (ii) we know $G\quotient Z(G)$ is abelian. This implies $[G,G] \leq  Z(G)$, which implies $[G,G]$ is either trivial or equal to $Z(G)$. Because $G$ is non-abelian, we know  $[G,G]$ can not be trivial. This implies $Z(G)=[G,G]$, as desired.
\end{proof}
\begin{question}{}{}
\begin{enumerate}[label=(\roman*)]
  \item Let $M,N$ be two normal subgroups of  $G$ with  $MN=G$. Prove that 
\begin{align*}
    G\quotient (M\cap N) \cong  (G\quotient M) \times (G\quotient N)
\end{align*} 
\item Let $H,K$ be two distinct subgroups of  $G$ of index  $2$. Prove that  $H\cap K$ is a normal subgroup with index $4$ and  $G \quotient (H\cap K)$ is not cyclic.  
\end{enumerate}
\end{question}
\begin{proof}
The map $G\quotient (M\cap N) \rightarrow (G \quotient M)\times (G\quotient N)$ defined by 
\begin{align}
\label{EQmg}
g(M\cap N) \mapsto (gM,gN)
\end{align}
is clearly a well-defined group homomorphism, since if $gM=hM$ and  $gN=hN$, then  $gh^{-1}\in M$ and $gh^{-1} \in N$, which implies $gh^{-1}\in M \cap N$, which implies $g(M\cap N)=h(M \cap N)$. Let $gM=M$ and  $gN=N$. Then  $g \in M \cap N$ and $g(M \cap N)=M \cap N$. Therefore \myref{map}{EQmg} is also injective. It remains to show \myref{map}{EQmg} is surjective. Fix $g,h \in G$. Write $g=mn$ and  $h=\tilde{m}\tilde{n}$. Clearly $gM=nM=\tilde{m}nM$ and  $hN=\tilde{m}N=\tilde{m}nN $. This implies that  \myref{mapping}{EQmg} maps $\tilde{m}n$ to $(gM,hN)$, as desired.\\

Because $H,K$ are both of index  $2$ in $G$, we know they are both normal in $G$. This by second isomorphism theorem implies  $HK$ forms a subgroup of  $G$. Because $H\neq K$, we know $HK$ properly contains $H$, which by finiteness of $G$ implies  the index of $HK$ is strictly less than  $H$, i.e., $HK=G$. Note that $H \cap K$ is normal since it is the intersection of normal subgroups. By part (i), we now have $G \quotient (H\cap K)\cong  (G \quotient H)\times (G\quotient K)\cong  \Z_2\times\Z_2$, which shows that $H \cap K$ has index $4$ and  $G \quotient (H\cap K)$ is cyclic.     
\end{proof}
\begin{question}{}{}
Let $G$ be a group of order  $pq$, where  $p>q$ are prime. 
\begin{enumerate}[label=(\roman*)]
  \item Show that there exists a unique subgroup of order $p$.  
  \item Suppose $a \in G$ with $o(a)=p$. Show that $\langle a\rangle \subseteq G$ is normal and for all $x \in G$, we have $x^{-1}ax =a^i$ for some $0<i<p$. 
\end{enumerate}
\end{question}
\begin{proof}
The third Sylow theorem stated that the number $n_p$ of Sylow $p$-subgroups satisfies 
\begin{align*}
n_p\equiv 1\pmod{p} \quad \text{ and }\quad n_p \mid q
\end{align*}
Because $p>q$, together they implies $n_p=1$. Since Sylow $p$-subgroups of  $G$ are exactly subgroups of order  $p$, we have proved (i).  \\

The third Sylow theorem also stated that $n_p= \abso{G:N_G(P)}$ for any Sylow $p$-subgroup  $P\leq G$. Therefore, $N_G(\langle a\rangle )=G$, i.e., $\langle a\rangle $ is normal in $G$. Fix  $x\in G$. It remains to prove $xax^{-1}\neq e$, which is a consequence of the fact that conjugacy (automorphism) preserves order.
\end{proof}
\begin{question}{}{}
Let $H,K$ be two subgroups of $G$ of coprime finite indices  $m,n$. Show that 
\begin{align*}
\operatorname{lcm}(m,n) \leq \abso{G:H \cap K} \leq mn
\end{align*}
\end{question}
\begin{proof}
Let $\Omega_{H\cap K},\Omega_$, and $\Omega_K$ respectively denote the set of left cosets of $H\cap K,H$, and $K$.  The map $\Omega_{H\cap K} \rightarrow \Omega_H \times \Omega_K$ defined by 
\begin{align}
\label{EQmgh}
g(H\cap K) \mapsto (gH,gK) 
\end{align}
is well defined since 
\begin{align*}
g(H\cap K)=l(H\cap K) \implies g^{-1}l \in H \cap K \implies gH=lH\text{ and }gK=lK
\end{align*}
such set map is injective since if $gH=lH$ and  $gK=lK$, then  $g^{-1}l \in H$ and $g^{-1}l\in K$, which implies $g(H \cap K)=l(H \cap K)$, as desired. From the injectivity of \myref{map}{EQmgh}, we have shown  index of  $H\cap K$ indeed have upper bound $mn$. \\


\end{proof}

\end{document}
