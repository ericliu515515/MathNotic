\documentclass{report}
%%%%%%%%%%%%%% macros.tex %%%%%%%%%%%%%%
% Place your custom macros here, if any.

%%%%%%%%%%%%%% letterfonts.tex %%%%%%%%%%%%%%
% Place your font setup here, if any.

%%%%%%%%%%%%%% preamble.tex %%%%%%%%%%%%%%
\usepackage[T1]{fontenc}
\usepackage{lmodern}
\usepackage{etoolbox}
\usepackage{pdfpages}
\usepackage{transparent}
\usepackage[utf8]{inputenc}
\usepackage[english]{babel}

% Page Setup
\usepackage[tmargin=2cm, rmargin=0.5in, lmargin=0.5in, bmargin=80pt, footskip=.2in]{geometry}

% Mathematics
\usepackage{amsmath,amsfonts,amsthm,amssymb,mathtools}
\usepackage{xfrac}
\usepackage[makeroom]{cancel}
\usepackage{enumitem}
\usepackage{nameref}
\usepackage{multicol,array}
\usepackage{tikz-cd}
\usepackage[ruled,vlined,linesnumbered]{algorithm2e}

% Colors
\usepackage[dvipsnames]{xcolor}
\definecolor{myg}{RGB}{56, 140, 70}
\definecolor{myb}{RGB}{45, 111, 177}
\definecolor{myr}{RGB}{199, 68, 64}
% Define more colors here...

% Hyperlinks
\usepackage{bookmark}
\usepackage{hyperref}
\hypersetup{
    pdftitle={Assignment},
    colorlinks=true, linkcolor=doc!90,
    bookmarksnumbered=true,
    bookmarksopen=true
}

% Figures and Graphics
\usepackage{import}
\usepackage{svg}
\newcommand{\incfig}[1]{%
    \def\svgwidth{\columnwidth}
    \import{./figures/}{#1.pdf_tex}
}

% Text-related
\usepackage{blindtext}
\usepackage{fontsize}
\changefontsize[14]{14}
\setlength{\parindent}{0pt}

% Theorems and Definitions
\usepackage{amsthm}
\renewcommand\qedsymbol{$\blacksquare$}

% Define a new theorem style
\newtheoremstyle{mytheoremstyle}% name
  {}% Space above
  {}% Space below
  {\sffamily}% Body font
  {}% Indent amount
  {\bfseries}% Theorem head font
  {.}% Punctuation after theorem head
  {.5em}% Space after theorem head
  {}% Theorem head spec (can be left empty, meaning ‘normal’)

% Apply the new theorem style to theorem-like environments
\theoremstyle{mytheoremstyle}
\newtheorem{theorem}{Theorem}[section]
\newtheorem{definition}{Definition}[section]
\newtheorem{corollary}{Corollary}[section]
\newtheorem{lemma}{Lemma}[section]
\newtheorem{axiom}{Axiom}[section]

% tcolorbox Setup
\usepackage[most,many,breakable]{tcolorbox}

% Define custom tcolorbox environments here...

%================================
% EXAMPLE BOX
%================================
\newtcbtheorem[definition]{Example}{Example}
{%
    colback = myexamplebg,
    breakable,
    colframe = myexamplefr,
    coltitle = myexampleti,
    boxrule = 1pt,
    sharp corners,
    detach title,
    before upper=\tcbtitle\par\smallskip,
    fonttitle = \bfseries,
    description font = \mdseries,
    separator sign none,
    description delimiters parenthesis,
}
{ex}

%================================
% Solution BOX
%================================
\makeatletter
\newtcolorbox{solution}{enhanced,
	breakable,
	colback=white,
	colframe=myg!80!black,
	attach boxed title to top left={yshift*=-\tcboxedtitleheight},
	title=Solution,
	boxed title size=title,
	boxed title style={%
			sharp corners,
			rounded corners=northwest,
			colback=tcbcolframe,
			boxrule=0pt,
		},
	underlay boxed title={%
			\path[fill=tcbcolframe] (title.south west)--(title.south east)
			to[out=0, in=180] ([xshift=5mm]title.east)--
			(title.center-|frame.east)
			[rounded corners=\kvtcb@arc] |-
			(frame.north) -| cycle;
		},
}
\makeatother

%================================
% Question BOX
%================================
\makeatletter
\newtcbtheorem{question}{Question}{enhanced,
	breakable,
	colback=white,
	colframe=myb!80!black,
	attach boxed title to top left={yshift*=-\tcboxedtitleheight},
	fonttitle=\bfseries,
	title={#2},
	boxed title size=title,
	boxed title style={%
			sharp corners,
			rounded corners=northwest,
			colback=tcbcolframe,
			boxrule=0pt,
		},
	underlay boxed title={%
			\path[fill=tcbcolframe] (title.south west)--(title.south east)
			to[out=0, in=180] ([xshift=5mm]title.east)--
			(title.center-|frame.east)
			[rounded corners=\kvtcb@arc] |-
			(frame.north) -| cycle;
		},
	#1
}{def}
\makeatother
\makeatletter
\newtcbtheorem{qstion}{Question}{enhanced,
    breakable,
    colback=white,
    colframe=mygr,
    attach boxed title to top left={yshift*=-\tcboxedtitleheight},
    fonttitle=\bfseries,
    title={#2},
    boxed title size=title,
    boxed title style={%
        sharp corners,
        rounded corners=northwest,
        colback=tcbcolframe,
        boxrule=0pt,
    },
    underlay boxed title={%
        \path[fill=tcbcolframe] (title.south west)--(title.south east)
        to[out=0, in=180] ([xshift=5mm]title.east)--
        (title.center-|frame.east)
        [rounded corners=\kvtcb@arc] |-
        (frame.north) -| cycle;
    },
    #1
}{def}
\makeatother

%%%%%%%%%%%%%%%%%%%%%%%%%%%%%%%%%%%%%%%%%%%
% TABLE OF CONTENTS
%%%%%%%%%%%%%%%%%%%%%%%%%%%%%%%%%%%%%%%%%%%
\usepackage{tikz}
\definecolor{doc}{RGB}{0,60,110}
\usepackage{titletoc}
\contentsmargin{0cm}
\titlecontents{chapter}[14pc]
{\addvspace{30pt}%
	\begin{tikzpicture}[remember picture, overlay]%
		\draw[fill=doc!60,draw=doc!60] (-7,-.1) rectangle (-0.9,.5);%
		\pgftext[left,x=-4.5cm,y=0.2cm]{\color{white}\Large\sc\bfseries Chapter\ \thecontentslabel};%
	\end{tikzpicture}\color{doc!60}\large\sc\bfseries}%
{}
{}
{\;\titlerule\;\large\sc\bfseries Page \thecontentspage
	\begin{tikzpicture}[remember picture, overlay]
		\draw[fill=doc!60,draw=doc!60] (2pt,0) rectangle (4,0.1pt);
	\end{tikzpicture}}%
\titlecontents{section}[3.7pc]
{\addvspace{2pt}}
{\contentslabel[\thecontentslabel]{2pc}}
{}
{\hfill\small \thecontentspage}
[]
\titlecontents*{subsection}[3.7pc]
{\addvspace{-1pt}\small}
{}
{}
{\ --- \small\thecontentspage}
[ \textbullet\ ][]

\makeatletter
\renewcommand{\tableofcontents}{
	\chapter*{%
	  \vspace*{-20\p@}%
	  \begin{tikzpicture}[remember picture, overlay]%
		  \pgftext[right,x=15cm,y=0.2cm]{\color{doc!60}\Huge\sc\bfseries \contentsname};%
		  \draw[fill=doc!60,draw=doc!60] (13,-.75) rectangle (20,1);%
		  \clip (13,-.75) rectangle (20,1);
		  \pgftext[right,x=15cm,y=0.2cm]{\color{white}\Huge\sc\bfseries \contentsname};%
	  \end{tikzpicture}}%
	\@starttoc{toc}}
\makeatother

\newcommand{\liff}{\llap{$\iff$}}
\newcommand{\rap}[1]{\rrap{\text{ (#1)}}}
\newcommand{\red}[1]{\textcolor{red}{#1}}
\newcommand{\blue}[1]{\textcolor{blue}{#1}}
\newcommand{\vi}[1]{\textcolor{violet}{#1}}
\newcommand{\teal}[1]{\textcolor{teal}{#1}}
\newcommand{\tCaC}{\text{ \CaC }}
\newcommand{\CaC}{\red{CaC} }
\newcommand{\As}[1]{Assume \red{#1}}
\newcommand{\vdone}{\vi{\text{ (done) }}}
\newcommand{\bdone}{\blue{\text{ (done) }}}
\newcommand{\tdone}{\teal{\text{ (done) }}}
\newcommand{\set}[1]{\{ #1 \}}
\newcommand{\inS}{\in S}
\newcommand{\inF}{\in\F}
\newcommand{\inE}{\in E}
\newcommand{\inA}{\in A}
\newcommand{\inB}{\in B}
\newcommand{\inC}{\in C}
\newcommand{\inU}{\in U}

\newcommand{\C}{\mathbb{C}}	
\renewcommand{\H}{\mathbb{H}}
\newcommand{\F}{\mathbb{F}}
\newcommand{\N}{\mathbb{N}}
\newcommand{\Q}{\mathbb{Q}}
\newcommand{\R}{\mathbb{R}}
\newcommand{\Z}{\mathbb{Z}}
\renewcommand{\P}{\mathbb{P}}
\renewcommand{\S}{\mathbb{S}}
\newcommand{\A}{\mathbb{A}}
\newcommand{\RP}{\R P}

%================================
% Abstract 
%================================

\usepackage{lipsum}

\renewenvironment{abstract}
 {% Start of environment
  \quotation
  \small
  \noindent
  \rule{\linewidth}{.5pt} % Draw the rule to match the linewidth
  \par\smallskip
  {\centering\bfseries\abstractname\par}\medskip
 }
 {% End of environment
  \par\noindent
  \rule{\linewidth}{.5pt} % Ensure the closing rule also matches
  \endquotation
 }


%================================
% Question BOX
%================================

\newcounter{qnum} % The counter name for Question box is qnum

\makeatletter
\newtcbtheorem[use counter=qnum]{question}{Question}{enhanced,
	breakable,
	colback=white,
	colframe=myb!80!black,
	attach boxed title to top left={yshift*=-\tcboxedtitleheight},
	fonttitle=\bfseries,
	title={#2},
	boxed title size=title,
	boxed title style={%
			sharp corners,
			rounded corners=northwest,
			colback=tcbcolframe,
			boxrule=0pt,
		},
	underlay boxed title={%
			\path[fill=tcbcolframe] (title.south west)--(title.south east)
			to[out=0, in=180] ([xshift=5mm]title.east)--
			(title.center-|frame.east)
			[rounded corners=\kvtcb@arc] |-
			(frame.north) -| cycle;
		},
	#1
}{def}
\makeatother
\makeatletter
\newtcbtheorem{qstion}{Question}{enhanced,
    breakable,
    colback=white,
    colframe=mygr,
    attach boxed title to top left={yshift*=-\tcboxedtitleheight},
    fonttitle=\bfseries,
    title={#2},
    boxed title size=title,
    boxed title style={%
        sharp corners,
        rounded corners=northwest,
        colback=tcbcolframe,
        boxrule=0pt,
    },
    underlay boxed title={%
        \path[fill=tcbcolframe] (title.south west)--(title.south east)
        to[out=0, in=180] ([xshift=5mm]title.east)--
        (title.center-|frame.east)
        [rounded corners=\kvtcb@arc] |-
        (frame.north) -| cycle;
    },
    #1
}{def}
\makeatother


% COMMUTATIVE DIAGRAM
%%%%%%%%%%%%%%%%%%%%%%%%%%%%%%%%%%%%%%%%%%%


\usepackage{tikz}

\usetikzlibrary{arrows} % Needed for custom arrow tips
\usetikzlibrary{arrows.meta} % Needed for custom arrow tips

\newcommand*{\DashedArrow}[1][]{\mathbin{\tikz [baseline=-0.25ex,-latex, dashed,#1] \draw [#1] (0pt,0.5ex) -- (1.3em,0.5ex);}}


\tikzcdset{
  doubletail/.style={
    arrows={tail reversed-tail}
  }
}
\tikzstyle{double_arrow} = [thick,<->,>=stealth]

\tikzset{
  broken double arrow/.style={
    thick,                    % line width
    dashed,                   % make it broken
    <->,                      % arrowheads at both ends
    >=Stealth[length=2pt,width=1pt] % thinner arrowheads           
  }
}




% TABLE OF CONTENTS
%%%%%%%%%%%%%%%%%%%%%%%%%%%%%%%%%%%%%%%%%%
\definecolor{doc}{RGB}{0,60,110}
\usepackage{titletoc}
\contentsmargin{0cm}
\titlecontents{chapter}[14pc]
{\addvspace{30pt}%
	\begin{tikzpicture}[remember picture, overlay]%
		\draw[fill=doc!60,draw=doc!60] (-7,-.1) rectangle (-0.9,.5);%
		\pgftext[left,x=-5.5cm,y=0.2cm]{\color{white}\Large\sc\bfseries Chapter\ \thecontentslabel};%
	\end{tikzpicture}\color{doc!60}\large\sc\bfseries}%
{}
{}
{\;\titlerule\;\large\sc\bfseries Page \thecontentspage
	\begin{tikzpicture}[remember picture, overlay]
		\draw[fill=doc!60,draw=doc!60] (2pt,0) rectangle (4,0.1pt);
	\end{tikzpicture}}%
\titlecontents{section}[3.7pc]
{\addvspace{2pt}}
{\contentslabel[\thecontentslabel]{3pc}}
{}
{\hfill\small \thecontentspage}
[]
\titlecontents*{subsection}[3.7pc]
{\addvspace{-1pt}\small}
{}
{}
{\ --- \small\thecontentspage}
[ \textbullet\ ][]

\makeatletter
\renewcommand{\tableofcontents}{
	\chapter*{%
	  \vspace*{-20\p@}%
	  \begin{tikzpicture}[remember picture, overlay]%
		  \pgftext[right,x=15cm,y=0.2cm]{\color{doc!60}\Huge\sc\bfseries \contentsname};%
		  \draw[fill=doc!60,draw=doc!60] (13,-.75) rectangle (20,1);%
		  \clip (13,-.75) rectangle (20,1);
		  \pgftext[right,x=15cm,y=0.2cm]{\color{white}\Huge\sc\bfseries \contentsname};%
	  \end{tikzpicture}}%
	\@starttoc{toc}}
\makeatother

\title{Suns}
\author{Eric Liu}
\date{}
\begin{document}
\maketitle
\newpage % or \cleardoublepage
% \pdfbookmark[<level>]{<title>}{<dest>}
\pdfbookmark[section]{\contentsname}{toc}

\tableofcontents
\pagebreak
\chapter{Groups}

\section{Group action}
Let $M$ be a set equipped with a binary operation $M\times M \rightarrow M$. We say $M$ is a \textbf{monoid} if the binary operation is associative and there exists a two-sided identity $e \in M$. 
\begin{Example}{}{}
Defining $(x,y)\mapsto y$, we see that the operation is associative and every element is a left identity, but no element is a right identity unless $\abso{M}=1$. This is an example why identity must be two-sided.  
\end{Example}

Because the identity of a monoid is defined to be two-sided, clearly it must be unique.  Suppose every element of monoid $M$ has a left inverse. Fix $x \in M$. Let $x^{-1}\in M$ be a left inverse of $x$. To see that  $x^{-1}$ is also a right inverse of $x$, let  $(x^{-1})^{-1}\in M$ be a left inverse of $x^{-1}$ and use  
\begin{align*}
  (x^{-1})^{-1}=(x^{-1})^{-1}e=(x^{-1})^{-1}(x^{-1}x)= ((x^{-1})^{-1}x^{-1})x= ex=x
\end{align*}
to deduce
\begin{align*}
xx^{-1}=(x^{-1})^{-1}x^{-1}= e
\end{align*}
In other words, if we require every element of a monoid $M$ to has a left inverse, then immediately every left inverse upgrades to a right inverse. In such case, we call $M$ a  \textbf{group}. Notice that inverses of elements of a group are clearly unique. 
\begin{Example}{}{}
  Another set of axioms for group is to require the operation to be associative, identity to be left, and the existence of left inverses. Under such condition, we see that if an element $y$  is \textbf{idempotent}, then it must be identity, since $y=(y^{-1}y)y=y^{-1}y=e$. Because of such, we see a left inverse is also a right inverse, since $(xx^{-1})(xx^{-1})=xe x^{-1}=xx^{-1}$. This then shows that the left identity is also a right identity, since $xe=x(x^{-1}x)=x$. 
\end{Example}
\begin{Example}{}{THwc} 
If the underlying set is finite, then weaker assumption can be made. For example, suppose only that the binary operation to be associative and that both cancellation holds: 
\begin{align*}
au=aw \implies u=w \quad \text{ and }\quad ua=wa \implies  u=w
\end{align*}
Because the set is finite, for all $a$, we may attach it with an natural number $n(a)$ such that  $a^{n(a)+1}=a$. Clearly, 
\begin{align*}
aa^{n(a)}b= a^{n(a)+1}b=ab=ab^{n(b)+1}=ab^{n(b)}b
\end{align*}
This then by cancellation laws implies $a^{n(a)}=b^{n(b)}$, which can be easily checked to be the identity.  
\end{Example}
\begin{Example}{}{}
By \myref{example}{ex:THwc}, we see that the set of nonzero integer relatively prime to $n$ and modulo $n$ forms a group under multiplication modulo $n$, called the \textbf{multiplicative group of integer modulo $n$}, or equivalently the unit group of the ring $\Z_n$. 
\end{Example}



Unlike the category of monoids, the category of groups behaves much better. Given two groups $G,H$ and a function  $\phi : G\rightarrow H$, if $\phi$ respects the binary operation, then $\phi$ also respects the identity:
\begin{align*}
e_H = (\phi (x)^{-1})\phi (x) = (\phi(x)^{-1}) \phi(x e_G) =  (\phi (x)^{-1} \phi (x)) \phi (e_G)=\phi (e_G)
\end{align*}
which implies that $\phi$ must also respect inverse. In such case, we call $\phi$ a \textbf{group homomorphism}. In this note, by a \textbf{subgroup} $H$ of $G$, we mean an injective group homomorphism $H \hookrightarrow G $. Clearly, a subset of $G$ forms a subgroup if and only if it is closed under both the binary operation and inverse. Note that one of the key basic property of subgroup $H \subseteq G$ is that if $g \not \in H$, then $hg \not \in H$, since otherwise $g=h^{-1}hg \in H$. \\


Let $S$ be a subset of $G$. The group of \textbf{words} in $S$:
\begin{align*}
\set{s_1^{\epsilon_1}\cdots s_n^{\epsilon _n}\in G: n\inn \cup  \set{0}\text{ and }s_i \in S\text{ and }\epsilon_i = \pm 1}
\end{align*}
is clearly the smallest subgroup of $G$ containing $S$. We say this subgroup is \textbf{generated} by $S$. If $G$ is generated by a single element, we say $G$ is \textbf{cyclic}. Let $x \in G$. The \textbf{order} of $G$ is the cardinality of $G$, and the order of  $x$ is the cardinality of the cyclic subgroup $\langle x\rangle \subseteq G$, or equivalently the infimum of the set of natural numbers $n$ that makes $x^n=e$. Clearly, finite cyclic groups of order $n$ are all isomorphic to  $\Z_n$.\\ 


Let $G$ be a group and $H$ a subgroup of $G$. The \textbf{right cosets} $Hx$ are defined by $Hx\triangleq \set{hx \in G: h \in H}$. Clearly, when we define an equivalence relation in $G$ by setting: 
\begin{align*}
x\sim  y \overset{\triangle}{\iff } xy^{-1} \in H
\end{align*}
the equivalence class $[x]$ coincides with the right coset $Hx$. Note that if we partition $G$ using \textbf{left cosets}, the equivalence relation being $x\sim  y \iff  x^{-1}y\in H$, then the two partitions need not to be identical. 
\begin{example}
Let $H\triangleq \set{e,(1,2)}\subseteq S_3$. The right cosets are 
\begin{align*}
H(2,3)=\set{(2,3),(1,2,3)}\quad \text{ and }\quad  H(1,3)=\set{(1,3),(1,3,2)}
\end{align*}
while the left cosets being
\begin{align*}
(2,3)H= \set{(2,3),(1,3,2)} \quad \text{ and }\quad (1,3)H= \set{(1,3),(1,2,3)}
\end{align*}
\qed
\end{example}
However, as one may verify, we have a well-defined bijection $xH\mapsto Hx^{-1}$ between the sets of left cosets and right cosets of $H$. Therefore, we may define the \textbf{index} $[G:H]$ of $H$ in  $G$ to be the cardinality of the collection of left cosets of $H$, without falling into the discussion of left and right. Moreover, let $K$ be a subgroup of  $H$, by axiom of choice, clearly we have: 
\begin{align*}
[G:K]= [G:H] \cdot [H:K]
\end{align*}
which gives \textbf{Lagrange's theorem}  
\begin{align*}
o(G)= [G:H] \cdot o(H)
\end{align*}
as a corollary.\\

Let $G$ be a group and $X$ a set. If we say  $G$ \textbf{acts on $X$ from left} we are defining a function $G \times X\rightarrow X$ such that 
\begin{enumerate}[label=(\roman*)]
  \item $e\cdot x=x$ for all $x\in X$. 
  \item $(gh)\cdot x= g \cdot (h \cdot x)$ for all $g,h \in G$. 
\end{enumerate}
Note that there is a difference between left action and right action, as $gh$ means $g \circ h$ in left action and means $h \circ g$ in right action.






Because groups admit inverses, a $G$-action is in fact a group homomorphism $G \rightarrow \operatorname{Sym}(X)$. The trivial action then correspond to the trivial group homomorphism.  An action is \textbf{faithful} if it is injective. \\


Show that $Z(G)\subseteq \operatorname{Ker} \theta$ if and only if $\theta$ is faithful.  \\


An action is \textbf{free} if $g\cdot x=x$ for a $x\in X$ implies $g=e$. Note that the isomorphism $\operatorname{Sym}(X)\rightarrow \operatorname{Sym}(X)$ is always injective but never free unless $\abso{X}\leq 2$. The action is \textbf{transitive} if for any $x,y \in X$, there always exists some $g \in G$ such that $y= g \cdot x$. An action is \textbf{regular} if it is both free and transitive.\\

Let $x \in X$. We call the set $G\cdot x\triangleq \set{g\cdot x \in X: g\in G }$ the \textbf{orbit} of $x$. Clearly the set $G_x$ of all elements of $G$ that fixes $x$ forms a group, called the \textbf{stabilizer subgroup} of $G$ with respect to $x$. Consider the action left. The fact that the obvious mapping between the set of left cosets of stabilizer subgroups of $G$ with respect to $x$ to the orbit of  $x$: 
\begin{align*}
\set{g G_x \subseteq G : g \in G} \longleftrightarrow G \cdot x
\end{align*}
forms a bijection is called the \textbf{orbit-stabilizer theorem}, which relates the index of the stabilizer subgroup of $x$ and the orbit of $x$:   \label{THost}
\begin{align*}
[G:G_x]=\abso{G \cdot x}
\end{align*}
\begin{example}
Let $H$ be a subgroup of  $G$, and let $H$ acts on  $G$ by right multiplication. Then the orbit of $x\in G$ is just the left coset $xH$, while the stabilizer subgroup $H_x$ is trivial, agreeing with  \customref{THost}{orbit-stabilizer theorem}.  
\end{example}






\begin{theorem}
\textbf{(Cauchy's theorem for finite group)}  Let $p\mid o(G)$. Then the number of elements of order divided by $p$ is a positive multiple of  $p$.  
\end{theorem}
\begin{proof}
The set $X$ of $p$-tuples  $(x_1,\dots ,x_p)$ that satisfies $x_1\cdots x_p=e$ clearly has cardinality $o(G)^{p-1}$. Consider the group action $\Z_p \rightarrow \operatorname{Sym}(X)$ defined by 
\begin{align*}
g \cdot (x_1,\dots ,x_p) \triangleq (x_p,x_1,\dots ,x_{p-1}),\quad \text{ where }\Z_p = \langle g\rangle 
\end{align*}
Notice that $x^p=e$ if and only if  $(x,\dots ,x) \in X$. Therefore the number of cardinality $1$ orbit equals to number of solution to $x^p=e$. By \customref{THost}{orbit-stabilizer theorem}, an orbit in $X$ either has cardinality $p$ or  $1$. Therefore, we may write
\begin{align*}
p\mid  o(G)^{p-1}=m+kp
\end{align*}
with $m$ the number of cardinality $1$ orbits and $k$ the number of cardinality  $p$ orbits. Clearly we have $p\mid m$, as desired.
\end{proof}


\section{Normalizer and centralizer}
Because the inverse of an injective group homomorphism forms a group homomorphism, we know the set $\operatorname{Aut}(G)$ of automorphisms of $G$ forms a group. We say $\pfi \in \operatorname{Aut}(G)$ is an \textbf{inner automorphism} if  $\pfi $ takes the form $x\mapsto  gxg^{-1}$ for some fixed $g \in G$. We say two elements  $x,y \in G$ are \textbf{conjugated} if there exists some inner automorphism that maps $x$ to $y$. Clearly conjugacy forms a equivalence relation. We call its classes \textbf{conjugacy classes}. 


\begin{equiv_def}
\textbf{(Normalize)}  
\end{equiv_def}

From the point of view of inner automorphism, we see that it is well-defined whether an element $g\in G$ \textbf{normalize} a subset $S\subseteq G$:
\begin{align*}
\set{gsg^{-1} \in G: s \in S} = S
\end{align*}
independent of left and right.  Because of the independence, For each subset $S\subseteq G$, we see that the set of elements $g\in G$ that normalize $S$ forms a group, called the \textbf{normalizer} of $S$. Note that if $g$ normalize  $S$, then $gS = Sg$.   \\







\begin{example}
  Consider $G\triangleq \operatorname{GL}_2(\R)$  and consider: 
\begin{align*}
H\triangleq \set{ \begin{pmatrix}
    1 & n \\
    0 & 1
\end{pmatrix} \in \operatorname{GL}_2(\R) : n \inz}  \quad \text{ and }\quad g\triangleq \begin{pmatrix} 
 2 & 0 \\
 0 & 1
\end{pmatrix} \in \operatorname{GL}_2(\R)
\end{align*}
Note that $gHg^{-1}\subset H $. In other words, inner automorphisms can maps a subgroup $H$ into a subgroup strictly contained by $H$ if  $G$ is infinite. 
\end{example}


\begin{equiv_def}
\textbf{(Normal subgroups)} Let $G$ be a group and $N$ a subgroup. We say $N$ is a \textbf{normal subgroup} of $G$ if any of the followings hold true: 
\begin{enumerate}[label=(\roman*)]
  \item $\pfi  (N) \subseteq N$  for all $\pfi  \in \operatorname{Inn}(G)$
  \item $\pfi  (N)=N$ for all $\pfi  \in \operatorname{Inn}(G)$ 
  \item $xN=Nx$ for all  $x \in G$.  
  \item The set of all left cosets of $N$ equals the set of all right cosets of $N$. 
  \item $N$ is a union of conjugacy classes. 
  \item For all $n \in N$ and $x \in G$, their \textbf{commutator} $nxn^{-1}x^{-1} \in G$ lies in $N$.  
  \item For all $x,y \in G$, we have $xy\in N \iff  yx \in N$. 
\end{enumerate}
\end{equiv_def}
\begin{proof}
  (i)$\implies $(ii): Let $\pfi  \in \operatorname{Inn}(G)$. By premise, $\pfi  (N)\subseteq N$ and $\pfi ^{-1}(N)\subseteq N$. Applying  $\pfi $ to both side of $\pfi ^{-1}(N)\subseteq N$, we have $\pfi (N)\subseteq N \subseteq \pfi  (N)$, as desired. \\

(ii)$\implies $(iii): Consider the automorphisms:  
\begin{align*}
 \pfi_{L,x}(g)= xg\quad \text{ and }\quad \pfi _{L,x^{-1}}(g)=x^{-1}g \quad \text{ and }\quad \pfi _{R,x}(g)= gx
\end{align*}
 Because $\pfi _{L,x^{-1}}\circ \pfi _{R,x} \in \operatorname{Inn}(G)$, by premise we have:
\begin{align*}
xN= \pfi _{L,x}(N)= \pfi _{L,x}\circ \pfi _{L,x^{-1}}\circ \pfi _{R,x}(N)= \pfi _{R,x}(N)=Nx 
\end{align*}

(iii)$\implies $(iv) is clear. (iv)$\implies $(iii): Let $x\in G$. By premise, there exists some $y\in G$ that makes $xN=Ny$. Let $x=ny$. The proof then follows from noting 
\begin{align*}
xN=Ny=N(n^{-1}x)=Nx
\end{align*}
(iii)$\implies $(v): Let $n \in N$ and $x \in G$. We are required to show $xnx^{-1} \in N$. Because $xN=NX$, we know  $xn=\tilde{n}x$ for some $\tilde{n}\in N$. This implies 
\begin{align*}
xnx^{-1}= \tilde{n}xx^{-1}=\tilde{n}\in N  
\end{align*}
(v)$\implies $(vi): Fix $n\in N$ and $x\in G$. By premise, $xn^{-1}x^{-1} \in N$. Therefore, $n(xn^{-1}x^{-1})\in N$, as desired.\\


(vi)$\implies $(vii): Let $xy \in N$. To see $yx$ also belong to $N$, observe: 
\begin{align*}
 (xy)^{-1}(yx) =(xy)^{-1}x^{-1}xyx=[xy,x] \in N
\end{align*}
(viii)$\implies $(i): Let $n \in N$ and $x\in G$. Because $(nx)x^{-1}=n \in N$, by premise we have $x^{-1}nx \in N$, as desired.
\end{proof}
\begin{equiv_def}
\textbf{(Normal closure)} Let $G$ be a group and $S\subseteq G$. The \textbf{normal closure} $\operatorname{ncl}_G(S)$ of $S$ in  $G$  refer to any one of the followings: 
\begin{enumerate}[label=(\roman*)]
  \item The smallest normal subgroup of $G$ containing $S$, which we know exists as the intersection of all normal subgroups of $G$ containing $S$.  
  \item The subgroup of $G$ generated by 
\begin{align*}
 \bigcup_{\pfi \in \operatorname{Inn}(G)} \set{\pfi (x) \in G:x \in S} 
\end{align*}   
\end{enumerate}
\end{equiv_def}
\begin{proof}
We are required to prove the subgroup of $G$ from (ii) is normal. Clearly, it is the set:
\begin{align*}
\set{g_1^{-1}x_1^{\epsilon _1}g_1 \cdots g_n^{-1}x_n^{\epsilon_n}g_n \in G: n\geq 0,x_i \in S,\epsilon_i =\pm 1, g_i \in G }
\end{align*}
Fix $g \in G$. The proof then follows from noting
\begin{align*}
g^{-1}\left(g_1^{-1}x_1^{\epsilon _1}g_1 \cdots g_n^{-1}x_n^{\epsilon_n}g_n\right)g= \left( \left(g_1g\right)^{-1}x_1 ^{\epsilon _1} \left(g _1g  \right) \right) \cdots   \left( \left(g _ng\right)^{-1}x_n ^{\epsilon _n} \left(g_ng  \right) \right) 
\end{align*}
\end{proof}

We denote the \textbf{centralizer} $C_G(S)\triangleq \set{g \in G: gsg^{-1}=s\text{ for all }s \in S}$. We call the centralizer of the whole group $Z(G)\triangleq C_G(G)$ \textbf{center}. Clearly $Z(G)$ forms an abelian subgroup of $G$, and every element of the center form a single conjugacy classes.     \\

For finite group $G$, we have the \textbf{class equation}
\begin{align*}
 \abso{G} = \abso{Z(G)}+ \sum \abso{G: C_G(x)}
\end{align*}
\label{THce}
where $x$ runs through conjugacy classes outside of  $Z(G)$. \\





Clearly $C_G(S)\subseteq N_G(S)$. 

\section{Isomorphism theorems}
 Let $G$ be a group and $N\subseteq  G$ a normal subgroup. We say a group homomorphism $ \pi :G \rightarrow   G\quotient N$ satisfies the \textbf{universal property of quotient group $G\quotient N$} if 
\begin{enumerate}[label=(\roman*)]
  \item it vanishes on $N$. \textbf{(Group condition)}
  \item for all group homomorphism $f:G\rightarrow H$ that vanishes on $N$ there exist a unique group homomorphism $\tilde{f} :G\quotient N\rightarrow H$ that makes the diagram: 
% https://q.uiver.app/#q=WzAsMyxbMiwwLCJHXFxxdW90aWVudCBOICJdLFswLDAsIkciXSxbMiwyLCJMIl0sWzAsMiwiZyJdLFsxLDIsImYiLDJdLFsxLDAsIiIsMCx7InN0eWxlIjp7ImhlYWQiOnsibmFtZSI6ImVwaSJ9fX1dXQ==
\[\begin{tikzcd}
	G && {G\quotient N } \\
	\\
	&& H 
	\arrow["\pi",two heads, from=1-1, to=1-3]
	\arrow["f"', from=1-1, to=3-3]
	\arrow["\tilde{f} ", from=1-3, to=3-3]
\end{tikzcd}\]
commute. \textbf{(Universality)}
\end{enumerate}
\begin{theorem}
\textbf{(The first isomorphism theorem for groups)} The group homomorphism $\pi : G \rightarrow G \quotient N$ is always surjective with kernel $N$. Let $f :G \rightarrow H$ be a group homomorphism. Then $\operatorname{ker}f$ is normal in $G$, and the induced homomorphism $\tilde{f}: G \quotient \operatorname{ker}f \rightarrow H$ is injective.   
\end{theorem}
\begin{proof}
The first part is an immediate consequence of construction of $G \quotient N$. However, it should be noted that such construction can be avoided. The fact that $\operatorname{ker}(\pi )=N$ can be proved by considering the permutation representation $G \rightarrow \operatorname{Sym}(\Omega)$, where $\Omega$ is the set of the cosets of $N$, and the fact that  $\pi $ is surjective is a consequence of $\tilde{\pi }=\id_{G\quotient N}$. \\

We clearly have $\operatorname{ker}f \trianglelefteq G$. The fact that $\tilde{f}:G \quotient \operatorname{ker}f \rightarrow H$ is injective follows from $\pi : G \rightarrow G \quotient \operatorname{ker}f$ being surjective with kernel $\operatorname{ker}f$. 
\end{proof}



Because the kernel of a group homomorphism is clearly normal, if $N$ is not normal, then there can not be a pair $G\rightarrow G\quotient N$ that satisfies the universal property.  If any things, this is the "reason" why normal subgroups are what meant to be quotiented in the category of group.  \\

Given $x,y \in G$, we often write  
\begin{align*}
[x,y]\triangleq xyx^{-1}y^{-1}\quad \text{ or }\quad [x,y]\triangleq x^{-1}y^{-1}xy
\end{align*}
and call $[x,y]$ the \textbf{commutator} of $x$ and  $y$. Independent of differences of the definition, we have $[x,y]\in N$ if and only if $xyN=yxN$. Again, independent of the definition, the \textbf{commutator subgroup} $[G,G]$ of $G$ is the subgroup generated by the commutators. It should be noted that given a normal subgroup $N$ of $G$, the quotient group  $G \quotient N$ is abelian if and only if $N$ contains the commutator subgroup of  $G$. 


\begin{example}
$G\triangleq S_3$. $S\triangleq  \langle (1,2)\rangle $ and $H\triangleq \langle (2,3)\rangle $. $SH$ doesn't form a group. $(2,3)(1,2)\not \in SH$. 
\end{example}
\begin{theorem}
\label{THsitg}
\textbf{(Second isomorphism theorem)} Let $H\leq G$. If $K$ is a subgroup of normalizer of $H$, then their product: 
\begin{align*}
HK\triangleq  \set{hk \in G: h \in H\text{ and } k \in K}
\end{align*}
forms a group (in fact, the subgroup generated by $H\cup K$) and is defined independent of left and right. Moreover, $H\trianglelefteq HK$ with $hkH=Hk$, and $H\cap K \trianglelefteq K$ with 
\begin{align*}
HK \quotient H \cong  K \quotient H \cap K \quad \text{ via } \quad  kH  \longleftrightarrow k (H \cap K) 
\end{align*}
\end{theorem}
\begin{proof}

\end{proof}
Third isomorphism theorem. 

Correspondence theorem.


Because $ \phi\circ \pfi _g \circ \phi^{-1} = \pfi _{\phi(g)}$, we know $\operatorname{Inn}(G)$ forms a normal subgroup of $\operatorname{Aut}(G)$. \\

\section{Sylow theorems}
\label{SECSt}
\begin{theorem}
\label{THfSt}
\textbf{(First and Third Sylow theorem, Wielandt's proofs)} Let $G$ be a finite group of order $p^mt$ with $\operatorname{gcd}(p,t)=1$. Let $r\leq m$. Then the number $n_p$ of $p$-subgroup with order $p^r$ satisfies 
\begin{align*}
n_p \equiv 1 \pmod{p}
\end{align*}
\end{theorem}
\begin{proof}
Let $X$ be the set of subset of  $G$ with cardinality $p^r$. Our goal is to find all elements of $X$ that forms a group. Clearly we may define a left $G$-action on  $X$ be setting 
 \begin{align*}
g \cdot \set{x_1,\dots ,x_{p^r}} \triangleq \set{gx_1,\dots ,gx_{p^r}}
\end{align*}
Let $\Gamma $ be an orbit. If $\Gamma $ contains a group, then we see that $\Gamma $ is the left coset space of that group,  containing exactly one group and satisfying $\abso{\Gamma }=p^{m-r}t$. If $\Gamma $ doesn't contain any group, there still exists some $S \in \Gamma $ such that $e \in S$, and clearly we will have $\operatorname{Stab}(S)\subseteq S$. Because $S$ isn't a group, we see  $p^r=\abso{S}>o(\operatorname{Stab}(S))$, which by \customref{THost}{orbit-stabilizer theorem} implies that $\abso{\Gamma }= [G: \operatorname{Stab}(S)]=p^{m-r+c}t$ for some $c\geq 1$. \\

In summary, by counting orbit, we have shown that: 
\begin{align*}
\binom{p^mt}{p^r} = \abso{X} = n_pp^{m-r}t + l p^{m-r+1}t ,\quad \text{ for some }l \inn
\end{align*}
Let $ut \equiv 1 \pmod{p}$. Recalling that $\binom{p^mt}{p^r}$ has  $p$-power  $p^{m-r}$, it remains to show 
\begin{align*}
u \cdot \frac{\binom{p^mt}{p^r}}{p^{m-r}}\equiv 1 \pmod{p}
\end{align*}
which follows from noting: 
\begin{align*}
u \cdot \frac{\binom{p^mt}{p^r}}{p^{m-r}}= ut \cdot \binom{p^mt-1}{p^r-1}  \equiv\binom{p^{m}t-1}{p^{r}-1}\equiv 1 \pmod{p}
\end{align*}
where the last equality follows from Lucas modulo binomial formula.    
\end{proof}
\begin{theorem}
\label{THcl}
\textbf{(Counting lemma for $p$-group)} Let $H$ be a  $p$-group acting on a finite set $\Omega$. Let $\Omega_0$ be the set of fixed points. Then 
\begin{align*}
\abso{\Omega} \equiv \abso{\Omega_0} \pmod{p}
\end{align*}
\end{theorem}
\begin{proof}
This is a consequence of \customref{THost}{orbit-stabilizer theorem}. 
\end{proof}
\begin{theorem}
\label{THsSt}
\textbf{(Second Sylow theorem)} Sylow $p$-subgroups are conjugated to each other.    
\end{theorem}
\begin{proof}
Let $H$ and $P$ be two Sylow $p$-subgroups of $G$, and let $H$ acts on left coset space of  $P$ by left multiplication. Because $P$ is Sylow, by \customref{THcl}{counting lemma for $p$-group}, we know the number of fixed points $gP$ is nonzero. Let $gP$ be a fixed point. We then see that, as desired, $g^{-1}hg  \in P$ for all $h \in H$, since  $hgP=gP$. 
\end{proof}
\begin{theorem}
\label{THtSt}
\textbf{(Remaining part of third  Sylow theorem)} Let $G$ be a finite group, and let $n_p$ be the number of Sylow  $p$-subgroup of  $G$. For all Sylow  $p$-subgroup $P$ of  $G$, we have 
 \begin{align*}
n_p =  [G:N(P)]
\end{align*}
\end{theorem}
\begin{proof}
This is a consequence of \customref{THsSt}{second Sylow theorem} and \customref{THost}{orbit stabilizer theorem}, where we note that when $G$ acts on  $\operatorname{Syl}_p(G)$ by conjugation we have $\operatorname{Stab}(P)=N(P)$.  
\end{proof}
\begin{example}
  Let $o(G)=pq$ with $p>q$ being prime. Because \customref{THfSt}{$n_p \equiv 1 \pmod{p}$} and \customref{THtSt}{$n_p \mid  o(G)=pq$}, we see $n_p=1$. \\

If  

  If $G$ is non-abelian, then we must have $q \mid  p-1$, since otherwise  
\end{example}
\section{Finitely generated abelian group}
\begin{equiv_def}
\label{EDidpfg}
  \textbf{(Internal direct products for groups)} Let $G$ be a group with normal subgroups  $N_1,\dots ,N_k$. We say $G$ is an \textbf{internal direct  products of $N_i$} if any of the followings hold true: 
\begin{enumerate}[label=(\roman*)]
  \item The natural map $N_1 \times \cdots \times N_k \rightarrow G$ forms a group isomorphism. 
  \item $N_1\cdots N_k=G$ and $N_i \cap \prod_{j\neq i}N_j=\set{e}$ for all $i$. 
  \item  Every $g \in G$ can be written uniquely as $\prod n_i$. 
\end{enumerate}
\end{equiv_def}
\begin{proof}
  (i)$\implies $(ii): Clearly we have $N_1 \cdots N_k = G$. Let $n_2 \cdots n_k \in N_1$. Because $n_2 \cdots n_k$ is both the image of $(n_2\cdots n_k,e,\dots ,e)$ and $(e,n_2,\dots ,n_k)$, by injectivity of the natural map, we know $n_2=\cdots =n_k =e$.  \\

 (ii)$\implies $(iii): The existence is clear. To see the uniqueness, observe that $\prod n_i = \prod \tilde{n}_i$ implies $(\tilde{n}_1)^{-1}n_1n_2 \cdots n_k=\tilde{n}_2 \cdots \tilde{n}_k  $

\end{proof}
\begin{example}
Let $G\triangleq \Z_4 \times \Z_2$. Clearly the direct product of $\langle (1,0)\rangle $ and $\langle (2,0)\rangle $ is isomorphic to $G$, but they do not form an internal direct product of  $G$. It is because of such, we must require  $N_1 \times \cdots \times N_k$ not only isomorphic to $G$, but moreover the natural way in \customref{EDidpfg}{definition of internal direct products for groups}.
\end{example}
\begin{theorem}
\textbf{(Fundamental theorem for finite abelian group)}
\end{theorem}
\begin{theorem}
\textbf{(Fundamental theorem for finitely generated abelian group)}
\end{theorem}
\section{Nilpotency and Solvability}
More than often, we care about the existence of \textbf{central series}
\begin{align*}
1 \trianglelefteq  \cdots \trianglelefteq  A_{n-1}\trianglelefteq   A_n \trianglelefteq  A_{n+1} \trianglelefteq    \cdots \trianglelefteq  G
\end{align*}
where we requires the successive quotient to be \textbf{central}, i.e., $[G,A_{n+1}]\leq A_n$, or equivalently, $A_{n+1} \quotient A_n \leq Z(G \quotient A_n)$, or equivalently, $xyA_n=yx A_n \in G \quotient A_n$ if one of $x,y$ is in  $A_{n+1}$. \\



To construct one, one can consider the \textbf{upper central series}, defining $G_{(n)}\triangleq [G,G_{(n-1)}]$ with $G_{(0)}\triangleq G$. This gives us 
\begin{align*}
\cdots  \trianglelefteq  G_{(2)}\trianglelefteq  G_{(1)} \trianglelefteq G
\end{align*}
Note that 

\begin{theorem}
\textbf{(Every subgroup of a nilpotent group is subnormal)} Let $G$ be a nilpotent group with $H \leq G$. Then $H$ is a subnormal subgroup of  $G$.   
\end{theorem}
\begin{proof}
Let 
\begin{align*}
1 \triangleleft  A_1 \triangleleft \cdots \triangleleft A_n = G
\end{align*}
\end{proof}
\begin{corollary}
\textbf{(Nilpotent group satisfies normalizer condition)} Let $G$ be a nilpotent group. If $H<G$, then $H <N_G(H)$. 
\end{corollary}



A group is said to be nilpotent if it admits a central series. 
\begin{equiv_def}
\textbf{(Nilpotency for finite group)} Let $G$ be a finite group. The followings are equivalent: 
\begin{enumerate}[label=(\roman*)]
  \item $G$ is nilpotent. 
  \item Sylow subgroups of $G$ are all normal. 
  \item Sylow subgroups of $G$ are all normal and they form an inner direct product equal to $G$. 
  \item    
\end{enumerate}
\end{equiv_def}
\begin{proof}

\end{proof}

\section{Remarkable Exercises}
\begin{question}{Wording problem}{}
Let $n\inz$ satisfies 
\begin{align*}
a^n=b^n\quad \text{ and }\quad a^{n-1}=b^{n-1}\quad \text{ and }\quad a^{n+1}=b^{n+1},\quad \text{ for all }a,b \in G
\end{align*}
Show that $G$ is abelian.  fkjladslkfjadsfadsfjadsklfjadslkfjdas;lkfj;lk jdflk;jasdl;kfjsldkajf;lksdajflkjasdkl;fjadsk;lfjadl;skjfakl;fj;aldkfjals;dkf
\end{question}
\begin{proof}
Deduce 
\begin{align*}
a^nb^n ab=(ab)^n (ab)=(ab)^{n+1}=a^{n+1}b^{n+1}
\end{align*}
which implies 
\begin{align*}
b^na=ab^n
\end{align*}
which implies 
\begin{align*}
  (a^{n-1}b^{n-1})ba= a^nb^n= (a^{n-1}b^{n-1})ab
\end{align*}
as desired. 
\end{proof}

\section{Exercises}
For question 1, recall that by class equation, $p$-group can not have trivial center, and recall that  $G\quotient N$ is abelian if and only if $[G,G]\leq N$. 
\begin{question}{}{}
Show that 
\begin{enumerate}[label=(\roman*)]
  \item If $H \quotient Z(H)$ is cyclic, then $H$ is abelian.  
  \item If $H$ is of order  $p^2$, then  $H$ is abelian.   
\end{enumerate}
From now on, suppose $G$ is non-abelian with order  $p^3$. 
\begin{enumerate}[label=(\roman*), start=3]
  \item $\abso{Z(G)}=p$. 
  \item $Z(G)=[G,G]$.  
\end{enumerate}
\end{question}
\begin{proof}
Let $a,b\in H$ and $H \quotient Z(H)= \langle hZ\rangle $. Write $a=h^nz_1$ and $b=h^mz_2$. Because  $z_1,z_2 \in Z(H)$, we may compute: 
\begin{align*}
ab=h^nz_1h^mz_2=h^{n+m}z_1z_2=ba
\end{align*}
as desired. \\

Let $\abso{H}=p^2$. Because $H$ is a  $p$-group, we know $Z(H)$ is nontrivial, therefore either $\abso{Z(H)}=p$ or $\abso{Z(H)}=p^2$. To see the former is impossible, just observe that if so, then $\abso{H\quotient Z(H)}=p$, which implies $H\quotient Z(H)$ is cyclic, which by part (i) implies $Z(H)=H$. \\

Because $G$ is non-abelian, we know  $\abso{Z(G)}\neq p^3$. Because $G$ is a  $p$-group, we know  $\abso{Z(G)}\neq 1$. Therefore, either $\abso{Z(G)}=p$ or $\abso{Z(G)}=p^2$. Part (i) tell us that $\abso{Z(G)}\neq p^2$, otherwise $G$ is abelian, a contradiction. We have shown  $\abso{Z(G)}=p$, as desired. \\

We now prove $Z(G)=[G,G]$. Because $\abso{Z(G)}=p$, by part (ii) we know $G\quotient Z(G)$ is abelian. This implies $[G,G] \leq  Z(G)$, which implies $[G,G]$ is either trivial or equal to $Z(G)$. Because $G$ is non-abelian, we know  $[G,G]$ can not be trivial. This implies $Z(G)=[G,G]$, as desired.
\end{proof}
\begin{question}{}{}
\begin{enumerate}[label=(\roman*)]
  \item Let $M,N$ be two normal subgroups of  $G$ with  $MN=G$. Prove that 
\begin{align*}
    G\quotient (M\cap N) \cong  (G\quotient M) \times (G\quotient N)
\end{align*} 
\item Let $H,K$ be two distinct subgroups of  $G$ of index  $2$. Prove that  $H\cap K$ is a normal subgroup with index $4$ and  $G \quotient (H\cap K)$ is not cyclic.  
\end{enumerate}
\end{question}
\begin{proof}
The map $G\quotient (M\cap N) \rightarrow (G \quotient M)\times (G\quotient N)$ defined by 
\begin{align}
\label{EQmg}
g(M\cap N) \mapsto (gM,gN)
\end{align}
is clearly a well-defined group homomorphism, since if $gM=hM$ and  $gN=hN$, then  $gh^{-1}\in M$ and $gh^{-1} \in N$, which implies $gh^{-1}\in M \cap N$, which implies $g(M\cap N)=h(M \cap N)$. Let $gM=M$ and  $gN=N$. Then  $g \in M \cap N$ and $g(M \cap N)=M \cap N$. Therefore \myref{map}{EQmg} is also injective. It remains to show \myref{map}{EQmg} is surjective. Fix $g,h \in G$. Write $g=mn$ and  $h=\tilde{m}\tilde{n}$. Clearly $gM=nM=\tilde{m}nM$ and  $hN=\tilde{m}N=\tilde{m}nN $. This implies that  \myref{mapping}{EQmg} maps $\tilde{m}n$ to $(gM,hN)$, as desired.\\

Because $H,K$ are both of index  $2$ in $G$, we know they are both normal in $G$. This by second isomorphism theorem implies  $HK$ forms a subgroup of  $G$. Because $H\neq K$, we know $HK$ properly contains $H$, which by finiteness of $G$ implies  the index of $HK$ is strictly less than  $H$, i.e., $HK=G$. Note that $H \cap K$ is normal since it is the intersection of normal subgroups. By part (i), we now have $G \quotient (H\cap K)\cong  (G \quotient H)\times (G\quotient K)\cong  \Z_2\times\Z_2$, which shows that $H \cap K$ has index $4$ and  $G \quotient (H\cap K)$ is cyclic.     
\end{proof}
\begin{question}{}{}
Let $G$ be a group of order  $pq$, where  $p>q$ are prime. 
\begin{enumerate}[label=(\roman*)]
  \item Show that there exists a unique subgroup of order $p$.  
  \item Suppose $a \in G$ with $o(a)=p$. Show that $\langle a\rangle \subseteq G$ is normal and for all $x \in G$, we have $x^{-1}ax =a^i$ for some $0<i<p$. 
\end{enumerate}
\end{question}
\begin{proof}
The third Sylow theorem stated that the number $n_p$ of Sylow $p$-subgroups satisfies 
\begin{align*}
n_p\equiv 1\pmod{p} \quad \text{ and }\quad n_p \mid q
\end{align*}
Because $p>q$, together they implies $n_p=1$. Since Sylow $p$-subgroups of  $G$ are exactly subgroups of order  $p$, we have proved (i).  \\

The third Sylow theorem also stated that $n_p= \abso{G:N_G(P)}$ for any Sylow $p$-subgroup  $P\leq G$. Therefore, $N_G(\langle a\rangle )=G$, i.e., $\langle a\rangle $ is normal in $G$. Fix  $x\in G$. It remains to prove $xax^{-1}\neq e$, which is a consequence of the fact that conjugacy (automorphism) preserves order.
\end{proof}
\begin{question}{}{}
Let $H,K$ be two subgroups of $G$ of coprime finite indices  $m,n$. Show that 
\begin{align*}
\operatorname{lcm}(m,n) \leq \abso{G:H \cap K} \leq mn
\end{align*}
\end{question}
\begin{proof}
Let $\Omega_{H\cap K},\Omega_H$, and $\Omega_K$ respectively denote the set of left cosets of $H\cap K,H$, and $K$.  The map $\Omega_{H\cap K} \rightarrow \Omega_H \times \Omega_K$ defined by 
\begin{align}
\label{EQmgh}
g(H\cap K) \mapsto (gH,gK) 
\end{align}
is well defined since 
\begin{align*}
g(H\cap K)=l(H\cap K) \implies g^{-1}l \in H \cap K \implies gH=lH\text{ and }gK=lK
\end{align*}

such set map is injective since if $gH=lH$ and  $gK=lK$, then  $g^{-1}l \in H$ and $g^{-1}l\in K$, which implies $g(H \cap K)=l(H \cap K)$, as desired. From the injectivity of \myref{map}{EQmgh}, we have shown  index of  $H\cap K$ indeed have upper bound $mn$. \\


Because 
\begin{align*}
\abso{G:H\cap K}=\abso{G:H}\cdot \abso{H:H\cap K}=\abso{G:K} \cdot \abso{K: H \cap K} 
\end{align*}
we know both  $n$ and  $m$ divides  $\abso{G:H \cap K}$, which gives the desired lower bound $\operatorname{lcm}(m,n)$. 
\end{proof}
\begin{question}{}{}
\begin{enumerate}[label=(\roman*)]
  \item Let $G$ be a group, $H\leq G$, and $x\in G$ of finite order. Prove that if $k$ is the smallest natural number that makes $x^k \in H$, then $k\mid o(x)$. 
  \item Let $G$ be a group and $N$ a normal subgroup of  $G$. Prove that  
\begin{align*}
o(gN)= \inf \set{k\inn: g^k \in N},\quad \text{ where }\inf \varnothing = \infty
\end{align*}
\item Let $G$ be a finite group, $H,N$ two subgroups of $G$ with $N$ normal. Show that if  $o(H)$ and $\abso{G:N}$ are coprime, then $H \leq N$. 
\end{enumerate}
\end{question}
\begin{proof}
  (i):  Let $a=qk+r\inn$ with $0\leq r<k$. If $x^a \in H$, then  $x^r=x^a\cdot (x^{k})^{-q}\in H$, which implies $r=0$. We have shown that $k$ divides all natural numbers  $a$ that makes  $x^a \in H$, which includes $o(x)$. \\

(ii): This is a simple observation that $(gN)^k=g^kN\in N\iff g^k \in N$. \\

(iii): By second isomorphism theorem, we know $\abso{HN: N}=\abso{H: H \cap N}$ which divides both $o(H)$ and $\abso{G:N}$. This by coprimality implies $\abso{H:H \cap N}=1$, which shows that $H\leq N$.  
\end{proof}
\begin{question}{}{}
Let $G$ be a finite group with Sylow $p$-subgroup $P$ and normal subgroup  $N$. Show that $P \cap N$ forms a Sylow $p$-subgroup of  $N$, and use such to deduce $N$ have index $p^{\nu _p(o(PN))-\nu _p(o(N))}$ in $PN$.
\end{question}
\begin{proof}
By second isomorphism theorem, we have 
\begin{align*}
o(PN)\cdot o(P\cap N)= o(P) \cdot o(N)
\end{align*}
Because $P$ is Sylow with $P \subseteq PN$, we know 
\begin{align*}
\nu_p(o(PN))=\nu _p (o(P))
\end{align*}
This shows that, indeed, $P\cap N$ forms a Sylow $p$-subgroup of $N$: 
\begin{align*}
\nu_p(o(P\cap N))= \nu _p (o(N))
\end{align*}
as desired. Because $P \cap N\leq P$ and because $P$ is Sylow, we know $o(P\cap N)$ is a power of $p$. It then follows that: 
\begin{align*}
\abso{PN: N}=  \frac{o(PN)}{o(N)}= \frac{o(P)}{o(P\cap N)}= p^{\nu_p(o(P))- \nu_p(o(P\cap N))}= p^{\nu_p(o(PN))-\nu_p(o(P))} 
\end{align*}
\end{proof}
\begin{question}{}{}
Prove that if $H$ is a Hall subgroup of  $G$ and  $N\trianglelefteq G$, then $H \cap N$ is a Hall subgroup of $N$ and  $HN\quotient N$ is a Hall subgroup of $G \quotient N$. 
\end{question}
\begin{proof}
The facts that: 
\begin{enumerate}[label=(\roman*)]
  \item By second isomorphism theorem, we have $\abso{N:H\cap N}=\abso{HN:H}$, which divides $\abso{G:H}$.   
  \item $o(H\cap N)\mid  o(H)$. 
  \item $o(H)$ and $\abso{G:H}$ are coprime. 
\end{enumerate}
implies $o(H\cap N)$ and $\abso{N:H\cap N}$ is coprime, i.e., $H\cap N$ is Hall in $N$.  \\

The facts that: 
\begin{enumerate}[label=(\roman*)]
  \item $o(HN\quotient N)= \frac{o(HN)}{o(N)}= \frac{o(H)}{o(H\cap N)}$ divides $o(H)$. (second isomorphism theorem) 
  \item $ \abso{(G\quotient N):(HN \quotient N)}= \abso{G:HN}$ divides $\abso{G:H}$. 
  \item  $o(H)$ and $\abso{G:H}$ are coprime.  
\end{enumerate}
implies $o(HN\quotient N)$  and $\abso{(G\quotient N): (HN\quotient N)}$ are coprime, i.e., $HN\quotient N$ is Hall in $G \quotient N$.



\end{proof}
\section{Exercises II}
\begin{question}{subgroup of $p$-group of index  $p$ is normal}{}
Prove that if $p$ is a prime and $o(G)=p^{\alpha }$ with $\alpha \inn$, then every subgroup $H$ of index $p$ is normal.\\

Deduce that every group of order $p^2$ has a normal subgroup of order  $p$. 
\end{question}
\begin{proof}
Let $G$ acts on the left cosets spaces $\Omega$ of $H$. We have a group homomorphism  $\phi: G \rightarrow \operatorname{Sym}(\Omega)$. Clearly we have $\operatorname{ker}\phi \subseteq H$. By first isomorphism theorem, we know 
\begin{align*}
\abso{G: \operatorname{ker}\phi} = o(\operatorname{Im} \phi) \mid  \operatorname{Sym}(\Omega)
\end{align*}
Noting that $\abso{\operatorname{Sym}\Omega}=p!$, we see $\operatorname{ker}\phi $ has index $\leq p$, which when combined with the fact $\operatorname{ker}\phi \subseteq H$ shows that $H=\operatorname{ker}\phi$, as desired. \\

Suppose $\alpha =2$. By \customref{THfS}{first Sylow theorem}, there is a subgroup of $G$ of order $p$. This subgroup is normal from what we have just proved.
\end{proof}
\begin{question}{}{}
Let $G$ be a group of odd order. Prove that for any $x\neq e \in G$, we have $\operatorname{Cl}(x)\neq \operatorname{Cl}(x^{-1})$. 
\end{question}
\begin{proof}
Assume for a contradiction that $\operatorname{Cl}(x)=\operatorname{Cl}(x^{-1})$. Because $(gxg^{-1})^{-1} = gx^{-1}g^{-1}\in \operatorname{Cl}(x^{-1})=\operatorname{Cl}(x)$, the inversion is well defined on $\operatorname{Cl}(x)$, and moreover clearly bijective. Because $o(G)$ is odd, we may pair up the elements of  $\operatorname{Cl}(x)$ via inversion to see $\abso{\operatorname{Cl}(x)}$ is even. This is impossible since by \customref{THost}{orbit-stabilizer theorem},  $\abso{\operatorname{Cl}(x)}$ is the index of some subgroup of $G$ .     
\end{proof}
\begin{question}{}{}
Let $o(G)=p^n$ with $n\geq 3$ and $o(Z(G))=p$. Prove that $G$ has a conjugacy class of size  $p$. 
\end{question}
\begin{proof}
  \customref{THce}{Class equation} stated that 
\begin{align}
\label{EQoG}
  o(G)= o(Z(G))+ \sum \abso{\operatorname{Cl}(x)}
\end{align}
and the \customref{THost}{orbit stabilizer theorem} shows that $\abso{\operatorname{Cl}(x)}$ is of order powers of $p$. If they are of $p$-powers  $\geq 2$, then  we see 
\begin{align*}
0\equiv o(G) \equiv p \equiv o(Z(G))+ \sum \abso{\operatorname{Cl}(x)}\pmod{p}
\end{align*}
a contradiction. 
\end{proof}
\begin{question}{}{}
Prove that if the center of $G$ is of index $n$, then every conjugacy class has at most  $n$ elements. 
\end{question}
\begin{proof}
Let $x \in G$. Because $Z(G) \subseteq C_G(x)$, by \customref{THost}{orbit-stabilizer theorem}, we have: 
\begin{align*}
\abso{\operatorname{Cl}(a)} = \abso{G : C_G(a)} \leq \abso{G:Z(G)}=n 
\end{align*}
\end{proof}
\begin{question}{}{}
Let $H,K\subseteq G$ be two finite subgroups. Show that 
\begin{align*}
\abso{HK}= \frac{o(H)o(K)}{ o(H \cap K)}
\end{align*}
\textbf{Remark}: The hint give a rigorous proof, but I prefer a heuristic one. 
\end{question}
\begin{proof}
Consider the right coset spaces $\Omega \triangleq \set{Hx: x \in G}$, and let $K$ acts on  $\Omega$ by right multiplication. Because $Hk =H$ if and only if $k \in H$, we know the stabilizer subgroup $K_H$ is identical to  $K \cap  H$. Therefore, by \customref{THost}{orbit-stabilizer theorem}, we have 
\begin{align*}
\frac{o(K)}{o(H \cap K)}= \abso{\set{Hk: k \in K}}
\end{align*}
Define an equivalence class in $K$ by setting  $k \sim \tilde{k}\overset{\triangle}{\iff } Hk = H \tilde{k}$. Pick a representative element our of each class and collect them into a set $T$. Clearly 
\begin{align*}
\abso{T}= \abso{\set{Hk:k \in K}}
\end{align*}
and we have a natural bijection $H \times T \rightarrow HK$. This finishes the proof. 
\end{proof}
\begin{question}{}{}
Let $G$ be a non-abelian group of order $21$. Prove that  $Z(G)=1$. 
\end{question}
\begin{proof}
If $o(Z(G))=3$ or $7$, then because  $G \quotient Z(G)$ is cylic  
\end{proof}
\begin{question}{}{}
Find all finite groups which have exactly two conjugacy classes. 
\end{question}
\begin{proof}
Let $G$ be a finite group that has exactly two conjugacy classes. One of the conjugacy class is $\set{e}$. Let $a$ be an element of the other class. By class equation and \customref{THost}{orbit-stabilizer theorem}, we have
\begin{align*}
 \abso{G}-1 =  \abso{\operatorname{Cl}(a)}  \mid  o(G)
\end{align*}
This implies $\abso{G}=2$, which implies $G=\Z_2$. 
\end{proof}
\begin{question}{}{}
Let $H$ be a subgroup of  $G$ and let 
 \begin{align*}
\bigcup_{ g \in G} gHg^{=1}=G
\end{align*}
Show that $H=G$. 
\end{question}

\section{Exercises III} 
\begin{question}{}{}
Let $o(G)=60$. Show that if $G$ is simple, then $G$ must have exactly $24$ elements of order $5$ and  $20$ elements of order $3$. 
\end{question}
\begin{proof}
By \customref{SECSt}{sylow}, we have 
\begin{align*}
n_5 \equiv 1 \pmod{5}\quad \text{ and }\quad n_5 \mid 12
\end{align*}
which by simplicity of $G$ implies $n_5=6$. The same argument gives us $n_3 \in \set{4,10}$. To see $n_3 \neq 4$, just recall that \customref{THsSt}{second sylow} stated that conjugacy action  $G\longrightarrow  \operatorname{Sym}(\operatorname{Syl}_3(G))\cong  S_{n_3}$ is nontrivial, and is therefore injective by simplicity of $G$. We now see that $n_3=4$ is too small to satisfies   
\begin{align*}
o(G)=60  \mid  n_3 !
\end{align*}
\end{proof}
\begin{question}{}{}
Let $o(G)=pqr$ with $p<q<r$ prime. Prove that $G$ has a normal Sylow $r$-subgroup  $H$. 
\end{question}
\begin{proof}
By \customref{SECst}{sylow} and counting arguments we know $1 \in \set{n_p,n_q,n_r}$. Therefore, if neither of $n_p$ and  $n_q$ is $1$, we are done. Suppose   $1 \in \set{n_p,n_q}$. Either way, we get a normal subgroup $N$ such that  $o(G \quotient N) \in \set{qr,pr}$. We also get a normal  $H \quotient N \in \operatorname{Syl}_r(G \quotient N)$. This give us a characteristic $K \in \operatorname{Syl}_r(H)$, which is normal in $G$. 
\end{proof}
\begin{question}{}{}
Let $o(G)=p^3q$ with $p,q$ prime. Show that one of the followings statement is true: 
 \begin{enumerate}[label=(\roman*)]
  \item $G$ has a normal Sylow  $p$-subgroup.  
  \item $G$ has a normal Sylow $q$-subgroup.  
  \item $p=2$,  $q=3$. 
\end{enumerate}
\end{question}
\begin{proof}
Suppose (i) and (ii) are both false. Then by \customref{SECSt}{sylow} we have $n_p=q$ and $p<q$. Because $p<q$, applying \customref{SECSt}{sylow} again we have  $n_q \in \set{p^2,p^3}$. Because $n_p>1$, by counting we see that $n_q \neq p^3$. Therefore $n_q=p^2$. Then by \customref{SECSt}{sylow}, $p^2=n_q \equiv 1 \pmod{q} $, which implies $q \mid  (p-1)(p+1)$. Because $p<q$ and $q$ is prime, we now see  $q=p+1$, which can only happens if  $p=2$ and  $q=3$.   
\end{proof}
\begin{question}{}{}
Show that no group of order $30$ is simple. 
\end{question}
\begin{proof}
Consider  $n_3$ and  $n_5$. We have  $n_5 \in \set{1,6}$ and $n_3 \in \set{1,10}$. If $n_5 \neq  1 \neq n_3$, then there are $24$ elements of order  $5$ and  $20$ elements of order $3$, impossible for a group of order $30$.   
\end{proof}
\begin{question}{}{}
Let $G$ be a finite group with sylow $p$-subgroup $P$ and normal subgroup  $N$. Show that $P \cap  N$ is $p$-sylow in $N$ and that $PN \quotient N$ is $p$-sylow in  $G \quotient N$. 
\end{question}
\begin{proof}
  \customref{THsitg}{Second isomorphism theorem} implies that 
\begin{align*}
o(PN)\cdot o(P \cap N) = o(P) \cdot o(N)  
\end{align*}
Because $P$ is  $p$-sylow, we know  $o(P)$ and $o(PN)$ has the same $p$-power, which implies that $o(P\cap N)$ and $o(N)$ has the same $p$-power, as desired. Again counting the $p$-power of  $PN \quotient  N \subseteq G \quotient N$, we see $PN\quotient N$ is  $p$-sylow.  
\end{proof}
\begin{question}{}{}
Let $G$ be a finite group, $H\leq G$ a subgroup with $[G:H]=n$. Show that: 
\begin{enumerate}[label=(\roman*)]
  \item For all subgroup $K\leq G$, we have $[H:H \cap K]\leq [G:K]$. 
  \item $[H: H \cap  H^g]\leq n$ for all $g \in G$. 
  \item If $H$ is a maximal proper subgroup of  $G$ and  $H$ is abelian, show that  $H \cap H^g \trianglelefteq G $  for all $g \not \in H$. 
  \item Suppose that $G$ is simple. If  $H$ is abelian and $n$ is prime, then $H=1$.  
\end{enumerate}
\end{question}
\begin{proof}
Let $H \quotient H \cap K$  and $G \quotient K$ denote left coset spaces. (i) is a consequence of verifying that the function 
\begin{align*}
H \quotient  H \cap  K \longrightarrow G \quotient K ;\quad h(H \cap K) \mapsto  hK
\end{align*}
is well-defined and injective. (ii) is then a corollary of (i). \\

We now prove (iii). Fix $g \not \in  H$. There are two cases: Either $H=H^g$ or $H \neq H^g$. For the first case, just observe that by maximality of $H$, we will have  $N_G(H)=G$. We now claim that $H \neq H^g \implies  H \cap H^g \subseteq Z(G)$. Because $H$ is abelian, we know  $H \cap H^g \leq Z(H)$. Clearly we also have $H \cap H^g \leq Z(H^g)$. We now have $H \cap H^g \leq Z(\langle H,H^g\rangle )$, where $\langle H,H^g\rangle =G$ by maximality of $H$, as desired.   \\

We now prove (iv). Clearly the primality of $n$ forces  $H$ to be a maximal proper subgroup of  $G$. Therefore by (iii), $H \cap H^g= 1$ for all $g \not \in H$.  This by (ii) implies $n \leq o(G)\leq n^2$. Write $o(G)\triangleq nk$ so $k \in \set{1,\dots ,n}$. We wish to show $k=1$. To see  $k\neq n$, just recall that if so, then $G$ would be abelian, contradicting to its simplicity. To see $k \not\in \set{2,\dots ,n-1}$, just observe that if so, then the unique $n$-Sylow subgroup would be proper, contradicting to simplicity of $G$.   
\end{proof}
\begin{question}{}{}
Let $G$ be a finite group with $P \in \operatorname{Syl}_p(G)$. Suppose that $N$ is a normal subgroup of $G$ with  $[G:N]= o(P)>1$. Show that 
\begin{enumerate}[label=(\roman*)]
  \item $N$ is the subset of  $G$ consisting of all elements of order not divisible by $p$. 
  \item If the elements of $G - N$ all has $p$-power order, then  $P= N_G(P)$. 
\end{enumerate}
\end{question}
\begin{proof}
Because $P$ is  $p$-sylow and $[G:N]=o(P)$, we know $p \nmid o(N)$.  This implies that no element of $N$ has order divisible by $p$. Let $g \in G$ with $p \nmid o(g)$. To see that $g \in N$, just observe that because $o(gN) \mid  o(g)$ and $o(gN)$ is a power of $p$, we have $o(gN)=1$. \\

Assume for a contradiction that $P< N_G(P)$. Then there exists some nontrivial sylow  $q$-subgroup  $Q$ of  $N_G(P)$ with $q \neq p$. By definition we have $[Q,P] \leq P$. By (i), $Q\leq N$. Therefore we also have $[Q,P] \leq N$. Coprimality of orders of $N$ and  $P$ now tell us that  $[Q,P]=1$. We now see that the product of two nontrivial elements $x \in Q,y \in P$ has order divisible by $pq$, a contradiction to the premise.   
\end{proof}
\section{Exercises IV}
\begin{question}{}{}
Show that the center of products is a product of centers: 
\begin{align*}
Z(G_1) \times \cdots \times Z(G_n) = Z (G_1 \times \cdots \times G_n)
\end{align*}
Deduce that a direct product of groups is abelian if and only if each of its factor is abelian.  
\end{question}
\begin{proof}
The "$\subseteq$" is clear. To see that 
\begin{align*}
g_1 \times \cdots \times g_n \in Z(G_1 \times \cdots \times G_n) \implies  g_i \in Z(G_i)
\end{align*}
just observe that if not, then 
\begin{align*}
  [g_1 \times \cdots \times g_n , e_1 \times \cdots  \times x_i \times \cdots \times e_n]\neq e \in \prod G_j 
\end{align*}
The second part then follows from noting 
\begin{align*}
Z(G_1 \times \cdots \times G_n)= G_1 \times \cdots \times G_n \iff  Z(G_i)=G_i,\quad \text{ for all }i
\end{align*}
\end{proof}
\begin{question}{}{}
Let $G\triangleq A_1 \times \cdots \times A_n$ and $B_i \trianglelefteq A_i$ for all $i$. Prove that  $B_1 \times \cdots \times B_n \trianglelefteq G$ and that 
\begin{align*}
\frac{A_1 \times \cdots \times A_n}{B_1 \times \cdots \times B_n} = \frac{A_1}{B_1} \times \cdots \times \frac{A_n}{B_n}
\end{align*}
\end{question}
\begin{proof}
\begin{align*}
  (g_1,\dots ,g_n) (b_1 ,\dots ,b_n )(g_1 , \dots ,g_n)^{-1} = (g_1b_1g_1^{-1},\dots ,g_nb_ng_n^{-1}) \in \prod B_i
\end{align*}
The second part require us to show that 
\begin{align*}
\prod \left( \frac{A_i}{B_i}\right) \longrightarrow \frac{\prod A_i}{\prod B_i};\quad \prod \left(\frac{a_i}{B_i} \right) \mapsto \frac{\prod a_i}{\prod B_i}
\end{align*}
is a well-defined group isomorphism, which boils down to showing that it is (i) well-defined, (ii) actually a homomorphism, (iii) injective, and (iv) surjective. To see it is injective, just observe that if $\prod a_i \in \prod B_i$, then $a_i \in B_i$ for all $i$, and therefore  $\prod \frac{a_i}{B_i}=e$. The rest are clear. 
\end{proof}
\begin{question}{}{}
Let $G$ be a finite abelian group with $m \mid  o(G)$. Show that $G$ has a subgroup of order $m$. 
\end{question}
\begin{proof}
This follows from noting that if $o(a)=p^n$, then $o(a^{p^{n-d}})=p^d$. (Ans also structure theorem for finite abelian group) 
\end{proof}
\begin{question}{}{}
Show that the subgroups and quotients of a nilpotent group $G$ are also nilpotent. 
\end{question}
\begin{proof}
Let $H$ be a subgroup of  $G$, and write 
\begin{align*}
0 = G_{(n)} \trianglelefteq  \cdots \trianglelefteq G_{(1)} \trianglelefteq G_{(0)}= G,\quad \text{ with }G_{(k)} \triangleq [G,G_{(k-1)}]
\end{align*}
To see that  
\begin{align*}
0 \leq H_n \leq  \cdots \leq H_1 \leq H
\end{align*}
form a central series, where $H_k\triangleq H \cap G_{(k)}$, just observe that  
\begin{align*}
[H,H \cap  G_{(k)}] \leq H \text{ and }[H,H \cap G_{(k)}] \leq [G,G_{(k)}] \leq G_{(k-1)}
\end{align*}
together implies 
\begin{align*}
[H,H_k] \leq H \cap G_{(k-1)} = H_{k-1}
\end{align*}
Let $N$ be a normal subgroup of  $G$, and let $m \leq n$ be the largest number such that $N\leq G_{(m)}$. It is clear that 
\begin{align*}
\frac{N}{N} \leq \frac{G_{(m)}}{N} \leq  \cdots \leq  \frac{G_{(1)}}{N} \leq \frac{G}{N}
\end{align*}
form a central series. 
\end{proof}
\begin{question}{}{}
Show that if $G \quotient Z(G)$ is nilpotent, then $G$ is nilpotent. 
\end{question}
\begin{proof}
Consider the central series 
\begin{align*}
 \frac{Z(G)}{Z(G)} \trianglelefteq \frac{G_1}{Z(G)} \trianglelefteq  \cdots \trianglelefteq  \frac{G_n}{Z(G)} = \frac{G}{Z(G)}
\end{align*}
Clearly we have the central series 
\begin{align*}
0 \trianglelefteq Z(G) \trianglelefteq G_1 \trianglelefteq  \cdots \trianglelefteq  G_n = G
\end{align*}
\end{proof}
\begin{question}{}{}
Let $o(G)=pqr$ with $p<q<r$ prime. Show that  $G$ is solvable. 
\end{question}
\begin{proof}
Recall that we have a normal subgroup $M \in \operatorname{Syl}_r(G)$. Then we have a normal subgroup $\frac{H}{M}\in \operatorname{Syl}_q(G)$. Then $1\trianglelefteq M\trianglelefteq  H \trianglelefteq  G$ forms the desired series.  
\end{proof}
\begin{question}{}{}
Show that a finite group $G$ is nilpotent if and only if every $a,b \in G$ that makes $\operatorname{gcd}(o(a),o(b))=1$ also makes $ab=ba$.   
\end{question}
\begin{proof}
($\implies $): Write $G=P_1 \times \cdots \times P_n$ with $P_i$ sylow. Clearly if the orders of $(x_1,\dots ,x_n)$ and $(y_1,\dots ,y_n)$  are coprime to each other, then for all $i$, we must have either  $x_i=e$ or  $y_i=e$. This implies the commutativity. \\

$(\impliedby)$: We need to show that Sylow subgroups of $G$ are normal. Let $P_1,\dots ,P_n$ each be a Sylow subgroup of $G$ with distinct  $p$. By premise, we see that $P_k \subseteq N_G(P_1)$ for all $k\geq 2$. This then implies $G=N_G(P_1)$, as desired. 
\end{proof}
\begin{question}{}{}
Let $G=HK$ be finite and  $S \leq G$ be a $p$-subgroup that contains some $p$-Sylow subgroup $P$ of $H$ and some $p$-Sylow subgroup $Q$ of  $K$. Show that  
\begin{enumerate}[label=(\roman*)]
  \item $S$ is $p$-Sylow in  $G$.  
  \item $S= (S \cap H)(S \cap K)$
\end{enumerate}
\end{question}
\begin{proof}
Because $P \cap Q \leq H \cap K$, we know $p$-part of 
 \begin{align*}
o(G)= \frac{o(H)o(K)}{o(H\cap K)}
\end{align*}
is smaller than  
\begin{align*}
 \frac{o(P)o(Q)}{o(P \cap Q)}= \abso{PQ} \leq o(S)
\end{align*}
 which can only happen if $S$ is Sylow with $\abso{PQ}=o(S)$. By definition, $ P \leq  S \cap H \leq H$. Because $S$ is a  $p$-group, we know  $S \cap H$ is also a $p$-group. Sylowness of $P \leq H$ then forces $S \cap H = P$. Similarly, we have $S \cap K= Q$. Now, to see $S = PQ$, just recall that $\abso{PQ}=o(S)$  
\end{proof}
\begin{question}{}{}
Let $M \trianglelefteq G$ and $N \trianglelefteq G$ with $M,N$ finite and nilpotent. Prove that  $MN$ is nilpotent. 
\end{question}
\begin{proof}
The proof follows form noting that if $S \in \operatorname{Syl}_p(MN)$, then by earlier questions, $S$ is uniquely determined by $S= (M\cap S)(N\cap S)$ with $M \cap S \in \operatorname{Syl}_p(M)$ and $N\cap S \in \operatorname{Syl}_p(N)$ uniquely determined.  
\end{proof}
\begin{question}{}{}
Let $G$ be finite with  $A,B \trianglelefteq G$ and $G \quotient A,G \quotient B$ solvable. Prove that $G \quotient (A \cap B)$ is solvable.   
\end{question}
\begin{proof}
\end{proof}

\end{document}

