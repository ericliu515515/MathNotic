\documentclass{report}
%%%%%%%%%%%%%% macros.tex %%%%%%%%%%%%%%
% Place your custom macros here, if any.

%%%%%%%%%%%%%% letterfonts.tex %%%%%%%%%%%%%%
% Place your font setup here, if any.

%%%%%%%%%%%%%% preamble.tex %%%%%%%%%%%%%%
\usepackage[T1]{fontenc}
\usepackage{lmodern}
\usepackage{etoolbox}
\usepackage{pdfpages}
\usepackage{transparent}
\usepackage[utf8]{inputenc}
\usepackage[english]{babel}

% Page Setup
\usepackage[tmargin=2cm, rmargin=0.5in, lmargin=0.5in, bmargin=80pt, footskip=.2in]{geometry}

% Mathematics
\usepackage{amsmath,amsfonts,amsthm,amssymb,mathtools}
\usepackage{xfrac}
\usepackage[makeroom]{cancel}
\usepackage{enumitem}
\usepackage{nameref}
\usepackage{multicol,array}
\usepackage{tikz-cd}
\usepackage[ruled,vlined,linesnumbered]{algorithm2e}

% Colors
\usepackage[dvipsnames]{xcolor}
\definecolor{myg}{RGB}{56, 140, 70}
\definecolor{myb}{RGB}{45, 111, 177}
\definecolor{myr}{RGB}{199, 68, 64}
% Define more colors here...

% Hyperlinks
\usepackage{bookmark}
\usepackage{hyperref}
\hypersetup{
    pdftitle={Assignment},
    colorlinks=true, linkcolor=doc!90,
    bookmarksnumbered=true,
    bookmarksopen=true
}

% Figures and Graphics
\usepackage{import}
\usepackage{svg}
\newcommand{\incfig}[1]{%
    \def\svgwidth{\columnwidth}
    \import{./figures/}{#1.pdf_tex}
}

% Text-related
\usepackage{blindtext}
\usepackage{fontsize}
\changefontsize[14]{14}
\setlength{\parindent}{0pt}

% Theorems and Definitions
\usepackage{amsthm}
\renewcommand\qedsymbol{$\blacksquare$}

% Define a new theorem style
\newtheoremstyle{mytheoremstyle}% name
  {}% Space above
  {}% Space below
  {\sffamily}% Body font
  {}% Indent amount
  {\bfseries}% Theorem head font
  {.}% Punctuation after theorem head
  {.5em}% Space after theorem head
  {}% Theorem head spec (can be left empty, meaning ‘normal’)

% Apply the new theorem style to theorem-like environments
\theoremstyle{mytheoremstyle}
\newtheorem{theorem}{Theorem}[section]
\newtheorem{definition}{Definition}[section]
\newtheorem{corollary}{Corollary}[section]
\newtheorem{lemma}{Lemma}[section]
\newtheorem{axiom}{Axiom}[section]

% tcolorbox Setup
\usepackage[most,many,breakable]{tcolorbox}

% Define custom tcolorbox environments here...

%================================
% EXAMPLE BOX
%================================
\newtcbtheorem[definition]{Example}{Example}
{%
    colback = myexamplebg,
    breakable,
    colframe = myexamplefr,
    coltitle = myexampleti,
    boxrule = 1pt,
    sharp corners,
    detach title,
    before upper=\tcbtitle\par\smallskip,
    fonttitle = \bfseries,
    description font = \mdseries,
    separator sign none,
    description delimiters parenthesis,
}
{ex}

%================================
% Solution BOX
%================================
\makeatletter
\newtcolorbox{solution}{enhanced,
	breakable,
	colback=white,
	colframe=myg!80!black,
	attach boxed title to top left={yshift*=-\tcboxedtitleheight},
	title=Solution,
	boxed title size=title,
	boxed title style={%
			sharp corners,
			rounded corners=northwest,
			colback=tcbcolframe,
			boxrule=0pt,
		},
	underlay boxed title={%
			\path[fill=tcbcolframe] (title.south west)--(title.south east)
			to[out=0, in=180] ([xshift=5mm]title.east)--
			(title.center-|frame.east)
			[rounded corners=\kvtcb@arc] |-
			(frame.north) -| cycle;
		},
}
\makeatother

%================================
% Question BOX
%================================
\makeatletter
\newtcbtheorem{question}{Question}{enhanced,
	breakable,
	colback=white,
	colframe=myb!80!black,
	attach boxed title to top left={yshift*=-\tcboxedtitleheight},
	fonttitle=\bfseries,
	title={#2},
	boxed title size=title,
	boxed title style={%
			sharp corners,
			rounded corners=northwest,
			colback=tcbcolframe,
			boxrule=0pt,
		},
	underlay boxed title={%
			\path[fill=tcbcolframe] (title.south west)--(title.south east)
			to[out=0, in=180] ([xshift=5mm]title.east)--
			(title.center-|frame.east)
			[rounded corners=\kvtcb@arc] |-
			(frame.north) -| cycle;
		},
	#1
}{def}
\makeatother
\makeatletter
\newtcbtheorem{qstion}{Question}{enhanced,
    breakable,
    colback=white,
    colframe=mygr,
    attach boxed title to top left={yshift*=-\tcboxedtitleheight},
    fonttitle=\bfseries,
    title={#2},
    boxed title size=title,
    boxed title style={%
        sharp corners,
        rounded corners=northwest,
        colback=tcbcolframe,
        boxrule=0pt,
    },
    underlay boxed title={%
        \path[fill=tcbcolframe] (title.south west)--(title.south east)
        to[out=0, in=180] ([xshift=5mm]title.east)--
        (title.center-|frame.east)
        [rounded corners=\kvtcb@arc] |-
        (frame.north) -| cycle;
    },
    #1
}{def}
\makeatother

%%%%%%%%%%%%%%%%%%%%%%%%%%%%%%%%%%%%%%%%%%%
% TABLE OF CONTENTS
%%%%%%%%%%%%%%%%%%%%%%%%%%%%%%%%%%%%%%%%%%%
\usepackage{tikz}
\definecolor{doc}{RGB}{0,60,110}
\usepackage{titletoc}
\contentsmargin{0cm}
\titlecontents{chapter}[14pc]
{\addvspace{30pt}%
	\begin{tikzpicture}[remember picture, overlay]%
		\draw[fill=doc!60,draw=doc!60] (-7,-.1) rectangle (-0.9,.5);%
		\pgftext[left,x=-4.5cm,y=0.2cm]{\color{white}\Large\sc\bfseries Chapter\ \thecontentslabel};%
	\end{tikzpicture}\color{doc!60}\large\sc\bfseries}%
{}
{}
{\;\titlerule\;\large\sc\bfseries Page \thecontentspage
	\begin{tikzpicture}[remember picture, overlay]
		\draw[fill=doc!60,draw=doc!60] (2pt,0) rectangle (4,0.1pt);
	\end{tikzpicture}}%
\titlecontents{section}[3.7pc]
{\addvspace{2pt}}
{\contentslabel[\thecontentslabel]{2pc}}
{}
{\hfill\small \thecontentspage}
[]
\titlecontents*{subsection}[3.7pc]
{\addvspace{-1pt}\small}
{}
{}
{\ --- \small\thecontentspage}
[ \textbullet\ ][]

\makeatletter
\renewcommand{\tableofcontents}{
	\chapter*{%
	  \vspace*{-20\p@}%
	  \begin{tikzpicture}[remember picture, overlay]%
		  \pgftext[right,x=15cm,y=0.2cm]{\color{doc!60}\Huge\sc\bfseries \contentsname};%
		  \draw[fill=doc!60,draw=doc!60] (13,-.75) rectangle (20,1);%
		  \clip (13,-.75) rectangle (20,1);
		  \pgftext[right,x=15cm,y=0.2cm]{\color{white}\Huge\sc\bfseries \contentsname};%
	  \end{tikzpicture}}%
	\@starttoc{toc}}
\makeatother

\newcommand{\liff}{\llap{$\iff$}}
\newcommand{\rap}[1]{\rrap{\text{ (#1)}}}
\newcommand{\red}[1]{\textcolor{red}{#1}}
\newcommand{\blue}[1]{\textcolor{blue}{#1}}
\newcommand{\vi}[1]{\textcolor{violet}{#1}}
\newcommand{\teal}[1]{\textcolor{teal}{#1}}
\newcommand{\tCaC}{\text{ \CaC }}
\newcommand{\CaC}{\red{CaC} }
\newcommand{\As}[1]{Assume \red{#1}}
\newcommand{\vdone}{\vi{\text{ (done) }}}
\newcommand{\bdone}{\blue{\text{ (done) }}}
\newcommand{\tdone}{\teal{\text{ (done) }}}
\newcommand{\set}[1]{\{ #1 \}}
\newcommand{\inS}{\in S}
\newcommand{\inF}{\in\F}
\newcommand{\inE}{\in E}
\newcommand{\inA}{\in A}
\newcommand{\inB}{\in B}
\newcommand{\inC}{\in C}
\newcommand{\inU}{\in U}

\newcommand{\C}{\mathbb{C}}	
\renewcommand{\H}{\mathbb{H}}
\newcommand{\F}{\mathbb{F}}
\newcommand{\N}{\mathbb{N}}
\newcommand{\Q}{\mathbb{Q}}
\newcommand{\R}{\mathbb{R}}
\newcommand{\Z}{\mathbb{Z}}
\renewcommand{\P}{\mathbb{P}}
\renewcommand{\S}{\mathbb{S}}
\newcommand{\A}{\mathbb{A}}
\newcommand{\RP}{\R P}


\title{Notes on Commutative Algebra}
\author{Eric Liu}
\date{}
\begin{document}
\maketitle
\newpage% or \cleardoublepage
% \pdfbookmark[<level>]{<title>}{<dest>}
\pdfbookmark[section]{\contentsname}{toc}

\tableofcontents
\pagebreak
\chapter{Untitled} 
\section{Rings and Ideals}
The precise meaning of the term \textbf{ring} varies across different books, depending on the context and purpose. In this note, the multiplication of a ring is always associative and commutative, and have an identity. The additive identity is denoted by $0$. From the axioms, we can straightforwardly show that $x\cdot 0 = 0$ for all $x$. Consequently, the multiplicative and additive identities are always distinct unless the ring contained only one element, called  \textbf{zero} in this case.\\

An \textbf{ideal} of a ring $R$ is an additive subgroup $I$ such that $ar \in I$ for all $a \in I,  r \in R$, or equivalently, the kernel of some \textbf{ring homomorphism}\footnote{Ring homomorphisms are mapping between two rings that respects addition, multiplication and  multiplicative identity.}. To see the equivalency, one simply construct the \textbf{quotient ring}\footnote{Consider the equivalence relation on $R$ defined by  $x\sim  y\overset{\triangle}{\iff } x-y \in I$} $R\quotient I$, under which the quotient map $\pi: R \rightarrow R \quotient  I$ is a surjective ring homomorphism whose kernel is the ideal $I$. Remarkably, the mapping defined by 
 \begin{align*}
\operatorname{Ideal }J\text{ of $R$ that contains }I \mapsto \set{[x]\in R\quotient I: x\in J}
\end{align*}
forms a bijection between the collection of the ideals of $R$ containing  $I$ and the collection of the ideals of $R\quotient I$. This fact is commonly referred to as \textbf{The Correspondence Theorem for Rings}. \\

A \textbf{unit} is an element that has a multiplicative inverse. Under our initial requirement that rings are commutative, for a non-zero ring $R$ to be a \textbf{field}, we only need all non-zero elements of $R$ to be units, or equivalently, the only ideals of $R$ to be $\set{0}$ or $R$ itself.\\

We use the term \textbf{proper} to describe strict set inclusion. By a \textbf{maximal ideal}, we mean a proper ideal $I$ contained by no other proper ideals, or equivalently\footnote{By the Correspondence Theorem for Rings.}, a proper ideal $I$ such that $R\quotient I$ is a field.\\

A \textbf{zero-divisor} is an element $x$ that has some non-zero element $y$ such that  $xy=0$. Again, under our initial requirement that rings are commutative, for a non-zero ring $R$ to be an  \textbf{integral domain}, we only need all non-zero elements to be zero-divisors. By a \textbf{prime ideal}, we mean a proper ideal $I$ such that the product of two elements belongs to $I$ only if one of them belong to $I$, or equivalently, a proper ideal $I$ such that $R\quotient I$ is an integral domain.  \\


There are many binary operations defined for ideals.  Given two ideals $I$ and $S$, we define their \textbf{sum} $I+S$ to be the set of all  $x+y$ where  $x\in I$ and $y \in S$, and define their \textbf{product} $IS$ to be the set of all finite sums $\sum x_iy_i$ where  $x_i \in I$ and $y_i\in S$. Note that the ideal multiplications are indeed distributive over addition, and they are both associative, so it make sense to write something like $I_1+I_2+I_3$ or $I_1I_2I_3$. Obviously, the intersection of ideals is still ideal, while the union of ideals generally are not. Moreover, we define their \textbf{quotient} $(I:S)$ to be the set of elements $x$ of $R$ such that $xy \in I$ for all $y \in S$. \\ 

For all subsets $S$ of some ring $R$, we may \textbf{generate} an ideal by setting it to be the set of all finite sum  $\sum rs$ such that $r\in R$ and $s \in S$, or equivalently, the smallest ideal of $R$ containing $S$. An ideal is called \textbf{principal} and denoted by $\langle x\rangle $ if it can be generated by a single element $x$. \\

An element $x$ is called \textbf{nilpotent} if $x^n=0$ for some  $n\inn$. The set of all nilpotent elements obviously form an ideal, which we call \textbf{nilradical} and denote by $\operatorname{Nil}(R)$. Here, we give a nice description of the nilradical. 
\begin{theorem}
\textbf{(Equivalent Definition for Nilradical)} We use the term \textbf{spectrum} of $R$ and the notation  $\operatorname{spec}(R)$ to denote the set of prime ideals of $R$. We have 
 \begin{align*}
\operatorname{Nil}(R)=\bigcap \operatorname{spec}(R)
\end{align*}
\end{theorem}
\begin{proof}
$\operatorname{Nil}(R)\subseteq \bigcap \operatorname{spec}(R)$ is obvious. Suppose $x \not\in \operatorname{Nil}(R)$. Let $\Sigma$ be the set of ideals $I$ such that $x^n\not\in I$ for all $n>0$. Because unions of chains in $\Sigma$ belong to $\Sigma$, by Zorn's Lemma, there exists some maximal element $I \in \Sigma$. Because $x\not \in I$, to close out the proof, we only have to show $I$ is prime.\\

Let $yz \in I$. Assume for a contradiction that $y\not\in I$ and $z\not\in I$. By maximality of $I$, both ideal $I+ \langle y\rangle$ and ideal $I+\langle z\rangle$ do not belong to $\Sigma$. This implies $x^n \in I+ \langle y\rangle$ and $x^m \in I + \langle z\rangle $ for some $n,m>0$, which cause a contradiction to $I \in \Sigma$, since $x^{n+m} \in I + \langle yz\rangle =I$. 
\end{proof}

Let $I$ be an ideal of the ring $R$. By the term \textbf{radical} of $I$, we mean $\sqrt{I}\triangleq \set{ x \in R : x^n \in I \text{ for some } n > 0 }$, which is equivalent to  the preimage of $\operatorname{Nil}(R \quotient  I)$ under the quotient map and equivalent\footnote{This follows from the fact that the correspondence between the ideals of $R$ and the ideals of $R \quotient  I$ induces a bijection between $\operatorname{Spec}(R)$ and $\operatorname{Spec}(R \quotient  I)$.} to the intersection of all prime ideals of $R$ that contain $I$.
\section{Script 1}

I proved and gathered the propositions in my paragraphs. 
\begin{theorem}
\textbf{(Ideal Quotients are well defined)} If we define for each pair $I,S$ of ideals of $R$ their \textbf{ideal quotient} by
\begin{align*}
  (I:S)\triangleq \set{x\in R: xy \in I\text{ for all }y \in S }
\end{align*}
Then $(I:S)$ forms an ideal.  
\end{theorem}
\begin{proof}
To see $(I:S)$ is closed under addition, let $x,z \in I,y \in S,$ and observe 
\begin{align*}
  (x+z)y=xz+yz \in I
\end{align*}
To see $(I:S)$ is a multiplicative black hole, let $u \in (I:S),v \in R,s \in S$ and observe 
\begin{align*}
  (uv)s=v(us)\in I\text{ because }us \in I
\end{align*}
\end{proof}
\begin{theorem}
\textbf{(Description of annihilator)} Given some ideal $I$ of $R$, we use the notation $\operatorname{Ann}(I)$ to denote its \textbf{annihilator} $(\set{0}:I)$. We have 
\begin{align*}
\operatorname{Ann}(I)=\set{x\in R: xy=0\text{ for all }y \in I}
\end{align*}
\end{theorem}
\begin{proof}
Obvious.
\end{proof}
\begin{mdframed}
Given a principal ideal $\langle x\rangle $, we shall always denote its annihilator simply by $\operatorname{Ann}(x)$
\end{mdframed}
\begin{theorem}
\textbf{(Description of the set of zero-divisors)} If we denote  $D$ the set of zero-divisors of $R$, we have 
\begin{align*}
D = \bigcup_{x\neq 0 \in R} \operatorname{Ann}(x )
\end{align*}
\end{theorem}
\begin{proof}
If $d$ is a zero-divisor, then $d \in \operatorname{Ann}(s)$ for the $s\neq 0$ that divides $0$ with  $d$.  If $x\neq 0$ and $y \in \operatorname{Ann}(x)$, then $yx=0$.  
\end{proof}
\begin{theorem}
\textbf{(An example)} Let $R\triangleq \Z,I\triangleq \langle m\rangle$ and $S\triangleq \langle n\rangle $. We have 
\begin{align*}
  (I:S)=\langle q\rangle 
\end{align*}
Where  
\begin{align*}
q= \frac{m}{(m,n)}\text{ and }(m,n)\text{ is the highest common factor of $m$ and $n$. }
\end{align*}
\end{theorem}
\begin{proof}
To show $\langle q\rangle \subseteq (I:S)$, we only have to show $q \in (I:S)$. Let $p$ be arbitrary integer so $pn$ is an arbitrary element of  $S$. Note that 
\begin{align*}
m \mid mp \cdot \frac{n}{(m,n)} =  q (pn) \implies  q(pn) \in I
\end{align*}
Because $pn$ is an arbitrary element of  $S$, we have shown  $q\in (I:S)$. To show $(I:S)\subseteq \langle q\rangle $, let $p \in (I:S)$. Because $p \in (I:S)$, we know $pn\in I$. That is, 
\begin{align*}
m \mid  pn
\end{align*}
Dividing both side with $(m,n)$, we see  
\begin{align*}
q \mid p \cdot \frac{n}{(m,n)} 
\end{align*}
Because $q= \frac{m}{(m,n)}$ is by definition coprime with $\frac{n}{(m,n)}$, we can now deduce 
\begin{align*}
q \mid p 
\end{align*}
as desired.
\end{proof}
\begin{question}{}{}
Let $I,S,T,V_\alpha $ be ideals of ring $R$. Show 
\begin{enumerate}[label=(\alph*)]
  \item $I \subseteq (I:S)$. 
  \item $(I:S)S \subseteq I$. 
  \item $((I:S):T)=(I:ST)=((I:T):S)$. 
  \item $(\bigcap  V_\alpha : S)= \bigcap  (V_\alpha :S)$. 
  \item $(I: \sum V_\alpha )=\bigcap (I:V_\alpha )$. 
\end{enumerate}
\end{question}
\begin{proof}
Proposition (a) is obvious. Proposition (b) is also obvious once we reduce the problem into proving the single sum $xy$ belongs to  $I$ where  $x\in (I:S)$ and $y \in S$. For proposition (c), we first show 
\begin{align*}
\vi{((I:S):T)\subseteq (I:ST)}
\end{align*}
Because ideal is closed under addition, we only have to prove $xst\in I$ where $x\in ((I:S):T), s \in S\text{ and } t \in T$, which follows from noting $xt \in (I:S)$. $\vdone$. Note that  
\begin{align*}
\blue{(I:ST)\subseteq ((I:T):S)}
\end{align*}
is obvious. $\bdone$. Lastly, we show 
\begin{align*}
\vi{((I:T):S)\subseteq ((I:S):T)}
\end{align*}
Let $x\in ((I:T):S),t\in T$ and $s \in S$. We are required to show $xts \in I$, which is obvious since $xs \in (I:T)$. $\vdone$. Proposition (d) is obvious.  Let $x \in (I: \sum V_\alpha )$. Fix $\alpha $ and $r \in V_\alpha $. Because $r\in \sum V_\alpha $, we see $xr\in I$. Let $x$ be in the intersection, it is clear that $x \sum v_\alpha =\sum xv_\alpha \in I$ because $xv_\alpha \in I$.     
\end{proof}
\begin{theorem}
\textbf{(Radicals of ideals are well-defined)} If $I$ is an ideal of $R$, then the  \textbf{radical} of $I$ defined by 
 \begin{align*}
r(I)\triangleq  \set{x\in R: x^n\in I\text{ for some }n>0}
\end{align*}
is also an ideal. 
\end{theorem}
\begin{proof}
  To see $r(I)$ is closed under addition, let $x^n,y^m \in I$, and observe $(x+y)^{n+m}\in I$. To see $r(I)$ is a multiplicative black hole, let $x^n\in I,v \in R$ and observe $(xv)^n=x^nv^n \in I$. 
\end{proof}
\begin{theorem}
\textbf{(Description of Radicals)} Let $\pi : R \rightarrow R \quotient I$ be the quotient map. We have
\begin{align*}
r(I)= \pi ^{-1}(\operatorname{Nil}(R\quotient I)) 
\end{align*}
\end{theorem}
\begin{proof}
Obvious. 
\end{proof}
\begin{question}{}{}
  \begin{enumerate}[label=(\alph*)]
    \item $I \subseteq r(I)$. 
    \item $r(r(I))=r(I)$.  
    \item $r(IS)=r(I\cap S)=r(I)\cap r(S)$
    \item $r(I)=R \iff  I= R$. 
    \item $r(I+S)=r(r(I)+r(S))$. 
    \item If $I$ is prime, then  $r(I^n)=I$ for all $n>0$. 
  \end{enumerate}
\end{question}
\begin{proof}
Proposition (a) and (b) are obvious. The proposition
\begin{align*}
r(IS)\subseteq r(I\cap S) 
\end{align*}
follows from $IS \subseteq I \cap S$. The propositions 
\begin{align*}
r(I \cap S)\subseteq r(I)\cap r(S)\text{ and }r(I) \cap r(S) \subseteq r(IS)
\end{align*}
are clear, thus proving proposition (c). The proposition 
\begin{align*}
I=R \implies  r(I)=R 
\end{align*}
is clear, and its converse follows from $1\in r(I)\implies 1=1^n \in I$, thus proving proposition (d). The proposition 
\begin{align*}
r(I+S)\subseteq r(r(I)+r(S))
\end{align*}
is clear. Let $x^n=y+z$ where  $y^m \in I$ and $z^p \in S$. We see $x^{n(m+p)}\in I+S$. We have shown 
\begin{align*}
r(r(I)+r(S)) \subseteq r(I+S)
\end{align*}
thus proving proposition (e). Let $I$ be prime. We know $I \subseteq r(I)$. To see the converse, let $x^n \in I$. Because $I$ is prime, either  $x$ or  $x^{n-1}$ belongs to $I$. If  $x$ does not belong to  $I$, then  $x^{n-1}$ belongs to $I$, which implies either $x\in I$ or $x^{n-2}\in I$. Applying the same argument repeatedly, we see $x \in I$, thus proving $r(I)\subseteq I$. Because 
\begin{align*}
I \supseteq I^2 \supseteq I^3 \supseteq I^4 \supseteq \cdots   
\end{align*}
we know
\begin{align*}
r(I) \supseteq r(I^2) \supseteq r(I^3) \supseteq r(I^4) \supseteq \cdots  
\end{align*}
Because 
\begin{align*}
x^n \in I \implies  x^{nk} \in I^k \text{ for all }k \inn
\end{align*}
We now also have 
\begin{align*}
r(I) \subseteq r(I^k)\text{ for all }k  \inn
\end{align*}
This proved proposition (e). 
\end{proof}
\begin{theorem}
\textbf{(Description of radical)}  Let $I$ be an ideal of $R$. 
\begin{align*}
r(I)= \bigcap \set{S \in \operatorname{spec}(R): I \subseteq S}
\end{align*}
\end{theorem}

\section{archived}

There are essentially two distinct substructures of a ring. A subset of a ring is called a \textbf{subring} if it is closed under addition and multiplication and contains the multiplicative identity. 

Because the union of a chain of proper ideals is still a proper ideal\footnote{No proper ideals contain $1$.}, we may apply \textbf{Zorn's Lemma} to show that a \textbf{maximal ideal}\footnote{By a maximal ideal, we mean a proper ideal contained by no other proper ideal.} always exists. Equivalently, we may define a proper ideal $I$ to be maximal if and only if $R\quotient I$ is a field.\\ 
\end{document}
