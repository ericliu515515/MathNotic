\documentclass{report}

%%%%%%%%%%%%%% preamble.tex %%%%%%%%%%%%%%
\usepackage[T1]{fontenc}
\usepackage{etoolbox}
% Page Setup
\usepackage[letterpaper, tmargin=2cm, rmargin=0.5in, lmargin=0.5in, bmargin=80pt, footskip=.2in]{geometry}
\usepackage{adjustbox}
\usepackage{graphicx}
\usepackage{tikz}
\usepackage{mathrsfs}
\usepackage{mdframed}

% Create a new toggle
\newtoggle{firstsection}

% Redefine the \chapter command to reset the toggle for each new chapter
\let\oldchapter\chapter
\renewcommand{\chapter}{\toggletrue{firstsection}\oldchapter}

% Redefine the \section command to check the toggle
\let\oldsection\section
\renewcommand{\section}{
    \iftoggle{firstsection}
    {\togglefalse{firstsection}} % If it's the first section, just switch off the toggle for next sections
    {\clearpage} % If it's not the first section, start a new page
    \oldsection
}

% Abstract Design

\usepackage{lipsum}

\renewenvironment{abstract}
 {% Start of environment
  \quotation
  \small
  \noindent
  \rule{\linewidth}{.5pt} % Draw the rule to match the linewidth
  \par\smallskip
  {\centering\bfseries\abstractname\par}\medskip
 }
 {% End of environment
  \par\noindent
  \rule{\linewidth}{.5pt} % Ensure the closing rule also matches
  \endquotation
 }

% Mathematics
\usepackage{amsmath,amsfonts,amsthm,amssymb,mathtools}
\usepackage{xfrac}
\usepackage[makeroom]{cancel}
\usepackage{enumitem}
\usepackage{nameref}
\usepackage{multicol,array}
\usepackage{tikz-cd}
\usepackage{array}
\usepackage{multirow}% http://ctan.org/pkg/multirow
\usepackage{graphicx}

% Colors
\usepackage[dvipsnames]{xcolor}
\definecolor{myg}{RGB}{56, 140, 70}
\definecolor{myb}{RGB}{45, 111, 177}
\definecolor{myr}{RGB}{199, 68, 64}
% Define more colors here...
\definecolor{olive}{HTML}{6B8E23}
\definecolor{orange}{HTML}{CC5500}
\definecolor{brown}{HTML}{8B4513}
% Hyperlinks
\usepackage{bookmark}
\usepackage[colorlinks=true,linkcolor=blue,urlcolor=blue,citecolor=blue,anchorcolor=blue]{hyperref}
\usepackage{xcolor}
\hypersetup{
    colorlinks,
    linkcolor={red!50!black},
    citecolor={blue!50!black},
    urlcolor={blue!80!black}
}

% Text-related
\usepackage{blindtext}
\usepackage{fontsize}
\changefontsize[14]{14}
\setlength{\parindent}{0pt}
\linespread{1.2}

% Theorems and Definitions
\usepackage{amsthm}
\renewcommand\qedsymbol{$\blacksquare$}

% Define a new theorem style
\newtheoremstyle{mytheoremstyle}% name
  {}% Space above
  {}% Space below
  {}% Body font
  {}% Indent amount
  {\bfseries}% Theorem head font
  {.}% Punctuation after theorem head
  {.5em}% Space after theorem head
  {}% Theorem head spec (can be left empty, meaning ‘normal’)

% Apply the new theorem style to theorem-like environments
\theoremstyle{mytheoremstyle}

\newtheorem{theorem}{Theorem}[section]  
\newtheorem{definition}[theorem]{Definition} 
\newtheorem{lemma}[theorem]{Lemma}  
\newtheorem{corollary}[theorem]{Corollary}
\newtheorem{axiom}[theorem]{Axiom}
\newtheorem{example}[theorem]{Example}
\newtheorem{equiv_def}[theorem]{Equivalent Definition}

% tcolorbox Setup
\usepackage[most,many,breakable]{tcolorbox}
\tcbuselibrary{theorems}

% Define custom tcolorbox environments here...

%================================
% EXAMPLE BOX
%================================
% After you have defined the style and other theorem environments
\definecolor{myexamplebg}{RGB}{245, 245, 245} % Very light grey for background
\definecolor{myexamplefr}{RGB}{120, 120, 120} % Medium grey for frame
\definecolor{myexampleti}{RGB}{60, 60, 60}    % Darker grey for title

\newtcbtheorem[]{Example}{Example}{
    colback=myexamplebg,
    breakable,
    colframe=myexamplefr,
    coltitle=myexampleti,
    boxrule=1pt,
    sharp corners,
    detach title,
    before upper=\tcbtitle\par\vspace{-20pt}, % Reduced the space after the title
    fonttitle=\bfseries,
    description font=\mdseries,
    separator sign none,
    description delimiters={}{}, % No delimiters around the title
}{ex}
%================================
% Solution BOX
%================================
\makeatletter
\newtcolorbox{solution}{enhanced,
	breakable,
	colback=white,
	colframe=myg!80!black,
	attach boxed title to top left={yshift*=-\tcboxedtitleheight},
	title=Solution,
	boxed title size=title,
	boxed title style={%
			sharp corners,
			rounded corners=northwest,
			colback=tcbcolframe,
			boxrule=0pt,
		},
	underlay boxed title={%
			\path[fill=tcbcolframe] (title.south west)--(title.south east)
			to[out=0, in=180] ([xshift=5mm]title.east)--
			(title.center-|frame.east)
			[rounded corners=\kvtcb@arc] |-
			(frame.north) -| cycle;
		},
}
\makeatother

% %================================
% % Question BOX
% %================================
\makeatletter
\newtcbtheorem{question}{Question}{enhanced,
	breakable,
	colback=white,
	colframe=myb!80!black,
	attach boxed title to top left={yshift*=-\tcboxedtitleheight},
	fonttitle=\bfseries,
	title={#2},
	boxed title size=title,
	boxed title style={%
			sharp corners,
			rounded corners=northwest,
			colback=tcbcolframe,
			boxrule=0pt,
		},
	underlay boxed title={%
			\path[fill=tcbcolframe] (title.south west)--(title.south east)
			to[out=0, in=180] ([xshift=5mm]title.east)--
			(title.center-|frame.east)
			[rounded corners=\kvtcb@arc] |-
			(frame.north) -| cycle;
		},
	#1
}{question}
\makeatother

%%%%%%%%%%%%%%%%%%%%%%%%%%%%%%%%%%%%%%%%%%%
% TABLE OF CONTENTS
%%%%%%%%%%%%%%%%%%%%%%%%%%%%%%%%%%%%%%%%%%%


\usepackage{tikz}
\definecolor{doc}{RGB}{0,60,110}
\usepackage{titletoc}
\contentsmargin{0cm}
\titlecontents{chapter}[14pc]
{\addvspace{30pt}%
	\begin{tikzpicture}[remember picture, overlay]%
		\draw[fill=doc!60,draw=doc!60] (-7,-.1) rectangle (-0.9,.5);%
		\pgftext[left,x=-5.5cm,y=0.2cm]{\color{white}\Large\sc\bfseries Chapter\ \thecontentslabel};%
	\end{tikzpicture}\color{doc!60}\large\sc\bfseries}%
{}
{}
{\;\titlerule\;\large\sc\bfseries Page \thecontentspage
	\begin{tikzpicture}[remember picture, overlay]
		\draw[fill=doc!60,draw=doc!60] (2pt,0) rectangle (4,0.1pt);
	\end{tikzpicture}}%
\titlecontents{section}[3.7pc]
{\addvspace{2pt}}
{\contentslabel[\thecontentslabel]{3pc}}
{}
{\hfill\small \thecontentspage}
[]
\titlecontents*{subsection}[3.7pc]
{\addvspace{-1pt}\small}
{}
{}
{\ --- \small\thecontentspage}
[ \textbullet\ ][]

\makeatletter
\renewcommand{\tableofcontents}{
	\chapter*{%
	  \vspace*{-20\p@}%
	  \begin{tikzpicture}[remember picture, overlay]%
		  \pgftext[right,x=15cm,y=0.2cm]{\color{doc!60}\Huge\sc\bfseries \contentsname};%
		  \draw[fill=doc!60,draw=doc!60] (13,-.75) rectangle (20,1);%
		  \clip (13,-.75) rectangle (20,1);
		  \pgftext[right,x=15cm,y=0.2cm]{\color{white}\Huge\sc\bfseries \contentsname};%
	  \end{tikzpicture}}%
	\@starttoc{toc}}
\makeatother

\newcommand{\liff}{\llap{$\iff$}}
\newcommand{\rap}[1]{\rrap{\text{ (#1)}}}
\newcommand{\red}[1]{\textcolor{red}{#1}}
\newcommand{\blue}[1]{\textcolor{blue}{#1}}
\newcommand{\vi}[1]{\textcolor{violet}{#1}}
\newcommand{\olive}[1]{\textcolor{olive}{#1}}
\newcommand{\teal}[1]{\textcolor{teal}{#1}}
\newcommand{\brown}[1]{\textcolor{brown}{#1}}
\newcommand{\orange}[1]{\textcolor{orange}{#1}}
\newcommand{\tCaC}{\text{ \CaC }}
\newcommand{\CaC}{\red{CaC} }
\newcommand{\As}[1]{Assume \red{#1}}
\newcommand{\vdone}{\vi{\text{ (done) }}}
\newcommand{\bdone}{\blue{\text{ (done) }}}
\newcommand{\tdone}{\teal{\text{ (done) }}}
\newcommand{\odone}{\olive{\text{ (done) }}}
\newcommand{\bodone}{\brown{\text{ (done) }}}
\newcommand{\ordone}{\orange{\text{ (done) }}}
\newcommand{\ld}{\lambda}
\newcommand{\vecta}[1]{\textbf{#1}}
\newcommand{\set}[1]{\left\{ #1 \right\}}
\newcommand{\bset}[1]{\Big\{ #1 \Big\}}
\newcommand{\inR}{\in\R}
\newcommand{\inn}{\in\N}
\newcommand{\inz}{\in\Z}
\newcommand{\inr}{\in\R}
\newcommand{\inc}{\in\C}
\newcommand{\inq}{\in\Q}
\newcommand{\norm}[1]{\| #1 \|}
\newcommand{\bnorm}[1]{\Big\| #1 \Big\|}
\newcommand{\gen}[1]{\langle #1 \rangle}
\newcommand{\abso}[1]{\left|#1\right|}
\newcommand{\myref}[2]{\hyperref[#2]{#1\ \ref*{#2}}}
\newcommand{\customref}[2]{\hyperref[#1]{#2}}
\newcommand{\power}[1]{\mathcal{P}(#1)}
\newcommand{\dcup}{\mathbin{\dot{\cup}}}
\newcommand{\diam}[1]{\text{diam}\, #1}
\newcommand{\at}{\Big|}
\newcommand{\quotient}{\diagup}
\let\originalphi\phi % Store the original \phi in \originalphi
\renewcommand{\phi}{\varphi} % Redefine \phi to \varphi
\newcommand{\pfi}{\originalphi} % Define \pfi to display the original \phi
\newcommand{\diota}{\dot{\iota}}
\newcommand{\Log}{\operatorname{Log}}
\newcommand{\id}{\text{\textbf{id}}}
\usepackage{amsmath}

\makeatletter
\NewDocumentCommand{\extp}{e{^}}{%
  \mathop{\mathpalette\extp@{#1}}\nolimits
}
\NewDocumentCommand{\extp@}{mm}{%
  \bigwedge\nolimits\IfValueT{#2}{^{\extp@@{#1}#2}}%
  \IfValueT{#1}{\kern-2\scriptspace\nonscript\kern2\scriptspace}%
}
\newcommand{\extp@@}[1]{%
  \mkern
    \ifx#1\displaystyle-1.8\else
    \ifx#1\textstyle-1\else
    \ifx#1\scriptstyle-1\else
    -0.5\fi\fi\fi
  \thinmuskip
}
\makeatletter
\usepackage{pifont}
\makeatletter
\newcommand\Pimathsymbol[3][\mathord]{%
  #1{\@Pimathsymbol{#2}{#3}}}
\def\@Pimathsymbol#1#2{\mathchoice
  {\@Pim@thsymbol{#1}{#2}\tf@size}
  {\@Pim@thsymbol{#1}{#2}\tf@size}
  {\@Pim@thsymbol{#1}{#2}\sf@size}
  {\@Pim@thsymbol{#1}{#2}\ssf@size}}
\def\@Pim@thsymbol#1#2#3{%
  \mbox{\fontsize{#3}{#3}\Pisymbol{#1}{#2}}}
\makeatother
% the next two lines are needed to avoid LaTeX substituting upright from another font
\input{utxmia.fd}
\DeclareFontShape{U}{txmia}{m}{n}{<->ssub * txmia/m/it}{}
% you may also want
\DeclareFontShape{U}{txmia}{bx}{n}{<->ssub * txmia/bx/it}{}
% just in case
%\DeclareFontShape{U}{txmia}{l}{n}{<->ssub * txmia/l/it}{}
%\DeclareFontShape{U}{txmia}{b}{n}{<->ssub * txmia/b/it}{}
% plus info from Alan Munn at https://tex.stackexchange.com/questions/290165/how-do-i-get-a-nicer-lambda?noredirect=1#comment702120_290165
\newcommand{\pilambdaup}{\Pimathsymbol[\mathord]{txmia}{21}}
\renewcommand{\lambda}{\pilambdaup}
\renewcommand{\tilde}{\widetilde}
\DeclareMathOperator*{\esssup}{ess\,sup}
\newcommand{\bluecheck}{}%
\DeclareRobustCommand{\bluecheck}{%
  \tikz\fill[scale=0.4, color=blue]
  (0,.35) -- (.25,0) -- (1,.7) -- (.25,.15) -- cycle;%
}


\usepackage{tikz}
\newcommand*{\DashedArrow}[1][]{\mathbin{\tikz [baseline=-0.25ex,-latex, dashed,#1] \draw [#1] (0pt,0.5ex) -- (1.3em,0.5ex);}}

\newcommand{\C}{\mathbb{C}}	
\newcommand{\F}{\mathbb{F}}
\newcommand{\N}{\mathbb{N}}
\newcommand{\Q}{\mathbb{Q}}
\newcommand{\R}{\mathbb{R}}
\newcommand{\Z}{\mathbb{Z}}



\title{\Huge{NCKU 112.1}\\Sum of k-th Power}
\author{\huge{Eric Liu}}
\date{}
\begin{document}

\maketitle
\newpage% or \cleardoublepage
% \pdfbookmark[<level>]{<title>}{<dest>}
\pdfbookmark[section]{\contentsname}{toc}
\tableofcontents
\pagebreak
\chapter{Sum of 1,2,3th power}
\section{k=1}
\begin{theorem}
  Given a natural number $m$, We have the identity 
\begin{equation}
1+\cdots +m=\frac{m^2+m}{2}
\end{equation} 
\end{theorem} 
\begin{proof}
Observe that given any natural number $n$
\begin{align}
  n^2-( n-1 ) ^2 &= n^2-( n^2-2n+1 ) \\ &= 2n-1
\end{align} 
Then we can deduce an identity
\begin{align}
  n&=\frac{1}{2}[ n^2-( n-1 ) ^2+1 ]  
\end{align} 
Then
\begin{align}
  \sum_{n=1}^{m} n &= \sum_{n=1}^{m}\frac{1}{2}[n^2-( n-1 ) ^2+1] \\&= \frac{1}{2}[\sum_{n=1}^{m} n^2-\sum_{n=1}^{m}(n-1)^2+\sum_{n=1}^{m}1] \\&=\frac{1}{2}[ \sum_{n=1}^{m} n^2- \sum_{n=1}^{m-1} n^2+m ] \\&= \frac{m^2+m}{2}
\end{align} 
\end{proof} 
\section{k=2}%
\label{sec:power to 2}
\begin{theorem}
  Given a natural number $m$, We have the identity 
\begin{equation}
 1^2+\cdots +m^2= \frac{m^3}{3}+\frac{m^2}{2}+\frac{m}{6}
\end{equation} 
\end{theorem} 
\begin{proof}
Observe that given any natural number $n$
\begin{align}
  n^3-( n-1 ) ^3&=n^3-( n^3-3n^2+3n-1 ) \\&= 3n^2-3n+1
\end{align} 
Then we can deduce an identity
\begin{equation}
n^2=\frac{1}{3}[ n^3-( n-1 ) ^3+3n-1 ] 
\end{equation} 
Then 
\begin{align}
  \sum_{n=1}^{m} n^2&=\sum_{n=1}^{m}\frac{1}{3}[  n^3-( n-1 ) ^3+3n-1 ]\\&= \frac{m^3}{3}+\sum_{n=1}^{m} n-\frac{m}{3}               \\&=\frac{m^3}{3}+\frac{m^2}{2}+\frac{m}{6}
\end{align} 
\end{proof} 
\section{k=3}%
\begin{theorem}
  Given a natural number $m$, We have the identity 
\begin{equation}
1^3+\cdots +m^3=\frac{m^4}{4}+ \frac{m^3}{2}+\frac{m^2}{4}
\end{equation} 
\end{theorem} 
\begin{proof}
Observe that given any natural number $n$
\begin{align}
  n^{4}-( n-1 ) ^{4}&= n^{4}- ( n^{4}-4n^3+6n^2-4n+1 )\\&=4n^3-6n^2+4n-1 
\end{align} 
Then we can deduce an identity
\begin{equation}
 n^3=\frac{1}{4} [ n^4-( n-1 ) ^4+6n^2-4n+1 ]
\end{equation} 
Then
\begin{align}
  \sum_{n=1}^{m} n^3&=\frac{m^4}{4}+\frac{3}{2}\sum_{n=1}^{m} n^2-\sum_{n=1}^{m} n+\frac{m}{4} 
  \\&= \frac{m^4}{4}+\frac{3}{2}( \frac{m^3}{3}+\frac{m^2}{2}+\frac{m}{6} ) - (  \frac{m^2+m}{2}	) +\frac{m}{4}
  \\&=\frac{m^4}{4}+ \frac{m^3}{2}+\frac{m^2}{4}
\end{align} 
\end{proof} 
\chapter{Generalization}
\section{Summary}
So far, we have collected the following using the same method. It isn't difficult to show for all natural number $k$, the sum $\sum_{n=1}^{m} n^k$ can be expressed by a polynomial of $m$ of $k+1$ degree.
\begin{align}
  \sum_{n=1}^{m} n^0 &= m\\
  \sum_{n=1}^{m} n^1 &= \frac{m^2}{2}+\frac{m}{2}\\
  \sum_{n=1}^{m} n^2 &= \frac{m^3}{3}+\frac{m^2}{2}+\frac{m}{6}\\
  \sum_{n=1}^{m} n^3 &= \frac{m^4}{4}+\frac{m^3}{2}+\frac{m^2}{4}
\end{align} 
Now, we use the same method to give an inductive formula of sum of $k$-th power. Notice that this formula is very inefficient if $k$ is large.
\begin{theorem}
  Given a natural number $m$, We have the identity 
\begin{equation}
\sum_{n=1}^{m} n^k = \frac{1}{k+1}[m^{k+1}+\sum_{i=0}^{k-1} \binom{k+1}{i}(-1)^{k-i+1}\sum_{n=1}^{m} n^i ]
\end{equation}
\end{theorem}
\begin{proof}
\begin{align}
  n^{k+1}-(n-1)^{k+1}&= -\sum_{i=0}^{k} \binom{k+1}{i}n^i(-1)^{k+1-i}\\
                     &=\sum_{i=0}^{k} \binom{k+1}{i}n^i(-1)^{k-i}\\
                     &= \binom{k+1}{k}n^k(-1)^0+\sum_{i-0}^{k-1} \binom{k+1}{i}n^i(-1)^{k-i}\\
                     &= (k+1)n^k+\sum_{i=0}^{k-1} \binom{k+1}{i}n^i(-1)^{k-i}\\
  (k+1)n^k &= n^{k+1}-(n-1)^{k+1}+\sum_{i=0}^{k-1} \binom{k+1}{i}n^i(-1)^{k-i+1}\\
  n^k &=\frac{1}{k+1}[n^{k+1}-(n-1)^{k+1}+\sum_{i=0}^{k-1} \binom{k+1}{i} n^i(-1)^{k-i+1}]\\
  \sum_{n=1}^{m} n^k&=\frac{1}{k+1}[m^{k+1}+\sum_{n=1}^{m} \sum_{i=0}^{k-1} \binom{k+1}{i}n^i(-1)^{k-i+1}]\\
                    &= \frac{1}{k+1}[m^{k+1}+\sum_{i=0}^{k-1} \binom{k+1}{i}(-1)^{k-i+1}\sum_{n=1}^{m} n^i]
\end{align} 
\end{proof}

Now we show an intreseting property of the sum of $k$-th power.
\begin{theorem}
If $m,k$ are natural numbers, then $\sum_{n=1}^{m} n^k$ is a polynomial of $m$, where the sum of coefficient is $1$
\end{theorem}
\begin{proof}
Let $f$ be a function that maps a polynomial of $m$ to the sum of coefficients of the polynomial.
We prove our theorem by induction.\\

We know that $\sum_{n=1}^{m}n^0=m$, which finish the proof for base case. Given a natural number $r$, assume that $\forall u\leq r\in \N, f(\sum_{n=1}^{m}n^u)=1$.\\

Notice
\begin{align}
  0=(1-1)^{r+2} &= \sum_{i=0}^{r+2}\binom{r+2}{i}(1)^i(-1)^{r+2-i}\\
                &= [\sum_{i=0}^{r}\binom{r+2}{i}(-1)^{r-i}]-(r+2)+1\\
  r+1 &= \sum_{i=0}^{r}\binom{r+2}{i}(-1)^{r-i} 
\end{align} 
Notice that $f$ is a linear function. We use this fact to deduce
\begin{align}
  \sum_{n=1}^{m}n^{r+1} &= \frac{1}{r+2}[m^{r+2}+\sum_{i=0}^{r}\binom{r+2}{i}(-1)^{r-i}\sum_{n=1}^{m}n^i]\\
  f(\sum_{n=1}^{m}n^{r+1}) &=f(\frac{1}{r+2}[m^{r+2}+\sum_{i=0}^{r}\binom{r+2}{i}(-1)^{r-i}\sum_{n=1}^{m}n^i])\\
                           &= \frac{1}{r+2}[f(m^{r+2})+\sum_{i=0}^{r}\binom{r+2}{i}(-1)^{r-i}f(\sum_{n=1}^{m}n^i)]\\
                           &= \frac{1}{r+2}(1+\sum_{i=0}^{r}\binom{r+2}{i}(-1)^{r-i})\\
                           &= \frac{1}{r+2}(1+r+1)\\
                           &= 1
\end{align} 
\end{proof}
\end{document}
