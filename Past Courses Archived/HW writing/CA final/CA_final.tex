\documentclass{report}
%%%%%%%%%%%%%% macros.tex %%%%%%%%%%%%%%
% Place your custom macros here, if any.

%%%%%%%%%%%%%% letterfonts.tex %%%%%%%%%%%%%%
% Place your font setup here, if any.

%%%%%%%%%%%%%% preamble.tex %%%%%%%%%%%%%%
\usepackage[T1]{fontenc}
\usepackage{lmodern}
\usepackage{etoolbox}
\usepackage{pdfpages}
\usepackage{transparent}
\usepackage[utf8]{inputenc}
\usepackage[english]{babel}

% Page Setup
\usepackage[tmargin=2cm, rmargin=0.5in, lmargin=0.5in, bmargin=80pt, footskip=.2in]{geometry}

% Mathematics
\usepackage{amsmath,amsfonts,amsthm,amssymb,mathtools}
\usepackage{xfrac}
\usepackage[makeroom]{cancel}
\usepackage{enumitem}
\usepackage{nameref}
\usepackage{multicol,array}
\usepackage{tikz-cd}
\usepackage[ruled,vlined,linesnumbered]{algorithm2e}

% Colors
\usepackage[dvipsnames]{xcolor}
\definecolor{myg}{RGB}{56, 140, 70}
\definecolor{myb}{RGB}{45, 111, 177}
\definecolor{myr}{RGB}{199, 68, 64}
% Define more colors here...

% Hyperlinks
\usepackage{bookmark}
\usepackage{hyperref}
\hypersetup{
    pdftitle={Assignment},
    colorlinks=true, linkcolor=doc!90,
    bookmarksnumbered=true,
    bookmarksopen=true
}

% Figures and Graphics
\usepackage{import}
\usepackage{svg}
\newcommand{\incfig}[1]{%
    \def\svgwidth{\columnwidth}
    \import{./figures/}{#1.pdf_tex}
}

% Text-related
\usepackage{blindtext}
\usepackage{fontsize}
\changefontsize[14]{14}
\setlength{\parindent}{0pt}

% Theorems and Definitions
\usepackage{amsthm}
\renewcommand\qedsymbol{$\blacksquare$}

% Define a new theorem style
\newtheoremstyle{mytheoremstyle}% name
  {}% Space above
  {}% Space below
  {\sffamily}% Body font
  {}% Indent amount
  {\bfseries}% Theorem head font
  {.}% Punctuation after theorem head
  {.5em}% Space after theorem head
  {}% Theorem head spec (can be left empty, meaning ‘normal’)

% Apply the new theorem style to theorem-like environments
\theoremstyle{mytheoremstyle}
\newtheorem{theorem}{Theorem}[section]
\newtheorem{definition}{Definition}[section]
\newtheorem{corollary}{Corollary}[section]
\newtheorem{lemma}{Lemma}[section]
\newtheorem{axiom}{Axiom}[section]

% tcolorbox Setup
\usepackage[most,many,breakable]{tcolorbox}

% Define custom tcolorbox environments here...

%================================
% EXAMPLE BOX
%================================
\newtcbtheorem[definition]{Example}{Example}
{%
    colback = myexamplebg,
    breakable,
    colframe = myexamplefr,
    coltitle = myexampleti,
    boxrule = 1pt,
    sharp corners,
    detach title,
    before upper=\tcbtitle\par\smallskip,
    fonttitle = \bfseries,
    description font = \mdseries,
    separator sign none,
    description delimiters parenthesis,
}
{ex}

%================================
% Solution BOX
%================================
\makeatletter
\newtcolorbox{solution}{enhanced,
	breakable,
	colback=white,
	colframe=myg!80!black,
	attach boxed title to top left={yshift*=-\tcboxedtitleheight},
	title=Solution,
	boxed title size=title,
	boxed title style={%
			sharp corners,
			rounded corners=northwest,
			colback=tcbcolframe,
			boxrule=0pt,
		},
	underlay boxed title={%
			\path[fill=tcbcolframe] (title.south west)--(title.south east)
			to[out=0, in=180] ([xshift=5mm]title.east)--
			(title.center-|frame.east)
			[rounded corners=\kvtcb@arc] |-
			(frame.north) -| cycle;
		},
}
\makeatother

%================================
% Question BOX
%================================
\makeatletter
\newtcbtheorem{question}{Question}{enhanced,
	breakable,
	colback=white,
	colframe=myb!80!black,
	attach boxed title to top left={yshift*=-\tcboxedtitleheight},
	fonttitle=\bfseries,
	title={#2},
	boxed title size=title,
	boxed title style={%
			sharp corners,
			rounded corners=northwest,
			colback=tcbcolframe,
			boxrule=0pt,
		},
	underlay boxed title={%
			\path[fill=tcbcolframe] (title.south west)--(title.south east)
			to[out=0, in=180] ([xshift=5mm]title.east)--
			(title.center-|frame.east)
			[rounded corners=\kvtcb@arc] |-
			(frame.north) -| cycle;
		},
	#1
}{def}
\makeatother
\makeatletter
\newtcbtheorem{qstion}{Question}{enhanced,
    breakable,
    colback=white,
    colframe=mygr,
    attach boxed title to top left={yshift*=-\tcboxedtitleheight},
    fonttitle=\bfseries,
    title={#2},
    boxed title size=title,
    boxed title style={%
        sharp corners,
        rounded corners=northwest,
        colback=tcbcolframe,
        boxrule=0pt,
    },
    underlay boxed title={%
        \path[fill=tcbcolframe] (title.south west)--(title.south east)
        to[out=0, in=180] ([xshift=5mm]title.east)--
        (title.center-|frame.east)
        [rounded corners=\kvtcb@arc] |-
        (frame.north) -| cycle;
    },
    #1
}{def}
\makeatother

%%%%%%%%%%%%%%%%%%%%%%%%%%%%%%%%%%%%%%%%%%%
% TABLE OF CONTENTS
%%%%%%%%%%%%%%%%%%%%%%%%%%%%%%%%%%%%%%%%%%%
\usepackage{tikz}
\definecolor{doc}{RGB}{0,60,110}
\usepackage{titletoc}
\contentsmargin{0cm}
\titlecontents{chapter}[14pc]
{\addvspace{30pt}%
	\begin{tikzpicture}[remember picture, overlay]%
		\draw[fill=doc!60,draw=doc!60] (-7,-.1) rectangle (-0.9,.5);%
		\pgftext[left,x=-4.5cm,y=0.2cm]{\color{white}\Large\sc\bfseries Chapter\ \thecontentslabel};%
	\end{tikzpicture}\color{doc!60}\large\sc\bfseries}%
{}
{}
{\;\titlerule\;\large\sc\bfseries Page \thecontentspage
	\begin{tikzpicture}[remember picture, overlay]
		\draw[fill=doc!60,draw=doc!60] (2pt,0) rectangle (4,0.1pt);
	\end{tikzpicture}}%
\titlecontents{section}[3.7pc]
{\addvspace{2pt}}
{\contentslabel[\thecontentslabel]{2pc}}
{}
{\hfill\small \thecontentspage}
[]
\titlecontents*{subsection}[3.7pc]
{\addvspace{-1pt}\small}
{}
{}
{\ --- \small\thecontentspage}
[ \textbullet\ ][]

\makeatletter
\renewcommand{\tableofcontents}{
	\chapter*{%
	  \vspace*{-20\p@}%
	  \begin{tikzpicture}[remember picture, overlay]%
		  \pgftext[right,x=15cm,y=0.2cm]{\color{doc!60}\Huge\sc\bfseries \contentsname};%
		  \draw[fill=doc!60,draw=doc!60] (13,-.75) rectangle (20,1);%
		  \clip (13,-.75) rectangle (20,1);
		  \pgftext[right,x=15cm,y=0.2cm]{\color{white}\Huge\sc\bfseries \contentsname};%
	  \end{tikzpicture}}%
	\@starttoc{toc}}
\makeatother

\newcommand{\liff}{\llap{$\iff$}}
\newcommand{\rap}[1]{\rrap{\text{ (#1)}}}
\newcommand{\red}[1]{\textcolor{red}{#1}}
\newcommand{\blue}[1]{\textcolor{blue}{#1}}
\newcommand{\vi}[1]{\textcolor{violet}{#1}}
\newcommand{\teal}[1]{\textcolor{teal}{#1}}
\newcommand{\tCaC}{\text{ \CaC }}
\newcommand{\CaC}{\red{CaC} }
\newcommand{\As}[1]{Assume \red{#1}}
\newcommand{\vdone}{\vi{\text{ (done) }}}
\newcommand{\bdone}{\blue{\text{ (done) }}}
\newcommand{\tdone}{\teal{\text{ (done) }}}
\newcommand{\set}[1]{\{ #1 \}}
\newcommand{\inS}{\in S}
\newcommand{\inF}{\in\F}
\newcommand{\inE}{\in E}
\newcommand{\inA}{\in A}
\newcommand{\inB}{\in B}
\newcommand{\inC}{\in C}
\newcommand{\inU}{\in U}

\newcommand{\C}{\mathbb{C}}	
\renewcommand{\H}{\mathbb{H}}
\newcommand{\F}{\mathbb{F}}
\newcommand{\N}{\mathbb{N}}
\newcommand{\Q}{\mathbb{Q}}
\newcommand{\R}{\mathbb{R}}
\newcommand{\Z}{\mathbb{Z}}
\renewcommand{\P}{\mathbb{P}}
\renewcommand{\S}{\mathbb{S}}
\newcommand{\A}{\mathbb{A}}
\newcommand{\RP}{\R P}


\begin{document}
\newpage% or \cleardoublepage
% \pdfbookmark[<level>]{<title>}{<dest>}
\pdfbookmark[section]{\contentsname}{toc}
\pagebreak
\chapter{Complex Analysis Final} 
\section{Complex Analysis Final}
\begin{mdframed}
  (1)  (20 pts) Let $S^2 \subseteq \R^3$ be the closed unit sphere defined by the equation $x^2+y^2+z^2=1$. Let $N=(0,0,1)$ be the north pole of $S^2$ and  $S=(0,0,-1)$ be the south pole of $S^2$. Define 
  \begin{align*}
  U_N \triangleq S^2 \setminus \set{N} \text{ and }U_S = S^2 \setminus \set{S}
  \end{align*}
\end{mdframed}
\begin{question}{}{}
  (a) (5pts) Prove that $U_N$ and  $U_S$ are open subsets of  $S^2$. (Here $S^2$ is equipped with the subspace topology induced by  $\R^3$.) 
\end{question}
\begin{proof}
Fix $p \in U_N$. Because $N \not\in  U_N$, we know $\abso{p-N}>0$. Let  $\epsilon<\abso{p-N}$, so that the open ball $B_\epsilon (p)$ in $\R^3$ does not contain $N$. This give us 
 \begin{align*}
p \in B_\epsilon  (p)\cap S^2 \subseteq U_N
\end{align*}
Where $B_\epsilon (p)\cap S^2$ is open in $S^2$ by definition of subspace topology. We have shown that for each $p \in U_{N}$ there exists some subset $M_p \subseteq S^2$ open in $S^2$, containing $p$ and contained by  $U_N$. This implies  $U_N$ is open in  $S^2$.    \\


Fix $p \in U_S$. Because $S \not\in  U_S$, we know $\abso{p-S}>0$. Let  $\epsilon<\abso{p-S}$, so that the open ball $B_\epsilon (p)$ in $\R^3$ does not contain $S$. This give us 
 \begin{align*}
p \in B_\epsilon  (p)\cap S^2 \subseteq U_S
\end{align*}
Where $B_\epsilon (p)\cap S^2$ is open in $S^2$ by definition of subspace topology. We have shown that for each $p \in U_{S}$ there exists some subset $M_p \subseteq S^2$ open in $S^2$, containing $p$ and contained by  $U_S$. This implies  $U_S$ is open in  $S^2$.    
\end{proof}
\begin{question}{}{}
  (b) (5pts) Define $\phi_N:U_N\rightarrow \C$ and $\phi_S:U_S \rightarrow \C$ by 
  \begin{align*}
  \phi_N(a,b,c)\triangleq \frac{a+b i}{1-c}\text{ and }\phi_S (a,b,c)\triangleq \frac{a-b i}{1+c}
  \end{align*}
Prove that both $\phi_N$ and $\phi_S$ are homeomorphisms. 
\end{question}
\begin{proof}
The continuity of $\phi_N$ and $\phi_S$ is obvious. Suppose 
\begin{align*}
x+yi= \frac{a+b i}{1-c}= \phi_N (a,b,c)
\end{align*}
Multiply both side by $1-c$ 
 \begin{align}
\label{xyi}
   (1-c)(x+yi)=a+bi
\end{align}
This give us 
\begin{align*}
  (1-c)^2 (x^2+y^2) + c^2 = a^2+ b^2 + c^2 =1
\end{align*}
Which give us 
\begin{align*}
  (x^2+y^2+1)c^2 - 2(x^2+y^2) c + (x^2+y^2-1)=0
\end{align*}
By quadratic formula, 
\begin{align*}
c&= \frac{2(x^2+y^2)\pm \sqrt{4(x^2+y^2)^2 - 4(x^2+y^2+1)(x^2+y^2-1)} }{2(x^2+y^2+1)} \\
&=\frac{x^2+y^2 \pm 1}{x^2+y^2+1}= \frac{x^2+y^2-1}{x^2+y^2+1}
\end{align*}
Note that the last equality hold because $(a,b,c)\in U_N \implies c\neq 1$. From  \myref{Equation}{xyi}, we may now compute 
\begin{align*}
a=(1-c)x= \frac{2x}{x^2+y^2+1}\text{ and }b=(1-c)y= \frac{2y}{x^2+y^2+1}
\end{align*}
We have shown that $\phi_N$ is a bijection between $U_N$ and $\C$, and its inverse is exactly 
 \begin{align}
\label{phn}
\phi_N^{-1} (x+yi)= \frac{(2x,2y,x^2+y^2-1)}{x^2+y^2+1} 
\end{align}
The continuity of $\phi_N^{-1}:\C \rightarrow U_N$ is obvious. We have shown $\phi_N$ is indeed a homeomorphism. Now, suppose 
\begin{align}
\label{xyi2}
x+yi= \frac{a-b i}{1+c}= \phi_S(a,b,c)
\end{align}
Multiply both side by $1+c$ 
 \begin{align*}
   (1+c)(x+yi)=a-b i
\end{align*}
This give us 
\begin{align*}
  (1+c)^2(x^2+y^2)+c^2= a^2+ b^2 +c^2 =1
\end{align*}
Which give us 
\begin{align*}
  (x^2+y^2+1)c^2 + 2(x^2+y^2)c + (x^2+y^2-1)=0
\end{align*}
By quadratic formula 
\begin{align*}
c&= \frac{-2(x^2+y^2)\pm \sqrt{4(x^2+y^2)^2 -4(x^2+y^2+1)(x^2+y^2-1)} }{2(x^2+y^2+1)} \\
&= \frac{-x^2-y^2\pm 1}{x^2+y^2+1}= \frac{-x^2-y^2+1}{x^2+y^2+1}
\end{align*}
Note that the last equality hold because $(a,b,c) \in U_S \implies c\neq -1$. From \myref{Equation}{xyi2}, we may now compute 
\begin{align*}
a=(1+c)x= \frac{2x}{x^2+y^2+1}\text{ and }b=(1+c)y= \frac{2y}{x^2+y^2+1}
\end{align*}
We have shown $\phi_S$ is a bijection between $U_S$ and  $\C$, and its inverse is exactly 
 \begin{align*}
\phi_S^{-1}(x+yi)= \frac{(2x,2y,1-x^2-y^2)}{x^2+y^2+1}
\end{align*}
The continuity of $\phi_S^{-1}:\C \rightarrow U_S$ is obvious. We have shown $\phi_S$ is indeed a homeomorphism. 
\end{proof}
\begin{question}{}{}
  (c) (5pts) Prove that 
\begin{align*}
\phi_N (S)=\phi_S(N)=0 \text{ and }\phi_N (U_N \cap U_S)= \phi_S (U_N \cap U_S)=\C^*
\end{align*}
\end{question}
\begin{proof}
Compute 
\begin{align*}
\phi_N (S)= \phi_N (0,0,-1)= \frac{0+0i}{2}=0
\end{align*}
Compute 
\begin{align*}
\phi_S(N)= \phi_S(0,0,1)= \frac{0-0i}{2}= 0
\end{align*}
Compute 
\begin{align*}
U_N \cap  U_S = U_N \setminus \set{S} = U_S \setminus \set{N}
\end{align*}
It then follows from the fact $\phi_N$ maps $U_N$ into  $\C$ bijectively that 
\begin{align*}
\phi_N (U_N \cap U_S)= \phi_N (U_N \setminus \set{S})= \C \setminus \set{\phi_N (S)}=\C \setminus \set{0}=\C^*
\end{align*}
 Similarly, it follows from the fact $\phi_S$ maps $U_S$ into  $\C$ bijectively that 
\begin{align*}
\phi_N (U_N \cap U_S)= \phi_N (U_S \setminus \set{N})= \C \setminus \set{\phi_S (N)}=\C \setminus \set{0}=\C^*
\end{align*}
\end{proof}
\begin{question}{}{}
  (d) (5pts) Show that 
  \begin{align*}
  f= \phi_S \circ \phi_N^{-1} :\C^* \rightarrow \C^* 
  \end{align*}
  is a holomorphic function. 
\end{question}
\begin{proof}
Using \myref{Equation}{phn}, we may compute for all $x+yi \inc^*$ 
\begin{align*}
f(x+yi)&= \phi_S (\phi_N^{-1}(x+yi)) \\
&=\phi_S \Big(\frac{2x}{x^2+y^2+1}, \frac{2y}{x^2+y^2+1}, \frac{x^2+y^2-1}{x^2+y^2+1}\Big)  \\
&= \frac{\frac{2x}{x^2+y^2+1}- \frac{2iy}{x^2+y^2+1}}{1+ \frac{x^2+y^2-1}{x^2+y^2+1}} \\
&= \frac{2x-2iy}{2(x^2+y^2)}= \frac{x-iy}{x^2+y^2}
\end{align*}
Compute 
\begin{align*}
\frac{\partial }{\partial x}\operatorname{Re}f&= \frac{\partial }{\partial x} \frac{x}{x^2+y^2}= \frac{y^2-x^2}{(x^2+y^2)^2} \\
\frac{\partial }{\partial y}\operatorname{Im}f&=\frac{\partial }{\partial y} \frac{-y}{x^2+y^2}= \frac{y^2-x^2}{(x^2+y^2)^2} 
\end{align*}
Compute 
\begin{align*}
\frac{\partial }{\partial x} \operatorname{Im} f&= \frac{\partial }{\partial x} \frac{-y}{x^2+y^2}= \frac{2xy}{(x^2+y^2)^2} \\
\frac{\partial }{\partial y}\operatorname{Re}f &= \frac{\partial }{\partial y} \frac{x}{x^2+y^2}= \frac{-2xy}{(x^2+y^2)^2}
\end{align*}
We have shown that $f$ satisfy the Cauchy-Riemann criteria. Because both $\frac{\partial }{\partial x}\operatorname{Re}f,\frac{\partial }{\partial y}\operatorname{Re}f:\R^2 \setminus \set{\textbf{0}} \rightarrow \R$ are continuous, we know $\operatorname{Re}f:\R^2\setminus \set{\textbf{0}} \rightarrow \R$ is differentiable.  Because both $\frac{\partial }{\partial x}\operatorname{Im}f,\frac{\partial }{\partial y}\operatorname{Im}f:\R^2\setminus \set{\textbf{0}} \rightarrow \R$ are continuous, we know $\operatorname{Im}f:\R^2 \setminus \set{\textbf{0}} \rightarrow \R$ is differentiable. It now follows from the Cauchy-Riemann Theorem that $f:\C^*\rightarrow \C$ is indeed holomorphic. 
\end{proof}
\begin{mdframed}
  (2) (20 pts) We identify $\C$ with  $S^2 \setminus \set{N}$. Denote $N$ by $\infty$. Denote $S^2 $ by  $\C_{\infty}= \C \cup  \set{\infty}$. From the previous problem, we know that  $N$ is the point of  $U_S$ corresponding  to  $w=0$. \\

   Let $f$ be a function defined on $\abso{z}>R$ for some $R>0$. We say that  $\infty$ is a pole of $f$ of order $m$ if the function $g:D_{\epsilon }(0)\setminus 0 \rightarrow \C$ defined by 
   \begin{align*}
   g(z)\triangleq f(\frac{1}{z})
   \end{align*}
  has a pole of order $m$ at  $z=0$. 
\end{mdframed}
\begin{question}{}{}
(a) Prove that $f(z)=-3z+z^2$ has a pole at $\infty$. Find the residue and order of the pole of $f$ at  $\infty$. 
\end{question}
\begin{proof}
Compute 
\begin{align*}
g(z)= f(\frac{1}{z})= \frac{-3}{z}+ \frac{1}{z^2}= \frac{-3z+1}{z^2}
\end{align*}
Because 
\begin{align*}
\lim_{z\to 0}z^2 g(z)= \lim_{z\to 0} -3z+1 =1 \inc^* 
\end{align*}
We see $g$ has a pole of order $2$ at  $0$. Therefore,  $f$ has a pole of order $2$ at $\infty$. Note that the residue of $f$ at $\infty$ is defined to be  
\begin{align*}
  \operatorname{Res}(f,\infty)&=  - \operatorname{Res}\Big(\frac{1}{z^2}g(z),0\Big) \\
  &= - \operatorname{Res}\Big(\frac{-3z+1}{z^4},0\Big)=0 
\end{align*}
\end{proof}
\begin{question}{}{}
  (b) Prove that a function on $\C_{\infty}$ is meromorphic if and only if it is a rational function, i.e., $f$ is meromorphic if and only if 
   \begin{align*}
  f(z)= \frac{Q(z)}{P(z)}
  \end{align*}
for some complex polynomial $P,Q$ such that $P\neq 0 \in \C [z]$  and $P,Q$ share no roots. 
\end{question}
\begin{proof}

For the 'if' part, by fundamental theorem of algebra, we may write 
\begin{align*}
f(z)= \frac{(z-z_0')\cdots (z-z_q')}{(z-z_0)\cdots (z-z_p)}
\end{align*}
This implies that  $f$ has at most $p+2$ numbers of poles, and  $f$ is differentiable everywhere except for points at which $f$ has a pole. That is, $f$ is meromorphic on  $\C_{\infty}$.\\

For the 'only if' part, suppose $f:\C_{\infty} \setminus Z\rightarrow \C$ is meromorphic, where 
\begin{align*}
Z\text{ is the set of poles of $f$ }
\end{align*}
Because $f$ is either differentiable or has a pole at $\infty$, we know there exists $R>0$ such that  $f$ is defined on $\set{z\inc: \abso{z}>R}$. By letting $R$ be larger if necessary, we may WLOG suppose $Z\setminus \set{\infty}$ is contained by $D_R(0)$, the open disk centering origin with radius $R$. Define $g:D_R(0)\rightarrow \C$ by 
\begin{align*}
g(z)\triangleq \begin{cases}
  \frac{1}{f(z)}& \text{ if $z\not\in Z$ }\\
  0& \text{ if $z\in Z$ }
\end{cases}
\end{align*}
Because $Z$ is the set of poles of  $f$, we know  $g$ is holomorphic. Obviously, $g$ can not vanish identically, otherwise $f$ has a pole at $p$ for all  $p \in D_R(0)$. It then follows from Identity Theorem that $Z\setminus \set{\infty}$, the set on which $g$ vanish, has no limit points in $D_R(0)$. By repeating the same procedure for similarly defined $g:D_{R+\epsilon }(0)\rightarrow \C$, we may WLOG suppose $Z \setminus \set{\infty}$ has no limit points in $\C$. It then follows from  $Z \setminus \set{\infty}$ is bounded that $Z$ is finite, since otherwise  $Z$ has a limit point in  $\C$, by Hiene-Borel and the fact limit point compact and compact are equivalent for  $\C$. \\

Knowing that $Z\setminus \set{\infty}$ is finite, we may write $Z \setminus \set{\infty}=\set{z_1,\dots ,z_k}$. Let $n_1,\dots ,n_k$ be the order of these poles. If we define  
\begin{align*}
g(z)\triangleq  (z-z_1)^{n_1} \cdots (z-z_k)^{n_k} f(z)
\end{align*}
We know on $\C$,  $g$ can only have removable singularity, so $g$ is in fact an entire function. Write 
\begin{align}
\label{gz}
g(z)= \sum_{n=0}^{\infty}a_nz^n\text{ for all }z\inc
\end{align}
Now, let $N$ be the order of the pole of  $f$ at  $\infty$, where $N=0$ if  $f$ is differentiable at $\infty$. By definition of $g$, 
 \begin{align*}
g\text{ has a pole at $\infty$ of order }N+n_1+\cdots +n_k\triangleq M
\end{align*}
It now follows from 
\begin{align*}
g(\frac{1}{z})= \sum_{n=0}^{\infty} a_nz^{-n}\text{ for all }\abso{z}>0
\end{align*}
that $a_{M+n}=0$ for all $n>0$. Then from  \myref{Equation}{gz}, we see $g$ is in fact a polynomial 
\begin{align*}
g(z)=a_Mz^M+ \cdots + a_0
\end{align*}
It follows from definition of $g$ that
\begin{align*}
f(z)&= \frac{g(z)}{(z-z_1)^{n_1}\cdots (z-z_k)^{n_k}}\\
&= \frac{a_Mz^M + \cdots +a_0}{(z-z_1)^{n_1}\cdots (z-z_k)^{n_k}}\text{ is indeed rational }
\end{align*}
Two things to note here: First, because 
\begin{align*}
  g(z)= (z-z_1)^{n_1}\cdots (z-z_k)^{n_k}f(z)\text{ and }z_j\text{ is of order $n_j$ for }f
\end{align*}
We have 
\begin{align*}
\lim_{z\to z_j} g(z)\neq 0
\end{align*}
That is,  $g$ indeed shares no roots with $(z-z-1)^{n_1}\cdots (z-z_k)^{n_k}$. Second, the argument holds true even if $Z\setminus \set{\infty}=\varnothing$. In such case, $P=1$ and $f$ is just  $g$.  
\end{proof}
\begin{question}{}{}
  (3) (15 pts) Let $f:\C\rightarrow \C$ be an entire function. Suppose that there exist two nonzero complex numbers $\omega_1$ and $\omega_2$ such that 
  \begin{enumerate}[label=(\alph*)]
    \item $\set{\omega_1,\omega_2}$ form a basis for $\C$ if $\C$ is viewed as a vector space over  $\R$.  
    \item $f(z+\omega_1)=f(z+\omega_2)=f(z)$ for all $z\inc$ 
  \end{enumerate}
Show that $f$ is constant. 
\end{question}
\begin{proof}
Let 
\begin{align*}
F\triangleq  \set{c_1\omega_1+ c_2\omega_2: 0\leq c_1,c_2 \leq 1}
\end{align*}
Because $F$ is by Hiene-Borel compact ($F$ is the closed parallelogram with vertices being $\set{0,\omega_1,\omega_2,\omega_1+\omega_2}$), and $f$ is continuous on $F$, we know by EVT $\abso{f}$ is bounded  by some $M>0$ on  $F$. Now, for all $z\inc$, because $\set{\omega_1,\omega_2}$ form a basis 
\begin{align*}
z=a_1\omega_1+a_2 \omega_2\text{ for some unique pair }a_1,a_2\inr
\end{align*}
By Euclidean algorithm, we may further write 
\begin{align*}
z= (n_1+c_1)\omega_1 + (n_2+c_2)\omega_2
\end{align*}
For some $n_1,n_2 \inz $ and $c_1,c_2 \in [0,1]$. It then follows from the premise that 
\begin{align*}
  \abso{f(z)}&=\abso{f((n_1+c_1)\omega_1+ (n_2+c_2)\omega_2)} \\
             &=\abso{f(c_1\omega_1+c_2\omega_2)}\leq M
\end{align*}
We have shown $f$ is bounded on the whole $\C$. Because $f$ is entire, it follows from the Liouville's Theorem that  $f$ is a constant.
\end{proof}
\begin{theorem}
\label{TL}
\textbf{(Truncated Laurent Series of $\cot (\pi  z)$)} Near $0$, we have 
\begin{align*}
\cot (\pi z)= \frac{1}{\pi  z}- \frac{\pi  z}{3}- \frac{(\pi  z)^3}{45}- \frac{2 (\pi z)^5}{945}- \frac{(\pi z)^7}{4725} + O(z^9)
\end{align*}
Near $n$, we have 
 \begin{align*}
\cot (\pi z)= \frac{1}{\pi (z-n)}- \frac{\pi  (z-n)}{3}- \frac{(\pi  (z-n))^3}{45}- \frac{2 (\pi (z-n))^5}{945}- \frac{(\pi (z-n))^7}{4725} + O((z-n)^9)
\end{align*}
\end{theorem}
\begin{proof}
Direct computation allow us to expand the Taylor series of $\tan$ at $0$ 
\begin{align*}
\tan(\pi z)=z+\frac{(\pi z)^3}{3}+\frac{2 (\pi  z)^5}{15}+\frac{17 (\pi z)^7}{315}+\frac{62
(\pi z)^9}{2835}+O\left(z^{11}\right)
\end{align*}
This implies 
\begin{align*}
  \cot(\pi z)=\frac 1{\pi z+\frac{(\pi  z)^3}{3}+\frac{2 (\pi  z)^5}{15}+\frac{17 (\pi z)^7}{315}+\frac{62 (\pi z)^9}{2835}+ O(z^{11})}
\end{align*}
Then by long division, we may compute 
\begin{align*}
\cot (\pi z)= \frac{1}{\pi  z}- \frac{\pi  z}{3}- \frac{(\pi  z)^3}{45}- \frac{2 (\pi z)^5}{945}- \frac{(\pi z)^7}{4725} + O(z^9)
\end{align*}
The truncated Laurent series around $n$ follows from the fact $\cot(\pi  z)$ is a function with period $1$.  
\end{proof}
\begin{theorem}
\label{IoD}
\textbf{(Two damned facts I can't prove)}  Let $\xi \not\inz$  and $f(z)\triangleq  \frac{1}{z-\xi}+ \frac{1}{z}$. There exists some sequence $(C_N)_{N=1}^{\infty}$ of closed contours such that the region enclosed by $C_N$ converge to  $\C$ and 
\begin{align*}
\lim_{N\to \infty} \int_{C_N}f(z)\cot (\pi  z)dz=0
\end{align*}
Also, 
\begin{align*}
\log \Big(\frac{\pi }{2} \Big)+ \sum_{n=1}^{\infty} \log (1- \frac{1}{4n^2})=0
\end{align*}
\end{theorem}
\begin{question}{}{}
(4) (20 pts) Prove the formula 
\begin{align*}
\sin \pi  z= \pi  z \prod _{n=1}^{\infty} \Big( 1-\frac{z^2}{n^2} \Big)
\end{align*}
using the following steps: 
\begin{enumerate}[label=(\alph*)]
  \item Consider $f(z)= \frac{1}{z- \xi}+ \frac{1}{z}$, and show that when $\xi \not\inz$, 
    \begin{align*}
    \pi  \cot (\pi  \xi)= \frac{1}{\xi}+ \sum_{n=1}^{\infty} \frac{2\xi}{\xi^2- n^2}
    \end{align*} 
    \item Integrate $\pi  \cot \pi z$ along a suitable contour to show that 
      \begin{align*}
      \log \sin \pi z= \log \pi z + \sum_{n=1}^{\infty} \log (1-\frac{z^2}{n^2})
      \end{align*}
      where $\log$ is chosen such that $\log 1=0$ in each term. 
\end{enumerate}
\end{question}
\begin{proof}
Fix $\xi \not\inz$ and define 
\begin{align*}
f(z)\triangleq \frac{1}{z-\xi}+ \frac{1}{z}= \frac{2z-\xi}{z(z-\xi)}
\end{align*}
Because  $\cot (\pi  z)$ has poles at $z=n\inz$ and they are all simple, we know $f(z)\cot (\pi  z)$ have simple poles at $z=n\inz^,z=\xi$, and have a double pole at $0$. Using \myref{Theorem}{TL}, we may compute 
 \begin{align*}
\operatorname{Res}(f(z)\cot (\pi  z),0)&= \operatorname{Res}\Big(\frac{\cot (\pi  z)}{z-\xi}, 0\Big)+ \operatorname{Res}\Big(\frac{\cot ( \pi z)}{z},0\Big) \\
&= \operatorname{Res}\Big(\frac{\cot (\pi  z)}{z-\xi}, 0\Big) \\
&=\lim_{z\to 0} \frac{z \cot (\pi  z)}{z-\xi}= \frac{\frac{1}{\pi }}{- \xi}= \frac{1}{- \pi \xi}
\end{align*}
And compute  
\begin{align*}
\operatorname{Res}(f(z)\cot (\pi  z), n)= \lim_{z\to n} f(z) z \cot (\pi  z)= \frac{2n - \xi}{\pi  n(n- \xi)} 
\end{align*}
And compute 
\begin{align*}
\operatorname{Res}(f(z)\cot (\pi  z), \xi)= \lim_{z\to \xi} (1+ \frac{z- \xi}{z}) \cot ( \pi  z)= \cot (\pi  \xi)
\end{align*}
It now follows from \myref{Theorem}{IoD} and Residue Theorem that 
\begin{align*}
0&= \lim_{N\to \infty} \int_{C_N} f(z) \cot (\pi  z) dz \\
&=\lim_{N\to \infty} \frac{1}{-\pi  \xi}+   \cot (\pi  \xi) +\sum_{0< \abso{n}\leq N} \frac{2n - \xi}{\pi  n (n- \xi)} 
\end{align*}
Therefore, 
\begin{align*}
  \cot (\pi  \xi)&= \frac{1}{\pi \xi}+ \sum_{\abso{n}>0} \frac{2n- \xi}{\pi n (\xi -n)}  \\
&=\frac{1}{\pi  \xi}+\frac{1}{\pi } \sum_{n=1}^{\infty} \Big( \frac{2n- \xi}{n (\xi-n)} + \frac{-2n-\xi}{-n(\xi+n)} \Big) \\
&= \frac{1}{\pi  \xi}+ \frac{1}{\pi }\sum_{n=1}^{\infty} \frac{(2n-\xi)(\xi+n)+(2n+\xi)(\xi-n)}{n(\xi^2-n^2)} \\
&=\frac{1}{\pi \xi} + \frac{1}{\pi }\sum_{n=1}^{\infty} \frac{2n\xi}{n(\xi^2-n^2)} \\
&=\frac{1}{\pi  \xi}+ \frac{1}{\pi }\sum_{n=1}^{\infty} \frac{2\xi}{\xi^2-n^2}
\end{align*}
Multiplying both side with $\pi $, we now have 
\begin{align}
\label{kk}
\pi  \cot (\pi  \xi)= \frac{1}{\xi}+ \sum_{n=1}^{\infty} \frac{2\xi}{\xi^2-n^2}
\end{align}
Now, note that 
\begin{align*}
\frac{d}{dz} \Big(\log \sin (\pi  z) \Big)&= \frac{\pi  \cos (\pi  z)}{\sin (\pi  z)}= \pi  \cot (\pi  z) \\
\frac{d}{dz} \Big( \log \pi  z \Big)&=\frac{1}{z} \\
\frac{d}{dz} \Big( \sum_{n=1}^{\infty} \log (1- \frac{z^2}{n^2}) \Big)&= \sum_{n=1}^{\infty} \frac{d}{dz} \log ( 1- \frac{z^2}{n^2}) = \sum_{n=1}^{\infty} \frac{\frac{-2z}{n^2}}{1- \frac{z^2}{n^2}}= \sum_{n=1}^{\infty} \frac{2z}{z^2-n^2}
\end{align*}
Fix $z\not\inz$, and let $\gamma $ be some contour that starts at $\frac{1}{2}$ and ends at $z$ without touching any integer. It follows from \myref{Theorem}{IoD} and fundamental theorem of calculus for complex function that 
\begin{align*}
\log (\sin (\pi z))&=\int_{\gamma } \pi  \cot (\pi  z)dz \\
&= \int_{\gamma }\Big( \frac{1}{z}+ \sum_{n=1}^{\infty} \frac{2z}{z^2-n^2} \Big) dz \\
&=\log (\pi  z)+ \sum_{n=1}^{\infty} \log (1-\frac{z^2}{n^2})
\end{align*}
Taking exponential on both side, we finally have 
\begin{align*}
\sin (\pi  z)= \pi  z \prod_{n=1}^{\infty} (1-\frac{z^2}{n^2})
\end{align*}
\end{proof}
\begin{question}{}{}
  (5) (15pts) If $\abso{a}>e$, use Rouche's Theorem to prove that the equation 
\begin{align*}
e^z= az^n  
\end{align*}
has $n$ roots with $\abso{z}<1$
\end{question}
\begin{proof}
Let $\textbf{D}$ be the unit disk centered at origin. Define entire $f,g:\C\rightarrow \C$ by 
\begin{align*}
f(z)\triangleq  az^n  \text{ and }g(z)\triangleq -e^z
\end{align*}
Because $e< \abso{a}$, on $\partial \textbf{D}$, we have 
\begin{align*}
\abso{g(z)}= \abso{e^z}=e^{\operatorname{Re}z} \leq e < \abso{a}= \abso{az^n}=\abso{f(z)}
\end{align*}
Therefore, by Rouche's Theorem,  $az^n-e^z=f+g$ has the same number of zeros in $\textbf{D}$ as  $f=az^n$. It is clear that $az^n$ only has zero $z=0$ with multiplicity $n$. Therefore, 
\begin{align*}
az^n-e^z\text{ has }n\text{ zeros in }\textbf{D}
\end{align*}
We have shown 
\begin{align*}
e^z=az^n\text{ has }n\text{ roots in }\textbf{D}
\end{align*}
\end{proof}
\begin{question}{}{}
  (6) (10pts) Let $f:\textbf{D}\rightarrow \C$ be a holomorphic function, where $\textbf{D}$ is the unit open disk centering $\textbf{D}$. If  $f(0)=0$ and $\abso{f(z)}\leq 1$ on $\textbf{D}$, prove that 
   \begin{enumerate}[label=(\alph*)]
    \item $\abso{f(z)}\leq \abso{z}$ for all $z\in \textbf{D}$. 
    \item $\abso{f'(0)}\leq 1$
  \end{enumerate}
\end{question}
\begin{proof}
Because the proposition is trivial for constant $f$, suppose $f$ is non-constant. Define $g:\textbf{D}\rightarrow \C$ by 
\begin{align*}
g(z)\triangleq \begin{cases}
  \frac{f(z)}{z}& \text{ if $z\neq 0$ }\\
  f'(0)& \text{ if $z=0$ }
\end{cases}
\end{align*}
Applying maximum modulus principle to $g$ on open disk centered at origin with radius  $r<1$, we have 
\begin{align*}
\abso{f'(0)}=\abso{g(0)}\leq \abso{g(z)}= \frac{\abso{f(z)}}{\abso{z}}\leq \frac{1}{r}\text{ for some }z\text{ such that }\abso{z}=r
\end{align*}
Letting $r\to 1$, this implies 
\begin{align*}
\abso{f'(0)}\leq 1
\end{align*}
Note that when $z=0$, 
 \begin{align*}
\abso{f(z)}=\abso{f(0)}=0 \leq 0= \abso{z}
\end{align*}



Fix $z\neq 0 \in \textbf{D}$. The maximum modulus principle implies that for each open disk $D_r$ centered at origin with radius  $r<1$ that contains $z$, there exists some $z_r\in \partial D_r$ such that 
 \begin{align*}
\abso{g(z)} \leq \abso{g(z_r)}= \frac{\abso{f(z_r)}}{\abso{z_r}} \leq \frac{1}{r}
\end{align*}
Letting $r\to 1$, we now have  
\begin{align*}
\frac{\abso{f(z)}}{\abso{z}}=\abso{g(z)}\leq 1
\end{align*}
Multiplying both side with $\abso{z}$, we now have 
\begin{align*}
\abso{f(z)}\leq \abso{z}\text{ for all }z\neq 0 \in \textbf{D}
\end{align*}
\end{proof}
\begin{theorem}
\label{OMT}
\textbf{(Open Mapping Theorem for Disk)} Let $U$ be an open disk and  $f:U \rightarrow \C$ be some non-constant holomorphic function. 
\begin{align*}
f(U)\text{ is open }
\end{align*}
\end{theorem}
\begin{proof}
Fix arbitrary $w_0\in f(U)$, and let $z_0\in U$ satisfy $f(z_0)=w_0$. Define holomorphic $g:U\rightarrow \C$ by $g(z)\triangleq f(z)-w_0$. Because $g$ is non-constant holomorphic, by Identity Theorem, the zeros of $g$ are isolated. Note that $z_0$ is a zero of  $g$. We may now let  $B$ be a closed disk centering  $z_0$ such that  $B$ is contained by $U$ and contains no other zero of  $g$. Because $\partial B\subseteq U$ is compact, by EVT, we may let 
\begin{align*}
a\triangleq \min_{\partial B}\abso{g}
\end{align*}
Note that $a>0$ because  $g$ has no zeros in  $\partial B \subseteq B$. Let $D$ be the open disk centering  $w_0$ with radius  $a$. Fix $w_1 \neq w_0\in D$. If we define $h:U\rightarrow \C$ by $h(z)\triangleq f(z)-w_1$, we see that for all $z\in \partial B$, we have 
\begin{align*}
\abso{g(z)-h(z)}= \abso{w_0-w_1}< a \leq \abso{g(z)}
\end{align*}
Therefore, by Rocuhe's Theorem, $h$ has some zero in  $B^\circ $. That is,  $f(z_1)=w_1$ for some $z_1 \in B^\circ $. Because $w_1$ is arbitrarily picked from  $D$, we have shown  $D \subseteq f(U)$. That is, $w_0$ is an interior point of  $f(U)$. Because $w_0$ is arbitrarily picked from $f(U)$. We have shown $f(U)$ is open. 
\end{proof}
\begin{question}{}{}
  (7) (20pts) Let $f:D\rightarrow \C$ be a non-constant holomorphic function defined on a domain $D$ contained in  $\C$. Prove that  $f$ is an open mapping. 
\end{question}
\begin{proof}
Let $E\subseteq D$ be open. We are required to show $f(E)$ is open. Because the set of open disk form a basis for $\C$, we may let 
\begin{align*}
E= \bigcup_{i \in I}U_i
\end{align*}
where $\set{U_i}_{i \in I}$ is a collection of open disk. It now follows from \myref{Theorem}{OMT} that 
\begin{align*}
f(E)= \bigcup_{i \in I} f(U_i)\text{ is open }
\end{align*}
because $f(E)$ is a union of open sets. 
\end{proof}
\end{document}
