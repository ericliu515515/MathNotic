\documentclass{report}

%%%%%%%%%%%%%% preamble.tex %%%%%%%%%%%%%%
\usepackage[T1]{fontenc}
\usepackage{etoolbox}
% Page Setup
\usepackage[letterpaper, tmargin=2cm, rmargin=0.5in, lmargin=0.5in, bmargin=80pt, footskip=.2in]{geometry}
\usepackage{adjustbox}
\usepackage{graphicx}
\usepackage{tikz}
\usepackage{mathrsfs}
\usepackage{mdframed}

% Create a new toggle
\newtoggle{firstsection}

% Redefine the \chapter command to reset the toggle for each new chapter
\let\oldchapter\chapter
\renewcommand{\chapter}{\toggletrue{firstsection}\oldchapter}

% Redefine the \section command to check the toggle
\let\oldsection\section
\renewcommand{\section}{
    \iftoggle{firstsection}
    {\togglefalse{firstsection}} % If it's the first section, just switch off the toggle for next sections
    {\clearpage} % If it's not the first section, start a new page
    \oldsection
}

% Abstract Design

\usepackage{lipsum}

\renewenvironment{abstract}
 {% Start of environment
  \quotation
  \small
  \noindent
  \rule{\linewidth}{.5pt} % Draw the rule to match the linewidth
  \par\smallskip
  {\centering\bfseries\abstractname\par}\medskip
 }
 {% End of environment
  \par\noindent
  \rule{\linewidth}{.5pt} % Ensure the closing rule also matches
  \endquotation
 }

% Mathematics
\usepackage{amsmath,amsfonts,amsthm,amssymb,mathtools}
\usepackage{xfrac}
\usepackage[makeroom]{cancel}
\usepackage{enumitem}
\usepackage{nameref}
\usepackage{multicol,array}
\usepackage{tikz-cd}
\usepackage{array}
\usepackage{multirow}% http://ctan.org/pkg/multirow
\usepackage{graphicx}

% Colors
\usepackage[dvipsnames]{xcolor}
\definecolor{myg}{RGB}{56, 140, 70}
\definecolor{myb}{RGB}{45, 111, 177}
\definecolor{myr}{RGB}{199, 68, 64}
% Define more colors here...
\definecolor{olive}{HTML}{6B8E23}
\definecolor{orange}{HTML}{CC5500}
\definecolor{brown}{HTML}{8B4513}
% Hyperlinks
\usepackage{bookmark}
\usepackage[colorlinks=true,linkcolor=blue,urlcolor=blue,citecolor=blue,anchorcolor=blue]{hyperref}
\usepackage{xcolor}
\hypersetup{
    colorlinks,
    linkcolor={red!50!black},
    citecolor={blue!50!black},
    urlcolor={blue!80!black}
}

% Text-related
\usepackage{blindtext}
\usepackage{fontsize}
\changefontsize[14]{14}
\setlength{\parindent}{0pt}
\linespread{1.2}

% Theorems and Definitions
\usepackage{amsthm}
\renewcommand\qedsymbol{$\blacksquare$}

% Define a new theorem style
\newtheoremstyle{mytheoremstyle}% name
  {}% Space above
  {}% Space below
  {}% Body font
  {}% Indent amount
  {\bfseries}% Theorem head font
  {.}% Punctuation after theorem head
  {.5em}% Space after theorem head
  {}% Theorem head spec (can be left empty, meaning ‘normal’)

% Apply the new theorem style to theorem-like environments
\theoremstyle{mytheoremstyle}

\newtheorem{theorem}{Theorem}[section]  
\newtheorem{definition}[theorem]{Definition} 
\newtheorem{lemma}[theorem]{Lemma}  
\newtheorem{corollary}[theorem]{Corollary}
\newtheorem{axiom}[theorem]{Axiom}
\newtheorem{example}[theorem]{Example}
\newtheorem{equiv_def}[theorem]{Equivalent Definition}

% tcolorbox Setup
\usepackage[most,many,breakable]{tcolorbox}
\tcbuselibrary{theorems}

% Define custom tcolorbox environments here...

%================================
% EXAMPLE BOX
%================================
% After you have defined the style and other theorem environments
\definecolor{myexamplebg}{RGB}{245, 245, 245} % Very light grey for background
\definecolor{myexamplefr}{RGB}{120, 120, 120} % Medium grey for frame
\definecolor{myexampleti}{RGB}{60, 60, 60}    % Darker grey for title

\newtcbtheorem[]{Example}{Example}{
    colback=myexamplebg,
    breakable,
    colframe=myexamplefr,
    coltitle=myexampleti,
    boxrule=1pt,
    sharp corners,
    detach title,
    before upper=\tcbtitle\par\vspace{-20pt}, % Reduced the space after the title
    fonttitle=\bfseries,
    description font=\mdseries,
    separator sign none,
    description delimiters={}{}, % No delimiters around the title
}{ex}
%================================
% Solution BOX
%================================
\makeatletter
\newtcolorbox{solution}{enhanced,
	breakable,
	colback=white,
	colframe=myg!80!black,
	attach boxed title to top left={yshift*=-\tcboxedtitleheight},
	title=Solution,
	boxed title size=title,
	boxed title style={%
			sharp corners,
			rounded corners=northwest,
			colback=tcbcolframe,
			boxrule=0pt,
		},
	underlay boxed title={%
			\path[fill=tcbcolframe] (title.south west)--(title.south east)
			to[out=0, in=180] ([xshift=5mm]title.east)--
			(title.center-|frame.east)
			[rounded corners=\kvtcb@arc] |-
			(frame.north) -| cycle;
		},
}
\makeatother

% %================================
% % Question BOX
% %================================
\makeatletter
\newtcbtheorem{question}{Question}{enhanced,
	breakable,
	colback=white,
	colframe=myb!80!black,
	attach boxed title to top left={yshift*=-\tcboxedtitleheight},
	fonttitle=\bfseries,
	title={#2},
	boxed title size=title,
	boxed title style={%
			sharp corners,
			rounded corners=northwest,
			colback=tcbcolframe,
			boxrule=0pt,
		},
	underlay boxed title={%
			\path[fill=tcbcolframe] (title.south west)--(title.south east)
			to[out=0, in=180] ([xshift=5mm]title.east)--
			(title.center-|frame.east)
			[rounded corners=\kvtcb@arc] |-
			(frame.north) -| cycle;
		},
	#1
}{question}
\makeatother

%%%%%%%%%%%%%%%%%%%%%%%%%%%%%%%%%%%%%%%%%%%
% TABLE OF CONTENTS
%%%%%%%%%%%%%%%%%%%%%%%%%%%%%%%%%%%%%%%%%%%


\usepackage{tikz}
\definecolor{doc}{RGB}{0,60,110}
\usepackage{titletoc}
\contentsmargin{0cm}
\titlecontents{chapter}[14pc]
{\addvspace{30pt}%
	\begin{tikzpicture}[remember picture, overlay]%
		\draw[fill=doc!60,draw=doc!60] (-7,-.1) rectangle (-0.9,.5);%
		\pgftext[left,x=-5.5cm,y=0.2cm]{\color{white}\Large\sc\bfseries Chapter\ \thecontentslabel};%
	\end{tikzpicture}\color{doc!60}\large\sc\bfseries}%
{}
{}
{\;\titlerule\;\large\sc\bfseries Page \thecontentspage
	\begin{tikzpicture}[remember picture, overlay]
		\draw[fill=doc!60,draw=doc!60] (2pt,0) rectangle (4,0.1pt);
	\end{tikzpicture}}%
\titlecontents{section}[3.7pc]
{\addvspace{2pt}}
{\contentslabel[\thecontentslabel]{3pc}}
{}
{\hfill\small \thecontentspage}
[]
\titlecontents*{subsection}[3.7pc]
{\addvspace{-1pt}\small}
{}
{}
{\ --- \small\thecontentspage}
[ \textbullet\ ][]

\makeatletter
\renewcommand{\tableofcontents}{
	\chapter*{%
	  \vspace*{-20\p@}%
	  \begin{tikzpicture}[remember picture, overlay]%
		  \pgftext[right,x=15cm,y=0.2cm]{\color{doc!60}\Huge\sc\bfseries \contentsname};%
		  \draw[fill=doc!60,draw=doc!60] (13,-.75) rectangle (20,1);%
		  \clip (13,-.75) rectangle (20,1);
		  \pgftext[right,x=15cm,y=0.2cm]{\color{white}\Huge\sc\bfseries \contentsname};%
	  \end{tikzpicture}}%
	\@starttoc{toc}}
\makeatother

\newcommand{\liff}{\llap{$\iff$}}
\newcommand{\rap}[1]{\rrap{\text{ (#1)}}}
\newcommand{\red}[1]{\textcolor{red}{#1}}
\newcommand{\blue}[1]{\textcolor{blue}{#1}}
\newcommand{\vi}[1]{\textcolor{violet}{#1}}
\newcommand{\olive}[1]{\textcolor{olive}{#1}}
\newcommand{\teal}[1]{\textcolor{teal}{#1}}
\newcommand{\brown}[1]{\textcolor{brown}{#1}}
\newcommand{\orange}[1]{\textcolor{orange}{#1}}
\newcommand{\tCaC}{\text{ \CaC }}
\newcommand{\CaC}{\red{CaC} }
\newcommand{\As}[1]{Assume \red{#1}}
\newcommand{\vdone}{\vi{\text{ (done) }}}
\newcommand{\bdone}{\blue{\text{ (done) }}}
\newcommand{\tdone}{\teal{\text{ (done) }}}
\newcommand{\odone}{\olive{\text{ (done) }}}
\newcommand{\bodone}{\brown{\text{ (done) }}}
\newcommand{\ordone}{\orange{\text{ (done) }}}
\newcommand{\ld}{\lambda}
\newcommand{\vecta}[1]{\textbf{#1}}
\newcommand{\set}[1]{\left\{ #1 \right\}}
\newcommand{\bset}[1]{\Big\{ #1 \Big\}}
\newcommand{\inR}{\in\R}
\newcommand{\inn}{\in\N}
\newcommand{\inz}{\in\Z}
\newcommand{\inr}{\in\R}
\newcommand{\inc}{\in\C}
\newcommand{\inq}{\in\Q}
\newcommand{\norm}[1]{\| #1 \|}
\newcommand{\bnorm}[1]{\Big\| #1 \Big\|}
\newcommand{\gen}[1]{\langle #1 \rangle}
\newcommand{\abso}[1]{\left|#1\right|}
\newcommand{\myref}[2]{\hyperref[#2]{#1\ \ref*{#2}}}
\newcommand{\customref}[2]{\hyperref[#1]{#2}}
\newcommand{\power}[1]{\mathcal{P}(#1)}
\newcommand{\dcup}{\mathbin{\dot{\cup}}}
\newcommand{\diam}[1]{\text{diam}\, #1}
\newcommand{\at}{\Big|}
\newcommand{\quotient}{\diagup}
\let\originalphi\phi % Store the original \phi in \originalphi
\renewcommand{\phi}{\varphi} % Redefine \phi to \varphi
\newcommand{\pfi}{\originalphi} % Define \pfi to display the original \phi
\newcommand{\diota}{\dot{\iota}}
\newcommand{\Log}{\operatorname{Log}}
\newcommand{\id}{\text{\textbf{id}}}
\usepackage{amsmath}

\makeatletter
\NewDocumentCommand{\extp}{e{^}}{%
  \mathop{\mathpalette\extp@{#1}}\nolimits
}
\NewDocumentCommand{\extp@}{mm}{%
  \bigwedge\nolimits\IfValueT{#2}{^{\extp@@{#1}#2}}%
  \IfValueT{#1}{\kern-2\scriptspace\nonscript\kern2\scriptspace}%
}
\newcommand{\extp@@}[1]{%
  \mkern
    \ifx#1\displaystyle-1.8\else
    \ifx#1\textstyle-1\else
    \ifx#1\scriptstyle-1\else
    -0.5\fi\fi\fi
  \thinmuskip
}
\makeatletter
\usepackage{pifont}
\makeatletter
\newcommand\Pimathsymbol[3][\mathord]{%
  #1{\@Pimathsymbol{#2}{#3}}}
\def\@Pimathsymbol#1#2{\mathchoice
  {\@Pim@thsymbol{#1}{#2}\tf@size}
  {\@Pim@thsymbol{#1}{#2}\tf@size}
  {\@Pim@thsymbol{#1}{#2}\sf@size}
  {\@Pim@thsymbol{#1}{#2}\ssf@size}}
\def\@Pim@thsymbol#1#2#3{%
  \mbox{\fontsize{#3}{#3}\Pisymbol{#1}{#2}}}
\makeatother
% the next two lines are needed to avoid LaTeX substituting upright from another font
\input{utxmia.fd}
\DeclareFontShape{U}{txmia}{m}{n}{<->ssub * txmia/m/it}{}
% you may also want
\DeclareFontShape{U}{txmia}{bx}{n}{<->ssub * txmia/bx/it}{}
% just in case
%\DeclareFontShape{U}{txmia}{l}{n}{<->ssub * txmia/l/it}{}
%\DeclareFontShape{U}{txmia}{b}{n}{<->ssub * txmia/b/it}{}
% plus info from Alan Munn at https://tex.stackexchange.com/questions/290165/how-do-i-get-a-nicer-lambda?noredirect=1#comment702120_290165
\newcommand{\pilambdaup}{\Pimathsymbol[\mathord]{txmia}{21}}
\renewcommand{\lambda}{\pilambdaup}
\renewcommand{\tilde}{\widetilde}
\DeclareMathOperator*{\esssup}{ess\,sup}
\newcommand{\bluecheck}{}%
\DeclareRobustCommand{\bluecheck}{%
  \tikz\fill[scale=0.4, color=blue]
  (0,.35) -- (.25,0) -- (1,.7) -- (.25,.15) -- cycle;%
}


\usepackage{tikz}
\newcommand*{\DashedArrow}[1][]{\mathbin{\tikz [baseline=-0.25ex,-latex, dashed,#1] \draw [#1] (0pt,0.5ex) -- (1.3em,0.5ex);}}

\newcommand{\C}{\mathbb{C}}	
\newcommand{\F}{\mathbb{F}}
\newcommand{\N}{\mathbb{N}}
\newcommand{\Q}{\mathbb{Q}}
\newcommand{\R}{\mathbb{R}}
\newcommand{\Z}{\mathbb{Z}}



\title{\Huge{NCKU 112.1}\\Apostol}
\author{\huge{Eric Liu}}
\date{}
\begin{document}
\maketitle
\newpage% or \cleardoublepage
% \pdfbookmark[<level>]{<title>}{<dest>}
\pdfbookmark[section]{\contentsname}{toc}
\tableofcontents
\pagebreak
\chapter{Introduction}
\section{Historical Introduction}
\subsection{Archimedes' method of exhaustion for the area of a parabolic segment}
Before we proceed to a systematic treatment of integral calculus, we first introduce Archimedes' method of exhaustion.\\

Given the curve $f(x)=x^2$, we want to find the "area" bounded by the $x$-axis, the curve, the line $x=0$ ant the line $x=b$.\\

We divide the base of the bounded segment into  $n$ equal part, each of length $\frac{b}{n}$. The points of subdivision correspond to the following values of $x$ :
\begin{equation}
  0,\frac{b}{n},\frac{2b}{n},\cdots ,\frac{(n-1)b}{n}, \frac{nb}{n}=b
\end{equation}
Now, we can draw the inner and outer rectangles. Let us denote by $S_n$ the sum of the areas of the outer rectangles and $s_n$ the sum of those of inner rectangles.\\

We compute
\begin{align}
  S_n &= \sum_{k=1}^{n} (\frac{b}{n})(\frac{kb}{n})^2 \quad\text{ (Notice  $(\frac{kb}{n})^2=f(\frac{kb}{n})$) }\\
      &= \frac{b^3}{n^3}\sum_{k=1}^{n}k^2\\
      &= \frac{b^3}{n^3}(\frac{n^3}{3}+\frac{n^2}{2}+\frac{n}{6})\\
\end{align}
and compute
\begin{align}
  s_n &= \frac{b^3}{n^3}\sum_{k=1}^{n-1}k^2\\
      &= \frac{b^3}{n^3}(\frac{n^3}{3}-\frac{n^2}{2}+\frac{n}{6})
\end{align} 

Notice that we haven't rigorously define the concept of "inner and outer" rectangle. So we can only define $s_n\text{ and }S_n$ as the value we just compute, and fill the gap between geometry truth and our definition with intuition.\\

Denote by $A$ the area of the segment. We then by our intuition say for all $n\in \N$, $s_n\leq A\leq S_n$. Although we can't compute $A$ the way we compute the area of a rectangle a triangle, we actually have a way to not just approximate  $A$, but to get the exact value of  $A$.
\begin{theorem} 
Given the sequences $S_n=\sum_{k=1}^{n}(\frac{b}{n})(\frac{kb}{n})^2$ and $s_n=\sum_{k=1}^{n-1}(\frac{b}{n})(\frac{kb}{n})^2$, we have
\begin{equation}
\forall n\in\N, s_n\leq A\leq S_n\iff A=\frac{b^3}{3}
\end{equation}
\end{theorem}
\begin{proof}
First we observe that for all $n\in \N$
\begin{gather}
  s_n\leq  A\leq S_n\\
\liff  \frac{b^3}{n^3}(\frac{n^3}{3}-\frac{n^2}{2}+\frac{n}{6})\leq A\leq  \frac{b^3}{n^3}(\frac{n^3}{3}+\frac{n^2}{2}+\frac{n}{6})\\ 
  \llap{$\iff $}\frac{n^3}{3}-\frac{n^2}{2}+\frac{n}{6} \leq (\frac{n^3}{b^3})A\leq  \frac{n^3}{3}+\frac{n^2}{2}+\frac{n}{6}
\end{gather} 
$(\longleftarrow)$\\

Observe
\begin{gather}
  \frac{n^3}{3}-\frac{n^2}{2}+\frac{n}{6} \leq \frac{n^3}{3}\leq \frac{n^3}{3}+\frac{n^2}{2}+\frac{n}{6} \\
  \llap{$\iff$}-\frac{n^2}{2}+\frac{n}{6} \leq 0\leq \frac{n^2}{2}+\frac{n}{6}
\end{gather} 
This is true and can be verified by induction.\\

$(\longrightarrow)$\\

We first show \vi{$S_n<\frac{b^3}{3}+\frac{b^3}{n}$}.
\begin{gather}
  \frac{1}{3n} < 1\\
 \liff  \frac{1}{6n^2}< \frac{1}{2n}\\
  \liff \frac{1}{2n}+\frac{1}{6n^2} <\frac{1}{n}\\
 \liff  \frac{b^3}{n^3}(\frac{n^2}{2}+\frac{n}{6})<\frac{b^3}{n} \\ \liff S_n=\frac{b^3}{n^3}(\frac{n^3}{3}+\frac{n^2}{2}+\frac{n}{6}) < \frac{b^3}{3}+\frac{b^3}{n}\text{ \vi{(done)} }
\end{gather} 

\As{$A>\frac{b^3}{3}$}. Find an $n$ such that $n>\frac{b^3}{A-\frac{b^3}{3}}$
\begin{gather}
n>\frac{b^3}{A-\frac{b^3}{3}}\\
\liff A-\frac{b^3}{3}>\frac{b^3}{n}\\
\liff A>\frac{b^3}{3}+\frac{b^3}{n}>S_n \text{ \CaC }
\end{gather}
We now show \blue{ $s_n>\frac{b^3}{3}-\frac{b^3}{n}$}
\begin{gather}
  -\frac{1}{2n} <\frac{1}{6n^2}\\
\liff   -\frac{1}{n}< \frac{1}{-2n}+\frac{1}{6n^2} \\
 \liff  -\frac{b^3}{n^3} < \frac{b^3}{-2n}+\frac{b^3}{6n^2} \\
\liff   \frac{b^3}{3}-\frac{b^3}{n}< \frac{b^3}{n^3}(\frac{n^3}{3}-\frac{n^2}{2}+\frac{n}{6})=s_n \text{ \blue{(done)} }
\end{gather} 
\As{$A<\frac{b^3}{3}$}. Find an $n$ such that $n>\frac{b^3}{\frac{b^3}{3}-A}$
\begin{gather}
n>\frac{b^3}{\frac{b^3}{3}-A}\\
\liff \frac{b^3}{3}-A>\frac{b^3}{n}\\
\liff s_n>\frac{b^3}{3}-\frac{b^3}{n}>A \text{ \CaC }
\end{gather}
\end{proof}
\subsection{Exercises}
\begin{question}{}{}
Given two sequences $s_n=\frac{b^4}{n^4}[1^3+\cdots +(n-1)^3]$, $S_n=\frac{b^4}{n^4}[1^3+\cdots+ n^3]$ and assuming inequality $1+\cdots +(n-1)^3<\frac{n^4}{4}<1+\cdots +n^3$.\\

Prove that
\begin{equation}
\forall n\in \N,s_n<A<S_n \iff A=\frac{b^4}{4}
\end{equation}
\end{question}

\begin{solution}
$(\longleftarrow)$\\

Observe
\begin{gather}
1+\cdots +(n-1)^3<\frac{n^4}{4}<1+\cdots +n^3 \\
\liff \frac{n^4}{b^4}(s_n)<\frac{n^4}{4}<\frac{n^4}{b^4}(S_n) \\
\liff s_n<\frac{b^4}{4}<S_n
\end{gather}
$(\longrightarrow)$\\

We first show  \vi{$S_n<\frac{b^4}{4}+\frac{b^4}{n}$}
\begin{gather}
1+\cdots +(n-1)^3<\frac{n^4}{4}\\
\liff 1+\cdots +n^3<\frac{n^4}{4}+n^3\\
\liff \frac{n^4}{b^4}(S_n)<\frac{n^4}{4}+n^3\\
\liff S_n<\frac{b^4}{4}+\frac{b^4}{n} \text{ \vi{(done)} }
\end{gather}
\As{$A>\frac{b^4}{4}$}. Find an $n$ such that $\frac{b^4}{A-\frac{b^4}{4}}<n$. Observe
\begin{gather}
\frac{b^4}{A-\frac{b^4}{4}}<n\\
\liff \frac{b^4}{n}<A-\frac{b^4}{4} \\
\liff S_n<\frac{b^4}{4}+\frac{b^4}{n}<A \text{ \CaC }
\end{gather}
We now show  \blue{$s_n>\frac{b^4}{4}-\frac{b^4}{n}$}
\begin{gather}
\frac{n^4}{4}<1+\cdots +n^3\\
\liff \frac{n^4}{4}-n^3<1+\cdots +(n-1)^3\\
\liff \frac{n^4}{4}-n^3<\frac{n^4}{b^4}(s_n)\\
\liff \frac{b^4}{4}-\frac{b^4}{n}<s_n \blue{\text{ (done) }} 
\end{gather}
\As{$A<\frac{b^4}{4}$}. Find an $n$ such that $\frac{b^4}{\frac{b^4}{4}-A}<n$. Observe
\begin{gather}
\frac{b^4}{\frac{b^4}{4}-A}<n\\
\liff \frac{b^4}{n}<\frac{b^4}{4}-A\\
\liff A<\frac{b^4}{4}-\frac{b^4}{n}<s_n \text{ \CaC }
\end{gather}
\end{solution}
\begin{question}{}{}
Use the same method to find the area when $f(x)=ax^3+c$
\end{question}
\begin{solution}
We first show \vi{$s_n=bc+\frac{b^4}{n^4}a(1+\cdots +(n-1)^3)$}. Observe
\begin{align}
  s_n&=\sum_{k=0}^{n-1}\frac{b}{n}f(\frac{bk}{n})\\
     &= \sum_{k=0}^{n-1} \frac{b}{n}(a \frac{b^3k^3}{n^3}+c) \\
     &= \sum_{k=0}^{n-1} \frac{b^4}{n^4}ak^3+\frac{bc}{n}\\
     &= bc+ \frac{b^4}{n^4}a \sum_{k=0}^{n-1}k^3 \\
     &= bc+\frac{b^4}{n^4}a(1+\cdots +(n-1)^3) \vi{\text{ (done) }}
\end{align}
Also, we see
\begin{align}
  S_n&= \sum_{k=1}^{n}\frac{b}{n}f(\frac{bk}{n})\\ 
&= bc+\frac{b^4}{n^4}a \sum_{k=1}^{n}k^3 \\
&\blue{= bc+\frac{b^4}{n^4}a(1+\cdots +n^3)\text{ (done) }}
\end{align}
Now we prove
\begin{equation}
\forall n\in \N, s_n<A<S_n\iff A=bc+\frac{ab^4}{4}
\end{equation}
$(\longleftarrow)$\\

Observe
\begin{gather}
1+\cdots +(n-1)^3<\frac{n^4}{4}<1+\cdots n^3\\
\liff \frac{b^4}{n^4}a(1+\cdots +(n-1)^3)<\frac{ab^4}{4}<\frac{b^4}{n^4}a(1+\cdots +n^3)\\
\liff s_n<bc+\frac{ab^4}{4}<S_n
\end{gather}
$(\longrightarrow)$\\

We first show \vi{$S_n<bc+\frac{ab^4}{4}+\frac{ab^4}{n}$} and \blue{$s_n>bc+\frac{ab^4}{4}-\frac{ab^4}{n}$}. Observe
\begin{gather}
1+\cdots +(n-1)^3<\frac{n^4}{4}\\
\liff 1+\cdots +n^3<\frac{n^4}{4}+n^3\\
\liff bc+\frac{b^4}{n^4}a(1+\cdots +n^3)<bc+\frac{b^4}{n^4}a(\frac{n^4}{4}+n^3)\\
\liff S_n<bc+\frac{ab^4}{4}+\frac{ab^4}{n} \vi{\text{ (done) }}
\end{gather}
and observe
\begin{gather}
\frac{n^4}{4}<1+\cdots +n^4\\
\liff \frac{n^4}{4}-n^3<1+\cdots +(n-1)^3\\
\liff bc+\frac{b^4}{n^4}a(\frac{n^4}{4}-n^3)<bc+\frac{b^4}{n^4}a(1+\cdots +(n-1)^3)\\
\liff bc+\frac{ab^4}{4}-\frac{ab^4}{n}<s_n\blue{\text{ (done) }}
\end{gather}
\As{$A>bc+\frac{ab^4}{4}$}, and find an $n$ such that $\frac{ab^4}{A-bc-\frac{ab^4}{4}}<n$. Observe
\begin{gather}
\frac{ab^4}{A-bc-\frac{ab^4}{4}}<n\\
\liff \frac{ab^4}{n}<A-bc-\frac{ab^4}{4}\\
\liff S_n<bc+\frac{ab^4}{4}+\frac{ab^4}{n}<A\tCaC
\end{gather}

\As{$A<bc+\frac{ab^4}{4}$}, and find an $n$ such that $\frac{ab^4}{bc+\frac{ab^4}{4}-A}<n$. Observe
\begin{gather}
\frac{ab^4}{bc+\frac{ab^4}{4}-A}<n\\
\liff \frac{ab^4}{n}<bc+\frac{ab^4}{4}-A\\
\liff A<bc+\frac{ab^4}{4}-\frac{ab^4}{n}<s_n\tCaC
\end{gather}
\end{solution}
\begin{question}{}{}
given inequalities 
\begin{equation}
\forall n,k\in \N,  1^k+\cdots +(n-1)^k<\frac{n^{k+1}}{k+1}<1^k+\cdots +n^k
\end{equation}
and sequences
\begin{equation}
s_n=\frac{b^{k+1}}{n^{k+1}}(1^k+\cdots +(n-1)^k)\text{ and }S_n=\frac{b^{k+1}}{n^{k+1}}(1^k+\cdots +n^k)
\end{equation}
Show that for all $k\inn$
\begin{equation}
\forall n\inn, s_n<A<S_n\iff A=\frac{b^{k+1}}{k+1}
\end{equation}
\end{question}
\\
\begin{solution}
  $(\longleftarrow)$\\
  
  
Observe that for all $n\inn$
\begin{gather}
1^k+\cdots +(n-1)^k<\frac{n^{k+1}}{k+1}<1^k+\cdots +n^k\\
\liff \frac{b^{k+1}}{n^{k+1}}(1^k+\cdots +(n-1)^k)<\frac{b^{k+1}}{n^{k+1}}\frac{n^{k+1}}{k+1}<\frac{b^{k+1}}{n^{k+1}}(1^k+\cdots +n^{k})\\
\liff s_n<\frac{b^{k+1}}{k+1}<S_n 
\end{gather}
$(\longrightarrow)$\\

We first show \vi{$S_n<\frac{b^{k+1}}{k+1}+\frac{b^{k+1}}{n}$} and \blue{$s_n>\frac{b^{k+1}}{k+1}-\frac{b^{k+1}}{n}$}. Observe that for all $n\inn$
\begin{gather}
1^k+\cdots +(n-1)^k<\frac{n^{k+1}}{k+1}\\
\liff 1^k+\cdots +n^k<\frac{n^{k+1}}{k+1}+n^k\\
\liff \frac{b^{k+1}}{n^{k+1}}(1^k+\cdots +n^k)<\frac{b^{k+1}}{n^{k+1}}(\frac{n^{k+1}}{k+1}+n^k)\\
\liff S_n< \frac{b^{k+1}}{k+1}+\frac{b^{k+1}}{n} \vdone 
\end{gather}
and
\begin{gather}
\frac{n^{k+1}}{k+1}<1+\cdots +n^k\\
\liff \frac{n^{k+1}}{k+1}-n^{k}<1+\cdots +(n-1)^k\\
\liff \frac{b^{k+1}}{n^{k+1}}(\frac{n^{k+1}}{k+1}-n^k)<\frac{b^{k+1}}{n^{k+1}}(1+\cdots +(n-1)^k)\\
\liff \frac{b^{k+1}}{k+1}-\frac{b^{k+1}}{n}<s_n \bdone
\end{gather}
\As{$A>\frac{b^{k+1}}{k+1}$}. We find an $n$ such that $\frac{b^{k+1}}{A-\frac{b^{k+1}}{k+1}}<n$. Observe
\begin{gather}
\frac{b^{k+1}}{A-\frac{b^{k+1}}{k+1}}<n\\
\liff \frac{b^{k+1}}{n}<A-\frac{b^{k+1}}{k+1}\\
\liff S_n<\frac{b^{k+1}}{n}+\frac{b^{k+1}}{k+1}<A \text{ \CaC }
\end{gather}
\As{$A<\frac{b^{k+1}}{k+1}$}. We find an $n$ such that $n>\frac{b^{k+1}}{\frac{b^{k+1}}{k+1}-A}$. Observe
\begin{gather}
\frac{b^{k+1}}{\frac{b^{k+1}}{k+1}-A}<n\\
\liff \frac{b^{k+1}}{n}<\frac{b^{k+1}}{k+1}-A\\
\liff A<\frac{b^{k+1}}{k+1}-\frac{b^{k+1}}{n}<s_n \tCaC 
\end{gather}
\end{solution}
\section{Some Basic Concepts of the Theory of Sets}
\subsection{Exercises}
\begin{question}{}{}
Use the roster notation to designate the following sets of real numbers.
\begin{enumerate}[label=(\alph*)]
  \item $A=\{x:x^2-1=0\}$ 
  \item $B=\{x:(x-1)^2=0\}$ 
  \item $C=\{x:x+8=9\}$
  \item $D=\{x:x^3-2x^2+x=2\}$ 
  \item $E=\{x:(x+8)^2=9^2\}$ 
  \item $F=\{x:(x^2+16x)^2=17^2\}$
\end{enumerate}
\end{question}
\begin{question}{}{}
For the sets in Exercise 1, note that $B\subseteq A$. List all the inclusion relation $\subseteq$ that hold among the sets $A,B,C,D,E,F$
\end{question}
\begin{question}{}{}
Let $A=\set{1},B=\set{1,2}$. Discuss the validity of the following statements
\begin{enumerate}[label=(\alph*)]
  \item $A\subset B$
    \item $1\in A$
      \item $A\subseteq B$ 
        \item $1\subseteq A$
          \item $A\in B$
            \item $1 \subset B$\\
\end{enumerate}
\end{question}
\begin{question}{}{}
Solve Exercise 3 if $A=\set{1}$ and $B=\set{\set{1},1}$
\end{question}
\begin{question}{}{}
Given the set $S-\set{1,2,3,4}$. Display all subsets of $S$.
\end{question}
\section{A Set of Axioms  for the Real-Number System}
\subsection{The order axioms}
The below axiom is to define the positive real number set, and we will use the positive real number set to define order.
\begin{axiom}
  \begin{equation}
  x\inr^+ \text{ and }y\inr^+\implies x+y\inr^+ \text{ and } xy\inr^+
  \end{equation}
\end{axiom}
\begin{axiom}
If $x\neq 0$, either $x\inr^+$ or $-x\inr^+$, but not both.
\end{axiom}
\begin{axiom}
$0\not\inr^+$
\end{axiom}
\begin{definition}
  \begin{equation}
  \text{  $x$ is \textbf{positive} }\iff x\inr^+
  \end{equation}
\end{definition}
\begin{definition}
\begin{equation}
\text{  $x$ is \textbf{negative} }\iff -x\inr^+
\end{equation}
\end{definition}
\begin{theorem}
 \begin{equation}
 x\text{ is \textbf{positive} }\iff -x\text{ is \textbf{negative} }
 \end{equation}
\end{theorem}
\begin{corollary}
\begin{equation}
x\text{ is \textbf{negative} }\iff -x\text{ is \textbf{positive} }
\end{equation}
\end{corollary}
\begin{theorem}
 If $x$ is a real number, then either $x=0$ or $x$ is positive or $x$ is negative. Only one of them hold true.
\end{theorem}
\begin{definition}
  \begin{equation}
  \text{  $x$ is \textbf{less than}  $y$} \iff x<y\iff y-x \text{ is positive }
  \end{equation}
\end{definition}
\begin{theorem}
\begin{equation}
x\text{ is \textbf{positive}}\iff 0<x
\end{equation} 
\end{theorem}
\begin{corollary}
\begin{equation}
  \text{  $x$ is\textbf{ negative}}\iff x<0
\end{equation}
\end{corollary}
\begin{definition}
  \begin{equation}
 y\text{ is \textbf{greater than} }x\iff y>x\iff x<y \iff x\text{ is less than $y$ }
  \end{equation}
\end{definition}
\begin{corollary}
\begin{equation}
x\text{ is \textbf{positive} }\iff x>0
\end{equation}
\end{corollary}
\begin{corollary}
\begin{equation}
x\text{ is \textbf{negative} }\iff 0>x
\end{equation}
\end{corollary}
\begin{definition}
  \begin{equation}
\text{ $x$ is \textbf{less than or equal to} $y$ }\iff x\leq y\iff x<y\text{ or }x=y
  \end{equation}
\end{definition} 
\begin{definition}
  \begin{equation}
  y\text{ is  \textbf{greater than or equal to} }x\iff y\geq x\iff y>x\text{ or }y=x
  \end{equation}
\end{definition}
\begin{theorem}
\begin{equation}
x\leq y\iff y\geq x
\end{equation} 
\end{theorem}
\begin{definition}
We say $x$ is \textbf{nonnegative} if $0\leq x$
\end{definition}
\begin{theorem}
If  $a,b$ are two real numbers, then exactly one of the three relations $a<b$, $a=b$, $a>b$ holds. 
\end{theorem}
\begin{proof}
Let $x=b-a$. Either $x=0$ or $x\neq 0$, but not both. If $x=0$, then $b-a=0$, then $b=a$. If $x\neq 0$, then either $x>0$ or $x<0$, but not both. If $x>0$, then $b-a>0$, then $a<b$. If $x<0$, then $b-a<0$, then $a>b$.
\end{proof}
\begin{theorem}
\begin{equation}
a<b\text{ and }b<c\implies a<c
\end{equation}
\end{theorem}
\begin{proof}
 $a<b\text{ and }b<c\implies b-a>0\text{ and }c-b>0\implies (b-a)+(c-b)>0\implies c-a>0\implies a<c$ 
\end{proof}
\begin{theorem}
\begin{equation}
a<b\implies a+c<b+c
\end{equation}
\end{theorem}
\begin{proof}
 $a<b\implies b-a>0\implies (b+c)-(a+c)>0\implies a+c<b+c$ 
\end{proof}
\begin{theorem}
\begin{equation}
a<b\text{ and }c>0\implies ac<bc
\end{equation}
\end{theorem}
\begin{proof}
 $a<b\implies b-a>0\text{ and }c>0\implies c(b-a)>0\implies bc-ac>0\implies ac<bc$
\end{proof}
\begin{theorem}
\begin{equation}
a\neq 0\implies a^2>0
\end{equation}
\end{theorem}
\begin{proof}
  $a\neq 0\implies a>0\text{ or }a<0\implies a^2>0\text{ or }-a>0\implies a^2>0\text{ or }(-a)^2>0\implies a^2>0$
\end{proof}
\begin{theorem}
\begin{equation}
1>0
\end{equation}
\end{theorem}
\begin{proof}
\As{ $1<0$}, so $-1>0$. Arbitrarily pick a positive real number $a$. $a>0\text{ and }-a=(-1)a>0\tCaC$
\end{proof}
\begin{theorem}
\begin{equation}
a<b\text{ and }c<0\implies ac>bc
\end{equation}
\end{theorem}
\begin{proof}
 $a<b\text{ and }c<0\implies b-a>0\text{ and }-c>0\implies -c(b-a)>0\implies ac-bc>0\implies ac>bc$
\end{proof}
\begin{theorem}
\begin{equation}
a<b\implies -a>-b
\end{equation}
\end{theorem}
\begin{proof}
 $a<b\implies b-a>0\implies -a-(-b)>0\implies -a>-b$
\end{proof}
\begin{corollary}
\begin{equation}
a<0\implies -a>0
\end{equation}
\end{corollary}
\begin{theorem}
\begin{equation}
ab>0\implies a,b>0\text{ or }a,b<0
\end{equation}
\begin{proof}
  If $a>0$, \As{ $b<0$}, then $-ab=a(-b)>0\tCaC$. If $a<0$, \As{  $b>0$}, then $-ab=(-a)b>0\tCaC$
\end{proof}
\end{theorem}
\begin{theorem}
\begin{equation}
a<c\text{ and }b<d\implies a+b<c+d
\end{equation}
\end{theorem}
\begin{proof}
 $a<c\text{ and }b<d\implies c-a>0\text{ and }d-b>0\implies c+d-(a+b)>0\implies a+b<c+d$
\end{proof}
\subsection{The Least-upper-bound Axiom}
\begin{definition}
Let $S$ be a set of real numbers and  $x$ be a real number.
\begin{equation}
 x\text{ is an \textbf{upper bound} of  }S\iff S\text{ is \textbf{bounded above} by }x\iff\forall y\in S, y\leq x  
\end{equation}
\begin{equation}
S\text{ is \textbf{unbounded above} }\iff \forall x\inr,\exists y\in S, x\leq y
\end{equation}
\begin{equation}
 x\text{ is an \textbf{lower bound} of  }S\iff S\text{ is \textbf{bounded below} by }x\iff\forall y\in S, y\geq  x  
\end{equation}
\begin{equation}
S\text{ is \textbf{unbounded below} }\iff \forall x\inr,\exists y\in S, x\geq  y
\end{equation}
\begin{equation}
s=\max S\iff s\text{ is the \textbf{maximum element} of }S\iff s\in S\text{ and }\forall y\in S,y\leq s
\end{equation}
\begin{equation}
s=\min S\iff s\text{ is the \textbf{minimum element} of }S\iff s\in S\text{ and }\forall y\in S,y\geq s
\end{equation}
\end{definition}
\begin{definition}
    \begin{multline}
     x=\sup S\iff x\text{ is an \textbf{least upper bound} of }S\iff x\text{ is an \textbf{supremum} of }S\iff \\
 x\text{ is an upper bound of }S\text{ and no number less than $x$ is an upper bound of $S$}\\
 \iff \forall y\in S, y\leq x\text{ and }\forall z<x,\exists y\in S, z<y
      \end{multline}
\end{definition}
\begin{definition}
    \begin{multline}
x=\inf S\iff  x\text{ is an \textbf{greatest lower bound} of }S\iff x\text{ is an \textbf{infimum} of }S\iff \\
 x\text{ is an lower bound of }S\text{ and no number greater than $x$ is an lower bound of $S$}\\
 \iff \forall y\in S, y\geq x\text{ and }\forall z>x,\exists y\in S, z> y
      \end{multline}
\end{definition}
\begin{theorem}
An bounded above set $S$ have exactly one least upper bound
\end{theorem}
\begin{proof}
\As{ $x$ and $y$ are two different least upper bound of $S$}. WOLG, let $x<y$. Because $y$ is an least upper bound of $S$, we know $\exists s\in S, x< s$ \CaC to that $x$ is an upper bound of $S$
\end{proof}
\begin{corollary}
An bounded below set $S$ have exactly one greatest lower bound.
\end{corollary}
\begin{axiom}
\textbf{(Real Numbers Set is a completed order filed)} Every nonempty set $S$ of real numbers which is bounded above has a supremum; that is, there is a real number $B$ such that $B=\sup S$.
\end{axiom}
\begin{theorem}
Every nonempty set $S$ of real number  which is bounded below has an infimum
\end{theorem}
\begin{proof}
  Define $-S:=\set{-s:s\inS}$. We know $-S$ is nonempty since $S$ is nonempty. So by completeness axiom, there exists a real number $B$ such that $\forall x\in -S, x\leq B$ and $\forall y<B,\exists x\in -S, y<x$. Then we know $\forall x\inS,-x\leq B$, which implies $\forall x\inS, x\geq -B$; that is, $-B$ is an lower bound of $S$. Also we know  $\forall y>-B,\exists x\inS,x<y$, which implies that $-B$ is an infimum of $-S$
\end{proof}
\end{document}
