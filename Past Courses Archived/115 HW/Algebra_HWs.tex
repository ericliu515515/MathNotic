\documentclass{report}
%%%%%%%%%%%%%% macros.tex %%%%%%%%%%%%%%
% Place your custom macros here, if any.

%%%%%%%%%%%%%% letterfonts.tex %%%%%%%%%%%%%%
% Place your font setup here, if any.

%%%%%%%%%%%%%% preamble.tex %%%%%%%%%%%%%%
\usepackage[T1]{fontenc}
\usepackage{lmodern}
\usepackage{etoolbox}
\usepackage{pdfpages}
\usepackage{transparent}
\usepackage[utf8]{inputenc}
\usepackage[english]{babel}

% Page Setup
\usepackage[tmargin=2cm, rmargin=0.5in, lmargin=0.5in, bmargin=80pt, footskip=.2in]{geometry}

% Mathematics
\usepackage{amsmath,amsfonts,amsthm,amssymb,mathtools}
\usepackage{xfrac}
\usepackage[makeroom]{cancel}
\usepackage{enumitem}
\usepackage{nameref}
\usepackage{multicol,array}
\usepackage{tikz-cd}
\usepackage[ruled,vlined,linesnumbered]{algorithm2e}

% Colors
\usepackage[dvipsnames]{xcolor}
\definecolor{myg}{RGB}{56, 140, 70}
\definecolor{myb}{RGB}{45, 111, 177}
\definecolor{myr}{RGB}{199, 68, 64}
% Define more colors here...

% Hyperlinks
\usepackage{bookmark}
\usepackage{hyperref}
\hypersetup{
    pdftitle={Assignment},
    colorlinks=true, linkcolor=doc!90,
    bookmarksnumbered=true,
    bookmarksopen=true
}

% Figures and Graphics
\usepackage{import}
\usepackage{svg}
\newcommand{\incfig}[1]{%
    \def\svgwidth{\columnwidth}
    \import{./figures/}{#1.pdf_tex}
}

% Text-related
\usepackage{blindtext}
\usepackage{fontsize}
\changefontsize[14]{14}
\setlength{\parindent}{0pt}

% Theorems and Definitions
\usepackage{amsthm}
\renewcommand\qedsymbol{$\blacksquare$}

% Define a new theorem style
\newtheoremstyle{mytheoremstyle}% name
  {}% Space above
  {}% Space below
  {\sffamily}% Body font
  {}% Indent amount
  {\bfseries}% Theorem head font
  {.}% Punctuation after theorem head
  {.5em}% Space after theorem head
  {}% Theorem head spec (can be left empty, meaning ‘normal’)

% Apply the new theorem style to theorem-like environments
\theoremstyle{mytheoremstyle}
\newtheorem{theorem}{Theorem}[section]
\newtheorem{definition}{Definition}[section]
\newtheorem{corollary}{Corollary}[section]
\newtheorem{lemma}{Lemma}[section]
\newtheorem{axiom}{Axiom}[section]

% tcolorbox Setup
\usepackage[most,many,breakable]{tcolorbox}

% Define custom tcolorbox environments here...

%================================
% EXAMPLE BOX
%================================
\newtcbtheorem[definition]{Example}{Example}
{%
    colback = myexamplebg,
    breakable,
    colframe = myexamplefr,
    coltitle = myexampleti,
    boxrule = 1pt,
    sharp corners,
    detach title,
    before upper=\tcbtitle\par\smallskip,
    fonttitle = \bfseries,
    description font = \mdseries,
    separator sign none,
    description delimiters parenthesis,
}
{ex}

%================================
% Solution BOX
%================================
\makeatletter
\newtcolorbox{solution}{enhanced,
	breakable,
	colback=white,
	colframe=myg!80!black,
	attach boxed title to top left={yshift*=-\tcboxedtitleheight},
	title=Solution,
	boxed title size=title,
	boxed title style={%
			sharp corners,
			rounded corners=northwest,
			colback=tcbcolframe,
			boxrule=0pt,
		},
	underlay boxed title={%
			\path[fill=tcbcolframe] (title.south west)--(title.south east)
			to[out=0, in=180] ([xshift=5mm]title.east)--
			(title.center-|frame.east)
			[rounded corners=\kvtcb@arc] |-
			(frame.north) -| cycle;
		},
}
\makeatother

%================================
% Question BOX
%================================
\makeatletter
\newtcbtheorem{question}{Question}{enhanced,
	breakable,
	colback=white,
	colframe=myb!80!black,
	attach boxed title to top left={yshift*=-\tcboxedtitleheight},
	fonttitle=\bfseries,
	title={#2},
	boxed title size=title,
	boxed title style={%
			sharp corners,
			rounded corners=northwest,
			colback=tcbcolframe,
			boxrule=0pt,
		},
	underlay boxed title={%
			\path[fill=tcbcolframe] (title.south west)--(title.south east)
			to[out=0, in=180] ([xshift=5mm]title.east)--
			(title.center-|frame.east)
			[rounded corners=\kvtcb@arc] |-
			(frame.north) -| cycle;
		},
	#1
}{def}
\makeatother
\makeatletter
\newtcbtheorem{qstion}{Question}{enhanced,
    breakable,
    colback=white,
    colframe=mygr,
    attach boxed title to top left={yshift*=-\tcboxedtitleheight},
    fonttitle=\bfseries,
    title={#2},
    boxed title size=title,
    boxed title style={%
        sharp corners,
        rounded corners=northwest,
        colback=tcbcolframe,
        boxrule=0pt,
    },
    underlay boxed title={%
        \path[fill=tcbcolframe] (title.south west)--(title.south east)
        to[out=0, in=180] ([xshift=5mm]title.east)--
        (title.center-|frame.east)
        [rounded corners=\kvtcb@arc] |-
        (frame.north) -| cycle;
    },
    #1
}{def}
\makeatother

%%%%%%%%%%%%%%%%%%%%%%%%%%%%%%%%%%%%%%%%%%%
% TABLE OF CONTENTS
%%%%%%%%%%%%%%%%%%%%%%%%%%%%%%%%%%%%%%%%%%%
\usepackage{tikz}
\definecolor{doc}{RGB}{0,60,110}
\usepackage{titletoc}
\contentsmargin{0cm}
\titlecontents{chapter}[14pc]
{\addvspace{30pt}%
	\begin{tikzpicture}[remember picture, overlay]%
		\draw[fill=doc!60,draw=doc!60] (-7,-.1) rectangle (-0.9,.5);%
		\pgftext[left,x=-4.5cm,y=0.2cm]{\color{white}\Large\sc\bfseries Chapter\ \thecontentslabel};%
	\end{tikzpicture}\color{doc!60}\large\sc\bfseries}%
{}
{}
{\;\titlerule\;\large\sc\bfseries Page \thecontentspage
	\begin{tikzpicture}[remember picture, overlay]
		\draw[fill=doc!60,draw=doc!60] (2pt,0) rectangle (4,0.1pt);
	\end{tikzpicture}}%
\titlecontents{section}[3.7pc]
{\addvspace{2pt}}
{\contentslabel[\thecontentslabel]{2pc}}
{}
{\hfill\small \thecontentspage}
[]
\titlecontents*{subsection}[3.7pc]
{\addvspace{-1pt}\small}
{}
{}
{\ --- \small\thecontentspage}
[ \textbullet\ ][]

\makeatletter
\renewcommand{\tableofcontents}{
	\chapter*{%
	  \vspace*{-20\p@}%
	  \begin{tikzpicture}[remember picture, overlay]%
		  \pgftext[right,x=15cm,y=0.2cm]{\color{doc!60}\Huge\sc\bfseries \contentsname};%
		  \draw[fill=doc!60,draw=doc!60] (13,-.75) rectangle (20,1);%
		  \clip (13,-.75) rectangle (20,1);
		  \pgftext[right,x=15cm,y=0.2cm]{\color{white}\Huge\sc\bfseries \contentsname};%
	  \end{tikzpicture}}%
	\@starttoc{toc}}
\makeatother

\newcommand{\liff}{\llap{$\iff$}}
\newcommand{\rap}[1]{\rrap{\text{ (#1)}}}
\newcommand{\red}[1]{\textcolor{red}{#1}}
\newcommand{\blue}[1]{\textcolor{blue}{#1}}
\newcommand{\vi}[1]{\textcolor{violet}{#1}}
\newcommand{\teal}[1]{\textcolor{teal}{#1}}
\newcommand{\tCaC}{\text{ \CaC }}
\newcommand{\CaC}{\red{CaC} }
\newcommand{\As}[1]{Assume \red{#1}}
\newcommand{\vdone}{\vi{\text{ (done) }}}
\newcommand{\bdone}{\blue{\text{ (done) }}}
\newcommand{\tdone}{\teal{\text{ (done) }}}
\newcommand{\set}[1]{\{ #1 \}}
\newcommand{\inS}{\in S}
\newcommand{\inF}{\in\F}
\newcommand{\inE}{\in E}
\newcommand{\inA}{\in A}
\newcommand{\inB}{\in B}
\newcommand{\inC}{\in C}
\newcommand{\inU}{\in U}

\newcommand{\C}{\mathbb{C}}	
\renewcommand{\H}{\mathbb{H}}
\newcommand{\F}{\mathbb{F}}
\newcommand{\N}{\mathbb{N}}
\newcommand{\Q}{\mathbb{Q}}
\newcommand{\R}{\mathbb{R}}
\newcommand{\Z}{\mathbb{Z}}
\renewcommand{\P}{\mathbb{P}}
\renewcommand{\S}{\mathbb{S}}
\newcommand{\A}{\mathbb{A}}
\newcommand{\RP}{\R P}

%================================
% Abstract 
%================================

\usepackage{lipsum}

\renewenvironment{abstract}
 {% Start of environment
  \quotation
  \small
  \noindent
  \rule{\linewidth}{.5pt} % Draw the rule to match the linewidth
  \par\smallskip
  {\centering\bfseries\abstractname\par}\medskip
 }
 {% End of environment
  \par\noindent
  \rule{\linewidth}{.5pt} % Ensure the closing rule also matches
  \endquotation
 }


%================================
% Question BOX
%================================

\newcounter{qnum} % The counter name for Question box is qnum

\makeatletter
\newtcbtheorem[use counter=qnum]{question}{Question}{enhanced,
	breakable,
	colback=white,
	colframe=myb!80!black,
	attach boxed title to top left={yshift*=-\tcboxedtitleheight},
	fonttitle=\bfseries,
	title={#2},
	boxed title size=title,
	boxed title style={%
			sharp corners,
			rounded corners=northwest,
			colback=tcbcolframe,
			boxrule=0pt,
		},
	underlay boxed title={%
			\path[fill=tcbcolframe] (title.south west)--(title.south east)
			to[out=0, in=180] ([xshift=5mm]title.east)--
			(title.center-|frame.east)
			[rounded corners=\kvtcb@arc] |-
			(frame.north) -| cycle;
		},
	#1
}{def}
\makeatother
\makeatletter
\newtcbtheorem{qstion}{Question}{enhanced,
    breakable,
    colback=white,
    colframe=mygr,
    attach boxed title to top left={yshift*=-\tcboxedtitleheight},
    fonttitle=\bfseries,
    title={#2},
    boxed title size=title,
    boxed title style={%
        sharp corners,
        rounded corners=northwest,
        colback=tcbcolframe,
        boxrule=0pt,
    },
    underlay boxed title={%
        \path[fill=tcbcolframe] (title.south west)--(title.south east)
        to[out=0, in=180] ([xshift=5mm]title.east)--
        (title.center-|frame.east)
        [rounded corners=\kvtcb@arc] |-
        (frame.north) -| cycle;
    },
    #1
}{def}
\makeatother


% COMMUTATIVE DIAGRAM
%%%%%%%%%%%%%%%%%%%%%%%%%%%%%%%%%%%%%%%%%%%


\usepackage{tikz}

\usetikzlibrary{arrows} % Needed for custom arrow tips
\usetikzlibrary{arrows.meta} % Needed for custom arrow tips

\newcommand*{\DashedArrow}[1][]{\mathbin{\tikz [baseline=-0.25ex,-latex, dashed,#1] \draw [#1] (0pt,0.5ex) -- (1.3em,0.5ex);}}


\tikzcdset{
  doubletail/.style={
    arrows={tail reversed-tail}
  }
}
\tikzstyle{double_arrow} = [thick,<->,>=stealth]

\tikzset{
  broken double arrow/.style={
    thick,                    % line width
    dashed,                   % make it broken
    <->,                      % arrowheads at both ends
    >=Stealth[length=2pt,width=1pt] % thinner arrowheads           
  }
}




% TABLE OF CONTENTS
%%%%%%%%%%%%%%%%%%%%%%%%%%%%%%%%%%%%%%%%%%
\definecolor{doc}{RGB}{0,60,110}
\usepackage{titletoc}
\contentsmargin{0cm}
\titlecontents{chapter}[14pc]
{\addvspace{30pt}%
	\begin{tikzpicture}[remember picture, overlay]%
		\draw[fill=doc!60,draw=doc!60] (-7,-.1) rectangle (-0.9,.5);%
		\pgftext[left,x=-5.5cm,y=0.2cm]{\color{white}\Large\sc\bfseries Chapter\ \thecontentslabel};%
	\end{tikzpicture}\color{doc!60}\large\sc\bfseries}%
{}
{}
{\;\titlerule\;\large\sc\bfseries Page \thecontentspage
	\begin{tikzpicture}[remember picture, overlay]
		\draw[fill=doc!60,draw=doc!60] (2pt,0) rectangle (4,0.1pt);
	\end{tikzpicture}}%
\titlecontents{section}[3.7pc]
{\addvspace{2pt}}
{\contentslabel[\thecontentslabel]{3pc}}
{}
{\hfill\small \thecontentspage}
[]
\titlecontents*{subsection}[3.7pc]
{\addvspace{-1pt}\small}
{}
{}
{\ --- \small\thecontentspage}
[ \textbullet\ ][]

\makeatletter
\renewcommand{\tableofcontents}{
	\chapter*{%
	  \vspace*{-20\p@}%
	  \begin{tikzpicture}[remember picture, overlay]%
		  \pgftext[right,x=15cm,y=0.2cm]{\color{doc!60}\Huge\sc\bfseries \contentsname};%
		  \draw[fill=doc!60,draw=doc!60] (13,-.75) rectangle (20,1);%
		  \clip (13,-.75) rectangle (20,1);
		  \pgftext[right,x=15cm,y=0.2cm]{\color{white}\Huge\sc\bfseries \contentsname};%
	  \end{tikzpicture}}%
	\@starttoc{toc}}
\makeatother

\title{Suns}
\author{Eric Liu}
\date{}
\begin{document}
\maketitle
\newpage % or \cleardoublepage
% \pdfbookmark[<level>]{<title>}{<dest>}
\pdfbookmark[section]{\contentsname}{toc}

\tableofcontents
\pagebreak
\chapter{HWs}
\section{HW1}
For question 1, recall that by class equation, $p$-group has nontrivial center. 
\begin{question}{}{}
Show that 
\begin{enumerate}[label=(\roman*)]
  \item If $H \quotient Z(H)$ is cyclic, then $H$ is abelian.  
  \item If $H$ is of order  $p^2$, then  $H$ is abelian.   
\end{enumerate}
From now on, suppose $G$ is non-abelian with order  $p^3$. 
\begin{enumerate}[label=(\roman*), start=3]
  \item $o(Z(G))=p$. 
  \item $Z(G)=G^{(1)}$.  
\end{enumerate}
\end{question}
\begin{proof}
Let $a,b\in H$ and $H \quotient Z(H)\triangleq  \langle hZ(H)\rangle $. Write $a\triangleq h^nz_1$ and $b\triangleq h^mz_2$. Because  $z_1,z_2 \in Z(H)$, we may compute: 
\begin{align*}
ab=h^nz_1h^mz_2=h^{n+m}z_1z_2=ba
\end{align*}
as desired. Let $o(H)=p^2$. Because $H$ is a $p$-group, we know the center of $H$ is nontrivial. Therefore  the order of its center is $\in \set{p,p^2,}$. To see that its order isn't $p$, just observe that if so, then by (i), $H$ is abelian, which contradicts to  $o(Z(H))=p$. We have shown $o(Z(H))=p^2=o(H)$, so $H$ is abelian. \\

Because $G$ is non-abelian with order $p^3$, we know $o(Z(G))\in \set{p,p^2}$. Part (i) tell us that $o(Z(G))\neq p^2$, so $o(Z(G))=p$. \\



We now prove $Z(G)=G^{(1)}$. Because $o(Z(G))=p$, by part (ii) we know $G\quotient Z(G)$ is abelian, which implies $G^{(1)} \leq  Z(G)$, which implies $G^{(1)}$ is either trivial or equal to $Z(G)$. Because $G$ is non-abelian, we know  $G^{(1)}$ is nontrivial. Therefore $G^{(1)}=Z(G)$, as desired. 
\end{proof}
\begin{question}{}{}
\begin{enumerate}[label=(\roman*)]
  \item Let $M,N$ be two normal subgroups of  $G$ with  $MN=G$. Prove that 
\begin{align*}
    G\quotient (M\cap N) \cong  (G\quotient M) \times (G\quotient N)
\end{align*} 
\item Let $H,K$ be two distinct subgroups of  $G$ of index  $2$. Prove that  $H\cap K$ is a normal subgroup with index $4$ and  $G \quotient (H\cap K)$ is not cyclic.  
\end{enumerate}
\end{question}
\begin{proof}
The map $G\quotient (M\cap N) \rightarrow (G \quotient M)\times (G\quotient N)$ defined by 
\begin{align}
\label{EQmg}
g(M\cap N) \mapsto (gM,gN)
\end{align}
is clearly a well-defined group homomorphism, since if $gM=hM$ and  $gN=hN$, then  $gh^{-1}\in M$ and $gh^{-1} \in N$, which implies $gh^{-1}\in M \cap N$, which implies $g(M\cap N)=h(M \cap N)$. Let $gM=M$ and  $gN=N$. Then  $g \in M \cap N$ and $g(M \cap N)=M \cap N$. Therefore \myref{map}{EQmg} is also injective. It remains to show \myref{map}{EQmg} is surjective. Fix $g,h \in G$. Write $g=mn$ and  $h=\tilde{m}\tilde{n}$. Clearly $gM=nM=\tilde{m}nM$ and  $hN=\tilde{m}N=\tilde{m}nN $. This implies that  \myref{mapping}{EQmg} maps $\tilde{m}n$ to $(gM,hN)$, as desired.\\

Because $H,K$ are both of index  $2$ in $G$, we know they are both normal in $G$. This by second isomorphism theorem implies  $HK$ forms a subgroup of  $G$. Because $H\neq K$, we know $HK$ properly contains $H$, which by finiteness of $G$ implies  the index of $HK$ is strictly less than  $H$, i.e., $HK=G$. Note that $H \cap K$ is normal since it is the intersection of normal subgroups. By part (i), we now have $G \quotient (H\cap K)\cong  (G \quotient H)\times (G\quotient K)\cong  \Z_2\times\Z_2$, which shows that $H \cap K$ has index $4$ and  $G \quotient (H\cap K)$ is cyclic.     
\end{proof}
\begin{question}{}{}
Let $G$ be a group of order  $pq$, where  $p>q$ are prime. 
\begin{enumerate}[label=(\roman*)]
  \item Show that there exists a unique subgroup of order $p$.  
  \item Suppose $a \in G$ with $o(a)=p$. Show that $\langle a\rangle \subseteq G$ is normal and for all $x \in G$, we have $x^{-1}ax =a^i$ for some $0<i<p$. 
\end{enumerate}
\end{question}
\begin{proof}
The third Sylow theorem stated that the number $n_p$ of Sylow $p$-subgroups satisfies 
\begin{align*}
n_p\equiv 1\pmod{p} \quad \text{ and }\quad n_p \mid q
\end{align*}
Because $p>q$, together they implies $n_p=1$. Since Sylow $p$-subgroups of  $G$ are exactly subgroups of order  $p$, we have proved (i).  \\

The third Sylow theorem also stated that $n_p= \abso{G:N_G(P)}$ for any Sylow $p$-subgroup  $P\leq G$. Therefore, $N_G(\langle a\rangle )=G$, i.e., $\langle a\rangle $ is normal in $G$. Fix  $x\in G$. It remains to prove $xax^{-1}\neq e$, which is a consequence of the fact that conjugacy (automorphism) preserves order.
\end{proof}
\begin{question}{}{}
Let $H,K$ be two subgroups of $G$ of finite indices  $m,n$. Show that 
\begin{align*}
\operatorname{lcm}(m,n) \leq \abso{G:H \cap K} \leq mn
\end{align*}
\end{question}
\begin{proof}
Let $\Omega_{H\cap K},\Omega_H$, and $\Omega_K$ respectively denote the set of left cosets of $H\cap K,H$, and $K$.  The map $\Omega_{H\cap K} \rightarrow \Omega_H \times \Omega_K$ defined by 
\begin{align}
\label{EQmgh}
g(H\cap K) \mapsto (gH,gK) 
\end{align}
is well defined since 
\begin{align*}
g(H\cap K)=l(H\cap K) \implies g^{-1}l \in H \cap K \implies gH=lH\text{ and }gK=lK
\end{align*}

such set map is injective since if $gH=lH$ and  $gK=lK$, then  $g^{-1}l \in H$ and $g^{-1}l\in K$, which implies $g(H \cap K)=l(H \cap K)$, as desired. From the injectivity of \myref{map}{EQmgh}, we have shown  index of  $H\cap K$ indeed have upper bound $mn$. \\


Because 
\begin{align*}
\abso{G:H\cap K}=\abso{G:H}\cdot \abso{H:H\cap K}=\abso{G:K} \cdot \abso{K: H \cap K} 
\end{align*}
we know both  $n$ and  $m$ divides  $\abso{G:H \cap K}$, which gives the desired lower bound $\operatorname{lcm}(m,n)$. 
\end{proof}
\begin{question}{}{}
\begin{enumerate}[label=(\roman*)]
  \item Let $G$ be a group, $H\leq G$, and $x\in G$ of finite order. Prove that if $k$ is the smallest natural number that makes $x^k \in H$, then $k\mid o(x)$. 
  \item Let $G$ be a group and $N$ a normal subgroup of  $G$. Prove that  
\begin{align*}
o(gN)= \inf \set{k\inn: g^k \in N},\quad \text{ where }\inf \varnothing = \infty
\end{align*}
\item Let $G$ be a finite group, $H,N$ two subgroups of $G$ with $N$ normal. Show that if  $o(H)$ and $\abso{G:N}$ are coprime, then $H \leq N$. 
\end{enumerate}
\end{question}
\begin{proof}
  (i):  Let $a=qk+r\inn$ with $0\leq r<k$. If $x^a \in H$, then  $x^r=x^a\cdot (x^{k})^{-q}\in H$, which implies $r=0$. We have shown that $k$ divides all natural numbers  $a$ that makes  $x^a \in H$, which includes $o(x)$. \\

(ii): This is a simple observation that $(gN)^k=g^kN\in N\iff g^k \in N$. \\

(iii): By second isomorphism theorem, we know $\abso{HN: N}=\abso{H: H \cap N}$ which divides both $o(H)$ and $\abso{G:N}$. This by coprimality implies $\abso{H:H \cap N}=1$, which shows that $H\leq N$.  
\end{proof}
\begin{question}{}{}
Let $G$ be a finite group with Sylow $p$-subgroup $P$ and normal subgroup  $N$. Show that $P \cap N$ forms a Sylow $p$-subgroup of  $N$, and use such to deduce $N$ have index $p^{\nu _p(o(PN))-\nu _p(o(N))}$ in $PN$.
\end{question}
\begin{proof}
By second isomorphism theorem, we have 
\begin{align*}
o(PN)\cdot o(P\cap N)= o(P) \cdot o(N)
\end{align*}
Because $P$ is Sylow with $P \subseteq PN$, we know 
\begin{align*}
\nu_p(o(PN))=\nu _p (o(P))
\end{align*}
This shows that, indeed, $P\cap N$ forms a Sylow $p$-subgroup of $N$: 
\begin{align*}
\nu_p(o(P\cap N))= \nu _p (o(N))
\end{align*}
as desired. Because $P \cap N\leq P$ and because $P$ is Sylow, we know $o(P\cap N)$ is a power of $p$. It then follows that: 
\begin{align*}
\abso{PN: N}=  \frac{o(PN)}{o(N)}= \frac{o(P)}{o(P\cap N)}= p^{\nu_p(o(P))- \nu_p(o(P\cap N))}= p^{\nu_p(o(PN))-\nu_p(o(P))} 
\end{align*}
\end{proof}
\begin{question}{}{}
Prove that if $H$ is a Hall subgroup of  $G$ and  $N\trianglelefteq G$, then $H \cap N$ is a Hall subgroup of $N$ and  $HN\quotient N$ is a Hall subgroup of $G \quotient N$. 
\end{question}
\begin{proof}
The facts that: 
\begin{enumerate}[label=(\roman*)]
  \item By second isomorphism theorem, we have $\abso{N:H\cap N}=\abso{HN:H}$, which divides $\abso{G:H}$.   
  \item $o(H\cap N)\mid  o(H)$. 
  \item $o(H)$ and $\abso{G:H}$ are coprime. 
\end{enumerate}
implies $o(H\cap N)$ and $\abso{N:H\cap N}$ is coprime, i.e., $H\cap N$ is Hall in $N$.  \\

The facts that: 
\begin{enumerate}[label=(\roman*)]
  \item $o(HN\quotient N)= \frac{o(HN)}{o(N)}= \frac{o(H)}{o(H\cap N)}$ divides $o(H)$. (second isomorphism theorem) 
  \item $ \abso{(G\quotient N):(HN \quotient N)}= \abso{G:HN}$ divides $\abso{G:H}$. 
  \item  $o(H)$ and $\abso{G:H}$ are coprime.  
\end{enumerate}
implies $o(HN\quotient N)$  and $\abso{(G\quotient N): (HN\quotient N)}$ are coprime, i.e., $HN\quotient N$ is Hall in $G \quotient N$.



\end{proof}
\section{HW2}
\setcounter{qnum}{0}
\begin{question}{subgroup of $p$-group of index  $p$ is normal}{}
Prove that if $p$ is a prime and $o(G)=p^{\alpha }$ with $\alpha \inn$, then every subgroup $H$ of index $p$ is normal.\\

Deduce that every group of order $p^2$ has a normal subgroup of order  $p$. 
\end{question}
\begin{proof}
Let $G$ acts on the left cosets spaces $\Omega$ of $H$. We have a group homomorphism  $\phi: G \rightarrow \operatorname{Sym}(\Omega)$. Clearly we have $\operatorname{ker}\phi \subseteq H$. By first isomorphism theorem, we know 
\begin{align*}
\abso{G: \operatorname{ker}\phi} = o(\operatorname{Im} \phi) \mid  \operatorname{Sym}(\Omega)
\end{align*}
Noting that $\abso{\operatorname{Sym}\Omega}=p!$, we see $\operatorname{ker}\phi $ has index $\leq p$, which when combined with the fact $\operatorname{ker}\phi \subseteq H$ shows that $H=\operatorname{ker}\phi$, as desired. \\

Suppose $\alpha =2$. By \customref{THfS}{first Sylow theorem}, there is a subgroup of $G$ of order $p$. This subgroup is normal from what we have just proved.
\end{proof}
\begin{question}{}{}
Let $G$ be a group of odd order. Prove that for any $x\neq e \in G$, we have $\operatorname{Cl}(x)\neq \operatorname{Cl}(x^{-1})$. 
\end{question}
\begin{proof}
Assume for a contradiction that $\operatorname{Cl}(x)=\operatorname{Cl}(x^{-1})$. Because $(gxg^{-1})^{-1} = gx^{-1}g^{-1}\in \operatorname{Cl}(x^{-1})=\operatorname{Cl}(x)$, the inversion is well defined on $\operatorname{Cl}(x)$, and moreover clearly bijective. Because $o(G)$ is odd, we may pair up the elements of  $\operatorname{Cl}(x)$ via inversion to see $\abso{\operatorname{Cl}(x)}$ is even. This is impossible since by \customref{THost}{orbit-stabilizer theorem},  $\abso{\operatorname{Cl}(x)}$ is the index of some subgroup of $G$ .     
\end{proof}
\begin{question}{}{}
Let $o(G)=p^n$ with $n\geq 3$ and $o(Z(G))=p$. Prove that $G$ has a conjugacy class of size  $p$. 
\end{question}
\begin{proof}
  \customref{THce}{Class equation} stated that 
\begin{align}
\label{EQoG}
  o(G)= o(Z(G))+ \sum \abso{\operatorname{Cl}(x)}
\end{align}
and the \customref{THost}{orbit stabilizer theorem} shows that $\abso{\operatorname{Cl}(x)}$ is of order powers of $p$. If they are of $p$-powers  $\geq 2$, then  we see 
\begin{align*}
0\equiv o(G) \equiv p \equiv o(Z(G))+ \sum \abso{\operatorname{Cl}(x)}\pmod{p}
\end{align*}
a contradiction. 
\end{proof}
\begin{question}{}{}
Prove that if the center of $G$ is of index $n$, then every conjugacy class has at most  $n$ elements. 
\end{question}
\begin{proof}
Let $x \in G$. Because $Z(G) \subseteq C_G(x)$, by \customref{THost}{orbit-stabilizer theorem}, we have: 
\begin{align*}
\abso{\operatorname{Cl}(a)} = \abso{G : C_G(a)} \leq \abso{G:Z(G)}=n 
\end{align*}
\end{proof}
\begin{question}{}{}
Let $H,K\subseteq G$ be two finite subgroups. Show that 
\begin{align*}
\abso{HK}= \frac{o(H)o(K)}{ o(H \cap K)}
\end{align*}
\textbf{Remark}: The hint give a rigorous proof, but I prefer a heuristic one. 
\end{question}
\begin{proof}
Consider the right coset spaces $\Omega \triangleq \set{Hx: x \in G}$, and let $K$ acts on  $\Omega$ by right multiplication. Because $Hk =H$ if and only if $k \in H$, we know the stabilizer subgroup $K_H$ is identical to  $K \cap  H$. Therefore, by \customref{THost}{orbit-stabilizer theorem}, we have 
\begin{align*}
\frac{o(K)}{o(H \cap K)}= \abso{\set{Hk: k \in K}}
\end{align*}
Define an equivalence class in $K$ by setting  $k \sim \tilde{k}\overset{\triangle}{\iff } Hk = H \tilde{k}$. Pick a representative element our of each class and collect them into a set $T$. Clearly 
\begin{align*}
\abso{T}= \abso{\set{Hk:k \in K}}
\end{align*}
and we have a natural bijection $H \times T \rightarrow HK$. This finishes the proof. 
\end{proof}
\begin{question}{}{}
Let $G$ be a non-abelian group of order $21$. Prove that  $Z(G)=1$. 
\end{question}
\begin{proof}
If $o(Z(G))=3$ or $7$, then because  $G \quotient Z(G)$ is cylic  
\end{proof}
\begin{question}{}{}
Find all finite groups which have exactly two conjugacy classes. 
\end{question}
\begin{proof}
Let $G$ be a finite group that has exactly two conjugacy classes. One of the conjugacy class is $\set{e}$. Let $a$ be an element of the other class. By class equation and \customref{THost}{orbit-stabilizer theorem}, we have
\begin{align*}
 \abso{G}-1 =  \abso{\operatorname{Cl}(a)}  \mid  o(G)
\end{align*}
This implies $\abso{G}=2$, which implies $G=\Z_2$. 
\end{proof}
\begin{question}{}{}
Let $H$ be a subgroup of  $G$ and let 
 \begin{align*}
\bigcup_{ g \in G} gHg^{=1}=G
\end{align*}
Show that $H=G$. 
\end{question}

\section{Exercises III} 
\begin{question}{}{}
Let $o(G)=60$. Show that if $G$ is simple, then $G$ must have exactly $24$ elements of order $5$ and  $20$ elements of order $3$. 
\end{question}
\begin{proof}
By \customref{SECSt}{sylow}, we have 
\begin{align*}
n_5 \equiv 1 \pmod{5}\quad \text{ and }\quad n_5 \mid 12
\end{align*}
which by simplicity of $G$ implies $n_5=6$. The same argument gives us $n_3 \in \set{4,10}$. To see $n_3 \neq 4$, just recall that \customref{THsSt}{second sylow} stated that conjugacy action  $G\longrightarrow  \operatorname{Sym}(\operatorname{Syl}_3(G))\cong  S_{n_3}$ is nontrivial, and is therefore injective by simplicity of $G$. We now see that $n_3=4$ is too small to satisfies   
\begin{align*}
o(G)=60  \mid  n_3 !
\end{align*}
\end{proof}
\begin{question}{}{}
Let $o(G)=pqr$ with $p<q<r$ prime. Prove that $G$ has a normal Sylow $r$-subgroup  $H$. 
\end{question}
\begin{proof}
By \customref{SECst}{sylow} and counting arguments we know $1 \in \set{n_p,n_q,n_r}$. Therefore, if neither of $n_p$ and  $n_q$ is $1$, we are done. Suppose   $1 \in \set{n_p,n_q}$. Either way, we get a normal subgroup $N$ such that  $o(G \quotient N) \in \set{qr,pr}$. We also get a normal  $H \quotient N \in \operatorname{Syl}_r(G \quotient N)$. This give us a characteristic $K \in \operatorname{Syl}_r(H)$, which is normal in $G$. 
\end{proof}
\begin{question}{}{}
Let $o(G)=p^3q$ with $p,q$ prime. Show that one of the followings statement is true: 
 \begin{enumerate}[label=(\roman*)]
  \item $G$ has a normal Sylow  $p$-subgroup.  
  \item $G$ has a normal Sylow $q$-subgroup.  
  \item $p=2$,  $q=3$. 
\end{enumerate}
\end{question}
\begin{proof}
Suppose (i) and (ii) are both false. Then by \customref{SECSt}{sylow} we have $n_p=q$ and $p<q$. Because $p<q$, applying \customref{SECSt}{sylow} again we have  $n_q \in \set{p^2,p^3}$. Because $n_p>1$, by counting we see that $n_q \neq p^3$. Therefore $n_q=p^2$. Then by \customref{SECSt}{sylow}, $p^2=n_q \equiv 1 \pmod{q} $, which implies $q \mid  (p-1)(p+1)$. Because $p<q$ and $q$ is prime, we now see  $q=p+1$, which can only happens if  $p=2$ and  $q=3$.   
\end{proof}
\begin{question}{}{}
Show that no group of order $30$ is simple. 
\end{question}
\begin{proof}
Consider  $n_3$ and  $n_5$. We have  $n_5 \in \set{1,6}$ and $n_3 \in \set{1,10}$. If $n_5 \neq  1 \neq n_3$, then there are $24$ elements of order  $5$ and  $20$ elements of order $3$, impossible for a group of order $30$.   
\end{proof}
\begin{question}{}{}
Let $G$ be a finite group with sylow $p$-subgroup $P$ and normal subgroup  $N$. Show that $P \cap  N$ is $p$-sylow in $N$ and that $PN \quotient N$ is $p$-sylow in  $G \quotient N$. 
\end{question}
\begin{proof}
  \customref{THsitg}{Second isomorphism theorem} implies that 
\begin{align*}
o(PN)\cdot o(P \cap N) = o(P) \cdot o(N)  
\end{align*}
Because $P$ is  $p$-sylow, we know  $o(P)$ and $o(PN)$ has the same $p$-power, which implies that $o(P\cap N)$ and $o(N)$ has the same $p$-power, as desired. Again counting the $p$-power of  $PN \quotient  N \subseteq G \quotient N$, we see $PN\quotient N$ is  $p$-sylow.  
\end{proof}
\begin{question}{}{}
Let $G$ be a finite group, $H\leq G$ a subgroup with $[G:H]=n$. Show that: 
\begin{enumerate}[label=(\roman*)]
  \item For all subgroup $K\leq G$, we have $[H:H \cap K]\leq [G:K]$. 
  \item $[H: H \cap  H^g]\leq n$ for all $g \in G$. 
  \item If $H$ is a maximal proper subgroup of  $G$ and  $H$ is abelian, show that  $H \cap H^g \trianglelefteq G $  for all $g \not \in H$. 
  \item Suppose that $G$ is simple. If  $H$ is abelian and $n$ is prime, then $H=1$.  
\end{enumerate}
\end{question}
\begin{proof}
Let $H \quotient H \cap K$  and $G \quotient K$ denote left coset spaces. (i) is a consequence of verifying that the function 
\begin{align*}
H \quotient  H \cap  K \longrightarrow G \quotient K ;\quad h(H \cap K) \mapsto  hK
\end{align*}
is well-defined and injective. (ii) is then a corollary of (i). \\

We now prove (iii). Fix $g \not \in  H$. There are two cases: Either $H=H^g$ or $H \neq H^g$. For the first case, just observe that by maximality of $H$, we will have  $N_G(H)=G$. We now claim that $H \neq H^g \implies  H \cap H^g \subseteq Z(G)$. Because $H$ is abelian, we know  $H \cap H^g \leq Z(H)$. Clearly we also have $H \cap H^g \leq Z(H^g)$. We now have $H \cap H^g \leq Z(\langle H,H^g\rangle )$, where $\langle H,H^g\rangle =G$ by maximality of $H$, as desired.   \\

We now prove (iv). Clearly the primality of $n$ forces  $H$ to be a maximal proper subgroup of  $G$. Therefore by (iii), $H \cap H^g= 1$ for all $g \not \in H$.  This by (ii) implies $n \leq o(G)\leq n^2$. Write $o(G)\triangleq nk$ so $k \in \set{1,\dots ,n}$. We wish to show $k=1$. To see  $k\neq n$, just recall that if so, then $G$ would be abelian, contradicting to its simplicity. To see $k \not\in \set{2,\dots ,n-1}$, just observe that if so, then the unique $n$-Sylow subgroup would be proper, contradicting to simplicity of $G$.   
\end{proof}
\begin{question}{}{}
Let $G$ be a finite group with $P \in \operatorname{Syl}_p(G)$. Suppose that $N$ is a normal subgroup of $G$ with  $[G:N]= o(P)>1$. Show that 
\begin{enumerate}[label=(\roman*)]
  \item $N$ is the subset of  $G$ consisting of all elements of order not divisible by $p$. 
  \item If the elements of $G - N$ all has $p$-power order, then  $P= N_G(P)$. 
\end{enumerate}
\end{question}
\begin{proof}
Because $P$ is  $p$-sylow and $[G:N]=o(P)$, we know $p \nmid o(N)$.  This implies that no element of $N$ has order divisible by $p$. Let $g \in G$ with $p \nmid o(g)$. To see that $g \in N$, just observe that because $o(gN) \mid  o(g)$ and $o(gN)$ is a power of $p$, we have $o(gN)=1$. \\

Assume for a contradiction that $P< N_G(P)$. Then there exists some nontrivial sylow  $q$-subgroup  $Q$ of  $N_G(P)$ with $q \neq p$. By definition we have $[Q,P] \leq P$. By (i), $Q\leq N$. Therefore we also have $[Q,P] \leq N$. Coprimality of orders of $N$ and  $P$ now tell us that  $[Q,P]=1$. We now see that the product of two nontrivial elements $x \in Q,y \in P$ has order divisible by $pq$, a contradiction to the premise.   
\end{proof}
\section{Exercises IV}
\begin{question}{}{}
Show that the center of products is a product of centers: 
\begin{align*}
Z(G_1) \times \cdots \times Z(G_n) = Z (G_1 \times \cdots \times G_n)
\end{align*}
Deduce that a direct product of groups is abelian if and only if each of its factor is abelian.  
\end{question}
\begin{proof}
The "$\subseteq$" is clear. To see that 
\begin{align*}
g_1 \times \cdots \times g_n \in Z(G_1 \times \cdots \times G_n) \implies  g_i \in Z(G_i)
\end{align*}
just observe that if not, then 
\begin{align*}
  [g_1 \times \cdots \times g_n , e_1 \times \cdots  \times x_i \times \cdots \times e_n]\neq e \in \prod G_j 
\end{align*}
The second part then follows from noting 
\begin{align*}
Z(G_1 \times \cdots \times G_n)= G_1 \times \cdots \times G_n \iff  Z(G_i)=G_i,\quad \text{ for all }i
\end{align*}
\end{proof}
\begin{question}{}{}
Let $G\triangleq A_1 \times \cdots \times A_n$ and $B_i \trianglelefteq A_i$ for all $i$. Prove that  $B_1 \times \cdots \times B_n \trianglelefteq G$ and that 
\begin{align*}
\frac{A_1 \times \cdots \times A_n}{B_1 \times \cdots \times B_n} = \frac{A_1}{B_1} \times \cdots \times \frac{A_n}{B_n}
\end{align*}
\end{question}
\begin{proof}
\begin{align*}
  (g_1,\dots ,g_n) (b_1 ,\dots ,b_n )(g_1 , \dots ,g_n)^{-1} = (g_1b_1g_1^{-1},\dots ,g_nb_ng_n^{-1}) \in \prod B_i
\end{align*}
The second part require us to show that 
\begin{align*}
\prod \left( \frac{A_i}{B_i}\right) \longrightarrow \frac{\prod A_i}{\prod B_i};\quad \prod \left(\frac{a_i}{B_i} \right) \mapsto \frac{\prod a_i}{\prod B_i}
\end{align*}
is a well-defined group isomorphism, which boils down to showing that it is (i) well-defined, (ii) actually a homomorphism, (iii) injective, and (iv) surjective. To see it is injective, just observe that if $\prod a_i \in \prod B_i$, then $a_i \in B_i$ for all $i$, and therefore  $\prod \frac{a_i}{B_i}=e$. The rest are clear. 
\end{proof}
\begin{question}{}{}
Let $G$ be a finite abelian group with $m \mid  o(G)$. Show that $G$ has a subgroup of order $m$. 
\end{question}
\begin{proof}
This follows from noting that if $o(a)=p^n$, then $o(a^{p^{n-d}})=p^d$. (Ans also structure theorem for finite abelian group) 
\end{proof}
\begin{question}{}{}
Show that the subgroups and quotients of a nilpotent group $G$ are also nilpotent. 
\end{question}
\begin{proof}
Let $H$ be a subgroup of  $G$, and write 
\begin{align*}
0 = G_{(n)} \trianglelefteq  \cdots \trianglelefteq G_{(1)} \trianglelefteq G_{(0)}= G,\quad \text{ with }G_{(k)} \triangleq [G,G_{(k-1)}]
\end{align*}
To see that  
\begin{align*}
0 \leq H_n \leq  \cdots \leq H_1 \leq H
\end{align*}
form a central series, where $H_k\triangleq H \cap G_{(k)}$, just observe that  
\begin{align*}
[H,H \cap  G_{(k)}] \leq H \text{ and }[H,H \cap G_{(k)}] \leq [G,G_{(k)}] \leq G_{(k-1)}
\end{align*}
together implies 
\begin{align*}
[H,H_k] \leq H \cap G_{(k-1)} = H_{k-1}
\end{align*}
Let $N$ be a normal subgroup of  $G$, and let $m \leq n$ be the largest number such that $N\leq G_{(m)}$. It is clear that 
\begin{align*}
\frac{N}{N} \leq \frac{G_{(m)}}{N} \leq  \cdots \leq  \frac{G_{(1)}}{N} \leq \frac{G}{N}
\end{align*}
form a central series. 
\end{proof}
\begin{question}{}{}
Show that if $G \quotient Z(G)$ is nilpotent, then $G$ is nilpotent. 
\end{question}
\begin{proof}
Consider the central series 
\begin{align*}
 \frac{Z(G)}{Z(G)} \trianglelefteq \frac{G_1}{Z(G)} \trianglelefteq  \cdots \trianglelefteq  \frac{G_n}{Z(G)} = \frac{G}{Z(G)}
\end{align*}
Clearly we have the central series 
\begin{align*}
0 \trianglelefteq Z(G) \trianglelefteq G_1 \trianglelefteq  \cdots \trianglelefteq  G_n = G
\end{align*}
\end{proof}
\begin{question}{}{}
Let $o(G)=pqr$ with $p<q<r$ prime. Show that  $G$ is solvable. 
\end{question}
\begin{proof}
Recall that we have a normal subgroup $M \in \operatorname{Syl}_r(G)$. Then we have a normal subgroup $\frac{H}{M}\in \operatorname{Syl}_q(G)$. Then $1\trianglelefteq M\trianglelefteq  H \trianglelefteq  G$ forms the desired series.  
\end{proof}
\begin{question}{}{}
Show that a finite group $G$ is nilpotent if and only if every $a,b \in G$ that makes $\operatorname{gcd}(o(a),o(b))=1$ also makes $ab=ba$.   
\end{question}
\begin{proof}
($\implies $): Write $G=P_1 \times \cdots \times P_n$ with $P_i$ sylow. Clearly if the orders of $(x_1,\dots ,x_n)$ and $(y_1,\dots ,y_n)$  are coprime to each other, then for all $i$, we must have either  $x_i=e$ or  $y_i=e$. This implies the commutativity. \\

$(\impliedby)$: We need to show that Sylow subgroups of $G$ are normal. Let $P_1,\dots ,P_n$ each be a Sylow subgroup of $G$ with distinct  $p$. By premise, we see that $P_k \subseteq N_G(P_1)$ for all $k\geq 2$. This then implies $G=N_G(P_1)$, as desired. 
\end{proof}
\begin{question}{}{}
Let $G=HK$ be finite and  $S \leq G$ be a $p$-subgroup that contains some $p$-Sylow subgroup $P$ of $H$ and some $p$-Sylow subgroup $Q$ of  $K$. Show that  
\begin{enumerate}[label=(\roman*)]
  \item $S$ is $p$-Sylow in  $G$.  
  \item $S= (S \cap H)(S \cap K)$
\end{enumerate}
\end{question}
\begin{proof}
Because $P \cap Q \leq H \cap K$, we know $p$-part of 
 \begin{align*}
o(G)= \frac{o(H)o(K)}{o(H\cap K)}
\end{align*}
is smaller than  
\begin{align*}
 \frac{o(P)o(Q)}{o(P \cap Q)}= \abso{PQ} \leq o(S)
\end{align*}
 which can only happen if $S$ is Sylow with $\abso{PQ}=o(S)$. By definition, $ P \leq  S \cap H \leq H$. Because $S$ is a  $p$-group, we know  $S \cap H$ is also a $p$-group. Sylowness of $P \leq H$ then forces $S \cap H = P$. Similarly, we have $S \cap K= Q$. Now, to see $S = PQ$, just recall that $\abso{PQ}=o(S)$  
\end{proof}
\begin{question}{}{}
Let $M \trianglelefteq G$ and $N \trianglelefteq G$ with $M,N$ finite and nilpotent. Prove that  $MN$ is nilpotent. 
\end{question}
\begin{proof}
The proof follows form noting that if $S \in \operatorname{Syl}_p(MN)$, then by earlier questions, $S$ is uniquely determined by $S= (M\cap S)(N\cap S)$ with $M \cap S \in \operatorname{Syl}_p(M)$ and $N\cap S \in \operatorname{Syl}_p(N)$ uniquely determined.  
\end{proof}
\begin{question}{}{}
Let $G$ be finite with  $A,B \trianglelefteq G$ and $G \quotient A,G \quotient B$ solvable. Prove that $G \quotient (A \cap B)$ is solvable.   
\end{question}
\begin{proof}
\end{proof}

\end{document}

