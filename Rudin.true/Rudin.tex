\documentclass{report}

%%%%%%%%%%%%%% preamble.tex %%%%%%%%%%%%%%
\usepackage[T1]{fontenc}
\usepackage{etoolbox}
% Page Setup
\usepackage[letterpaper, tmargin=2cm, rmargin=0.5in, lmargin=0.5in, bmargin=80pt, footskip=.2in]{geometry}
\usepackage{adjustbox}
\usepackage{graphicx}
\usepackage{tikz}
\usepackage{mathrsfs}
\usepackage{mdframed}

% Create a new toggle
\newtoggle{firstsection}

% Redefine the \chapter command to reset the toggle for each new chapter
\let\oldchapter\chapter
\renewcommand{\chapter}{\toggletrue{firstsection}\oldchapter}

% Redefine the \section command to check the toggle
\let\oldsection\section
\renewcommand{\section}{
    \iftoggle{firstsection}
    {\togglefalse{firstsection}} % If it's the first section, just switch off the toggle for next sections
    {\clearpage} % If it's not the first section, start a new page
    \oldsection
}

% Abstract Design

\usepackage{lipsum}

\renewenvironment{abstract}
 {% Start of environment
  \quotation
  \small
  \noindent
  \rule{\linewidth}{.5pt} % Draw the rule to match the linewidth
  \par\smallskip
  {\centering\bfseries\abstractname\par}\medskip
 }
 {% End of environment
  \par\noindent
  \rule{\linewidth}{.5pt} % Ensure the closing rule also matches
  \endquotation
 }

% Mathematics
\usepackage{amsmath,amsfonts,amsthm,amssymb,mathtools}
\usepackage{xfrac}
\usepackage[makeroom]{cancel}
\usepackage{enumitem}
\usepackage{nameref}
\usepackage{multicol,array}
\usepackage{tikz-cd}
\usepackage{array}
\usepackage{multirow}% http://ctan.org/pkg/multirow
\usepackage{graphicx}

% Colors
\usepackage[dvipsnames]{xcolor}
\definecolor{myg}{RGB}{56, 140, 70}
\definecolor{myb}{RGB}{45, 111, 177}
\definecolor{myr}{RGB}{199, 68, 64}
% Define more colors here...
\definecolor{olive}{HTML}{6B8E23}
\definecolor{orange}{HTML}{CC5500}
\definecolor{brown}{HTML}{8B4513}
% Hyperlinks
\usepackage{bookmark}
\usepackage[colorlinks=true,linkcolor=blue,urlcolor=blue,citecolor=blue,anchorcolor=blue]{hyperref}
\usepackage{xcolor}
\hypersetup{
    colorlinks,
    linkcolor={red!50!black},
    citecolor={blue!50!black},
    urlcolor={blue!80!black}
}

% Text-related
\usepackage{blindtext}
\usepackage{fontsize}
\changefontsize[14]{14}
\setlength{\parindent}{0pt}
\linespread{1.2}

% Theorems and Definitions
\usepackage{amsthm}
\renewcommand\qedsymbol{$\blacksquare$}

% Define a new theorem style
\newtheoremstyle{mytheoremstyle}% name
  {}% Space above
  {}% Space below
  {}% Body font
  {}% Indent amount
  {\bfseries}% Theorem head font
  {.}% Punctuation after theorem head
  {.5em}% Space after theorem head
  {}% Theorem head spec (can be left empty, meaning ‘normal’)

% Apply the new theorem style to theorem-like environments
\theoremstyle{mytheoremstyle}

\newtheorem{theorem}{Theorem}[section]  
\newtheorem{definition}[theorem]{Definition} 
\newtheorem{lemma}[theorem]{Lemma}  
\newtheorem{corollary}[theorem]{Corollary}
\newtheorem{axiom}[theorem]{Axiom}
\newtheorem{example}[theorem]{Example}
\newtheorem{equiv_def}[theorem]{Equivalent Definition}

% tcolorbox Setup
\usepackage[most,many,breakable]{tcolorbox}
\tcbuselibrary{theorems}

% Define custom tcolorbox environments here...

%================================
% EXAMPLE BOX
%================================
% After you have defined the style and other theorem environments
\definecolor{myexamplebg}{RGB}{245, 245, 245} % Very light grey for background
\definecolor{myexamplefr}{RGB}{120, 120, 120} % Medium grey for frame
\definecolor{myexampleti}{RGB}{60, 60, 60}    % Darker grey for title

\newtcbtheorem[]{Example}{Example}{
    colback=myexamplebg,
    breakable,
    colframe=myexamplefr,
    coltitle=myexampleti,
    boxrule=1pt,
    sharp corners,
    detach title,
    before upper=\tcbtitle\par\vspace{-20pt}, % Reduced the space after the title
    fonttitle=\bfseries,
    description font=\mdseries,
    separator sign none,
    description delimiters={}{}, % No delimiters around the title
}{ex}
%================================
% Solution BOX
%================================
\makeatletter
\newtcolorbox{solution}{enhanced,
	breakable,
	colback=white,
	colframe=myg!80!black,
	attach boxed title to top left={yshift*=-\tcboxedtitleheight},
	title=Solution,
	boxed title size=title,
	boxed title style={%
			sharp corners,
			rounded corners=northwest,
			colback=tcbcolframe,
			boxrule=0pt,
		},
	underlay boxed title={%
			\path[fill=tcbcolframe] (title.south west)--(title.south east)
			to[out=0, in=180] ([xshift=5mm]title.east)--
			(title.center-|frame.east)
			[rounded corners=\kvtcb@arc] |-
			(frame.north) -| cycle;
		},
}
\makeatother

% %================================
% % Question BOX
% %================================
\makeatletter
\newtcbtheorem{question}{Question}{enhanced,
	breakable,
	colback=white,
	colframe=myb!80!black,
	attach boxed title to top left={yshift*=-\tcboxedtitleheight},
	fonttitle=\bfseries,
	title={#2},
	boxed title size=title,
	boxed title style={%
			sharp corners,
			rounded corners=northwest,
			colback=tcbcolframe,
			boxrule=0pt,
		},
	underlay boxed title={%
			\path[fill=tcbcolframe] (title.south west)--(title.south east)
			to[out=0, in=180] ([xshift=5mm]title.east)--
			(title.center-|frame.east)
			[rounded corners=\kvtcb@arc] |-
			(frame.north) -| cycle;
		},
	#1
}{question}
\makeatother

%%%%%%%%%%%%%%%%%%%%%%%%%%%%%%%%%%%%%%%%%%%
% TABLE OF CONTENTS
%%%%%%%%%%%%%%%%%%%%%%%%%%%%%%%%%%%%%%%%%%%


\usepackage{tikz}
\definecolor{doc}{RGB}{0,60,110}
\usepackage{titletoc}
\contentsmargin{0cm}
\titlecontents{chapter}[14pc]
{\addvspace{30pt}%
	\begin{tikzpicture}[remember picture, overlay]%
		\draw[fill=doc!60,draw=doc!60] (-7,-.1) rectangle (-0.9,.5);%
		\pgftext[left,x=-5.5cm,y=0.2cm]{\color{white}\Large\sc\bfseries Chapter\ \thecontentslabel};%
	\end{tikzpicture}\color{doc!60}\large\sc\bfseries}%
{}
{}
{\;\titlerule\;\large\sc\bfseries Page \thecontentspage
	\begin{tikzpicture}[remember picture, overlay]
		\draw[fill=doc!60,draw=doc!60] (2pt,0) rectangle (4,0.1pt);
	\end{tikzpicture}}%
\titlecontents{section}[3.7pc]
{\addvspace{2pt}}
{\contentslabel[\thecontentslabel]{3pc}}
{}
{\hfill\small \thecontentspage}
[]
\titlecontents*{subsection}[3.7pc]
{\addvspace{-1pt}\small}
{}
{}
{\ --- \small\thecontentspage}
[ \textbullet\ ][]

\makeatletter
\renewcommand{\tableofcontents}{
	\chapter*{%
	  \vspace*{-20\p@}%
	  \begin{tikzpicture}[remember picture, overlay]%
		  \pgftext[right,x=15cm,y=0.2cm]{\color{doc!60}\Huge\sc\bfseries \contentsname};%
		  \draw[fill=doc!60,draw=doc!60] (13,-.75) rectangle (20,1);%
		  \clip (13,-.75) rectangle (20,1);
		  \pgftext[right,x=15cm,y=0.2cm]{\color{white}\Huge\sc\bfseries \contentsname};%
	  \end{tikzpicture}}%
	\@starttoc{toc}}
\makeatother

\newcommand{\liff}{\llap{$\iff$}}
\newcommand{\rap}[1]{\rrap{\text{ (#1)}}}
\newcommand{\red}[1]{\textcolor{red}{#1}}
\newcommand{\blue}[1]{\textcolor{blue}{#1}}
\newcommand{\vi}[1]{\textcolor{violet}{#1}}
\newcommand{\olive}[1]{\textcolor{olive}{#1}}
\newcommand{\teal}[1]{\textcolor{teal}{#1}}
\newcommand{\brown}[1]{\textcolor{brown}{#1}}
\newcommand{\orange}[1]{\textcolor{orange}{#1}}
\newcommand{\tCaC}{\text{ \CaC }}
\newcommand{\CaC}{\red{CaC} }
\newcommand{\As}[1]{Assume \red{#1}}
\newcommand{\vdone}{\vi{\text{ (done) }}}
\newcommand{\bdone}{\blue{\text{ (done) }}}
\newcommand{\tdone}{\teal{\text{ (done) }}}
\newcommand{\odone}{\olive{\text{ (done) }}}
\newcommand{\bodone}{\brown{\text{ (done) }}}
\newcommand{\ordone}{\orange{\text{ (done) }}}
\newcommand{\ld}{\lambda}
\newcommand{\vecta}[1]{\textbf{#1}}
\newcommand{\set}[1]{\left\{ #1 \right\}}
\newcommand{\bset}[1]{\Big\{ #1 \Big\}}
\newcommand{\inR}{\in\R}
\newcommand{\inn}{\in\N}
\newcommand{\inz}{\in\Z}
\newcommand{\inr}{\in\R}
\newcommand{\inc}{\in\C}
\newcommand{\inq}{\in\Q}
\newcommand{\norm}[1]{\| #1 \|}
\newcommand{\bnorm}[1]{\Big\| #1 \Big\|}
\newcommand{\gen}[1]{\langle #1 \rangle}
\newcommand{\abso}[1]{\left|#1\right|}
\newcommand{\myref}[2]{\hyperref[#2]{#1\ \ref*{#2}}}
\newcommand{\customref}[2]{\hyperref[#1]{#2}}
\newcommand{\power}[1]{\mathcal{P}(#1)}
\newcommand{\dcup}{\mathbin{\dot{\cup}}}
\newcommand{\diam}[1]{\text{diam}\, #1}
\newcommand{\at}{\Big|}
\newcommand{\quotient}{\diagup}
\let\originalphi\phi % Store the original \phi in \originalphi
\renewcommand{\phi}{\varphi} % Redefine \phi to \varphi
\newcommand{\pfi}{\originalphi} % Define \pfi to display the original \phi
\newcommand{\diota}{\dot{\iota}}
\newcommand{\Log}{\operatorname{Log}}
\newcommand{\id}{\text{\textbf{id}}}
\usepackage{amsmath}

\makeatletter
\NewDocumentCommand{\extp}{e{^}}{%
  \mathop{\mathpalette\extp@{#1}}\nolimits
}
\NewDocumentCommand{\extp@}{mm}{%
  \bigwedge\nolimits\IfValueT{#2}{^{\extp@@{#1}#2}}%
  \IfValueT{#1}{\kern-2\scriptspace\nonscript\kern2\scriptspace}%
}
\newcommand{\extp@@}[1]{%
  \mkern
    \ifx#1\displaystyle-1.8\else
    \ifx#1\textstyle-1\else
    \ifx#1\scriptstyle-1\else
    -0.5\fi\fi\fi
  \thinmuskip
}
\makeatletter
\usepackage{pifont}
\makeatletter
\newcommand\Pimathsymbol[3][\mathord]{%
  #1{\@Pimathsymbol{#2}{#3}}}
\def\@Pimathsymbol#1#2{\mathchoice
  {\@Pim@thsymbol{#1}{#2}\tf@size}
  {\@Pim@thsymbol{#1}{#2}\tf@size}
  {\@Pim@thsymbol{#1}{#2}\sf@size}
  {\@Pim@thsymbol{#1}{#2}\ssf@size}}
\def\@Pim@thsymbol#1#2#3{%
  \mbox{\fontsize{#3}{#3}\Pisymbol{#1}{#2}}}
\makeatother
% the next two lines are needed to avoid LaTeX substituting upright from another font
\input{utxmia.fd}
\DeclareFontShape{U}{txmia}{m}{n}{<->ssub * txmia/m/it}{}
% you may also want
\DeclareFontShape{U}{txmia}{bx}{n}{<->ssub * txmia/bx/it}{}
% just in case
%\DeclareFontShape{U}{txmia}{l}{n}{<->ssub * txmia/l/it}{}
%\DeclareFontShape{U}{txmia}{b}{n}{<->ssub * txmia/b/it}{}
% plus info from Alan Munn at https://tex.stackexchange.com/questions/290165/how-do-i-get-a-nicer-lambda?noredirect=1#comment702120_290165
\newcommand{\pilambdaup}{\Pimathsymbol[\mathord]{txmia}{21}}
\renewcommand{\lambda}{\pilambdaup}
\renewcommand{\tilde}{\widetilde}
\DeclareMathOperator*{\esssup}{ess\,sup}
\newcommand{\bluecheck}{}%
\DeclareRobustCommand{\bluecheck}{%
  \tikz\fill[scale=0.4, color=blue]
  (0,.35) -- (.25,0) -- (1,.7) -- (.25,.15) -- cycle;%
}


\usepackage{tikz}
\newcommand*{\DashedArrow}[1][]{\mathbin{\tikz [baseline=-0.25ex,-latex, dashed,#1] \draw [#1] (0pt,0.5ex) -- (1.3em,0.5ex);}}

\newcommand{\C}{\mathbb{C}}	
\newcommand{\F}{\mathbb{F}}
\newcommand{\N}{\mathbb{N}}
\newcommand{\Q}{\mathbb{Q}}
\newcommand{\R}{\mathbb{R}}
\newcommand{\Z}{\mathbb{Z}}



\title{\Huge{NCKU 112.1}\\Rudin}
\author{\huge{Eric Liu}}
\date{}
\begin{document}
\maketitle
\newpage% or \cleardoublepage
% \pdfbookmark[<level>]{<title>}{<dest>}
\pdfbookmark[section]{\contentsname}{toc}
\tableofcontents
\pagebreak
\chapter{The Real and Complex Number System}
\section{Introduction}
\fbox{\begin{minipage}{39em}
In this section, we will define the concept of ordered sets, and give a close look of the completeness property of real numbers, by showing the "uncompleteness" of rational numbers. First, we prove an elementary and classic theorem of rational numbers.        
\end{minipage}}
\begin{theorem}
\label{1.1.1}
There exists no rational $p$ such that $p^2=2$
\end{theorem}
\begin{proof}
\As{there is, and we write $p$ in the form $p=\frac{a}{b}$, where $a,b\inz\text{ and one of $a,b$ is odd}$}. Observe that $p^2=2\implies a^2=2b^2\implies 2\text{ divides }a\implies 2\text{ divides }b\tCaC$
\end{proof}

\fbox{\begin{minipage}{39em}
For the sake of our discussion below, we will use $\Q^+$ instead of $\Q$ as our universal. Now, we divide the "rational numbers line" in half at the point $\sqrt{2}$, and have two subdivisions $A=\set{x\inq^+ : x^2<2},B=\set{x\inq^+:x^2>2}$ 
\end{minipage}}

\begin{axiom}
\label{1.1.2}
Let $S$ be an ordered set and $X$ be a subset of $S$. Axiomatically define
\begin{equation}
\max X\in X\text{ and }\forall x\in X, x\leq \max X
\end{equation}
\begin{equation}
\min X\in X \text{ and }\forall x\in X, \min X\leq x 
\end{equation}
\end{axiom}
\begin{theorem}
\label{1.1.3}
  $\max A\text{ and }\min B$ both doesn't exist.
\end{theorem}
\begin{proof}
We wish to construct a function $q(p)$ on $\Q^+$ such that for all $p \inq^+$, we have $p^2<2\implies p^2<q^2<2$ and have $2<p^2\implies 2<q^2<p^2$. Notice $p<q\iff p^2<q^2$, so we can translate the wanted property of $q$ into
\begin{align}
  p^2<2 & \implies p<q \text{ and }q^2<2\\
  p^2>2 & \implies q<p\text{ and }2<q^2
\end{align}
Let $a,b$ be two function of $p$ on $\Q^+$. To satisfy the above properties, we can let the below equation first be true and then solve for $a,b$.
\begin{align}
  q-p&=\frac{2-p^2}{a}\\
  q^2-2 &= \frac{p^2-2}{b}
\end{align}
Now we solve for $a,b$
\begin{gather}
q-p=\frac{2-p^2}{a}\text{ and }q^2-2=\frac{p^2-2}{b}
\implies q= \frac{2-p^2}{a}+p\text{ and }q^2=\frac{p^2-2}{b}+2\\
\liff [\frac{2-p^2}{a}+p]^2=\frac{p^2-2}{b}+2\\
\liff \frac{(2-p^2)^2}{a^2}+\frac{2p(2-p^2)}{a}+p^2=\frac{p^2-2}{b}+2\\
\liff (p^2-2)^2(\frac{1}{a^2})+(p^2-2)(-\frac{2p}{a}+1-\frac{1}{b})=0\\
\liff \frac{p^2-2}{a^2}-\frac{2p}{a}+1-\frac{1}{b}=0\text{ (because $p^2-2\neq 0$) }\\
\liff \frac{p^2-2-2ap+a^2}{a^2}=\frac{1}{b}\\
\liff b=\frac{a^2}{p^2-2-2ap+a^2}=\frac{a^2}{(p-a)^2-2}
\end{gather}
Define $a:=p+c$ where $c^2>2$, and we are finished.
\end{proof}
\fbox{\begin{minipage}{39em}
Now, we come back to define the concept of ordered set.
\end{minipage}}
\begin{definition}
\label{1.1.4}
\textbf{(Ordered Set Axioms)} $S$ is an ordered set if there is a relation $\sim$ on it that satisfy 
\begin{equation}
\forall x\in S, x\not\sim x
\end{equation}
\begin{equation}
\forall x,y\in S, x\neq y\implies x\sim y \text{ exclusively or }y\sim x
\end{equation}
\begin{equation}
  \forall x,y,z\in S, x\sim y \text{ and }y\sim z\implies x\sim z
\end{equation}
From now on, we write this relation as $<$ and if we write  $x>y$, we mean $y<x$.
\end{definition}
\fbox{\begin{minipage}{39em}
To discuss the uncompleteness of rational numbers, which is an ordered set, we first define a few concepts brought by concept of ordered set.
\end{minipage}}
\begin{definition}
\label{1.1.5}
Let $S$ be an ordered set and $E\subseteq S$. $E$ is bounded above if 
\begin{equation}
\exists a\in S, \forall b\in E, a\geq b 
\end{equation}
In this case, we say $a$ is an upper bound of $E$ and $E$ is bounded above by $a$. On the other hand, $E$ is bounded below by $c$ if
\begin{equation}
\forall b\in E, c\leq b
\end{equation}
\end{definition}
\fbox{\begin{minipage}{39em}
Before we give the definition of supremum, aka least upper bound, we first prove a theorem about it.
\end{minipage}}
\begin{theorem}
\label{1.1.6}
If $x$ is the smallest upper bound of a bounded above nonempty set  $E$, then any number smaller than  $x$ is not an upper bound of  $E$, and every upper bound of  $E$ is greater than or equal  to $x$.
\end{theorem}

\begin{proof}

Let $A$ be the set of upper bounds of  $E$. Arbitrarily pick an $m$ such that $m<x$. \As{$\forall z\in E, m>z$}. We see $m\in A$, and because $x=\min A$, we see $x\leq m\tCaC$. \As{$\exists n\in A, n<x$}. Then $\exists z\in E, n<z$, which implies $n\not\in A\tCaC$
\end{proof}
\begin{corollary}
\label{1.1.7}
 Any number smaller than the greatest lower bound is a lower bound, and any number greater than the greatest lower bound is not a lower bound.
\end{corollary}
\begin{definition}
\label{1.1.8}
\textbf{(Definition of Supremum and Infimum)} Let $A,B$ respectively be the set of upper bounds of  $E$ and the set of lower bounds of  $E$. We define
\begin{equation}
\sup E:=\min A
\end{equation}
and
\begin{equation}
\inf E:=\max B
\end{equation}
if they ever exist.
\end{definition}
\begin{theorem}
\label{1.1.9}
Let $A$ be a subset of ordered set  $S$. If  $\max A$ exists, then $\max A=\sup A$. Similarly, if $\min A$ exists, then $\min A=\inf A$
\end{theorem}
\begin{proof}
$\max A$ is an upper bound and $\min  A$ is an lower bound hold true by definition. Any number smaller than $\max A$ can not be an upper bound since $\max A\inA$. Similarly, and number greater than $\min A$ can not be an lower bound since $\min A\inA$  
\end{proof}
\fbox{\begin{minipage}{39em}
Now, we look back to our subdivisions $A=\set{x\inq^+: x^2<2},B=\set{x\inq^+: x^2>2}$. We first show that $B$ is exactly the set of all upper bounds of $A$.
\end{minipage}}
\begin{theorem}
\label{1.1.10}
Given $A=\set{x\inq^+: x^2<2},B=\set{x\inq^+: x^2>2}$. We have
\begin{equation}
B=\set{x\inq^+ : \forall y\in A, y\leq x}
\end{equation}
\end{theorem}
\begin{proof}
Arbitrarily pick $x\in B$, and we see $\forall y\in A, y^2<2<x^2$. Then $\forall y\in A,y<x$. This implies $B\subseteq \set{x\inq^+:\forall y\in A, y\leq x}$. Let $x$ satisfy $\forall y\in A, y\leq x$. \As{$x^2<2$}. We immediately see $x=\max A$, but $\max A$ doesn't exist \CaC
\end{proof}
\fbox{\begin{minipage}{39em}
    By \myref{Theorem}{1.1.3}, we then can see that $\sup A$ does not exist. Now, we give this idea a name.
\end{minipage}}
\begin{definition}
\label{1.1.11}
  \textbf{(Definition of Completed Ordered Set)}
An ordered set $S$ satisfy least-upper-bound property if
 \begin{equation}
E\subseteq S\text{ and }E\neq \varnothing\implies \sup E\text{ exists }
\end{equation}
Also, we say $S$ is completed.
\end{definition}
\fbox{\begin{minipage}{39em}
$\sup A$ does not exists indicate that $\Q$ as an ordered set doesn't satisfy the least-upper-bound property. Before we close this section, we reveal the face of the twin brother of the least-upper-bound property. In fact, it is more like property itself in the mirror, since they are equivalent.
\end{minipage}}
\begin{theorem}
\label{1.1.12}
  \textbf{(LUB$\iff $GLB)} $S$ satisfy the least-upper-bound property if and only if  $S$ satisfy the greatest-lower-bound property.
\end{theorem}
\begin{proof}
  From left to right, consider a bounded below set $E$.  Let $L$ be the set of lower bounds of  $E$. We know $\sup L$ exists and every elements of $E$ is an upper bound of $L$, so $\sup L$ is an lower bound of $E$. Then $\sup L=\inf E$. The other direction use the same method. 
\end{proof}
\section{Ordered Fields}
\fbox{\begin{minipage}{39em}
In this section, we first give the definition of ordered fields, and prove basic result concerning positivity. Notice in this section that $x,y,z$ are all in  $\F$.
\end{minipage}}
\begin{axiom}
\label{1.2.1}
\textbf{(Ordered Field Axioms)} An ordered field $\F$ is a field that is not only a ordered set, but also satisfy the following axioms
\begin{equation}
y<z\implies x+y<x+z
\end{equation}
\begin{equation}
x>0\text{ and }y>0\implies xy>0
\end{equation}
\end{axiom}
\begin{theorem}
\label{1.2.2}
\textbf{(Negate reverse positivity)} $(x>0\iff -x<0)\text{ and }(x<0\iff -x>0)$
\end{theorem}
\begin{proof}
  Observe $x>0\implies x+(-x)>0+(-x)\implies -x<0$, and $x<0\implies x+(-x)<0+(-x)\implies -x>0$. Clearly, $x=0\implies -x=0$. Then the Theorem follows.
\end{proof}

\begin{theorem}
\label{1.2.3}
  
\textbf{(Multiply a negative number reverse positivity)} Given $y<0$, we have
\begin{equation}
  (x>0\iff xy<0)\text{ and }(x<0\iff xy>0)
\end{equation}
\end{theorem}
\begin{proof}
Observe $x>0\implies x(-y)>0\implies -xy>0\implies xy<0$, and 
$x<0\implies (-x)(-y)>0\implies xy>0$. Clearly, $x=0\implies xy=0$. Then the Theorem follows. 
\end{proof}
\begin{theorem}
\label{1.2.4}
\textbf{(Multiply on both side)}  Given $y<z$, we have
\begin{equation}
  (x>0\iff xy<xz)\text{ and }(x<0\iff xy>xz)
\end{equation}
\end{theorem}
\begin{proof}
First observe $y<z\implies z-y>0$. Now observe $x>0\implies x(z-y)>0\implies xz-xy>0\implies xz>xy$. 
Observe  $x<0\implies x(z-y)<0\implies xz<xy$. Obviously $x=0\implies  xy=xz$. Then the Theorem follows.
\end{proof}
\begin{theorem}
\label{1.2.5}
  \textbf{(Squares are nonnegative)} $x\neq 0\implies x^2>0$
\end{theorem}
\begin{proof}
  If $x>0$, then $x^2>0$ follows from the axiom. If $x<0$, then $x^2>0$ follows from \myref{Theorem}{1.2.2}
\end{proof}

\begin{corollary}
\label{1.2.6}
$1>0$
\end{corollary}

\begin{theorem}
\label{1.2.7}
\textbf{(Inverse preserve positivity)} $(x>0\iff \frac{1}{x}>0)\text{ and }(x<0\iff \frac{1}{x}<0)$ 
\end{theorem}
\begin{proof}
If $x>0$ but $\frac{1}{x}<0$, then $1<0$. The same logic applies to  $\frac{1}{x}>0\implies x>0$. Because $\frac{1}{0}$ does not exist, the theorem follows.
\end{proof}
\fbox{\begin{minipage}{39em}
If we were to define $x^{-2}:=x^{-1}x^{-1}$, we must realize $0^{-1}$ does not exist and realize we have not yet prove some common inequalities concerning integer powers. Notice that the following inequalities require base to be positive.  
\end{minipage}}

\begin{definition}
\label{1.2.8}
\textbf{(Definition of Inverse)} For all nonzero $x$ and naturals $p$, we define $x^{-p}:=(x^{-1})^p$, and define $x^0:=1$ 
\end{definition}
\begin{theorem}
\label{1.2.9}
  
\textbf{(Inequality when base is fixed)} Given a positive $a$ and two integer  $x,y$ where  $x<y$, we have
\begin{equation}
\begin{cases}
  a^x<a^y\iff a>1\\
  a^x=a^y \iff a=1 \\
  a^x>a^y \iff 0<a<1  
\end{cases}
\end{equation}
\end{theorem}
\begin{proof}
By \myref{Theorem}{1.2.4}, observe $1<a\iff  1<a<a^2\iff   1<a<a^2<a^3\iff  \cdots  \iff  1<a<\cdots <a^{y-x}\iff  a^x<a^y$. If $a=1$, we know  $a^x=1=a^y$. Again by \myref{Theorem}{1.2.4}, observe $0<a<1\iff  a^2<a<1\iff  \cdots \iff  a^{y-x}<\cdots <1\iff  a^{y}<a^{x}$. 
\end{proof}
\begin{theorem}
\label{1.2.10}
\textbf{(Inequality when integer power is fixed)} Given $0<b<c$ and  $z\inz$, we have 
\begin{equation}
\begin{cases}
  b^z<c^z \iff 0<z\\
  b^z=c^z \iff 0=z\\
  b^z>c^z\iff 0>z
\end{cases}
\end{equation}
\end{theorem}
\begin{proof}
  If $z>0$, then by \myref{Theorem}{1.2.4}, observe $0<b<c\implies 0<b^2<bc<c^2\implies b^3<b^2c<bc^2<c^3\implies \cdots \implies b^z<c^z $. Obviously, $0=z\implies b^z=1=c^z$. If $z<0$, then $b^z=(\frac{1}{b})^{-z}$ and $c^z=(\frac{1}{c})^{-z}$ by definition. Observe $b<c\implies 1=\frac{1}{b}b<\frac{1}{b}c=\frac{c}{b}\implies \frac{1}{c}<\frac{1}{b}$. By the logic above, we deduce $c^z=(\frac{1}{c})^{-z}<(\frac{1}{b})^{-z}=b^z$. Then the Theorem follows. 
\end{proof}
\begin{theorem}
\label{1.2.11}
\textbf{(Positivity of integer power)} If $a>0$, then $\forall x\inz,a^x>0$. If  $a<0$, then  $a^x>0\iff 2|x$
\end{theorem}
\begin{proof}
The former result is a direct consequence of \myref{Axiom}{1.2.1} and \myref{Theorem}{1.2.7}. The latter result is a direct consequence of \myref{Theorem}{1.2.3} and \myref{Theorem}{1.2.5}  
\end{proof}
\fbox{\begin{minipage}{39em}

Notice that if the base is negative, the corresponding inequalities can all be deduced by the above theorems with a little effort, although the results are quite messy. \\

Now we prove some arithmetic properties concerning nonzero base, unlike the above inequalities concerning only positive base. Notice that those properties of natural power inherited by integer power can be proven inductively, and that the base $x$ can be any nonzero number.        
\end{minipage}}
\begin{theorem}  
\label{1.2.12}
 \textbf{(Integer Power addition written in multiplication)} For all nonzero $x$ and integers $p,q$, we have 
 \begin{equation}
x^{p+q}=x^px^{q}
\end{equation}
\end{theorem}
\begin{proof}
  If $p,q$ are both positive, the Theorem is proven by induction. If $p,q$ are both negative, observe $x^px^q=(x^{-1})^{-p}(x^{-1})^{-q}=(x^{-1})^{-(p+q)}=x^{p+q}$, where the last equality hold true because $p+q<0$. If $p>0>q\text{ and }p+q>0$, observe $x^px^q=x^p(x^{-1})^{-q}=x^{p-(-q)}=x^{p+q}$. If $p>0>q\text{ and }p+q<0$, observe $x^px^q=x^p(x^{-1})^{-q}=(x^{-1})^{-(q+p)}=x^{p+q}$, where the last equality hold true because $p+q<0.$ 
\end{proof}
\begin{theorem}
\label{1.2.13}
\textbf{(Integer Power multiplication written in power of power of base)} For all nonzero $x$ and integers $p,q$, we have
\begin{equation}
  x^{pq}=(x^p)^q=(x^q)^p
\end{equation}
\end{theorem}
\begin{proof}
  If $p,q$ are both positive or if any of $p,q$ are zero, the proof is trivial. If $p<0<q$, observe $(x^p)^q=((x^{-1})^{-p})^q=(x^{-1})^{-pq}=x^{pq}$, where the last equality hold true because $pq<0$, and observe $(x^q)^p=((x^q)^{-1})^{-p}=((x^{-1})^{q})^{-p}=(x^{-1})^{-qp}=x^{qp}$, where the second equality hold true can be proven by induction.    
\end{proof}
\begin{theorem}
\label{1.2.14}
\textbf{(Multiplication of two number raised to the same integer power)} For all nonzero $x,y$ and integer  $p$, we have
 \begin{equation}
x^py^p=(xy)^p
\end{equation}
\end{theorem}
\begin{proof}
If $p\geq 0$, the proof is trivial. If $p<0$, then $x^py^p=(x^{-1})^{-p}(y^{-1})^{-p}=(x^{-1}y^{-1})^{-p}=((xy)^{-1})^{-p}=(xy)^p$.
\end{proof}

\section{Real Numbers Field}
\fbox{\begin{minipage}{39em}
Although the title of this section is "Real Numbers Field", here, we will not construct the real numbers field, nor use any common property of real numbers. In fact, we will not even use the symbol $\R$ in this section, since we are merely proving theorems about an ordered field with least-upper-bound property. We don't know if there exists any ordered field with least-upper-property. Let's say there does; yet, we don't know if such structure is unique. Let's say it is unique; yet, we don't know if that structure have relation with $\R$. Here, we will use the symbol $\F$ to denote an ordered field with least-upper-bound property. One should realize that we can use algorithm to define a subset containing $1\inF$ that is isomorphic to $\N$, and thereby we abuse the notation to denote that subset $\N$. A subfield of $\F$ isomorphic to $\Q$ can also be defined after we define $\Z$, so we also thereby abuse the notation to denote that subfield  $\Q$.         
\end{minipage}}
\begin{theorem}
\label{1.3.1}
$\N$ is unbounded above.
\end{theorem}
\begin{proof}
 \As{$\N$ is bounded above}. Because $1>0$, we know $\sup \N -1<\sup \N$. Then $\sup \N -1$ is not an upper bound of $\N$. Arbitrarily pick any $m \inn$ greater than $\sup \N -1$. We see $m>\sup \N-1\implies m+1>\sup \N$, where $m+1\inn\tCaC$
\end{proof}
\begin{corollary}
\label{1.3.2}
Both $\Z\text{ and }\Q$ are unbounded both above and below.
\end{corollary}
\begin{corollary}
\label{1.3.3}
  
\textbf{(Divided by $1$)} Given any $x\inF$, there exists $n\inz$ such that $n\leq x<n+1$
\end{corollary}
\begin{proof}
If $x>0$, let $S=\set{n\inn: n>x}$. Notice $S=\varnothing$ implies $\N$ is bounded above by $x$, so $S$ is nonempty. Then by well-ordering principle, we know $\min S$ exists. We now show \vi{$\min S-1\leq x<\min S$}. Observe that $\min S\in S\implies x<\min S$. \As{$\min S-1>x$}. We immediately see $\min S-1\in S\tCaC\vdone$.\\

If $x<0$, let $S=\set{n\inn:n\geq -x}$. Again, $S=\varnothing$ implies $\N$ is bounded above by $-x$, so $S$ is nonempty. Then by well-ordering principle, we know $\min S$ exists. We now show \blue{$-\min S\leq x<-\min S+1$}. Observe that $\min S\inS\implies \min S\geq -x\implies x\geq -\min S$. \As{$-\min S+1\leq x$}. Then $\min S-1\geq -x>0$; thus $\min S-1\inS\tCaC\bdone$\\

If $x=0$, then we let $n=0$. 
\end{proof}
\begin{theorem}
\label{1.3.4}
\textbf{(Archimedean Property)} Given $x,y\inF\text{ and }0<x$, there exists $n\inn$ such that $nx>y$ 
\end{theorem}
\begin{proof}
 Because $\N$ is unbounded above, we know  $\frac{y}{x}$ can not be an upper bound of $\N$, so we know  $\exists n\inn,n>\frac{y}{x} $. Then because $x>0$, we can deduce $nx>y$.
\end{proof}
\begin{theorem}
\label{1.3.5}
\textbf{($\Q$ is dense in $\F$)} Given $x,y\inF\text{ and }x<y$, we know there exists $p \inq$ such that $x<p<y$
\end{theorem}
\begin{proof}
Every rational, positive or negative, can be expressed in the form $\frac{m}{n}$ for some integer $m$ and naturals $n$. We seek to find some integer $m$ and $n$ such that $x<\frac{m}{n}<y$. Notice that $x<\frac{m}{n}<y\iff nx<m<ny$. Because $m$ has to be an integer, we know for $nx<m<ny$ to hold true, we must first have $ny-nx>1$. Because $y-x>0$, by Archimedean Property, there exists $n\inn$ such that $ny-nx=n(y-x)>1$. By \hyperref[1.3.3]{Corollary 1.3.3}, we know there exists $m \inz$ such that $m\leq  ny<m+1$.\\

Notice $m=ny$ if and only if $ y\inq$. So we can split the proof into two cases.\\

\vi{Case 1: $y\inq$}\\

We see that the set $\set{r\inq:r<y}$ have supremum $y$, since $y\inq$. Then $x<y$ tell us $x$ is not an upper bound of the set, then we can pick some rational $r$ in the set greater than $x$, so $x<r<y\vdone$.\\

\blue{Case 2: $y\not\inq$}\\ 

We know $m<ny<m+1$. $ny<m+1$ tell us $nx<ny-1<m$, so $nx<m<ny\bdone$ 
\end{proof}
\begin{theorem}
\label{1.3.6}
\textbf{(Positive root of power uniquely exists)} For all natural $n$ and $y>0$, there exists a unique positive $x$ such that $x^n=y$
\end{theorem}
\begin{proof}
By \hyperref[1.2.10]{Theroem 1.2.10}, we know two different positive numbers $0<x<x'$ are different when raised to the power of $n$, being $0<x^n<(x')^{n}$, so if such positive power exists, it must be unique.\\

We have handled the uniqueness part of the proof. Denote $E:= \set{m \inF^+: m^n<y}\text{ and }x:=\sup E$. Now we do the existence part by proving \vi{$x$ exists} and  \blue{$x^n=y$}.\\

To show \vi{ $x=\sup \set{m \inF^+: m^n<y}$ exists}, we only have to show the set $\set{m \inF^+: m^n<y}$ is nonempty and bounded above. In other word, we wish to construct function $a\inF^+$ and $b$ of $y$ such that for all positive input $y>0$, we have $a^n<y$ and $(0<m^n<y\longrightarrow m<b)$. In the followings, the domain of $a$ and $b$ are only positives.\\

First we construct $a$. By \myref{Theorem}{1.2.9}, we know if $a<\min \set{1,y}$ , then $a^n<a<y$, so we construct $a$ such that $0<a<\min \set{1,y}$. Notice that $a$ must be positive because we are constructing a number in $E$, where $E$ contain only positives. Express $a$ in the form  $a=\frac{p}{q}$ where $p,q$ are both function of $y$. In the process of  construction, We must be careful to make sure $a$ exists for all positive $y$.\\

To satisfy $0<a$, we need only guarantee $p,q$ are always of the same sign for all positive $y$. If such $p,q$ exists, we can change both sign of  $p,q$ when they are negative, and get two positive function. So, we can just require $p,q$ to be positive for all positive $y$.\\      

To satisfy $a<1$, observe $a<1\iff \frac{p}{q}<1\iff p<q$. The easiest construction is to let $q=p+c$ where $c$ is positive.\\

To satisfy $a<y$, observe $a<y\iff \frac{p}{q}<y\iff p(y-1)+cy=qy-p>0$. The easiest construction is to let $p(y-1)+cy=y^2$, which is possible, if we let $p=y$ and $c=1$. In this case $p=y>0$ and $q=y+1>0$, and  $a=\frac{y}{y+1}$. We finished proving $E$ is nonempty. \textbf{(Notice $c=y^2,p=y^3,q=y^3+y^2,a=\frac{y^3}{y^3+y^2}$ also do the trick)}\\

Now we construct $b$. By \myref{Theorem}{1.2.10}, we know if $0<b\text{ and }0<m^n<b^n$, then $m<b$, so we construct $b$ such that $y<b^n$ which lead to $0<m^n<y<b^n$ if  $m^n<y$. Because $y>0$, this is fairly easy. Simply let $b=y+1$, so we have $b>1\text{ and }b>y$; thus by \myref{Theorem}{1.2.9}, we have $b^n>b>y$, finishing proving $E$ is bounded above, where  $b=y+1$ is an upper bound. $\vdone$ \\

To show \blue{$x^n=y$}, we show $x^n\geq y$ and $x^n\leq y$. We will assume that $x^n<y$ or $x^n>y$, but before we do such, let's see what property from which can we possibly draw contradiction. Notice that because we just prove the existence of the supremum of $E$, we haven't use the fact that $x=\sup E$ in anywhere of our proof. We know
\begin{equation}
x=\sup E\iff \forall d>0,
\begin{cases}
x+d\not\in E\text{ ($x$ is an upper bound) }\\
\text{ and }\\
x-d \text{ is not an upper bound of $E$ (the \emph{least} upper bound)}
\end{cases}    
\end{equation}
So, you see, we wish to construct  a small and positive $h\text{ and }k$ such that if we assume $x^n<y\text{ or }x^n>y$ we can draw $x+h\in E$ or $x-k$ is an upper bound of $E$. \textbf{(We are going to assume $\sup E$ is smaller or greater than $\sqrt[n]{y}$)} \\

Observe $x+h\inE\iff (x+h)^n<y\iff (x+h)^n-x^n<y-x^n$, and observe $x-k$ is an upper bound of $E\iff (m^n<y\longrightarrow m<x-k)\iff (m\geq x-k\longrightarrow m^n\geq y)\iff (m\geq x-k\longrightarrow x^n-m^n\leq x^n-y)$.\\

Notice that the act of  subtracting  $x^n$ at the both side of the inequality play an important role in our proof: not only does the act allow us to use the identity $a^n-b^n=(a-b)(a^{n-1}+a^{n-2}b+\cdots +b^{n-1})$, and the act also tell us between $x+h \in E\text{ and }x-k$ is an upper bound of $E$, which contradiction statement should we draw from $x^n>y$. If $y-x^n<0$, then  $y-x^n<0<(x+h)^n-x^n$, so we can not possibly draw $x+h\inE$ from $x^n>y$. \textbf{(If $\sup E>\sqrt[y]{n}$, then it is too big, we can find a smaller upper bound of $E$)}\\

\As{$x^n>y$}. We wish to construct positive $k$ such that \teal{$m\geq x-k \longrightarrow x^n-m^n\leq x^n-y$}, so we can draw the contradiction $x-k$ is an upper bound of $E$. \\ 

Notice that $m\geq x-k\implies x^n-m^n\leq x^n-(x-k)^n$, so if $x^n-(x-k)^n\leq x^n-y$, our proof at this part is finished. Now our job is to single out the $k$ in the inequality to give an condition such that $x^n-(x-k)^n\leq x^n-y$ hold if the condition hold. It is easy to see that computing the polynomial of $k$ on left hand side of inequality and to prove such positive $k$ exists for all $n$ is almost impossible. Thus, we take a possible but actually non-existing risk in our next step of the proof. Use the $a^n-b^n$ identity and that $x-k<x$ to deduce
\begin{equation}
x^n-(x-k)^n=k(x^{n-1}+x^{n-2}(x-k)+\cdots +(x-k)^{n-1})\leq knx^{n-1}
\end{equation}
So, if we have $knx^{n-1}\leq x^n-y$, which is equivalent to $k\leq \frac{x^n-y}{nx^{n-1}}$, our proof is partially finished. Notice that $x^n-y>0\text{ and }nx^{n-1}>0$, so $\frac{x^n-y}{nx^{n-1}}>0$; thus the positive $k$ exists.  $\tdone\tCaC$\textbf{(The reason I use the word "possible risk" is that if we use an identity that show us $x^n-(x-k)^n$ smaller than a quantity greater than $x^n-y$, the proof can not be done.)} \\

\As{$x^n<y$}. We wish to construct positive $h$ such that \teal{$(x+h)^n-x^n<y-x^n$}, so we can draw the contradiction $x+h\in E$.\\

Again, we use the same identity to deduce
 \begin{equation}
   (x+h)^n-x^n=h((x+h)^{n-1}+(x+h)^{n-2}x+\cdots +x^{n-1})<hn(x+h)^{n-1}
\end{equation}
To single out the $h$ in the  $hn(x+h)^{n-1}$, notice that we can take the risk to add the constraint $h<1$ at the end of our construction to have $(x+h)^n-x^n<hn(x+h)^{n-1}<hn(x+1)^{n-1}$. Then, if we have $hn(x+1)^{n-1}<y-x^n$, which is equivalent to $h<\frac{y-x^n}{n(x+1)^{n-1}} $, our proof is finished. To sum up, any $h$ satisfy  $h<\min \set{1,\frac{y-x^n}{n(x+1)^{n-1}}}$ does the trick, and such $h$ exists, since  $0<\min \set{y-x^n,n(x+1)^{n-1}}$. $\tdone\tCaC\bdone$   
\end{proof}
\fbox{\begin{minipage}{39em}
If you want a proof with less of my commentary, here you go.
\end{minipage}}
\begin{proof}
  Observe $\min \set{1,\frac{y}{2}}\in E$ and $\max \set{1,y}$ is an upper bound of $E$, so $\sup E$ exists.  Denote $x:=\sup E$. \As{$x^n>y$}. Observe $x-k\text{ is an upper bound of $E$ }\iff m^n<y\longrightarrow m<x-k\iff m\geq x-k\longrightarrow m^n\geq y$. We know $(x-k)^n\geq y\iff x^n-(x-k)^n\leq x^n-y$, and know $x^n-(x-k)^n\leq knx^{n-1}$. If we let $0<k\leq \frac{x^n-y}{nx^{n-1}}$, then $x-k$ is an upper bound of  $E\tCaC.$ \As{$x^n<y$}. Observe $x+h\inE\iff (x+h)^n-x^n<y-x^n$. We know that $(x+h)^n-x^n<hn(x+h)^{n-1}$ and that if $h<1$, we have  $hn(x+h)^{n-1}<hn(x+1)^{n-1}$. So we let  $h=\min \set{1,\frac{y-x^n}{n(x+1)^{n-1}}}$, and we can see $x+h\inE\tCaC$. 
\end{proof}
\begin{definition}
\label{1.3.7}
The number $x$ in \hyperref[1.3.6]{Theorem 1.3.6} is written  $x=\sqrt[n]{y}$ or $x=y^{\frac{1}{n}}$.
\end{definition}
\fbox{\begin{minipage}{39em}
The above theorem is by far the trickiest we have seen. Often the theorem is proven only in special case $\sqrt{2}$ as a classical example in the first class of analysis. Here, we prove a general result. The following theorem is to make sure the definition of rational power make sense . In last section we prove some inequalities concerning integer power. Here, we prove those inequalities are inherited by rational power. Of course, the arithmetic properties of rational power, also inherited from those of integer power, will be proven after these inequalities.   \\ 


About the coverage of definition, notice that we didn't and won't define $y^{\frac{1}{n}}$ when $y$ is negative. Also, notice that for all nonzero rational $s$, we can define $0^s=0$. 
\end{minipage}}
\begin{theorem}
\label{1.3.8}
Given $m,p\inz\text{ and }n,q\inn\text{ and }a>0\text{ and }\frac{m}{n}=\frac{p}{q}$, we have 
\begin{equation}
(a^m)^{\frac{1}{n}}=(a^p)^{\frac{1}{q}}
\end{equation}
\end{theorem}
\begin{proof}
Observe
\begin{align}
  x&= (a^m)^{\frac{1}{n}}\\
  \liff x^n&=a^m\\
  \liff (x^n)^q&=(a^m)^q\\
\liff x^{nq}&= a^{mq}=a^{np} \rap{because $n,m,q\inz$}\\
  \liff (x^q)^n&=(a^p)^n \rap{again, because $n,p,q\inz$}\\
  \liff x^q&=a^p\rap{by \hyperref[1.3.6]{Theorem 1.3.6}}\\
  \liff x&=(a^p)^{\frac{1}{q}}\rap{by \hyperref[1.3.6]{Theroem 1.3.6}}\\
  \liff (a^m)^{\frac{1}{n}}&=x=(a^p)^{\frac{1}{q}} 
\end{align}
\end{proof}
\begin{definition}
\label{1.3.9}
\textbf{(Definition of Rational Powers)} Given a rational $r=\frac{p}{q}$, where $q\inn$. For all $a$, we define $a^r=(a^p)^{\frac{1}{q}}$
\end{definition}
\begin{theorem}
\label{1.3.10}
\textbf{(Inequality when base is fixed)} Given a positive $a$ and two rational  $x,y$ where  $x<y$, we have
 \begin{equation}
\begin{cases}
  a^x<a^y\iff a>1\\
  a^x=a^y\iff a=1\\
  a^x>a^y\iff 0<a<1
\end{cases}
\end{equation}
\end{theorem}
\begin{proof}
  Express $x,y$ in the form $x=\frac{q}{p},y=\frac{n}{m}$ where $q,n\inz\text{ and }m,p\inn$. Notice $x<y\implies \frac{q}{p}<\frac{n}{m}\implies mq<np$. Observe
  \begin{align}
    a^x<a^y &\implies a^{\frac{q}{p}}<a^{\frac{n}{m}}\\
  &\implies a^q=(a^{\frac{q}{p}})^p<(a^{\frac{n}{m}})^p\rap{by \myref{Theorem}{1.2.10}}\\
  &\implies a^{mq}=(a^q)^m<((a^{\frac{n}{m}})^p)^m=((a^{\frac{n}{m}})^m)^p=(a^n)^p=a^{np}\\
  &\implies a>1\rap{by \myref{Theorem}{1.2.9}} 
  \end{align}
  Notice that the above implication still hold true if $<$ is replaced by $>$ and $>$ is replaced by  $<$, and notice that the above implication still hold true if $<$ and  $>$ are all replaced by $=$, so we in fact have three implications. The Theorem follows from the three implication.
   
\end{proof}
\begin{theorem}
\label{1.3.11}
\textbf{(Inequality when rational power is fixed)} Given $0<b<c$ and  $z\inq$, we have
\begin{equation}
\begin{cases}
  b^z<c^z \iff 0<z\\
  b^z=c^z \iff 0=z\\
  b^z>c^z\iff 0>z
\end{cases}
\end{equation}
\end{theorem}
\begin{proof}
Express $z=\frac{q}{p}$, where $q$ is an integer and $p$ is a natural. Notice $q\text{ and }z$ are of the same sign. Observe by \myref{Theorem}{1.2.10}, we have $b^{\frac{q}{p}}<c^{\frac{q}{p}}\implies b^q=(b^{\frac{q}{p}})^p<(c^{\frac{q}{p}})^p=c^q\implies 0<q$.\\

  Notice that the above implication still hold true if $<$ is replaced by $>$ and $>$ is replaced by  $<$, and notice that the above implication still hold true if $<$ and  $>$ are all replaced by $=$, so we in fact have three implications. The Theorem follows from the three implication.
\end{proof}
\fbox{\begin{minipage}{39em}
Now we prove the arithmetic properties of rational power concerning only positive base. Notice there is no definition of a negative raised to the power of a rational. \red{From Here}
\end{minipage}}
\begin{theorem}
\label{1.3.12}
 \textbf{(Rational Power addition written in multiplication, inherited form \myref{Theorem}{1.2.12} and \myref{Theorem}{1.2.13})} Given $r,s\inq\text{ and }a>0$, we have
\begin{equation}
a^{r+s}=a^ra^s
\end{equation}
\end{theorem}
\begin{proof}
Express $r,s$ in the form $r=\frac{p}{q},s=\frac{m}{n}$ where $q,n\inn\text{ and }p,m \inz$. Observe
\begin{align}
  (a^{r+s})^{nq}&=(a^{\frac{np+mq}{nq}})^{nq}\\
  &=a^{np+mq} 
\end{align}
And observe
\begin{align}
  (a^ra^s)^{nq}&= (a^{\frac{p}{q}}a^{\frac{m}{n}})^{nq} \\
  &= (a^{\frac{p}{q}})^{nq}(a^{\frac{m}{n}})^{nq}\rap{because $nq\inn$}\\
  &=((a^{\frac{p}{q}})^q)^n((a^{\frac{m}{n}})^n)^q\\
&=(a^p)^n(a^m)^q\\
&= a^{np+mq}=(a^{r+s})^{nq}\rap{\myref{Theorem}{1.2.12}\text{ and }\myref{Theorem}{1.2.13}} 
\end{align}
Because the rational power of a positive must also be positive, we can deduce $a^ra^s=a^{r+s}$ by \hyperref[1.3.6]{Theorem 1.3.6}
\end{proof}
\begin{theorem}
\label{1.3.13}
\textbf{(Rational Power of Rational Power written in power multiplication, inherited from \myref{Theorem}{1.2.13})} Given $r,s\inq\text{ and }a>0$, we have
\begin{equation}
  (a^r)^s=a^{rs}
\end{equation}
\end{theorem}
\begin{proof}
Express $r,s$ in the form $r=\frac{p}{q},s=\frac{m}{n}$ where $q,n\inn\text{ and }p,m \inz$. Observe
\begin{align}
  (a^{rs})^{nq}&=(a^{\frac{mp}{nq}})^{nq}\\
  &= a^{mp}
\end{align}
And observe
\begin{align}
  ((a^r)^s)^{nq}&=(((a^{\frac{p}{q}})^{\frac{m}{n}})^n)^q\\
  &=((a^{\frac{p}{q}})^m)^q\\
 &=(a^{\frac{p}{q}})^{mq}\rap{\myref{Theorem}{1.2.13}}\\
  &=((a^{\frac{p}{q}})^q)^m \rap{\myref{Theorem}{1.2.13}}\\
  &=(a^p)^m=a^{mp}=(a^{rs})^{nq} \rap{\myref{Theorem}{1.2.13}} 
\end{align}
\end{proof}
\begin{corollary}
\label{1.3.14}
$a^{\frac{q}{p}}=(a^{\frac{1}{p}})^q$
\end{corollary}
\section{Irrational Power}
\fbox{\begin{minipage}{39em}
    After the rational power, we now try to define irrational power. In most of the textbooks, irrational power as a rigorous definition often come after the definition of Euler number, but here, we define the irrational power with a technique not so advanced and to be fair quite cumbersome compared to the approaches in most textbooks.\\

    The definition will be split into two parallel part, each definition have a lemma beforehand to guarantee the definition exists. The properties inherited from those of rational power will be proven as soon as it can be proven, and the common inequalities will be proven last. Notice that we still don't define $y^{\frac{1}{n}}$ when $y$ is negative.
\end{minipage}}
\begin{lemma}
\label{1.4.1}
Given $b>1$, for $x\inF$, define  $B(x):=\set{b^t:t\inq,t\leq x}$. Then $\forall x\inF, \sup B(x)$ exists.
\end{lemma}
\begin{proof}
For all $x$,  we know $B(x)$ is nonempty, since if not, $\Q$ is bounded below. Because $\Q$ is nonbounded above, we can pick a rational  $y$ greater than $x$, and observe that $t\leq x\implies t\leq y$. Then because $b>1$, we deduce  $\forall b^t\inB(x),b^t\leq b^y $; thus $b^y$ is an upper bound of  $B(x)$.      
\end{proof}
\begin{definition}
\label{1.4.2}
  \textbf{(The First Half of Irrational Power Definition)} For all $b,x\inF$ where $b>1$, we define $b^x:=\sup B(x)$, where $B$ is the function in \hyperref[1.4.1]{Lemma 1.4.1}. 
\end{definition}
\fbox{\begin{minipage}{39em}
The above definition is in fact not very appropriate, since we have already define $b^x$ where  $x\inq$. We don't know if the above definition is consistent with our old definition. The next theorem is to show it is.  
\end{minipage}}
\begin{theorem}
\label{1.4.3}
  \textbf{(Consistency of power definitions)} If $b>1\text{ and }r\inq$, then $b^r=\sup B(r)$
\end{theorem}
\begin{proof}
Because $1<b$, we know  $b^r=\max  B(r)$, then $\sup  B(r)=\max  B(r)=b^r$   
\end{proof}
\fbox{\begin{minipage}{39em}
Now, we are going to prove that the definition of real power do inherit the properties of rational power (i.e. \hyperref[1.3.10]{Theorem 1.3.10} and \hyperref[1.3.11]{Theorem 1.3.11}), but before that, we will prove a general result about supremum.    
\end{minipage}}
\begin{lemma}
\label{1.4.4}
Let $A,B$ and be two bounded above subset of  $\F$, containing only nonnegative numbers. Define $AB:=\set{ab:a\inA,b\inB}$. We have    
\begin{equation}
\sup AB=\sup A \sup B
\end{equation}
\end{lemma}
\begin{proof}
Because $A,B$ contain only nonnegative numbers, we know $a\leq \sup A$ and $b\leq \sup B$ implies $ab\leq \sup A \sup B$, so $\sup A \sup B$ is an upper bound of $AB$. Now we show \vi{$\sup A \sup B$ is the \textit{least} upper bound of $AB$}.\\

\As{there is an upper bound $x$ of $AB$ smaller than $\sup A \sup B$}. Express $x$ in the form  $x=\sup A \frac{x}{\sup A}$. Because $x<\sup A \sup B$, we know $\frac{x}{\sup A}<\sup B$, which implies there is a number  $b\inB$ greater than $\frac{x}{\sup A}$. Observe $b>\frac{x}{\sup A}\implies \frac{x}{b}<\sup A$, which implies that there is a number $a\in A$ greater than  $\frac{x}{b}$. Observe  $\frac{x}{b}<a\implies x<ab \tCaC\vdone$  
\end{proof}
\begin{theorem}
\label{1.4.5}
\textbf{(Power addition written in multiplication, inherited from Theorem 1.3.6 and Theorem 1.3.7)} Given $r,s\inF$ and $b>1$, we have
\begin{equation}
  b^{r+s}=b^rb^s
\end{equation}
\end{theorem}
\begin{proof}
We prove \vi{$B(r+s)=\set{xy:x\inB(r)\text{ and }y\inB(s)}$}. Denote the set on right side $E$. Observe that $xy\in E \implies \exists q\leq r\inq,x=b^q\text{ and }\exists m\leq s\inq,y=b^m\implies xy=b^{q+m}\leq b^{r+s}\implies xy\in B(r+s)$. Then we know $E\subseteq B(r+s)$. Given $b^d\in B(r+s)$, we know $d\leq r+s$, so if we express $b^d=b^rb^{d-r}$, we are sure $b^{d-r}\in B(s)$ and $b^r\in B(r)$, which implies $b^d\in E$. Then we can deduce $B(r+s)\subseteq E$.$\vdone$\\

By \hyperref[1.4.4]{Lemma 1.4.4}, our proof is finished.
\end{proof}
\fbox{\begin{minipage}{39em}
Before we prove $(b^r)^s=b^{rs}$ if $b>1$, we must finish the definition of the reals power, since $b^r$ may be less than $1$. Notice that the function $A$ below, is identical to $B$ above.
\end{minipage}}
\begin{lemma}
\label{1.4.6}
  Given $0<a<1$, for all $x\inF$, define $A(x):=\set{a^t:t\inq,t\leq x}$. Then $\forall x\inF, \inf A(x)$ exists.
\end{lemma}
\begin{proof}
For all $x$,  we know $A(x)$ is nonempty, since if not, $\Q$ is bounded below. Because $\Q$ is nonbounded above, we can pick a rational  $y$ greater than $x$, and observe that $t\leq x\implies t\leq y$. Then because $0<a<1$, we deduce  $\forall a^t\inA(x),a^t\geq a^y $; thus $a^y$ is an lower bound of  $A(x)$.      
\end{proof}
\begin{definition}
\label{1.4.7}
\textbf{(The Second Half of Real Power Definition)} For all $a,x\inF$ where $0<a<1$, we define  $a^x:=\inf  A(x)$, where $A$ is the function in Lemma above.
\end{definition}
\begin{theorem}
\label{1.4.8}
\textbf{(Consistency of power definitions)} If $0<a<1$ and  $r\inq$, then $a^r=\inf A(r)$
\end{theorem}
\begin{proof}
Because $0<a<1$, we know  $a^r=\min A(r)$, then $\inf A(r)=\min A(r)=a^r$   
\end{proof}
\begin{lemma}
\label{1.4.9}
Let $A,B$ and be two bounded below subset of  $\F$, containing only nonnegative numbers. Define $AB:=\set{ab:a\inA,b\inB}$. We have    
\begin{equation}
\inf  AB=\inf  A \inf  B
\end{equation}
\end{lemma}
\begin{proof}
Because $A,B$ contain only nonnegative numbers, we know $a\geq \inf  A$ and $b\geq \inf  B$ implies $ab\geq \inf  A \inf  B$, so $\inf  A \inf  B$ is an lower bound of $AB$. Now we show \vi{$\inf  A \inf  B$ is the \textit{greatest} lower bound of $AB$}.\\

If $\inf A=0$, we know $\inf AB=0$, since if not, we can arbitrarily pick nonzero $b\inB$ and see for all $a\inA$, we have $a>(\inf AB)b>0$. \As{there is an lower bound $x$ of $AB$ greater than $\inf  A \inf  B$}. Express $x$ in the form  $x=\inf  A \frac{x}{\inf  A}$. Because $x>\inf  A \inf  B$, we know $\frac{x}{\inf  A}>\inf  B$, which implies there is a number  $b\inB$ less than $\frac{x}{\inf  A}$. Observe $b<\frac{x}{\inf  A}\implies \frac{x}{b}>\inf  A$, which implies that there is a number $a\in A$ less than  $\frac{x}{b}$. Observe  $\frac{x}{b}>a\implies x>ab \tCaC\vdone$  
\end{proof}
\begin{theorem}
\label{1.4.10}
\textbf{(Power addition written in multiplication, inherited from \hyperref[1.3.10]{Theorem 1.3.10} and \hyperref[1.3.11]{Theorem 1.3.11})} Given $r,s\inF$ and $0<a<1$, we have
\begin{equation}
  a^{r+s}=a^ra^s
\end{equation}
\end{theorem}
\begin{proof}
We prove \vi{$A(r+s)=\set{xy:x\inA(r)\text{ and }y\inA(s)}$}. Denote the set on right side $E$. Observe that $xy\in E \implies \exists q\leq r\inq\text{ and }m\leq s\inq, x=a^q\text{ and }y=a^m\implies xy=a^{q+m}\geq a^{r+s}\implies xy\in A(r+s)$. Then we know $E\subseteq A(r+s)$. Given $a^d\in A(r+s)$, we know $d\leq r+s$, so if we express $a^d=a^ra^{d-r}$, we are sure $a^{d-r}\in A(s)$ and $a^r\in A(r)$, which implies $a^d\in E$. Then we can deduce $A(r+s)\in E$.$\vdone$\\

By \hyperref[1.4.9]{Lemma 1.4.9}, our proof is finished.
\end{proof}
\begin{theorem}
\label{1.4.11}
\textbf{(Power of Power written in power multiplication, inherited from \hyperref[1.3.11]{Theorem 1.3.11})} Given $r,s\inF\text{ and }a>0$, we have
\begin{equation}
  (a^r)^s=a^{rs}
\end{equation}
\end{theorem}
\begin{proof}
We split the proof in eight main cases: $\begin{cases}
  1) & a>1\text{ and }r>0\text{ and }s>0 \\
  2) & a>1 \text{ and }r<0\text{ and }s>0\\
  3) & a<1\text{ and }r>0\text{ and }s>0\\
  4)& a<1\text{ and }r<0\text{ and }s>0
\end{cases}$\\

The boundary cases where $a=1\text{ or }a^r=1$ is easy to handle, which we do after the main cases.\\

\vi{Case 1: $a>1\text{ and }r>0\text{ and }s>0$}\\

Denote $x:= a^r=\sup \set{a^t:t\inq,t\leq r}$, and denote $E:=\set{x^u:u\inq,u\leq s}$. We seek to prove $a^{rs}=\sup E$. \As{there exists $x^u\inE$  greater than $a^{rs}$}. Observe that $a^{rs}<x^u\implies \begin{cases}
  a^r<a^{r \frac{s}{u}}<x & \text{ if $s>0$ }\\

\end{cases}$
\end{proof}
\section{Logarithm} 
\section{Exercises}
\begin{question}{}{}
Given a nonzero rational $r$, prove that $r+x$ and $rx$ are irrational. 
\end{question}
\begin{question}{}{}
Prove that no rational $r$ satisfy  $r^2=12$
\end{question}
\begin{question}{}{}
Let $E$ be a nonempty subset of an ordered set; suppose $\alpha$ is a lower bound and $\beta$ is an upper bound of $E$. Prove $\alpha<\beta$ 
\end{question}
\begin{question}{}{}
Let $A$ be a nonempty subset of real numbers which is bounded below.  Define $-A:=\set{-x: x\inA}$. Prove that
\begin{equation}
\inf A=-\sup (-A)
\end{equation}
\end{question}
\end{document}
