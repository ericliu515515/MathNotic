\documentclass{report}

%%%%%%%%%%%%%% macros.tex %%%%%%%%%%%%%%
% Place your custom macros here, if any.

%%%%%%%%%%%%%% letterfonts.tex %%%%%%%%%%%%%%
% Place your font setup here, if any.

%%%%%%%%%%%%%% preamble.tex %%%%%%%%%%%%%%
\usepackage[T1]{fontenc}
\usepackage{lmodern}
\usepackage{etoolbox}
\usepackage{pdfpages}
\usepackage{transparent}
\usepackage[utf8]{inputenc}
\usepackage[english]{babel}

% Page Setup
\usepackage[tmargin=2cm, rmargin=0.5in, lmargin=0.5in, bmargin=80pt, footskip=.2in]{geometry}

% Mathematics
\usepackage{amsmath,amsfonts,amsthm,amssymb,mathtools}
\usepackage{xfrac}
\usepackage[makeroom]{cancel}
\usepackage{enumitem}
\usepackage{nameref}
\usepackage{multicol,array}
\usepackage{tikz-cd}
\usepackage[ruled,vlined,linesnumbered]{algorithm2e}

% Colors
\usepackage[dvipsnames]{xcolor}
\definecolor{myg}{RGB}{56, 140, 70}
\definecolor{myb}{RGB}{45, 111, 177}
\definecolor{myr}{RGB}{199, 68, 64}
% Define more colors here...

% Hyperlinks
\usepackage{bookmark}
\usepackage{hyperref}
\hypersetup{
    pdftitle={Assignment},
    colorlinks=true, linkcolor=doc!90,
    bookmarksnumbered=true,
    bookmarksopen=true
}

% Figures and Graphics
\usepackage{import}
\usepackage{svg}
\newcommand{\incfig}[1]{%
    \def\svgwidth{\columnwidth}
    \import{./figures/}{#1.pdf_tex}
}

% Text-related
\usepackage{blindtext}
\usepackage{fontsize}
\changefontsize[14]{14}
\setlength{\parindent}{0pt}

% Theorems and Definitions
\usepackage{amsthm}
\renewcommand\qedsymbol{$\blacksquare$}

% Define a new theorem style
\newtheoremstyle{mytheoremstyle}% name
  {}% Space above
  {}% Space below
  {\sffamily}% Body font
  {}% Indent amount
  {\bfseries}% Theorem head font
  {.}% Punctuation after theorem head
  {.5em}% Space after theorem head
  {}% Theorem head spec (can be left empty, meaning ‘normal’)

% Apply the new theorem style to theorem-like environments
\theoremstyle{mytheoremstyle}
\newtheorem{theorem}{Theorem}[section]
\newtheorem{definition}{Definition}[section]
\newtheorem{corollary}{Corollary}[section]
\newtheorem{lemma}{Lemma}[section]
\newtheorem{axiom}{Axiom}[section]

% tcolorbox Setup
\usepackage[most,many,breakable]{tcolorbox}

% Define custom tcolorbox environments here...

%================================
% EXAMPLE BOX
%================================
\newtcbtheorem[definition]{Example}{Example}
{%
    colback = myexamplebg,
    breakable,
    colframe = myexamplefr,
    coltitle = myexampleti,
    boxrule = 1pt,
    sharp corners,
    detach title,
    before upper=\tcbtitle\par\smallskip,
    fonttitle = \bfseries,
    description font = \mdseries,
    separator sign none,
    description delimiters parenthesis,
}
{ex}

%================================
% Solution BOX
%================================
\makeatletter
\newtcolorbox{solution}{enhanced,
	breakable,
	colback=white,
	colframe=myg!80!black,
	attach boxed title to top left={yshift*=-\tcboxedtitleheight},
	title=Solution,
	boxed title size=title,
	boxed title style={%
			sharp corners,
			rounded corners=northwest,
			colback=tcbcolframe,
			boxrule=0pt,
		},
	underlay boxed title={%
			\path[fill=tcbcolframe] (title.south west)--(title.south east)
			to[out=0, in=180] ([xshift=5mm]title.east)--
			(title.center-|frame.east)
			[rounded corners=\kvtcb@arc] |-
			(frame.north) -| cycle;
		},
}
\makeatother

%================================
% Question BOX
%================================
\makeatletter
\newtcbtheorem{question}{Question}{enhanced,
	breakable,
	colback=white,
	colframe=myb!80!black,
	attach boxed title to top left={yshift*=-\tcboxedtitleheight},
	fonttitle=\bfseries,
	title={#2},
	boxed title size=title,
	boxed title style={%
			sharp corners,
			rounded corners=northwest,
			colback=tcbcolframe,
			boxrule=0pt,
		},
	underlay boxed title={%
			\path[fill=tcbcolframe] (title.south west)--(title.south east)
			to[out=0, in=180] ([xshift=5mm]title.east)--
			(title.center-|frame.east)
			[rounded corners=\kvtcb@arc] |-
			(frame.north) -| cycle;
		},
	#1
}{def}
\makeatother
\makeatletter
\newtcbtheorem{qstion}{Question}{enhanced,
    breakable,
    colback=white,
    colframe=mygr,
    attach boxed title to top left={yshift*=-\tcboxedtitleheight},
    fonttitle=\bfseries,
    title={#2},
    boxed title size=title,
    boxed title style={%
        sharp corners,
        rounded corners=northwest,
        colback=tcbcolframe,
        boxrule=0pt,
    },
    underlay boxed title={%
        \path[fill=tcbcolframe] (title.south west)--(title.south east)
        to[out=0, in=180] ([xshift=5mm]title.east)--
        (title.center-|frame.east)
        [rounded corners=\kvtcb@arc] |-
        (frame.north) -| cycle;
    },
    #1
}{def}
\makeatother

%%%%%%%%%%%%%%%%%%%%%%%%%%%%%%%%%%%%%%%%%%%
% TABLE OF CONTENTS
%%%%%%%%%%%%%%%%%%%%%%%%%%%%%%%%%%%%%%%%%%%
\usepackage{tikz}
\definecolor{doc}{RGB}{0,60,110}
\usepackage{titletoc}
\contentsmargin{0cm}
\titlecontents{chapter}[14pc]
{\addvspace{30pt}%
	\begin{tikzpicture}[remember picture, overlay]%
		\draw[fill=doc!60,draw=doc!60] (-7,-.1) rectangle (-0.9,.5);%
		\pgftext[left,x=-4.5cm,y=0.2cm]{\color{white}\Large\sc\bfseries Chapter\ \thecontentslabel};%
	\end{tikzpicture}\color{doc!60}\large\sc\bfseries}%
{}
{}
{\;\titlerule\;\large\sc\bfseries Page \thecontentspage
	\begin{tikzpicture}[remember picture, overlay]
		\draw[fill=doc!60,draw=doc!60] (2pt,0) rectangle (4,0.1pt);
	\end{tikzpicture}}%
\titlecontents{section}[3.7pc]
{\addvspace{2pt}}
{\contentslabel[\thecontentslabel]{2pc}}
{}
{\hfill\small \thecontentspage}
[]
\titlecontents*{subsection}[3.7pc]
{\addvspace{-1pt}\small}
{}
{}
{\ --- \small\thecontentspage}
[ \textbullet\ ][]

\makeatletter
\renewcommand{\tableofcontents}{
	\chapter*{%
	  \vspace*{-20\p@}%
	  \begin{tikzpicture}[remember picture, overlay]%
		  \pgftext[right,x=15cm,y=0.2cm]{\color{doc!60}\Huge\sc\bfseries \contentsname};%
		  \draw[fill=doc!60,draw=doc!60] (13,-.75) rectangle (20,1);%
		  \clip (13,-.75) rectangle (20,1);
		  \pgftext[right,x=15cm,y=0.2cm]{\color{white}\Huge\sc\bfseries \contentsname};%
	  \end{tikzpicture}}%
	\@starttoc{toc}}
\makeatother

\newcommand{\liff}{\llap{$\iff$}}
\newcommand{\rap}[1]{\rrap{\text{ (#1)}}}
\newcommand{\red}[1]{\textcolor{red}{#1}}
\newcommand{\blue}[1]{\textcolor{blue}{#1}}
\newcommand{\vi}[1]{\textcolor{violet}{#1}}
\newcommand{\teal}[1]{\textcolor{teal}{#1}}
\newcommand{\tCaC}{\text{ \CaC }}
\newcommand{\CaC}{\red{CaC} }
\newcommand{\As}[1]{Assume \red{#1}}
\newcommand{\vdone}{\vi{\text{ (done) }}}
\newcommand{\bdone}{\blue{\text{ (done) }}}
\newcommand{\tdone}{\teal{\text{ (done) }}}
\newcommand{\set}[1]{\{ #1 \}}
\newcommand{\inS}{\in S}
\newcommand{\inF}{\in\F}
\newcommand{\inE}{\in E}
\newcommand{\inA}{\in A}
\newcommand{\inB}{\in B}
\newcommand{\inC}{\in C}
\newcommand{\inU}{\in U}

\newcommand{\C}{\mathbb{C}}	
\renewcommand{\H}{\mathbb{H}}
\newcommand{\F}{\mathbb{F}}
\newcommand{\N}{\mathbb{N}}
\newcommand{\Q}{\mathbb{Q}}
\newcommand{\R}{\mathbb{R}}
\newcommand{\Z}{\mathbb{Z}}
\renewcommand{\P}{\mathbb{P}}
\renewcommand{\S}{\mathbb{S}}
\newcommand{\A}{\mathbb{A}}
\newcommand{\RP}{\R P}


\title{\Huge{NCKU 112.1}\\Discrete Math}
\author{\huge{Eric Liu}}
\date{}
\begin{document}
\maketitle
\newpage% or \cleardoublepage
% \pdfbookmark[<level>]{<title>}{<dest>}


\setcounter{chapter}{-1}
\pdfbookmark[section]{\contentsname}{toc}
\tableofcontents
\pagebreak
\chapter{General Counting Methods for Arrangements and Selection}
\section{Practical Identity}
\begin{theorem}
\label{0.1.1}
\textbf{(Fundamental Identity)} We have
\begin{equation}
\binom{n}{k}=\binom{n-1}{k}+\binom{n-1}{k-1}
\end{equation}
\end{theorem}
\begin{proof}
Notice that  $\binom{n}{k}$ represent amount of ways to pick $k$ numbers from the set  $\set{1,\dots ,n}$. There are two possibilities:
\begin{equation}
\begin{cases}
  1\text{ is picked }\\
  1\text{ is not picked }
\end{cases}
\end{equation}
The amount of ways to pick $k$ numbers from the set  $\set{1,\dots ,n}$ when $1$ is mandatory to pick is $\binom{n-1}{k-1}$, and the amount of ways when $1$ is mandatory not to pick is  $\binom{n-1}{k}$
\end{proof}
\fbox{\begin{minipage}{39em}
The above identity is the most important in the sense that it allow us to deduce other identities with induction.
\end{minipage}}
\begin{theorem}
\label{0.1.2}
\textbf{(First Identity)} We have
\begin{equation*}
\binom{n}{k}\binom{k}{m}=\binom{n}{m}\binom{n-m}{k-m}
\end{equation*}
\end{theorem}
\begin{proof}
Consider possibility of picking subsets $I\subseteq \set{1,\dots ,n}$ of cardinality $k$  and subsets $I_1\subseteq I$ of cardinality $m$. The amount of possibilities equals to $\binom{n}{k}\binom{k}{m}$.\\

We can first pick the subset $I_1$ which has  $\binom{n}{m}$ possibilities. We can then pick the subset $I$ by picking  $k-m$ amount of numbers in  $\set{1,\dots ,n}\setminus I_1$ and add $I_1$ to have  $I$. There are  $\binom{n-m}{k-m}$ ways to do such, as $\abso{\set{1,\dots ,n}\setminus I_1}=n-m$. 
\end{proof}
\begin{theorem}
\label{0.1.3}
\textbf{(Second Identity)} We have
\begin{equation*}
\sum_{k=0}^n \binom{n}{k}=2^n
\end{equation*}
\end{theorem}
\begin{proof}
We have 
\begin{align*}
2^n&=(1+1)^n\\
&=\sum_{k=0}^n \binom{n}{k}1^k 1^n\\
&=\sum_{k=0}^n \binom{n}{k}
\end{align*}
\end{proof}
\fbox{\begin{minipage}{39em}
The first identity rely on the usage of intuition of picking $r$ objects from $n$ distinct objects.\\

The second identity use binomial theorem.  
\end{minipage}}
\begin{theorem}
\textbf{(Identity When Both Arguments are Growing)} We have 
\begin{equation*}
\sum_{k=0}^r \binom{n+k}{k}=\binom{n+r+1}{r}
\end{equation*}
\end{theorem}
\begin{proof}
Base case: $r=0$
 \begin{align*}
\sum_{k=0}^r \binom{n+k}{k}&=\sum_{k=0}^0 \binom{n}{0}\\
&=1\\
&=\binom{n+1}{0}\\
&=\binom{n+r+1}{r}
\end{align*}
Induction case: suppose 
\begin{equation*}
\sum_{k=0}^s \binom{n+k}{k}=\binom{n+s+1}{s}
\end{equation*}
Observe
\begin{align*}
\binom{n+s+2}{s+1}&=\binom{n+s+1}{s+1}+\binom{n+s+1}{s}\\
&=\sum_{k=0}^s \binom{n+k}{k}+\binom{n+s+1}{s+1}\\
&=\sum_{k=0}^{s+1} \binom{n+k}{k}
\end{align*}
\end{proof}
\begin{corollary}
\textbf{(Putting at most $r$ different things in  $n$ barrels)} We have
\begin{equation*}
\sum_{k=0}^r H^{n}_k=\sum_{k=0}^r \binom{\left(n-1 \right)+ k}{k}=\binom{n+r}{r}=H_r^{n+1}
\end{equation*}
\end{corollary}
\begin{theorem}
\textbf{(Identity When Only The Larger Argument is Growing)} We have
\begin{equation*}
\sum_{k=0}^{n-r} \binom{r+k}{r}=\binom{n+1}{r+1}
\end{equation*}
\end{theorem}
\begin{proof}
Base case: $n=r$ 
\begin{equation*}
\sum_{k=0}^{n-r}\binom{r+k}{r}=\binom{r}{r}=1=\binom{r+1}{r+1}=\binom{n+1}{r+1}
\end{equation*}
Induction case: Suppose
 \begin{equation*}
\sum_{k=0}^{n-s} \binom{s+k}{s}=\binom{n+1}{s+1}
\end{equation*}
Observe 
\begin{align*}
\binom{n+2}{s+1}&=\binom{n+1}{s+1}+\binom{n+1}{s}\\
&=\sum_{k=0}^{n-s} \binom{s+k}{s}+\binom{n+1}{s}\\
&=\sum_{k=0}^{n+1-s}\binom{s+k}{s}
\end{align*}
\end{proof}
\fbox{\begin{minipage}{39em}
The above two identities can be applied when the arguments are growing, notice that the second identity is an identity when the smaller argument is growing.
\end{minipage}}
\begin{theorem}
\textbf{(Fifth Identity)} We have
\begin{equation*}
\sum_{k=0}^n \binom{n}{k}^2 =\binom{2n}{n}
\end{equation*}
\end{theorem}
\begin{proof}
Imagine an $n\times n$ grid. Imagine we wish to go from the lower left corner to top right corner without going left or low.\\

The amount of path that cross $(0,n)$ is $\binom{n}{0}^2$ and the amount of path that cross $(1,n-1)$ is $\binom{n}{1}^2$,\dots .
\end{proof}
\begin{theorem}
\textbf{(Identity When Multiplying Two Binomial where the Sum of the Smaller Arguments is Fixed)} 
\begin{equation*}
\sum_{k=0}^r \binom{m}{k}\binom{n}{r-k}=\binom{m+n}{r}
\end{equation*}
\end{theorem}
\begin{proof}
The right hand side is to pick $r$ objects from  $m+n$ distinct object. The left hand side is also the same, but done so by first picking from  $m$ then from  $n$
\end{proof}
\begin{corollary}
\textbf{(identity When Multiplying Two Binomials where the Gap of the Smaller Arguments is Fixed)}
\begin{equation*}
\sum_{k=0}^m \binom{m}{k}\binom{n}{r+k}=\binom{m+n}{m+r}
\end{equation*}
\end{corollary}
\begin{proof}
\begin{align*}
\sum_{k=0}^m \binom{m}{k}\binom{n}{r+k}&=\sum_{k=0}^m \binom{m}{m-k}\binom{n}{r+k}\\
&=\sum_{u=0}^m \binom{m}{u}\binom{n}{m+r-u}\text{ where $u=m-k$}\\
&=\binom{m+n}{m+r}
\end{align*}
\end{proof}

\chapter{Generating Function} 
\section{Modeling of Generating Function}
\fbox{\begin{minipage}{39em}
Chapter 6 has 3 question, 6.4 has 1 question.
\end{minipage}}
\begin{theorem}
\textbf{(Putting Same Object into Distinct Barrels)} Given 
\begin{equation*}
\sum_{i=1}^n e_i=r
\end{equation*}
There are 
\begin{equation*}
H^n_r:=\binom{n+r-1}{n-1}=\binom{n+r-1}{r}\text{ amount of solutions }
\end{equation*}

\end{theorem}
\section{Calculation of Generating Function}
\begin{theorem}
\textbf{(Important Identity)} We have
\begin{equation*}
\frac{1}{1-x}=\sum_{k=0}^\infty x^k
\end{equation*}
\begin{equation*}
\frac{1}{(1-x)^n}=(\sum_{k=0}^\infty x^k)^n=\sum_{k=0}^\infty H^n_k x^k=\sum_{k=0}^\infty \binom{n+k-1}{k}x^k
\end{equation*}
\begin{equation*}
  \sum_{k=0}^n x^k=\frac{1-x^{n+1}}{1-x}
\end{equation*}
\end{theorem}
\section{Exponential Generating Function}
\begin{theorem}
\textbf{($r$ Arrangement with and without Repetition of $n$ objects)}
\begin{equation*}
\text{ This means the numbers of ways of selecting $r$ object from distinct $n$ object and rearrange them  }
\end{equation*}
\end{theorem}
\begin{theorem}
\textbf{(Taylor Expansion)}
\begin{equation*}
e^nx=
\end{equation*}
\end{theorem}
\chapter{HW}
\section{HW3}
\begin{question}{}{}
How many numbers between 0 and 10000 have a sum of digit
\begin{enumerate}[label=(\alph*)]
  \item Equal to 7?\\
  \item Less than or equal to 7?\\
  \item Equal to 13
\end{enumerate}
\end{question}
\begin{proof}
Represent the number by 
\begin{equation*}
e_0e_1e_2e_3e_4
\end{equation*}
where if the number is $237$, we have
 \begin{equation*}
e_0=0\text{ and }e_1=0\text{ and }e_2=2\text{ and }e_3=3\text{ and }e_4=7
\end{equation*}
And of course we have the constraint
\begin{equation*}
0\leq e_1,e_2,e_3,e_4\leq 9\text{ and }e_0\in \set{0,1}\text{ and }e_0=1\implies e_1=e_2=e_3=e_4=0
\end{equation*}
Then the first question is equivalent to asking the amount of solution of 
\begin{equation*}
e_0+e_1+e_2+e_3+e_4=7
\end{equation*}
If $e_1=1$, clearly there is no solution. If  $e_0=0$, there exists $H^4_7=\binom{10}{7}=120$ amount of solutions.\\

The second question is equivalent to asking the amount of solution of 
\begin{equation*}
e_0+e_1+e_2+e_3+e_4\leq 7
\end{equation*}
If $e_0=1$, clearly there exists only one solution. If  $e_0=0$, there exists
\begin{equation*}
\sum_{k=0}^7 H^4_k=\sum_{k=0}^7 \binom{k+3}{k}=\binom{11}{7}=330
\end{equation*}
So 
\begin{equation*}
\text{ The amount of solutions are }331
\end{equation*}
The third question is equivalent to asking the amount of solution of 
\begin{equation*}
e_0+e_1+e_2+e_3+e_4=13
\end{equation*}
Clearly, we can not have $e_0=1$, so our question has become 
\begin{equation*}
e_1+e_2+e_3+e_4=13
\end{equation*}
where the constrain is 
\begin{equation*}
0\leq e_1,e_2,e_3,e_4\leq 9
\end{equation*}
If we remove the 9 upper bound constrain, the amount of solutions is then 
\begin{equation*}
H^4_{13}=560
\end{equation*}
Adding the constrain back, we need to remove those solutions that doesn't satisfy the constrain, i.e. $e_j>9$ for some  $j\in \set{1,2,3,4}$.\\

Clearly if $e_j>9$, then no other digit would be greater than $9$. The amount of solutions that should be removed are
\begin{equation*}
4(H^3_{13-10}+H^3_{13-11}+H^3_{13-12}+H^3_{13-13})=4(\sum_{k=0}^3 H^3_k)=4(\sum_{k=0}^3 \binom{k+2}{k})=4\binom{6}{3}=80
\end{equation*}
Then the amount of solutions are 
\begin{equation*}
560-80=480\text{ ways }
\end{equation*}
\end{proof}
\begin{question}{}{}
Evaluate
\begin{equation*}
\sum_{k=1}^n \binom{n}{k}\binom{n}{k-1}
\end{equation*}
\end{question}
\begin{proof}
\begin{align*}
\sum_{k=1}^n \binom{n}{k}\binom{n}{k-1}&=\sum_{k=1}^n \binom{n+1}{k}\\
                                                    &=\sum_{k=0}^{n+1} \binom{n+1}{k}-\binom{n+1}{0}-\binom{n+1}{n+1}\\
                                                    &=\sum_{k=0}^{n+1} \binom{n+1}{k}-2\\
                                                    &=2^{n+1}-2
\end{align*}
\end{proof}
\begin{question}{}{}
Show that the generating function for the number of integer solutions to 
\begin{equation*}
e_1+e_2+e_3+e_4=r,0\leq e_1\leq e_2\leq e_3\leq e_4
\end{equation*}
is
\begin{equation*}
  \left(1+x+x^2+ \cdots \right)\left(1+x^2+x^4+\cdots \right)\left(1+x^3+x^6+\cdots \right)\left(1+x^4+x^8+\cdots \right)
\end{equation*}
\end{question}
\begin{proof}
Define
\begin{equation*}
d_1:=e_2-e_1\text{ and }d_2:=e_3-e_2\text{ and }d_3:=e_4-e_3
\end{equation*}
Then we have
\begin{equation*}
e_2=e_1+d_1\text{ and }e_3=e_1+d_1+d_2\text{ and }e_4=e_1+d_1+d_2+d_3
\end{equation*}
The question is thus reduced to finding the generating function for 
\begin{equation*}
4e_1+3d_1+2d_2+d_3=r, \set{e_1,d_1,d_2,d_3}\in \N \cup \set{0}
\end{equation*}
Which is 
\begin{equation*}
\left(1+x^4+x^8+\cdots \right)\left(1+x^3+x^6+\cdots \right)\left(1+x^2+x^4+\cdots \right)\left(1+x+x^2+ \cdots \right)
\end{equation*}
\end{proof}
\begin{question}{}{}
Use the equation
\begin{equation*}
\frac{\left(1-x^2 \right)^n}{\left(1-x \right)^n}=\left(1+x \right)^n
\end{equation*}
to show that
\begin{equation*}
\sum_{k=0}^{\frac{m}{2}}\left(-1 \right)^k \binom{n}{k} \binom{n+m-2k-1}{n-1}=\binom{n}{m}\text{ $m\leq n\text{ and }m$ even }
\end{equation*}
\end{question}
\begin{proof}
Observe that 
\begin{equation*}
\binom{n}{m}\text{ is the coefficient of $x^m$ in } \left(1+x \right)^n
\end{equation*}
So we only have to prove that 
\begin{equation*}
\sum_{k=0}^{\frac{m}{2}}\left(-1 \right)^k \binom{n}{k} \binom{n+m-2k-1}{n-1}\text{ is the coefficient of $x^m$ in }\frac{\left(1-x^2 \right)^n}{\left(1-x \right)^n}
\end{equation*}
Observe that 
\begin{equation*}
  \left(1-x^2 \right)^n=\sum_{k=0}^n \left(-1 \right)^k \binom{n}{k} x^{2k}
\end{equation*}
and that 
\begin{equation*}
  \frac{1}{\left(1-x \right)^n}=\sum_{u=0}^\infty \binom{u+n-1}{u} x^u
\end{equation*}
Then because $m$ is even, we can compute the coefficient by summing $k\in \left[0,\frac{m}{2} \right]$ and $u=m-2k\text{ from $\left(2k+u=m \right)$ }$, which tell us that the coefficient is 
 \begin{equation*}
 \sum_{k=0}^{\frac{m}{2}}\left(-1 \right)^k \binom{n}{k}\binom{m-2k+n-1}{m-2k}
\end{equation*}
which equals to 
\begin{equation*}
\sum_{k=0}^{\frac{m}{2}} \left(-1 \right)^k \binom{n}{k} \binom{n+m-2k-1}{n-1}
\end{equation*}



\end{proof}
\begin{question}{}{}
Show that 
\begin{equation*}
2\left(1-x \right)^{-3}\left[\left(1-x \right)^{-3}+\left(1+x \right)^{-3} \right]
\end{equation*}
is the generating function for the number of ways to toss $r$ identical dice and obtain even sum.
\end{question}
\begin{proof}
The generating function of ways to toss $r$ identical dice is 
\begin{equation*}
  \left( \sum_{k=1}^6 x^k\right)^r
\end{equation*}
which equals to
\begin{equation*}
  \left(\frac{1-x^7}{1-x} \right)^r
\end{equation*}
Then the generating function for the number of ways to toss $r$ identical dice and obtain even sum is
\begin{equation*}
  \frac{1}{2}\left[ \left(\frac{1-x^7}{1-x} \right)^r - \left(\frac{1+x^7}{1+x}\right)^r\right]
\end{equation*}
\end{proof}
\end{document}
