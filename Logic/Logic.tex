\documentclass{report}

%%%%%%%%%%%%%% preamble.tex %%%%%%%%%%%%%%
\usepackage[T1]{fontenc}
\usepackage{etoolbox}
% Page Setup
\usepackage[letterpaper, tmargin=2cm, rmargin=0.5in, lmargin=0.5in, bmargin=80pt, footskip=.2in]{geometry}
\usepackage{adjustbox}
\usepackage{graphicx}
\usepackage{tikz}
\usepackage{mathrsfs}
\usepackage{mdframed}

% Create a new toggle
\newtoggle{firstsection}

% Redefine the \chapter command to reset the toggle for each new chapter
\let\oldchapter\chapter
\renewcommand{\chapter}{\toggletrue{firstsection}\oldchapter}

% Redefine the \section command to check the toggle
\let\oldsection\section
\renewcommand{\section}{
    \iftoggle{firstsection}
    {\togglefalse{firstsection}} % If it's the first section, just switch off the toggle for next sections
    {\clearpage} % If it's not the first section, start a new page
    \oldsection
}

% Abstract Design

\usepackage{lipsum}

\renewenvironment{abstract}
 {% Start of environment
  \quotation
  \small
  \noindent
  \rule{\linewidth}{.5pt} % Draw the rule to match the linewidth
  \par\smallskip
  {\centering\bfseries\abstractname\par}\medskip
 }
 {% End of environment
  \par\noindent
  \rule{\linewidth}{.5pt} % Ensure the closing rule also matches
  \endquotation
 }

% Mathematics
\usepackage{amsmath,amsfonts,amsthm,amssymb,mathtools}
\usepackage{xfrac}
\usepackage[makeroom]{cancel}
\usepackage{enumitem}
\usepackage{nameref}
\usepackage{multicol,array}
\usepackage{tikz-cd}
\usepackage{array}
\usepackage{multirow}% http://ctan.org/pkg/multirow
\usepackage{graphicx}

% Colors
\usepackage[dvipsnames]{xcolor}
\definecolor{myg}{RGB}{56, 140, 70}
\definecolor{myb}{RGB}{45, 111, 177}
\definecolor{myr}{RGB}{199, 68, 64}
% Define more colors here...
\definecolor{olive}{HTML}{6B8E23}
\definecolor{orange}{HTML}{CC5500}
\definecolor{brown}{HTML}{8B4513}
% Hyperlinks
\usepackage{bookmark}
\usepackage[colorlinks=true,linkcolor=blue,urlcolor=blue,citecolor=blue,anchorcolor=blue]{hyperref}
\usepackage{xcolor}
\hypersetup{
    colorlinks,
    linkcolor={red!50!black},
    citecolor={blue!50!black},
    urlcolor={blue!80!black}
}

% Text-related
\usepackage{blindtext}
\usepackage{fontsize}
\changefontsize[14]{14}
\setlength{\parindent}{0pt}
\linespread{1.2}

% Theorems and Definitions
\usepackage{amsthm}
\renewcommand\qedsymbol{$\blacksquare$}

% Define a new theorem style
\newtheoremstyle{mytheoremstyle}% name
  {}% Space above
  {}% Space below
  {}% Body font
  {}% Indent amount
  {\bfseries}% Theorem head font
  {.}% Punctuation after theorem head
  {.5em}% Space after theorem head
  {}% Theorem head spec (can be left empty, meaning ‘normal’)

% Apply the new theorem style to theorem-like environments
\theoremstyle{mytheoremstyle}

\newtheorem{theorem}{Theorem}[section]  
\newtheorem{definition}[theorem]{Definition} 
\newtheorem{lemma}[theorem]{Lemma}  
\newtheorem{corollary}[theorem]{Corollary}
\newtheorem{axiom}[theorem]{Axiom}
\newtheorem{example}[theorem]{Example}
\newtheorem{equiv_def}[theorem]{Equivalent Definition}

% tcolorbox Setup
\usepackage[most,many,breakable]{tcolorbox}
\tcbuselibrary{theorems}

% Define custom tcolorbox environments here...

%================================
% EXAMPLE BOX
%================================
% After you have defined the style and other theorem environments
\definecolor{myexamplebg}{RGB}{245, 245, 245} % Very light grey for background
\definecolor{myexamplefr}{RGB}{120, 120, 120} % Medium grey for frame
\definecolor{myexampleti}{RGB}{60, 60, 60}    % Darker grey for title

\newtcbtheorem[]{Example}{Example}{
    colback=myexamplebg,
    breakable,
    colframe=myexamplefr,
    coltitle=myexampleti,
    boxrule=1pt,
    sharp corners,
    detach title,
    before upper=\tcbtitle\par\vspace{-20pt}, % Reduced the space after the title
    fonttitle=\bfseries,
    description font=\mdseries,
    separator sign none,
    description delimiters={}{}, % No delimiters around the title
}{ex}
%================================
% Solution BOX
%================================
\makeatletter
\newtcolorbox{solution}{enhanced,
	breakable,
	colback=white,
	colframe=myg!80!black,
	attach boxed title to top left={yshift*=-\tcboxedtitleheight},
	title=Solution,
	boxed title size=title,
	boxed title style={%
			sharp corners,
			rounded corners=northwest,
			colback=tcbcolframe,
			boxrule=0pt,
		},
	underlay boxed title={%
			\path[fill=tcbcolframe] (title.south west)--(title.south east)
			to[out=0, in=180] ([xshift=5mm]title.east)--
			(title.center-|frame.east)
			[rounded corners=\kvtcb@arc] |-
			(frame.north) -| cycle;
		},
}
\makeatother

% %================================
% % Question BOX
% %================================
\makeatletter
\newtcbtheorem{question}{Question}{enhanced,
	breakable,
	colback=white,
	colframe=myb!80!black,
	attach boxed title to top left={yshift*=-\tcboxedtitleheight},
	fonttitle=\bfseries,
	title={#2},
	boxed title size=title,
	boxed title style={%
			sharp corners,
			rounded corners=northwest,
			colback=tcbcolframe,
			boxrule=0pt,
		},
	underlay boxed title={%
			\path[fill=tcbcolframe] (title.south west)--(title.south east)
			to[out=0, in=180] ([xshift=5mm]title.east)--
			(title.center-|frame.east)
			[rounded corners=\kvtcb@arc] |-
			(frame.north) -| cycle;
		},
	#1
}{question}
\makeatother

%%%%%%%%%%%%%%%%%%%%%%%%%%%%%%%%%%%%%%%%%%%
% TABLE OF CONTENTS
%%%%%%%%%%%%%%%%%%%%%%%%%%%%%%%%%%%%%%%%%%%


\usepackage{tikz}
\definecolor{doc}{RGB}{0,60,110}
\usepackage{titletoc}
\contentsmargin{0cm}
\titlecontents{chapter}[14pc]
{\addvspace{30pt}%
	\begin{tikzpicture}[remember picture, overlay]%
		\draw[fill=doc!60,draw=doc!60] (-7,-.1) rectangle (-0.9,.5);%
		\pgftext[left,x=-5.5cm,y=0.2cm]{\color{white}\Large\sc\bfseries Chapter\ \thecontentslabel};%
	\end{tikzpicture}\color{doc!60}\large\sc\bfseries}%
{}
{}
{\;\titlerule\;\large\sc\bfseries Page \thecontentspage
	\begin{tikzpicture}[remember picture, overlay]
		\draw[fill=doc!60,draw=doc!60] (2pt,0) rectangle (4,0.1pt);
	\end{tikzpicture}}%
\titlecontents{section}[3.7pc]
{\addvspace{2pt}}
{\contentslabel[\thecontentslabel]{3pc}}
{}
{\hfill\small \thecontentspage}
[]
\titlecontents*{subsection}[3.7pc]
{\addvspace{-1pt}\small}
{}
{}
{\ --- \small\thecontentspage}
[ \textbullet\ ][]

\makeatletter
\renewcommand{\tableofcontents}{
	\chapter*{%
	  \vspace*{-20\p@}%
	  \begin{tikzpicture}[remember picture, overlay]%
		  \pgftext[right,x=15cm,y=0.2cm]{\color{doc!60}\Huge\sc\bfseries \contentsname};%
		  \draw[fill=doc!60,draw=doc!60] (13,-.75) rectangle (20,1);%
		  \clip (13,-.75) rectangle (20,1);
		  \pgftext[right,x=15cm,y=0.2cm]{\color{white}\Huge\sc\bfseries \contentsname};%
	  \end{tikzpicture}}%
	\@starttoc{toc}}
\makeatother

\newcommand{\liff}{\llap{$\iff$}}
\newcommand{\rap}[1]{\rrap{\text{ (#1)}}}
\newcommand{\red}[1]{\textcolor{red}{#1}}
\newcommand{\blue}[1]{\textcolor{blue}{#1}}
\newcommand{\vi}[1]{\textcolor{violet}{#1}}
\newcommand{\olive}[1]{\textcolor{olive}{#1}}
\newcommand{\teal}[1]{\textcolor{teal}{#1}}
\newcommand{\brown}[1]{\textcolor{brown}{#1}}
\newcommand{\orange}[1]{\textcolor{orange}{#1}}
\newcommand{\tCaC}{\text{ \CaC }}
\newcommand{\CaC}{\red{CaC} }
\newcommand{\As}[1]{Assume \red{#1}}
\newcommand{\vdone}{\vi{\text{ (done) }}}
\newcommand{\bdone}{\blue{\text{ (done) }}}
\newcommand{\tdone}{\teal{\text{ (done) }}}
\newcommand{\odone}{\olive{\text{ (done) }}}
\newcommand{\bodone}{\brown{\text{ (done) }}}
\newcommand{\ordone}{\orange{\text{ (done) }}}
\newcommand{\ld}{\lambda}
\newcommand{\vecta}[1]{\textbf{#1}}
\newcommand{\set}[1]{\left\{ #1 \right\}}
\newcommand{\bset}[1]{\Big\{ #1 \Big\}}
\newcommand{\inR}{\in\R}
\newcommand{\inn}{\in\N}
\newcommand{\inz}{\in\Z}
\newcommand{\inr}{\in\R}
\newcommand{\inc}{\in\C}
\newcommand{\inq}{\in\Q}
\newcommand{\norm}[1]{\| #1 \|}
\newcommand{\bnorm}[1]{\Big\| #1 \Big\|}
\newcommand{\gen}[1]{\langle #1 \rangle}
\newcommand{\abso}[1]{\left|#1\right|}
\newcommand{\myref}[2]{\hyperref[#2]{#1\ \ref*{#2}}}
\newcommand{\customref}[2]{\hyperref[#1]{#2}}
\newcommand{\power}[1]{\mathcal{P}(#1)}
\newcommand{\dcup}{\mathbin{\dot{\cup}}}
\newcommand{\diam}[1]{\text{diam}\, #1}
\newcommand{\at}{\Big|}
\newcommand{\quotient}{\diagup}
\let\originalphi\phi % Store the original \phi in \originalphi
\renewcommand{\phi}{\varphi} % Redefine \phi to \varphi
\newcommand{\pfi}{\originalphi} % Define \pfi to display the original \phi
\newcommand{\diota}{\dot{\iota}}
\newcommand{\Log}{\operatorname{Log}}
\newcommand{\id}{\text{\textbf{id}}}
\usepackage{amsmath}

\makeatletter
\NewDocumentCommand{\extp}{e{^}}{%
  \mathop{\mathpalette\extp@{#1}}\nolimits
}
\NewDocumentCommand{\extp@}{mm}{%
  \bigwedge\nolimits\IfValueT{#2}{^{\extp@@{#1}#2}}%
  \IfValueT{#1}{\kern-2\scriptspace\nonscript\kern2\scriptspace}%
}
\newcommand{\extp@@}[1]{%
  \mkern
    \ifx#1\displaystyle-1.8\else
    \ifx#1\textstyle-1\else
    \ifx#1\scriptstyle-1\else
    -0.5\fi\fi\fi
  \thinmuskip
}
\makeatletter
\usepackage{pifont}
\makeatletter
\newcommand\Pimathsymbol[3][\mathord]{%
  #1{\@Pimathsymbol{#2}{#3}}}
\def\@Pimathsymbol#1#2{\mathchoice
  {\@Pim@thsymbol{#1}{#2}\tf@size}
  {\@Pim@thsymbol{#1}{#2}\tf@size}
  {\@Pim@thsymbol{#1}{#2}\sf@size}
  {\@Pim@thsymbol{#1}{#2}\ssf@size}}
\def\@Pim@thsymbol#1#2#3{%
  \mbox{\fontsize{#3}{#3}\Pisymbol{#1}{#2}}}
\makeatother
% the next two lines are needed to avoid LaTeX substituting upright from another font
\input{utxmia.fd}
\DeclareFontShape{U}{txmia}{m}{n}{<->ssub * txmia/m/it}{}
% you may also want
\DeclareFontShape{U}{txmia}{bx}{n}{<->ssub * txmia/bx/it}{}
% just in case
%\DeclareFontShape{U}{txmia}{l}{n}{<->ssub * txmia/l/it}{}
%\DeclareFontShape{U}{txmia}{b}{n}{<->ssub * txmia/b/it}{}
% plus info from Alan Munn at https://tex.stackexchange.com/questions/290165/how-do-i-get-a-nicer-lambda?noredirect=1#comment702120_290165
\newcommand{\pilambdaup}{\Pimathsymbol[\mathord]{txmia}{21}}
\renewcommand{\lambda}{\pilambdaup}
\renewcommand{\tilde}{\widetilde}
\DeclareMathOperator*{\esssup}{ess\,sup}
\newcommand{\bluecheck}{}%
\DeclareRobustCommand{\bluecheck}{%
  \tikz\fill[scale=0.4, color=blue]
  (0,.35) -- (.25,0) -- (1,.7) -- (.25,.15) -- cycle;%
}


\usepackage{tikz}
\newcommand*{\DashedArrow}[1][]{\mathbin{\tikz [baseline=-0.25ex,-latex, dashed,#1] \draw [#1] (0pt,0.5ex) -- (1.3em,0.5ex);}}

\newcommand{\C}{\mathbb{C}}	
\newcommand{\F}{\mathbb{F}}
\newcommand{\N}{\mathbb{N}}
\newcommand{\Q}{\mathbb{Q}}
\newcommand{\R}{\mathbb{R}}
\newcommand{\Z}{\mathbb{Z}}



\title{\Huge{NCKU 112.1}\\Discrete Math}
\author{\huge{Eric Liu}}
\date{}
\begin{document}
\maketitle
\newpage% or \cleardoublepage
% \pdfbookmark[<level>]{<title>}{<dest>}


\setcounter{chapter}{-1}
\pdfbookmark[section]{\contentsname}{toc}
\tableofcontents
\pagebreak
\chapter{General Counting Methods for Arrangements and Selection}
\section{Practical Identity}
\begin{theorem}
\label{0.1.1}
\textbf{(Fundamental Identity)} We have
\begin{equation}
\binom{n}{k}=\binom{n-1}{k}+\binom{n-1}{k-1}
\end{equation}
\end{theorem}
\begin{proof}
Notice that  $\binom{n}{k}$ represent amount of ways to pick $k$ numbers from the set  $\set{1,\dots ,n}$. There are two possibilities:
\begin{equation}
\begin{cases}
  1\text{ is picked }\\
  1\text{ is not picked }
\end{cases}
\end{equation}
The amount of ways to pick $k$ numbers from the set  $\set{1,\dots ,n}$ when $1$ is mandatory to pick is $\binom{n-1}{k-1}$, and the amount of ways when $1$ is mandatory not to pick is  $\binom{n-1}{k}$
\end{proof}
\fbox{\begin{minipage}{39em}
The above identity is the most important in the sense that it allow us to deduce other identities with induction.
\end{minipage}}
\begin{theorem}
\label{0.1.2}
\textbf{(First Identity)} We have
\begin{equation*}
\binom{n}{k}\binom{k}{m}=\binom{n}{m}\binom{n-m}{k-m}
\end{equation*}
\end{theorem}
\begin{proof}
Consider possibility of picking subsets $I\subseteq \set{1,\dots ,n}$ of cardinality $k$  and subsets $I_1\subseteq I$ of cardinality $m$. The amount of possibilities equals to $\binom{n}{k}\binom{k}{m}$.\\

We can first pick the subset $I_1$ which has  $\binom{n}{m}$ possibilities. We can then pick the subset $I$ by picking  $k-m$ amount of numbers in  $\set{1,\dots ,n}\setminus I_1$ and add $I_1$ to have  $I$. There are  $\binom{n-m}{k-m}$ ways to do such, as $\abso{\set{1,\dots ,n}\setminus I_1}=n-m$. 
\end{proof}
\begin{theorem}
\label{0.1.3}
\textbf{(Second Identity)} We have
\begin{equation*}
\sum_{k=0}^n \binom{n}{k}=2^n
\end{equation*}
\end{theorem}
\begin{proof}
We have 
\begin{align*}
2^n&=(1+1)^n\\
&=\sum_{k=0}^n \binom{n}{k}1^k 1^n\\
&=\sum_{k=0}^n \binom{n}{k}
\end{align*}
\end{proof}
\fbox{\begin{minipage}{39em}
The first identity rely on the usage of intuition of picking $r$ objects from $n$ distinct objects.\\

The second identity use binomial theorem.  
\end{minipage}}
\begin{theorem}
\textbf{(Identity When Both Arguments are Growing)} We have 
\begin{equation*}
\sum_{k=0}^r \binom{n+k}{k}=\binom{n+r+1}{r}
\end{equation*}
\end{theorem}
\begin{proof}
Base case: $r=0$
 \begin{align*}
\sum_{k=0}^r \binom{n+k}{k}&=\sum_{k=0}^0 \binom{n}{0}\\
&=1\\
&=\binom{n+1}{0}\\
&=\binom{n+r+1}{r}
\end{align*}
Induction case: suppose 
\begin{equation*}
\sum_{k=0}^s \binom{n+k}{k}=\binom{n+s+1}{s}
\end{equation*}
Observe
\begin{align*}
\binom{n+s+2}{s+1}&=\binom{n+s+1}{s+1}+\binom{n+s+1}{s}\\
&=\sum_{k=0}^s \binom{n+k}{k}+\binom{n+s+1}{s+1}\\
&=\sum_{k=0}^{s+1} \binom{n+k}{k}
\end{align*}
\end{proof}
\begin{corollary}
\textbf{(Putting at most $r$ different things in  $n$ barrels)} We have
\begin{equation*}
\sum_{k=0}^r H^{n}_k=\sum_{k=0}^r \binom{\left(n-1 \right)+ k}{k}=\binom{n+r}{r}=H_r^{n+1}
\end{equation*}
\end{corollary}
\begin{theorem}
\textbf{(Identity When Only The Larger Argument is Growing)} We have
\begin{equation*}
\sum_{k=0}^{n-r} \binom{r+k}{r}=\binom{n+1}{r+1}
\end{equation*}
\end{theorem}
\begin{proof}
Base case: $n=r$ 
\begin{equation*}
\sum_{k=0}^{n-r}\binom{r+k}{r}=\binom{r}{r}=1=\binom{r+1}{r+1}=\binom{n+1}{r+1}
\end{equation*}
Induction case: Suppose
 \begin{equation*}
\sum_{k=0}^{n-s} \binom{s+k}{s}=\binom{n+1}{s+1}
\end{equation*}
Observe 
\begin{align*}
\binom{n+2}{s+1}&=\binom{n+1}{s+1}+\binom{n+1}{s}\\
&=\sum_{k=0}^{n-s} \binom{s+k}{s}+\binom{n+1}{s}\\
&=\sum_{k=0}^{n+1-s}\binom{s+k}{s}
\end{align*}
\end{proof}
\fbox{\begin{minipage}{39em}
The above two identities can be applied when the arguments are growing, notice that the second identity is an identity when the smaller argument is growing.
\end{minipage}}
\begin{theorem}
\textbf{(Fifth Identity)} We have
\begin{equation*}
\sum_{k=0}^n \binom{n}{k}^2 =\binom{2n}{n}
\end{equation*}
\end{theorem}
\begin{proof}
Imagine an $n\times n$ grid. Imagine we wish to go from the lower left corner to top right corner without going left or low.\\

The amount of path that cross $(0,n)$ is $\binom{n}{0}^2$ and the amount of path that cross $(1,n-1)$ is $\binom{n}{1}^2$,\dots .
\end{proof}
\begin{theorem}
\textbf{(Identity When Multiplying Two Binomial where the Sum of the Smaller Arguments is Fixed)} 
\begin{equation*}
\sum_{k=0}^r \binom{m}{k}\binom{n}{r-k}=\binom{m+n}{r}
\end{equation*}
\end{theorem}
\begin{proof}
The right hand side is to pick $r$ objects from  $m+n$ distinct object. The left hand side is also the same, but done so by first picking from  $m$ then from  $n$
\end{proof}
\begin{corollary}
\textbf{(identity When Multiplying Two Binomials where the Gap of the Smaller Arguments is Fixed)}
\begin{equation*}
\sum_{k=0}^m \binom{m}{k}\binom{n}{r+k}=\binom{m+n}{m+r}
\end{equation*}
\end{corollary}
\begin{proof}
\begin{align*}
\sum_{k=0}^m \binom{m}{k}\binom{n}{r+k}&=\sum_{k=0}^m \binom{m}{m-k}\binom{n}{r+k}\\
&=\sum_{u=0}^m \binom{m}{u}\binom{n}{m+r-u}\text{ where $u=m-k$}\\
&=\binom{m+n}{m+r}
\end{align*}
\end{proof}

\chapter{Generating Function} 
\section{Modeling of Generating Function}
\fbox{\begin{minipage}{39em}
Chapter 6 has 3 question, 6.4 has 1 question.
\end{minipage}}
\begin{theorem}
\textbf{(Putting Same Object into Distinct Barrels)} Given 
\begin{equation*}
\sum_{i=1}^n e_i=r
\end{equation*}
There are 
\begin{equation*}
H^n_r:=\binom{n+r-1}{n-1}=\binom{n+r-1}{r}\text{ amount of solutions }
\end{equation*}

\end{theorem}
\section{Calculation of Generating Function}
\begin{theorem}
\textbf{(Important Identity)} We have
\begin{equation*}
\frac{1}{1-x}=\sum_{k=0}^\infty x^k
\end{equation*}
\begin{equation*}
\frac{1}{(1-x)^n}=(\sum_{k=0}^\infty x^k)^n=\sum_{k=0}^\infty H^n_k x^k=\sum_{k=0}^\infty \binom{n+k-1}{k}x^k
\end{equation*}
\begin{equation*}
  \sum_{k=0}^n x^k=\frac{1-x^{n+1}}{1-x}
\end{equation*}
\end{theorem}
\section{Exponential Generating Function}
\begin{theorem}
\textbf{($r$ Arrangement with and without Repetition of $n$ objects)}
\begin{center}
   \begin{minipage}{0.9\linewidth}  
       \centering
       This means the numbers of ways of selecting $r$ object from distinct $n$ object and rearrange them  
   \end{minipage}
\end{center}
\end{theorem}
\begin{theorem}
\textbf{(Taylor Expansion)}
\begin{equation*}
e^x=\sum_{n=0}^\infty \frac{x^n}{n!}
\end{equation*}
\end{theorem}
\begin{theorem}
\textbf{(The form of $r$-Term of Generating Function)}
\begin{equation*}
  (\sum_{e_1+e_2+e_3=r} \frac{r!}{e_1!e_2!e_3!}) \frac{x^r}{r!}
\end{equation*}
\end{theorem}
\chapter{Recurrence Relation}
\section{Solution of Homogeneous Relation}
\begin{theorem}
\textbf{(Solution of Second order Homogeneous Recurrence Relation)} Let $\alpha ,\beta $ be two constant. Suppose 
\begin{equation*}
a_n=\alpha a_{n-1}+\beta a_{n-2}
\end{equation*}
We have
\begin{equation*}
\exists A,B, a_n= A(\frac{\alpha +\sqrt{\alpha ^2+4\beta } }{2})^n+B(\frac{\alpha -\sqrt{\alpha ^2+4\beta } }{2})^n
\end{equation*}
\end{theorem}
\begin{proof}
Let $b_n$ and $c_n$ be sequence that satisfy the recurrence relation. We see 
\begin{align}
  (b+c)_n&=b_n+c_n\\
  &=\alpha b_{n-1}+\beta b_{n-2} +\alpha  c_{n-1}+\beta c_{n-2}\\
  &=\alpha (b+c)_{n-1}+\beta  (b+c)_{n-2}
\end{align}
and 
\begin{align}
\gamma b_n&=\gamma (b_{n-1}+b_{n-2})\\
&=\gamma b_{n-1}+\gamma b_{n-2}
\end{align}
We see that the set of sequences satisfying the recurrence relation is a vector space.\\

This vector space if of dimension $2$, as the sequence is determined by the initial condition $(a_1,a_2)$.\\ 

We wish to find a basis, in which the elements are geometric series.\\

Suppose a geometric series $a_n=x^n$ satisfy the recurrence relation. Then we have 
\begin{equation*}
x^n=\alpha x^{n-1}+\beta x^{n-2}
\end{equation*}
This give us 
\begin{equation*}
x^2-\alpha x-\beta =0
\end{equation*}
Solving the equation, we have
\begin{equation*}
x=\frac{\alpha \pm \sqrt{\alpha ^2+4\beta } }{2}
\end{equation*}
Then each sequence satisfying the recurrence relation can be written in the form 
\begin{equation*}
a_n= A(\frac{\alpha +\sqrt{\alpha ^2+4\beta } }{2})^n+B(\frac{\alpha -\sqrt{\alpha ^2+4\beta } }{2})^n
\end{equation*}

\end{proof}
\begin{corollary}
\textbf{(Solution of Homogeneous Recurrence Relation)} Suppose 
\begin{align*}
a_n=\sum_{k=1}^m c_ka_{n-k}
\end{align*}
The exact solution is clearly determined by the first $a_1,\dots ,a_m$. Then we know 
\begin{align*}
\text{ Set of solutions is a $m$-dimensional vector space }
\end{align*}
If lucky, the basis can be gained by solving the equation 
\begin{align*}
\alpha^m=\sum_{k=1}^{m}c_k\alpha^{k-1}
\end{align*}
to have basis 
\begin{align*}
\set{\set{\alpha_1^n}_{n\inn},\set{\alpha _2^n}_{n\inn},\dots ,\set{\alpha _m^n}_{n\inn}}
\end{align*}
Then we can express the general solution by 
\begin{align*}
a_n=\sum_{k=1}^m d_k \alpha_k^n
\end{align*}
\end{corollary}
\section{Solution of Inhomogeneous Recurrence Relation}
\begin{theorem}
\textbf{(General Solution of Inhomogeneous Recurrence Relation)} Consider the recurrence relation $a_n$ 
 \begin{align*}
a_n=\sum_{k=1}^m c_ka_{n-k} +f(n)
\end{align*}
If we have a particular solution $a_n^*$, and a solution to  $b_n$ to the homogeneous recurrence relation 
 \begin{align*}
b_n=\sum_{k=1}^m c_kb_{n-k}
\end{align*}
Then $(a+b)_n$ is a solution. To guess a particular solution $a_n^*$, we first have to observe the form of  $f(n)$. If $f(n)$ is a polynomial of degree $p$, then we can guess 
\begin{align*}
a_n^*=A_pn^p+A_{p-1}n^{p-1}+\cdots +A_1n+A_0
\end{align*}
\end{theorem}
\section{Solution with Generating Function}
\begin{question}{}{}
Let $a_{-1}=0$. Find a generating function
\begin{align*}
g(x)=\sum_{k=0}a_k x^k
\end{align*}
that satisfy the recurrence relation 
\begin{align}
\label{eq1}\forall n\geq 1,a_n=a_{n-1}+n
\end{align}
\end{question}
\begin{proof}
Multiplying \myref{Equation}{eq1} on both side with $x^n$, we have
\begin{align*}
a_nx^n=a_{n-1}x^n +nx^n
\end{align*}
Then we can rewrite $g(x)$ into 
\begin{align*}
g(x)&=\sum_{k=0}a_kx^k\\
&=a_0 +\sum_{k=1}a_kx^k\\
&=a_0 +\sum_{k=1}\big(a_{k-1}x^k+kx^k\big)\\
&=a_0+x\sum_{k=1}a_{k-1}x^{k-1} + \sum_{k=1} kx^k\\
&=a_0+x\sum _{k=1}a_{k-1}x^{k-1} +\sum_{k=1} kx^k\\
&=a_0+x\sum_{k=0}a_kx^k+\sum_{k=1} \binom{k}{1}x^k\\
&=a_0+xg(x)+\sum_{k=0}\binom{k+1}{1}x^{k+1}\\
&=a_0+xg(x)+ x\sum_{k=0} \binom{k+1}{k}x^k\\
&=a_0+xg(x)+x \sum_{k=0} H^2_k x^k\\
&=a_0+x g(x)+\frac{x}{(1-x)^2}
\end{align*}
In summary 
\begin{align*}
g(x)=a_0+xg(x)+\frac{x}{(1-x)^2}
\end{align*}
This give us 
\begin{align*}
g(x)=\frac{a_0}{1-x}+\frac{x}{(1-x)^3}
\end{align*}
Then we see 
\begin{align*}
g(x)&=a_0 \sum_{k=0}^\infty x^k+x \sum_{k=0}^\infty H^3_k x^k\\
&=a_0\sum _{k=0}^\infty x^k + \sum_{k=0}^\infty \binom{k+2}{k}x^{k+1}\\\
&=a_0 x_0 +\big(\sum_{k=1} a_0x^k+ \binom{k+1}{k-1}x^k\big)\\
&=a_0x_0+\sum_{k=1} \big (a_0+\binom{k+1}{k-1}\big)x^k\\
&=a_0x_0+\sum_{k=1} \big(a_0+\binom{k+1}{2}\big)x^k\\
&=\sum_{k=0} \big(a_0+\binom{k+1}{2})x^k
\end{align*}
\end{proof}
\begin{question}{}{}
Use generating function to solve the recurrence relation 
 \begin{align*}
\forall n\geq 2,a_n=a_{n-1}+a_{n-2}
\end{align*}
where the initial condition is 
\begin{align*}
a_0=a_1=1
\end{align*}
\end{question}
Suppose 
\begin{align*}
g(x)=\sum_{k=0}^\infty a_kx^k
\end{align*}
satisfy the recurrence relation. Then we have 
\begin{align*}
g(x)-a_0-a_1x&=\sum_{k=2}^\infty a_kx^k\\
&=\sum_{k=2}^\infty (a_{k-1}+a_{k-2})x^k\\
&=x\sum_{k=2}^\infty a_{k-1}x^{k-1}+x^2\sum_{k=2}^\infty a_{k-2}x^{k-2}\\
&=x\sum_{k=1}^\infty a_kx^k+x^2 \sum_{k=0}^\infty a_kx^k\\
&=x(g(x)-1)+x^2g(x)
\end{align*}
Then we have 
\begin{align*}
  (1-x-x^2)g(x)=1
\end{align*}
Notice
\begin{align*}
  1-x-x^2=-(x-\frac{-1+\sqrt{5} }{2})(x-\frac{-1-\sqrt{5} }{2})
\end{align*}
Then 
\begin{align*}
g(x)&=\frac{-1}{(x-\frac{-1+\sqrt{5} }{2})(x-\frac{-1-\sqrt{5} }{2})}\\
&=\frac{\frac{-1}{\sqrt{5} }}{x-\frac{-1+\sqrt{5} }{2}}+\frac{\frac{1}{\sqrt{5} }}{x-\frac{-1-\sqrt{5} }{2}}\\
&=\frac{\frac{1}{\sqrt{5}}}{\frac{-1+\sqrt{5} }{2}-x}+\frac{\frac{-1}{\sqrt{5} }}{\frac{-1-\sqrt{5} }{2}-x}\\
&=\frac{\frac{2}{\sqrt{5}(-1+\sqrt{5} ) }}{1-\frac{2}{-1+\sqrt{5} }x}+\frac{\frac{-2}{\sqrt{5}(-1-\sqrt{5} ) }}{1-\frac{2}{-1-\sqrt{5} }x}\\
&=\frac{2}{5-\sqrt{5} } \sum_{k=0}^\infty (\frac{2}{-1+\sqrt{5} }x)^k+\frac{-2}{-5-\sqrt{5} }\sum_{k=0}^\infty (\frac{2}{-1-\sqrt{5} }x)^k
\end{align*}
\chapter{HW}
\section{HW3}
\begin{question}{}{}
How many numbers between 0 and 10000 have a sum of digit
\begin{enumerate}[label=(\alph*)]
  \item Equal to 7?\\
  \item Less than or equal to 7?\\
  \item Equal to 13
\end{enumerate}
\end{question}
\begin{proof}
Represent the number by 
\begin{equation*}
e_0e_1e_2e_3e_4
\end{equation*}
where if the number is $237$, we have
 \begin{equation*}
e_0=0\text{ and }e_1=0\text{ and }e_2=2\text{ and }e_3=3\text{ and }e_4=7
\end{equation*}
And of course we have the constraint
\begin{equation*}
0\leq e_1,e_2,e_3,e_4\leq 9\text{ and }e_0\in \set{0,1}\text{ and }e_0=1\implies e_1=e_2=e_3=e_4=0
\end{equation*}
Then the first question is equivalent to asking the amount of solution of 
\begin{equation*}
e_0+e_1+e_2+e_3+e_4=7
\end{equation*}
If $e_1=1$, clearly there is no solution. If  $e_0=0$, there exists $H^4_7=\binom{10}{7}=120$ amount of solutions.\\

The second question is equivalent to asking the amount of solution of 
\begin{equation*}
e_0+e_1+e_2+e_3+e_4\leq 7
\end{equation*}
If $e_0=1$, clearly there exists only one solution. If  $e_0=0$, there exists
\begin{equation*}
\sum_{k=0}^7 H^4_k=\sum_{k=0}^7 \binom{k+3}{k}=\binom{11}{7}=330
\end{equation*}
So 
\begin{equation*}
\text{ The amount of solutions are }331
\end{equation*}
The third question is equivalent to asking the amount of solution of 
\begin{equation*}
e_0+e_1+e_2+e_3+e_4=13
\end{equation*}
Clearly, we can not have $e_0=1$, so our question has become 
\begin{equation*}
e_1+e_2+e_3+e_4=13
\end{equation*}
where the constrain is 
\begin{equation*}
0\leq e_1,e_2,e_3,e_4\leq 9
\end{equation*}
If we remove the 9 upper bound constrain, the amount of solutions is then 
\begin{equation*}
H^4_{13}=560
\end{equation*}
Adding the constrain back, we need to remove those solutions that doesn't satisfy the constrain, i.e. $e_j>9$ for some  $j\in \set{1,2,3,4}$.\\

Clearly if $e_j>9$, then no other digit would be greater than $9$. The amount of solutions that should be removed are
\begin{equation*}
4(H^3_{13-10}+H^3_{13-11}+H^3_{13-12}+H^3_{13-13})=4(\sum_{k=0}^3 H^3_k)=4(\sum_{k=0}^3 \binom{k+2}{k})=4\binom{6}{3}=80
\end{equation*}
Then the amount of solutions are 
\begin{equation*}
560-80=480\text{ ways }
\end{equation*}
\end{proof}
\begin{question}{}{}
Evaluate
\begin{equation*}
\sum_{k=1}^n \binom{n}{k}\binom{n}{k-1}
\end{equation*}
\end{question}
\begin{proof}
\begin{align*}
\sum_{k=1}^n \binom{n}{k}\binom{n}{k-1}&=\sum_{k=1}^n \binom{n+1}{k}\\
                                                    &=\sum_{k=0}^{n+1} \binom{n+1}{k}-\binom{n+1}{0}-\binom{n+1}{n+1}\\
                                                    &=\sum_{k=0}^{n+1} \binom{n+1}{k}-2\\
                                                    &=2^{n+1}-2
\end{align*}
\end{proof}
\begin{question}{}{}
Show that the generating function for the number of integer solutions to 
\begin{equation*}
e_1+e_2+e_3+e_4=r,0\leq e_1\leq e_2\leq e_3\leq e_4
\end{equation*}
is
\begin{equation*}
  \left(1+x+x^2+ \cdots \right)\left(1+x^2+x^4+\cdots \right)\left(1+x^3+x^6+\cdots \right)\left(1+x^4+x^8+\cdots \right)
\end{equation*}
\end{question}
\begin{proof}
Define
\begin{equation*}
d_1:=e_2-e_1\text{ and }d_2:=e_3-e_2\text{ and }d_3:=e_4-e_3
\end{equation*}
Then we have
\begin{equation*}
e_2=e_1+d_1\text{ and }e_3=e_1+d_1+d_2\text{ and }e_4=e_1+d_1+d_2+d_3
\end{equation*}
The question is thus reduced to finding the generating function for 
\begin{equation*}
4e_1+3d_1+2d_2+d_3=r, \set{e_1,d_1,d_2,d_3}\in \N \cup \set{0}
\end{equation*}
Which is 
\begin{equation*}
\left(1+x^4+x^8+\cdots \right)\left(1+x^3+x^6+\cdots \right)\left(1+x^2+x^4+\cdots \right)\left(1+x+x^2+ \cdots \right)
\end{equation*}
\end{proof}
\begin{question}{}{}
Use the equation
\begin{equation*}
\frac{\left(1-x^2 \right)^n}{\left(1-x \right)^n}=\left(1+x \right)^n
\end{equation*}
to show that
\begin{equation*}
\sum_{k=0}^{\frac{m}{2}}\left(-1 \right)^k \binom{n}{k} \binom{n+m-2k-1}{n-1}=\binom{n}{m}\text{ $m\leq n\text{ and }m$ even }
\end{equation*}
\end{question}
\begin{proof}
Observe that 
\begin{equation*}
\binom{n}{m}\text{ is the coefficient of $x^m$ in } \left(1+x \right)^n
\end{equation*}
So we only have to prove that 
\begin{equation*}
\sum_{k=0}^{\frac{m}{2}}\left(-1 \right)^k \binom{n}{k} \binom{n+m-2k-1}{n-1}\text{ is the coefficient of $x^m$ in }\frac{\left(1-x^2 \right)^n}{\left(1-x \right)^n}
\end{equation*}
Observe that 
\begin{equation*}
  \left(1-x^2 \right)^n=\sum_{k=0}^n \left(-1 \right)^k \binom{n}{k} x^{2k}
\end{equation*}
and that 
\begin{equation*}
  \frac{1}{\left(1-x \right)^n}=\sum_{u=0}^\infty \binom{u+n-1}{u} x^u
\end{equation*}
Then because $m$ is even, we can compute the coefficient by summing $k\in \left[0,\frac{m}{2} \right]$ and $u=m-2k\text{ from $\left(2k+u=m \right)$ }$, which tell us that the coefficient is 
 \begin{equation*}
 \sum_{k=0}^{\frac{m}{2}}\left(-1 \right)^k \binom{n}{k}\binom{m-2k+n-1}{m-2k}
\end{equation*}
which equals to 
\begin{equation*}
\sum_{k=0}^{\frac{m}{2}} \left(-1 \right)^k \binom{n}{k} \binom{n+m-2k-1}{n-1}
\end{equation*}



\end{proof}
\begin{question}{}{}
Show that 
\begin{equation*}
2\left(1-x \right)^{-3}\left[\left(1-x \right)^{-3}+\left(1+x \right)^{-3} \right]
\end{equation*}
is the generating function for the number of ways to toss $r$ identical dice and obtain even sum.
\end{question}
\begin{proof}
The generating function of ways to toss $r$ identical dice is 
\begin{equation*}
  \left( \sum_{k=1}^6 x^k\right)^r
\end{equation*}
which equals to
\begin{equation*}
  \left(\frac{1-x^7}{1-x} \right)^r
\end{equation*}
Then the generating function for the number of ways to toss $r$ identical dice and obtain even sum is
\begin{equation*}
  \frac{1}{2}\left[ \left(\frac{1-x^7}{1-x} \right)^r - \left(\frac{1+x^7}{1+x}\right)^r\right]
\end{equation*}
\end{proof}
\section{HW4}
\begin{question}{}{}
How many 10-letter words are there in which each of the letters e,n,r,s occur 
\begin{enumerate}[label=(\alph*)]
  \item At most once?
  \item At least once?
\end{enumerate}
\end{question}
\begin{proof}
The generating function for $(a)$ is 
\begin{align}
  (\frac{x^0}{0!}+\frac{x^1}{1!})^{4}\times (\sum_{n=0}^\infty \frac{x^n}{n!})^{22}&= (x+1)^4 e^{22x}\\
  &=(x^4+4x^3+6x^2+4x+1) \sum_{n=0}^\infty \frac{(22x)^n}{n!}
\end{align}
Then the 10th-coefficient is 
\begin{equation*}
  x^4 \times \frac{(22x)^6}{6!}+4x^3 \times \frac{(22x)^7}{7!}+6x^2 \times \frac{(22x)^8}{8!}+4x\times \frac{(22x)^9}{9!}+\frac{(22x)^{10}}{10!}
\end{equation*}
which is 
\begin{equation*}
  (22^6\times (7\times 8\times 9\times 10)+4\times22^7 \times 8 \times 9\times 10+6\times 22^8\times 9\times 10+4\times 22^9\times 10+22^{10})\times \frac{x^{10}}{10!}
\end{equation*}
The generating function for $(b)$ is 
\begin{align*}
  (\sum_{n=1}^\infty \frac{x^n}{n!})^4 \times (\sum_{n=0}^\infty \frac{x^n}{n!})^{22}&=(e^x-1)^4 e^{22x}\\
  &=(e^{4x}-4e^{3x}+6e^{2x}-4e^x+1)e^{22x}\\
  &=e^{26x}-4e^{25x}+6e^{24x}-4e^{23x}+e^{22x}\\
  &=\sum_{n=0}^\infty \frac{(26x)^n}{n!}-\sum_{n=0}^\infty \frac{4(25x)^n}{n!}+\sum_{n=0}^\infty \frac{6(24x)^n}{n!}-\sum_{n=0}^\infty \frac{4(23x)^n}{n!}+\sum_{n=0}^\infty \frac{(22x)^n}{n!}
\end{align*}
Then the $10$th-coefficient for $(b)$ is 
\begin{equation*}
  (26^{10}-25^{10}\times 4+24^{10}\times 6-23^{10}\times 4+22^{10})\times \frac{x^{10}}{10!}
\end{equation*}
\end{proof}
\begin{question}{}{}
Find a recurrence relation for the number of $n$-digit ternary $(0,1,2)$ sequence in which no $1$ appears anywhere to the right of any $2$.
\end{question}
\begin{proof}
Consider an $n$-digit sequence $\set{x_i}_{i=1}^n$ with desired property:
\begin{equation*}
\forall i, x_i\in \set{0,1,2}\text{ and }x_i=2\implies  x_{i+1}\neq 1 
\end{equation*}
If  $x_n\in \set{0,2}$, we see 
\begin{equation*}
\set{x_i}_{i=1}^{n-1}\text{ satisfy desired property }\implies   \set{x_i}_{i=1}^{n}\text{ satisfy desired property }
\end{equation*}
Notice that the converse of the above statement is clearly true, so we have 
\begin{equation*}
\set{x_i}_{i=1}^{n-1}\text{ satisfy desired property }\iff  \set{x_i}_{i=1}^n\text{ satisfy desired property }
\end{equation*}
If $x_n=1$, we see
\begin{equation*}
\set{x_i}_{i=1}^{n-1}\text{ satisfy desired property and $x_{n-1}\neq 2$}\implies \set{x_i}_{i=1}^n\text{ satisfy desired property }
\end{equation*}
Again notice that the converse of the above statement is clearly true under the hypothesis $x_n=1$
 \begin{equation*}
\set{x_i}_{i=1}^{n-1}\text{ satisfy desired property and $x_{n-1}\neq 2$}\iff  \set{x_i}_{i=1}^n\text{ satisfy desired property }
\end{equation*}
Also, if $x_{n-1}=2$, we see 
\begin{equation*}
\set{x_i}^{n-1}_{i=1}\text{ satisfy desired property}\iff  \set{x_i}^{n-2}_{i=1}\text{ satisfy desired property }
\end{equation*}
Now we reach to the recurrence relation
\begin{equation*}
a_n=2a_{n-1}+(a_{n-1}-a_{n-2})=3a_{n-1}-a_{n-2}
\end{equation*}
where the initial condition is clearly 
\begin{equation*}
a_1=3\text{ and }a_2=3+3+2=8
\end{equation*}
\end{proof}
\begin{question}{}{}
Find an solve a recurrence relation for the number of ways to make a pile of $n$ chips using red, white, and blue chips and such that no two red chips are together.
\end{question}
\begin{proof}
If the last chip is of the color white or blue, it is clear that the pile is legal if and only if the rest of pile is legal. It the last chip is of color red, the pile is legal if and only if the rest of the pile is legal and the second last chip is of the color white or blue. Then we reach to the recurrence relation
\begin{equation*}
  a_n=2a_{n-1}+2a_{n-2}
\end{equation*}
The sequence that satisfy the above homogeneous is a vector space of dimension $2$. Suppose a geometric sequence $a_n=\alpha^n$ is in the vector space. We have
\begin{equation*}
\alpha^n=2\alpha ^{n-1}+2\alpha ^{n-2}
\end{equation*}
Solving the quadratic equation, we have 
\begin{equation*}
\alpha =1 \pm \sqrt{3} 
\end{equation*}
Then we see the set of sequence 
\begin{equation*}
\set{\set{(1+\sqrt{3} )^n}_{n\inn},\set{(1-\sqrt{3} )^n}_{n\inn}}
\end{equation*}
form a basis of the vector space.\\

Now, let 
\begin{equation*}
a_n=A(1+\sqrt{3} )^n+B(1-\sqrt{3} )^n
\end{equation*}
Because $a_1=3,a_2=8$  ($\because$ simple computation), we solve the linear equation  
\begin{equation*}
  \begin{cases}
    A(1+\sqrt{3} )+B(1-\sqrt{3} )=3\\
    A(1+\sqrt{3} )^2+B(1-\sqrt{3} )^2=8
  \end{cases}
\end{equation*}
to have
\begin{equation*}
A=\frac{1}{2}+\frac{\sqrt{3} }{3}\text{ and }B=\frac{1}{2}-\frac{\sqrt{3} }{3}
\end{equation*}
We now have the solution 
\begin{equation*}
  a_n=(\frac{1}{2}+\frac{\sqrt{3} }{3}) (1+\sqrt{3} )^n +(\frac{1}{2}-\frac{\sqrt{3} }{3} )(1-\sqrt{3} )^n
\end{equation*}
\end{proof}
\begin{question}{}{}
Find a general solution to 
\begin{equation*}
a_n-5a_{n-1}+6a_{n-2}=2+3n
\end{equation*}
\end{question}
\begin{proof}
From the form of right hand side, we can guess that the form of particular solution is 
\begin{equation*}
a_n=xn+y
\end{equation*}
Then we can solve $x,y$ from 
\begin{align*}
&xn+y-5(x(n-1)+y)+6(x(n-2)+y)=2+3n\\
  \implies & n(x-5x+6x)+(y+5x-5y-12x+6y)=3n+2\\
  \implies  &(2x)n+(-7x+2y)=3n+2\\
  \implies &\begin{cases}
    2x=3\\
    -7x+2y=2
  \end{cases}\\
    \implies &x=\frac{3}{2}\text{ and }y=\frac{25}{4}
\end{align*}
Clearly, if $b_n,c_n$ satisfy 
 \begin{equation*}
\begin{cases}
  b_n-5b_{n-1}+6b_{n-2}=0\\
  c_n-5c_{n-1}+6c_{n-2}=3n+2
\end{cases}
\end{equation*}
we have 
\begin{equation*}
  (b+c)_n-5(b+c)_{n-1}+6(b+c)_{n-2}=3n+2
\end{equation*}
We have already solved  a particular $c_n=\frac{3}{2}n+\frac{25}{4}$. We now solve for second order homogenous recurrence relation 
\begin{equation*}
b_n=5b_{n-1}-6b_{n-2}
\end{equation*}
Consider 
\begin{equation*}
\alpha^2=5\alpha-6
\end{equation*}
We have 
\begin{equation*}
\alpha=2\text{ or }3
\end{equation*}
We now see 
\begin{equation*}
a_n=A 2^n+B 3^n+\frac{3}{2}n+\frac{25}{4}
\end{equation*}
\end{proof}
\section{Final}
\begin{theorem}
\textbf{(5.5.25)}

\begin{align*}
\binom{2n}{n}^2 = \sum_{k=0}^n \frac{(2n)!}{(k!)^2 (n-k)!^2}
\end{align*}
\end{theorem}
\begin{proof}
Expand 
\begin{align*}
\binom{2n}{n}^2 = \sum_{k=0}^n \binom{n}{k}^2 \binom{2n}{n}
\end{align*}
\end{proof}
\begin{theorem}
\textbf{(5.5.26)} Given $n\leq m$
\begin{align*}
\sum_{k=0}^n \binom{n}{k}\binom{m}{k}= \binom{n+m}{m}
\end{align*}
\end{theorem}
\begin{proof}
Consider walking from $(0,0)$ to $(m,n)$, and notice that we must cross the line $L:y+x=n$
\end{proof}
\begin{theorem}
\textbf{(5.5.28)} Show that 
\begin{align*}
\sum_{j=0}^k \binom{n+k-j-1}{k-j} \binom{m+j-1}{j}=\binom{n+m+k-2}{k}
\end{align*}
\end{theorem}
\begin{proof}
Notice the textbook is wrong on this one. Consider walking from $(0,0)$ to $(k,n-1+m-1)$, and notice that we must cross the line $L:y=n-1$
\end{proof}
\begin{theorem}
\textbf{(6.4.7)}
\begin{enumerate}[label=(\alph*)]
  \item $\frac{3^r+1}{2}$, $g(x)=e^{2x}(\frac{e^x+e^{-x}}{2})$ 
  \item $\frac{3^r+2+(-1)^r}{4}$, $g(x)=e^{x}(\frac{e^x+e^{-x}}{2})^2$ 
  \item $3^r-2^{r+1}+1,$ $g(x)=(e^x-1)^2 e^x$
\end{enumerate}
\end{theorem}
\begin{theorem}
\textbf{(6.4.10)}
\begin{enumerate}[label=(\alph*)]
  \item $3^r-3 \cdot 2 ^{r-2}\cdot \binom{r}{2} + 18 \binom{r}{4}$, $g(x)=(e^x-\frac{x^2}{2})^3$. The coefficient I compute is only for when $r>6$.  I'm not sure if the result remain valid for $0\leq r\leq 6$
  \item $\frac{3^k}{4}-2^k+\frac{3}{2}+\frac{(-1)^k}{4}+\binom{0}{k},g(x)=(\frac{e^x+e^{-x}}{2}-1)^2 e^x$
\end{enumerate}
\end{theorem}
\begin{theorem}
\textbf{(6.4.11)}
\begin{align*}
\frac{4^r + \binom{0}{r}}{2}, g(x)=(\frac{e^{2x}+e^{-2x}}{2})e^{2x}
\end{align*}
\end{theorem}
\begin{theorem}
\textbf{(Conjecture of Question 6.4.12)} If we ask for the total number of 0s, 1s and 2s are even, the generating function should be 
 \begin{align*}
g(x)=(\frac{e^{3x}+e^{-3x}}{2})e^x?
\end{align*}
\end{theorem}
\begin{theorem}
\textbf{(7.3.3)}
\begin{enumerate}[label=(\alph*)]
  \item $a_n=3a_{n-1}+4a_{n-2},a_0=a_1=1\implies a_n=\frac{1}{10}4^{n+1}+ \frac{-6}{10}(-1)^{n+1}$ 
  \item $a_n=a_{n-2},a_0=a_1=1\implies a_n=1$
  \item $a_n=2a_{n-1}-a_{n-2}, a_0=a_1=2\implies $ multiplicity is greater than $1$, won't be in exam. 
  \item $a_n=3a_{n-1}-3a_{n-2}+a_{n-3}, a_0=a_1=1,a_2=2\implies $ multiplicity is greater than $1$
\end{enumerate}
\end{theorem}
\begin{theorem}
\textbf{(7.4.9)}
\begin{enumerate}[label=(\alph*)]
  \item $a_n=3a_{n-1}-2,a_0=0\implies a_n=1-3^n$ 
  \item $a_n=2a_{n-1}+(-1)^n,a_0=2 \implies a_n=\frac{5}{3}2^n + \frac{(-1)^n}{3}$ 
  \item $a_n=2a_{n-1}+n ,a_0=1 \implies a_n=(-n-2)+3 \cdot 2^n$ 
  \item $a_n=2a_{n-1}+2n^2,a_0=3 \implies a_n=15 \cdot 2^n+ (-2n^2-8n-12)$
\end{enumerate}
\end{theorem}
\begin{theorem}
\textbf{(7.4.19)}
\begin{enumerate}[label=(\alph*)]
  \item $a_n^2=2a_{n-1}^2 +1, a_0=1 \implies a_n=\sqrt{2^{n+1}-1} $ 
  \item $a_n=-na_{n-1}+n!,a_0=1 \implies a_n=n!(\frac{1}{2}\cdot (-1)^n+\frac{1}{2})$ 
\end{enumerate}
Hint for (b): Let  $b_n=\frac{a_n}{n!}$
\end{theorem}
\begin{theorem}
\textbf{(Conjecture)} If the remaining term is of the form $r^n+n^m$ , we should guess particular solution by 
\begin{align*}
a_n= c_{m+1} r^n + \sum_{k=0}^m c_k n^k
\end{align*}
\end{theorem}
\begin{theorem}
\textbf{(7.5.3)} 
\begin{align*}
\forall n\geq 1,a_n=\sum_{i=0}^{n-1}a_i a_{n-1-i}, a_0=1 \implies g(x)=\frac{1}{\sqrt{1-x} }
\end{align*}
\begin{align*}
\forall n\geq 3, a_n=\sum_{i=2}^{n-2}a_i a_{n-i},a_0=a_1=a_2=1 \implies (g(x)-1-x)^2=g(x)-x^2-x-1
\end{align*}
\begin{align*}
\forall n\geq 2, a_n=\sum_{i=1}^{n-1} 2^i a_{n-i},a_0=a_1=1 \implies \frac{g(x)-1}{1-2x} =g(x)-x-1
\end{align*}
\end{theorem}
\begin{theorem}
\textbf{(7.5.8)}
\begin{align*}
\forall n\geq 3, a_n=3a_{n-1}-a_{n-3}, \begin{cases}
  a_0=-1\\
  a_1=0\\
  a_2=0
\end{cases}\implies g(x)=\frac{-3x-1}{x^3-3x+1}
\end{align*}
\end{theorem}
\begin{theorem}
\textbf{(8.1.35)}
\begin{align*}
25\text{ people bring just salad and $5$ bring just sandwiches }
\end{align*}
\end{theorem}
\begin{theorem}
\textbf{(8.1.36)}
\begin{align*}
2\text{ people take Greek and Hebrew }
\end{align*}
\end{theorem}
\begin{theorem}
\textbf{(8.3.2)}
\begin{enumerate}[label=(\alph*)]
  \item $6x^4+20x^3+21x^2+8x+1$
  \item $6x^4+17x^3+17x^2+7x+1$ 
  \item $4x^3+12x^2+7x+1$ 
  \item $4x^3+14x^2+8x+1$
\end{enumerate}
$H^6_9 - (6H^5_9 - 15 H^4_9 +20 H^3_9 - 15 H^2_9 + 6 H^1_9)$
\begin{align*}
  (1+3x)^3+2x(1+3x)^2+x^2()
\end{align*}
\end{theorem}
\end{document}
