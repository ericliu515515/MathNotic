\documentclass{report}
%%%%%%%%%%%%%% preamble.tex %%%%%%%%%%%%%%
\usepackage[T1]{fontenc}
\usepackage{etoolbox}
% Page Setup
\usepackage[letterpaper, tmargin=2cm, rmargin=0.5in, lmargin=0.5in, bmargin=80pt, footskip=.2in]{geometry}
\usepackage{adjustbox}
\usepackage{graphicx}
\usepackage{tikz}
\usepackage{mathrsfs}
\usepackage{mdframed}

% Create a new toggle
\newtoggle{firstsection}

% Redefine the \chapter command to reset the toggle for each new chapter
\let\oldchapter\chapter
\renewcommand{\chapter}{\toggletrue{firstsection}\oldchapter}

% Redefine the \section command to check the toggle
\let\oldsection\section
\renewcommand{\section}{
    \iftoggle{firstsection}
    {\togglefalse{firstsection}} % If it's the first section, just switch off the toggle for next sections
    {\clearpage} % If it's not the first section, start a new page
    \oldsection
}

% Abstract Design

\usepackage{lipsum}

\renewenvironment{abstract}
 {% Start of environment
  \quotation
  \small
  \noindent
  \rule{\linewidth}{.5pt} % Draw the rule to match the linewidth
  \par\smallskip
  {\centering\bfseries\abstractname\par}\medskip
 }
 {% End of environment
  \par\noindent
  \rule{\linewidth}{.5pt} % Ensure the closing rule also matches
  \endquotation
 }

% Mathematics
\usepackage{amsmath,amsfonts,amsthm,amssymb,mathtools}
\usepackage{xfrac}
\usepackage[makeroom]{cancel}
\usepackage{enumitem}
\usepackage{nameref}
\usepackage{multicol,array}
\usepackage{tikz-cd}
\usepackage{array}
\usepackage{multirow}% http://ctan.org/pkg/multirow
\usepackage{graphicx}

% Colors
\usepackage[dvipsnames]{xcolor}
\definecolor{myg}{RGB}{56, 140, 70}
\definecolor{myb}{RGB}{45, 111, 177}
\definecolor{myr}{RGB}{199, 68, 64}
% Define more colors here...
\definecolor{olive}{HTML}{6B8E23}
\definecolor{orange}{HTML}{CC5500}
\definecolor{brown}{HTML}{8B4513}
% Hyperlinks
\usepackage{bookmark}
\usepackage[colorlinks=true,linkcolor=blue,urlcolor=blue,citecolor=blue,anchorcolor=blue]{hyperref}
\usepackage{xcolor}
\hypersetup{
    colorlinks,
    linkcolor={red!50!black},
    citecolor={blue!50!black},
    urlcolor={blue!80!black}
}

% Text-related
\usepackage{blindtext}
\usepackage{fontsize}
\changefontsize[14]{14}
\setlength{\parindent}{0pt}
\linespread{1.2}

% Theorems and Definitions
\usepackage{amsthm}
\renewcommand\qedsymbol{$\blacksquare$}

% Define a new theorem style
\newtheoremstyle{mytheoremstyle}% name
  {}% Space above
  {}% Space below
  {}% Body font
  {}% Indent amount
  {\bfseries}% Theorem head font
  {.}% Punctuation after theorem head
  {.5em}% Space after theorem head
  {}% Theorem head spec (can be left empty, meaning ‘normal’)

% Apply the new theorem style to theorem-like environments
\theoremstyle{mytheoremstyle}

\newtheorem{theorem}{Theorem}[section]  
\newtheorem{definition}[theorem]{Definition} 
\newtheorem{lemma}[theorem]{Lemma}  
\newtheorem{corollary}[theorem]{Corollary}
\newtheorem{axiom}[theorem]{Axiom}
\newtheorem{example}[theorem]{Example}
\newtheorem{equiv_def}[theorem]{Equivalent Definition}

% tcolorbox Setup
\usepackage[most,many,breakable]{tcolorbox}
\tcbuselibrary{theorems}

% Define custom tcolorbox environments here...

%================================
% EXAMPLE BOX
%================================
% After you have defined the style and other theorem environments
\definecolor{myexamplebg}{RGB}{245, 245, 245} % Very light grey for background
\definecolor{myexamplefr}{RGB}{120, 120, 120} % Medium grey for frame
\definecolor{myexampleti}{RGB}{60, 60, 60}    % Darker grey for title

\newtcbtheorem[]{Example}{Example}{
    colback=myexamplebg,
    breakable,
    colframe=myexamplefr,
    coltitle=myexampleti,
    boxrule=1pt,
    sharp corners,
    detach title,
    before upper=\tcbtitle\par\vspace{-20pt}, % Reduced the space after the title
    fonttitle=\bfseries,
    description font=\mdseries,
    separator sign none,
    description delimiters={}{}, % No delimiters around the title
}{ex}
%================================
% Solution BOX
%================================
\makeatletter
\newtcolorbox{solution}{enhanced,
	breakable,
	colback=white,
	colframe=myg!80!black,
	attach boxed title to top left={yshift*=-\tcboxedtitleheight},
	title=Solution,
	boxed title size=title,
	boxed title style={%
			sharp corners,
			rounded corners=northwest,
			colback=tcbcolframe,
			boxrule=0pt,
		},
	underlay boxed title={%
			\path[fill=tcbcolframe] (title.south west)--(title.south east)
			to[out=0, in=180] ([xshift=5mm]title.east)--
			(title.center-|frame.east)
			[rounded corners=\kvtcb@arc] |-
			(frame.north) -| cycle;
		},
}
\makeatother

% %================================
% % Question BOX
% %================================
\makeatletter
\newtcbtheorem{question}{Question}{enhanced,
	breakable,
	colback=white,
	colframe=myb!80!black,
	attach boxed title to top left={yshift*=-\tcboxedtitleheight},
	fonttitle=\bfseries,
	title={#2},
	boxed title size=title,
	boxed title style={%
			sharp corners,
			rounded corners=northwest,
			colback=tcbcolframe,
			boxrule=0pt,
		},
	underlay boxed title={%
			\path[fill=tcbcolframe] (title.south west)--(title.south east)
			to[out=0, in=180] ([xshift=5mm]title.east)--
			(title.center-|frame.east)
			[rounded corners=\kvtcb@arc] |-
			(frame.north) -| cycle;
		},
	#1
}{question}
\makeatother

%%%%%%%%%%%%%%%%%%%%%%%%%%%%%%%%%%%%%%%%%%%
% TABLE OF CONTENTS
%%%%%%%%%%%%%%%%%%%%%%%%%%%%%%%%%%%%%%%%%%%


\usepackage{tikz}
\definecolor{doc}{RGB}{0,60,110}
\usepackage{titletoc}
\contentsmargin{0cm}
\titlecontents{chapter}[14pc]
{\addvspace{30pt}%
	\begin{tikzpicture}[remember picture, overlay]%
		\draw[fill=doc!60,draw=doc!60] (-7,-.1) rectangle (-0.9,.5);%
		\pgftext[left,x=-5.5cm,y=0.2cm]{\color{white}\Large\sc\bfseries Chapter\ \thecontentslabel};%
	\end{tikzpicture}\color{doc!60}\large\sc\bfseries}%
{}
{}
{\;\titlerule\;\large\sc\bfseries Page \thecontentspage
	\begin{tikzpicture}[remember picture, overlay]
		\draw[fill=doc!60,draw=doc!60] (2pt,0) rectangle (4,0.1pt);
	\end{tikzpicture}}%
\titlecontents{section}[3.7pc]
{\addvspace{2pt}}
{\contentslabel[\thecontentslabel]{3pc}}
{}
{\hfill\small \thecontentspage}
[]
\titlecontents*{subsection}[3.7pc]
{\addvspace{-1pt}\small}
{}
{}
{\ --- \small\thecontentspage}
[ \textbullet\ ][]

\makeatletter
\renewcommand{\tableofcontents}{
	\chapter*{%
	  \vspace*{-20\p@}%
	  \begin{tikzpicture}[remember picture, overlay]%
		  \pgftext[right,x=15cm,y=0.2cm]{\color{doc!60}\Huge\sc\bfseries \contentsname};%
		  \draw[fill=doc!60,draw=doc!60] (13,-.75) rectangle (20,1);%
		  \clip (13,-.75) rectangle (20,1);
		  \pgftext[right,x=15cm,y=0.2cm]{\color{white}\Huge\sc\bfseries \contentsname};%
	  \end{tikzpicture}}%
	\@starttoc{toc}}
\makeatother

\newcommand{\liff}{\llap{$\iff$}}
\newcommand{\rap}[1]{\rrap{\text{ (#1)}}}
\newcommand{\red}[1]{\textcolor{red}{#1}}
\newcommand{\blue}[1]{\textcolor{blue}{#1}}
\newcommand{\vi}[1]{\textcolor{violet}{#1}}
\newcommand{\olive}[1]{\textcolor{olive}{#1}}
\newcommand{\teal}[1]{\textcolor{teal}{#1}}
\newcommand{\brown}[1]{\textcolor{brown}{#1}}
\newcommand{\orange}[1]{\textcolor{orange}{#1}}
\newcommand{\tCaC}{\text{ \CaC }}
\newcommand{\CaC}{\red{CaC} }
\newcommand{\As}[1]{Assume \red{#1}}
\newcommand{\vdone}{\vi{\text{ (done) }}}
\newcommand{\bdone}{\blue{\text{ (done) }}}
\newcommand{\tdone}{\teal{\text{ (done) }}}
\newcommand{\odone}{\olive{\text{ (done) }}}
\newcommand{\bodone}{\brown{\text{ (done) }}}
\newcommand{\ordone}{\orange{\text{ (done) }}}
\newcommand{\ld}{\lambda}
\newcommand{\vecta}[1]{\textbf{#1}}
\newcommand{\set}[1]{\left\{ #1 \right\}}
\newcommand{\bset}[1]{\Big\{ #1 \Big\}}
\newcommand{\inR}{\in\R}
\newcommand{\inn}{\in\N}
\newcommand{\inz}{\in\Z}
\newcommand{\inr}{\in\R}
\newcommand{\inc}{\in\C}
\newcommand{\inq}{\in\Q}
\newcommand{\norm}[1]{\| #1 \|}
\newcommand{\bnorm}[1]{\Big\| #1 \Big\|}
\newcommand{\gen}[1]{\langle #1 \rangle}
\newcommand{\abso}[1]{\left|#1\right|}
\newcommand{\myref}[2]{\hyperref[#2]{#1\ \ref*{#2}}}
\newcommand{\customref}[2]{\hyperref[#1]{#2}}
\newcommand{\power}[1]{\mathcal{P}(#1)}
\newcommand{\dcup}{\mathbin{\dot{\cup}}}
\newcommand{\diam}[1]{\text{diam}\, #1}
\newcommand{\at}{\Big|}
\newcommand{\quotient}{\diagup}
\let\originalphi\phi % Store the original \phi in \originalphi
\renewcommand{\phi}{\varphi} % Redefine \phi to \varphi
\newcommand{\pfi}{\originalphi} % Define \pfi to display the original \phi
\newcommand{\diota}{\dot{\iota}}
\newcommand{\Log}{\operatorname{Log}}
\newcommand{\id}{\text{\textbf{id}}}
\usepackage{amsmath}

\makeatletter
\NewDocumentCommand{\extp}{e{^}}{%
  \mathop{\mathpalette\extp@{#1}}\nolimits
}
\NewDocumentCommand{\extp@}{mm}{%
  \bigwedge\nolimits\IfValueT{#2}{^{\extp@@{#1}#2}}%
  \IfValueT{#1}{\kern-2\scriptspace\nonscript\kern2\scriptspace}%
}
\newcommand{\extp@@}[1]{%
  \mkern
    \ifx#1\displaystyle-1.8\else
    \ifx#1\textstyle-1\else
    \ifx#1\scriptstyle-1\else
    -0.5\fi\fi\fi
  \thinmuskip
}
\makeatletter
\usepackage{pifont}
\makeatletter
\newcommand\Pimathsymbol[3][\mathord]{%
  #1{\@Pimathsymbol{#2}{#3}}}
\def\@Pimathsymbol#1#2{\mathchoice
  {\@Pim@thsymbol{#1}{#2}\tf@size}
  {\@Pim@thsymbol{#1}{#2}\tf@size}
  {\@Pim@thsymbol{#1}{#2}\sf@size}
  {\@Pim@thsymbol{#1}{#2}\ssf@size}}
\def\@Pim@thsymbol#1#2#3{%
  \mbox{\fontsize{#3}{#3}\Pisymbol{#1}{#2}}}
\makeatother
% the next two lines are needed to avoid LaTeX substituting upright from another font
\input{utxmia.fd}
\DeclareFontShape{U}{txmia}{m}{n}{<->ssub * txmia/m/it}{}
% you may also want
\DeclareFontShape{U}{txmia}{bx}{n}{<->ssub * txmia/bx/it}{}
% just in case
%\DeclareFontShape{U}{txmia}{l}{n}{<->ssub * txmia/l/it}{}
%\DeclareFontShape{U}{txmia}{b}{n}{<->ssub * txmia/b/it}{}
% plus info from Alan Munn at https://tex.stackexchange.com/questions/290165/how-do-i-get-a-nicer-lambda?noredirect=1#comment702120_290165
\newcommand{\pilambdaup}{\Pimathsymbol[\mathord]{txmia}{21}}
\renewcommand{\lambda}{\pilambdaup}
\renewcommand{\tilde}{\widetilde}
\DeclareMathOperator*{\esssup}{ess\,sup}
\newcommand{\bluecheck}{}%
\DeclareRobustCommand{\bluecheck}{%
  \tikz\fill[scale=0.4, color=blue]
  (0,.35) -- (.25,0) -- (1,.7) -- (.25,.15) -- cycle;%
}


\usepackage{tikz}
\newcommand*{\DashedArrow}[1][]{\mathbin{\tikz [baseline=-0.25ex,-latex, dashed,#1] \draw [#1] (0pt,0.5ex) -- (1.3em,0.5ex);}}

\newcommand{\C}{\mathbb{C}}	
\newcommand{\F}{\mathbb{F}}
\newcommand{\N}{\mathbb{N}}
\newcommand{\Q}{\mathbb{Q}}
\newcommand{\R}{\mathbb{R}}
\newcommand{\Z}{\mathbb{Z}}



\title{Notes on Linear Algebra}
\author{Eric Liu}
\date{}
\begin{document}
\maketitle
\newpage% or \cleardoublepage
% \pdfbookmark[<level>]{<title>}{<dest>}
\pdfbookmark[section]{\contentsname}{toc}

\tableofcontents
\pagebreak
\chapter{Linear Algebra Done Outrageous}
\section{Quick Recap}
\begin{abstract}
In this section, $V$ and $W$ are two vector spaces over the same field $\F$, $\F$ can be interpreted as either the field of real numbers or the field of complex numbers, and $S$ is subset of  $V$.  
\end{abstract}
\begin{mdframed}
We say  $S$ is 
\begin{enumerate}[label=(\alph*)]
  \item \textbf{linearly independent} if to write $0$ as finite linear combination of  $S$, the coefficients must all be zero.
  \item  $S$  \textbf{spans $V$} if all $v\in V$ can be written as finite linear combination of $S$. 
\end{enumerate}
We say $S$ is a \textbf{basis} if any of the following equivalent conditions hold true 
\begin{enumerate}[label=(\alph*)]
  \item $S$ is a maximum linearly independent subset of  $V$.  
  \item $S$ is linearly independent and spans $V$.  
  \item All elements of $V$ can be uniquely written as some finite linear combination of  $S$. 
\end{enumerate}
Suppose $V$ has a basis  of cardinality $n$. Using \textbf{Gauss elimination}, one see that all linear independent subset of $V$ must have cardinality not greater than $n$, and all basis must have cardinality $n$. Therefore, for vector space with some basis of finite cardinality, it make sense to refer to its \textbf{dimension}. Let $\operatorname{dim}(V)=n$. Each linearly independent subset $S$ of  $V$ of cardinality  $n$ must be a basis, otherwise we may construct a linearly independent set of cardinality greater than  $n$. By an algorithm, one see that 
\begin{enumerate}[label=(\alph*)]
  \item If $S$ spans  $V$ and has cardinality greater than  $n$, then there exists a subset of $S$ that is a basis of $V$.  
  \item If $B$ is a basis of  $V$ and  $S$ is linearly independent, then there exists some subset  $U$ of $B$ such that  $S \cup  U$ is a basis of $V$.   
\end{enumerate}
Let $F:V\rightarrow W$ be a linear map. It is clear that  both image and kernel of $F$ form some vector space. Therefore, it make sense to define the \textbf{rank} of $F$ by 
 \begin{align*}
\operatorname{rank}(F)\triangleq \operatorname{dim} (\operatorname{Im}F)
\end{align*}
By extending a basis of the kernel of $F$ to a basis of the whole  $V$, one can prove the  \textbf{Rank-Nullity Theorem} 
\begin{align*}
\operatorname{dim}(\operatorname{Ker}F)+ \operatorname{rank}(F)= \operatorname{dim}(V)
\end{align*}
We use $V^*$ to denote the vector space of linear map from  $V$ to  $\F$, the  \textbf{dual space} of $V$. Let $F:V\rightarrow W$ be a linear map. Its \textbf{dual map} $F^*:W^*\rightarrow V^*$ is defined by 
\begin{align*}
F^*(\phi)\triangleq \phi \circ F
\end{align*}
If $V$ is finite-dimensional with basis  $\set{v_1,\dots ,v_n}$, then its \textbf{dual basis} $\set{\phi_1,\dots ,\phi_n}$ is defined by 
\begin{align*}
\phi_i (v_j)\triangleq \delta^i_j
\end{align*}
And there exists a natural isomorphism between $V$ and its double dual  $(V^*)^*$ by identifying $v\in (V^*)^*$ as 
\begin{align*}
v(\phi) \triangleq \phi (v)
\end{align*}
Let $\set{w_1,\dots ,w_m}$ be a basis of $W$ with dual basis  $\set{\xi_i , \dots ,\xi_m}$. It is clear that
\begin{align*}
  \text{ The matrices of $F$ and  $F^*$ are transpose of each other. }
\end{align*}
This together with the observation that for finite dimensional $V,W$, we have 
 \begin{align*}
\operatorname{rank}(F)+ \operatorname{dim}(\operatorname{ker}F^*)= \operatorname{dim}(W)
\end{align*}
gain us a quick proof that the dimensions of column space and row space of a fixed matrix are always equal. 
\end{mdframed}
\begin{mdframed}
  Given some $n$-by-$n$ square matrix $A$, its \textbf{determinant} is by definition 
\begin{align*}
\operatorname{det}A\triangleq  \sum_{\sigma \in S_n} \Big(\operatorname{sgn}\sigma \prod_{k=1}^n  A_{\sigma (k),k} \Big)
\end{align*}
If one identify the space of $n$-by-$n$ square matrix over  $\F$ with  $(\F^n)^n$, determinant can be equivalently defined to be the unique alternating multilinear map from $(\F^n)^n$ to $\F$ such that 
 \begin{align*}
\operatorname{det}(I)=1
\end{align*}
By an algorithm, one see that, for each alternating multilinear map $F:(\F^n)^n\rightarrow \F$, we have 
\begin{align*}
F(B)= (\operatorname{det}B)\cdot F(I) \text{ for all }B.
\end{align*}
Therefore, if we observe for each $n$-by-$n$ square matrix $A$ over  $\F$ that the map $B\mapsto \operatorname{det}AB$ is indeed alternating multilinear, then we immediately have the celebrated result $\operatorname{det}AB=\operatorname{det}A\operatorname{det}B$ as a corollary. This multiplicative property of determinant allow us to well define determinant for each linear epimorphism $F$ over some finite-dimensional vector space by 
 \begin{align*}
\operatorname{det}F\triangleq  \operatorname{det}([F])
\end{align*}
where $[F]$ is the matrix representation of $F$ with respect to some basis. Moreover, because the inverse of a linear map is linear if exists, we see that for linear map $F$ and square matrix $A$, 
\begin{align*}
F\text{ is invertible }\iff \operatorname{det}F\neq 0 
\end{align*}
And 
\begin{align*}
A\text{ is invertible }\iff  \operatorname{det}A \neq 0
\end{align*}
Given some linear epimorphism $F$ over $V$, we say $v\in V$ is an \textbf{eigenvector} with respect to the \textbf{eigenvalue} $\ld \in \F$ if $v\neq 0$ and $F(v)=\ld v$. If there exists some basis of $V$ consisting eigenvectors  of $F$, we say  $F$ is  \textbf{diagonalizable}. Let $A$ be some square matrix, if there exists some  invertible square matrix  $P$ such that  $PAP^{-1}$ diagonal, we also say $A$ is  \textbf{diagonalizable}. Immediately, we see that if $V$ is finite dimensional, then with respect to any basis of $V$
\begin{align*}
[F]\text{ is diagonalizable }\iff F\text{ is diagonalizable }
\end{align*}
Let $V$ be finite dimensional. We define the \textbf{characteristic polynomial} of $F$ and $A$ by 
 \begin{align*}
\operatorname{det}(tI-F)\text{ and }\operatorname{det}(tI-A)
\end{align*}
Obviously, $\ld $ is an eigenvalue of $F$ if and only if $\ld $ is a root of the characteristic polynomial of $F$.
\end{mdframed}

\section{Jordan-Chevalley Decomposition}
\begin{abstract}
In this section, all vectors spaces are finite dimensional,  and we say that $V$ is a \textbf{direct sum} of some collection $\set{U_i}_{i \in I}$ of subspaces of $V$ and writes $V=\bigoplus _{i \in I}U_i$ if for each $v\in V$ there exists some unique tuple $(u_i)_{i \in I}$ such that $u_i \neq 0$ for finite number of $i \in I$ and $v=\sum u_i$. 
\end{abstract}
\begin{mdframed}
Given $F\in \operatorname{End}(V)$, we know the kernels of its powers is increasing 
\begin{align*}
\set{0}=\operatorname{Ker}F^0\subseteq \operatorname{Ker}F^1 \subseteq \operatorname{Ker}F^2 \subseteq \operatorname{Ker}F^3\subseteq \cdots 
\end{align*}
This sequence grows in good manner. If it stops growing at some points, say, $\operatorname{Ker}F^n=\operatorname{Ker}F^{n+1}$ for some $n\inz_0^+$, then it stops forever 
\begin{align*}
\operatorname{Ker}F^n = \operatorname{Ker}F^{n+1}= \operatorname{Ker}F^{n+2}= \operatorname{Ker}F^{n+3}= \cdots 
\end{align*}
In particular, by counting dimensions, we know that this  sequence must stops before reaching to the dimension of $V$ in the sense that  
 \begin{align}
\label{stop}
\operatorname{Ker}(F^{\operatorname{dim}V})=\operatorname{Ker}(F^{\operatorname{dim}V+1})
\end{align}
\myref{Equation}{stop} allows us to elegantly unify distinct notions within a single framework. For instance, for each eigenvalue \( \lambda \), we define the \textbf{generalized eigenspace} \( G(\lambda, F) \) as  
\begin{align*}
G(\lambda, F) \triangleq  \operatorname{Ker} (F - \lambda I)^{\operatorname{dim}V}
\end{align*}
Similarly, we say that \( F \) is \textbf{nilpotent} if  
\begin{align*}
\operatorname{Ker}F^{\operatorname{dim}V}=V
\end{align*}
\end{mdframed}
\begin{theorem}
\label{LIGE}
\textbf{(Linear Independence of Generalized Eigenspaces)} Let $F\in \operatorname{End}(V)$. If $v_1,\dots ,v_n$ are generalized eigenvectors with respect to distinct eigenvalues $\ld_1,\dots ,\ld _n$, then they are linearly independent. 
\end{theorem}
\begin{proof}
Suppose 
\begin{align}
\label{a1v1}
a_1v_1 + \cdots + a_nv_n=0
\end{align}
Because $v_1$ is a generalized eigenvector, we may let $k$ be the largest non-negative integer such that  $(F-\ld_1 I)^kv_1\neq 0$, so that  
\begin{align*}
w\triangleq (F-\ld_1 I)^k v_1\text{ is an eigenvector with eigenvalue  $\ld_1$. }
\end{align*}
Now, noting that the set of endomorphism $\set{(F-\ld_1I)^k,(F-\ld_2I)^{\operatorname{dim}V},\dots ,(F-\ld_nI)^{\operatorname{dim}V}}$  indeed commutes, we may apply the endomorphism $(F-\ld_1I)^k(F-\ld_2I)^{\operatorname{dim}V}\cdots (F-\ld_n I)^{\operatorname{dim}V}$ onto both sides of \myref{Equation}{a1v1} and get 
\begin{align*}
  0&=a_1(F-\ld_2I)^{\operatorname{dim}V}\cdots (F-\ld_nI)^{\operatorname{dim}V}w \\
  &=a_1(\ld_1-\ld_2)^{\operatorname{dim}V}\cdots (\ld_1-\ld_n)^{\operatorname{dim}V}w
\end{align*}
Which implies $a_1=0$. WLOG, we have shown $a_1=\cdots = a_n=0$.  
\end{proof}
\begin{mdframed}
\myref{Equation}{stop} together with Rank-Nullity Theorem give us a theoretically crucial decomposition of $V$ 
\begin{align}
\label{VFn}
V= \operatorname{Ker}F^{\operatorname{dim}V} \oplus  \operatorname{Im}F^{\operatorname{dim}V}
\end{align}
As we will later see, \myref{Decomposition}{VFn} plays a central role in the proof for \myref{Theorem}{DiGE}. 
\end{mdframed}
\begin{theorem}
\label{DiGE}
\textbf{(Decomposition into Generalized Eigenspaces)} Let $V$ be a finite dimensional complex vector space, let $F \in \operatorname{End}(V) $, and let $\set{\ld _1,\dots ,\ld _m}$ be the set of eigenvalues of $F$. We have 
 \begin{align*}
V= G(\ld_1,F)\oplus \cdots \oplus G(\ld_m,F)
\end{align*}
\end{theorem}
\begin{proof}
This is proved by induction on the dimension of $V$. The two base cases correspond to the 0-dimensional and 1-dimensional cases, both of which are trivial. Now, suppose that such a decomposition always exists for complex vector spaces of strictly smaller dimension than $V$ and their endomorphisms. By \myref{Equation}{VFn}, we may decompose $V$ into 
\begin{align*}
  V= G(\ld_1,F) \oplus U,\quad\text{ where }U=\operatorname{Im}(F-\ld_1)^{\operatorname{dim}V}. 
\end{align*}
Noting that $F$ and  $F-\ld_1I$ commutes, we conclude that $U$ is stable under $F$. Therefore, the restriction $F |_U$ defines an endomorphism  $F|_U \in \operatorname{End}(U)$. By inductive hypothesis, we may decompose $U$ into 
 \begin{align*}
U= G(\ld _2,F|_U)\oplus   \cdots \oplus  G(\ld _m,F|_U)
\end{align*}
WLOG, it remains to show 
\begin{align*}
G(\ld _2,F|_U)=G(\ld _2,F)
\end{align*}
Arbitrary select $v\in G(\ld _2,F)$. We may decompose $v=a_1v_1+\cdots + a_mv_m$, where $v_1\in G(\ld _1,F)$ and $v_n\in G(\ld _n ,F|_U)$ for $2\leq n\leq m$. Because $(F-\ld _2I)^{\operatorname{dim}V}v_2=0$, we know 
\begin{align*}
  (F-\ld_2I)^{\operatorname{dim}V}(a_1v_1+ a_3v_3+\cdots + a_mv_m)=0
\end{align*}
This implies $a_1v_1+a_3v_3+\cdots + a_mv_m \in G(\ld_2,F)$. It now follows from \myref{Theorem}{LIGE} that $a_1=a_3=\cdots = a_m=0$. We have shown $v\in G(\ld_2,F|_U)$.
\end{proof} 
\begin{mdframed}
Given some finite dimensional vector space $V$ and some $F \in \operatorname{End}(V)$, we are particularly concerned with the existence and uniqueness of the \textbf{Jordan-Chevalley decomposition} of $F$, i.e, some diagonalizable $S \in \operatorname{End}(V)$ such that 
\begin{enumerate}[label=(\alph*)]
  \item $N\triangleq F-S$ is nilpotent. 
  \item $S$ and  $N$ commute. 
\end{enumerate}
If $V$ is over $\C$, then \myref{Theorem}{DiGE} assert the existence of such decomposition by letting $S$ maps $v\in G(\ld_i,F)$ to $\ld_iv$. To see such decomposition is unique, let $S\in \operatorname{End}(V)$ be some Jordan-Chevalley decomposition of $F$ and decompose $V$ into 
\begin{align*}
V=V_1\oplus  \cdots \oplus  V_k
\end{align*}
where $V_i$ are the eigenspaces of $S$ corresponding to distinct eigenvalues $\ld_i$. Because $F,S$ commute and $F-S$ is nilpotent, we may conclude that  $\ld_i$ are indeed eigenvalues of $F$ and 
\begin{align*}
V_i \subseteq G(\ld_i,F)\text{ for all }i
\end{align*}
This together with \myref{Theorem}{DiGE} shows the uniqueness of Jordan-Chevalley decomposition of endomorphisms of finite dimensional vector spaces over $\C$. 
\end{mdframed}
\begin{theorem}
\label{JfNO}
\textbf{(Jordan forms of Nilpotent Operators)} Let $V$ be a finite-dimensional complex vector space, and let  $N \in \operatorname{End}(V)$ be nilpotent. There exists some $v_1,\dots ,v_m \in V$ and nonnegative integer $n_1,\dots ,n_m$ such that 
\begin{enumerate}[label=(\alph*)]
  \item $N^{n_1}v_1,\dots ,Nv_1,v_1,\dots ,N^{n_m}v_m,\dots ,Nv_m,v_m$ form a basis for $V$. 
  \item $N^{n_1+1}v_1=\cdots = N^{n_m+1}v_m=0$. 
\end{enumerate}
\end{theorem}
\begin{proof}
This is also proved by induction on the dimension of $V$. The two base cases correspond to the $0$-dimensional and $1$-dimensional cases, both of which are trivial. We now prove the inductive case. \\

Because $N$ is nilpotent, we know  $\operatorname{dim}(\operatorname{Im}N)< \operatorname{dim}V$. Noting that $\operatorname{Im}N$ is stable under $N$, we see $N|_{\operatorname{Im}N}\in \operatorname{End}(\operatorname{Im}N)$. From inductive hypothesis, we have $v_1,\dots ,v_n\in \operatorname{Im}N$ and nonnegative integer $k_1,\dots ,k_n$ such that 
\begin{align*}
  \set{N^{k_1}v_1,\dots ,Nv_1,v_1, \dots ,N^{k_n}v_n,\dots ,Nv_n,v_n}\text{ form a basis for $\operatorname{Im}N$. }
\end{align*}
and 
\begin{align*}
N^{k_1+1}v_1=\cdots = N^{k_n+1}v_n=0. 
\end{align*}
Because $v_1,\dots ,v_n \in \operatorname{Im}N$, we may let $u_1,\dots ,u_n \in V$ satisfy $v_j=Nu_j$ for all  $j$. To see 
\begin{align}
\label{setn}
  \set{N^{k_1+1}u_1 ,\dots ,Nu_1,u_1, \dots ,N^{k_n}u_n,\dots ,Nu_n,u_n}\text{ is linearly independent, }
\end{align}
suppose some finite linear combination equals to $0$. By applying $N$ to this finite linear combination, we see the only possible nonzero coefficients are those of $N^{k_j+1}u_j$, otherwise 
\begin{align*}
  \set{N^{k_1}v_1,\dots ,Nv_1,v_1,\dots ,N^{k_n}v_n,\dots ,Nv_n,v_n}\text{ would not be linearly independent. }
\end{align*}
Knowing that the only possible nonzero coefficients are those of $N^{k_j+1}u_j=N^{k_j}v_j$, we may conclude that even these coefficients are zero, since $\set{N^{k_1}v_1,\dots ,N^{k_n}v_n}$ is linearly independent in the first place. We may now expand \myref{Set}{setn} to a basis 
\begin{align}
\label{set1}
\set{N^{k_1+1}u_1,\dots ,Nu_1,u_1,\dots ,N^{k_n+1}u_n,\dots ,Nu_n,u_n,w_1,\dots ,w_p}
\end{align}
for $V$. Because $\set{N^{k_1+1}u_1,\dots ,N^2u_1,Nu_1,\dots ,N^{k_n+1}u_n,\dots ,N^2u_n,Nu_n}$ is a basis for  $\operatorname{Im}N$, we may subtract each $w_i$ with some element of 
\begin{align*}
\operatorname{span} \set{N^{k_1}u_1,\dots ,Nu_1,u_1,\dots , N^{k_n+1}u_n,\dots ,Nu_n,u_n}
\end{align*}
so that  \myref{Set}{set1} form a desired basis for $V$.    
\end{proof}
\begin{mdframed}
Let $V$ be some finite-dimensional complex vector space and $F\in \operatorname{End}(V)$. Let $S+N$ be the Jordan-Chevalley decomposition of  $F$. The basis for $V$ from  \myref{Theorem}{JfNO} is called a \textbf{Jordan basis}. If we express $F$ as a matrix with respect to this basis, we say that matrix is in its \textbf{Jordan form}. The Jordan form looks like 
\begin{align*}
\begin{bmatrix}
  J_1 & & \\
      & \ddots & \\
      & & J_r
\end{bmatrix}
\end{align*}
where each block matrix looks like 
\begin{align*}
J =  \begin{bmatrix}
  \ld_i & 1 & & \\
        & \ld_i & \ddots & \\
        & & \ddots & 1 \\
        & & & \ld_i
\end{bmatrix}
\end{align*}
Let $\ld_1,\dots ,\ld_m$ be the distinct eigenvalues of $F$, and let  $r_i$ be the dimension of the largest Jordan block with eigenvalue  $\ld_i$. It is clear that the \textbf{minimal polynomial} of $F$, some polynomial $m$ of smallest degree such that $m(F)=0$, take the form 
\begin{align*}
m(x)=(x-\ld_1)^{r_1}\cdots (x-\ld_m)^{r_m} 
\end{align*}
It is also clear that even if $V$ is over $\R$ instead of  $\C$, we have the  \textbf{Cayley-Hamilton Theorem}, that is, $p(F)=0$, where $p$ is the characteristic polynomial of  $F$.\\

Interestingly, we may also take a completely algebraic approach to construct the Jordan-Chevalley decomposition \( S, N \) and show that they are polynomials in \( F \), without ever invoking \myref{Theorem}{LIGE}, \myref{Theorem}{DiGE}, nor any result that depends on them. The argument proceeds as follows: \\

Because \( \operatorname{End}(V) \) is finite-dimensional, the minimal polynomial \( m \) exists. By the Fundamental Theorem of Algebra, we can factor \( m \) into linear terms:
\begin{align*}
m(x) = (x - \xi_1)^{r_1} \cdots (x - \xi_k)^{r_k}.
\end{align*}
Consider the polynomials \( f_i \) defined by
\begin{align*}
f_i(x) \triangleq \prod_{j=1;j\neq i}^{k} (x - \xi_j)^{r_j}.
\end{align*}
Since \( \set{f_1, \dots, f_k} \) are coprime, by Bézout's identity, we may find polynomials \( q_i \in \C[x] \) such that
\begin{align*}
1 = \sum_{i=1}^{k} q_i f_i.
\end{align*}
Define \( \pi_i \in \operatorname{End}(V) \) by \( \pi_i \triangleq (q_i f_i)(F) \). Because \( m \) divides \( f_i f_j \) for \( i \neq j \), we conclude that
\begin{align*}
\pi_i \pi_j = 0 \quad \text{for } i \neq j.
\end{align*}
This, together with \( \sum \pi_j = 1 \), shows that \( \pi_i \) are \textbf{projections}:
\begin{align*}
\pi_i^2 = \pi_i \quad \text{for all } i.
\end{align*}
It is clear that \( V \) is a direct sum of the images of these projections. Define
\begin{align*}
S \triangleq \sum_{i=1}^{k} \xi_i \pi_i.
\end{align*}
Since \( S \) is a polynomial in \( F \), we conclude that \( S \) and \( N \) commute. The fact that \( N \) is nilpotent follows from the definition of \( f_i \).
\end{mdframed}
\section{Spectral}
\begin{mdframed}
  By a \textbf{norm}, we mean some \textbf{positive-definite} functional $\norm{\cdot}:V\to \R$ that satisfies \textbf{absolute homogeneity} and \textbf{triangle inequality}. In this context, for $\norm{\cdot}$ to be positive-definite, it must satisfy 
\begin{align*}
  \norm{v}=0 \implies  v=0
\end{align*}
Observing that  
\begin{align*}
\norm{0}=\norm{0+v}\leq \norm{0}+\norm{v}\text{ for all $v\in V$ }
\end{align*}
we see norm must also be nonnegative.\\

By an \textbf{inner product}, we mean some \textbf{positive-definite}  map $\langle \cdot,\cdot\rangle : V^2 \rightarrow \F$ that is linear in the first argument and satisfies \textbf{conjugate symmetry}: 
\begin{align*}
\langle v,w\rangle = \overline{\langle w,v\rangle }\quad\text{for all }v,w \in V
\end{align*}
In this context, for $\langle \cdot,\cdot\rangle :V^2\rightarrow \F$ to be positive-definite, it has to satisfy 
\begin{align*}
\langle v,v\rangle >0\quad\text{for all }v\neq 0
\end{align*}
With an inner product equipped on $V$, we may discuss the  \textbf{orthogonality} of vectors. Given a countable set  $\set{v_1,v_2,v_3,\dots }$ of non-zero vectors, we use the term \textbf{Gram-Schmidt process} to refer to the process of defining 
\begin{align*}
e_n \triangleq v_n - \frac{\langle v_{n},e_{n-1}\rangle }{\norm{e_{n-1}}^2}e_{n-1} - \cdots - \frac{\langle v_n,e_1\rangle }{\norm{e_1}^2}e_1
\end{align*}
so that $\set{e_1,e_2,e_3,\dots}$ become orthogonal while 
\begin{align*}
\operatorname{span}(v_1,\dots ,v_n)=\operatorname{span}(e_1,\dots ,e_n)\quad\text{for all }n.
\end{align*}
Although implicit, Gram-Schmidt process may be considered one of the most important notions in inner product space. For example, the Gram-Schmidt process, together with \textbf{Pythagorean Theorem}, can be used to prove the \textbf{Cauchy-Schwarz Inequality}, which in term shows that the functional $f:V\rightarrow \R$ induced by 
\begin{align*}
f(v)\triangleq \sqrt{\langle v,v\rangle }  
\end{align*}
indeed satisfies triangle inequality, thus forming a norm. In addition, given an endomorphism $T$ of some finite-dimensional complex vector space $V$, by applying Gram-Schmidt process to a Jordan basis, we get an orthogonal basis under which $T$ becomes an upper triangular matrix, called its \textbf{Schur's form}.\\

Let $V,W$ be two inner product space over  $\F$, and let  $T:V\rightarrow W$ be some linear map. If some linear map $T^\dagger:W\rightarrow V$ satisfies 
\begin{align*}
\langle Tv,w\rangle = \langle v,T^\dagger w\rangle \quad\text{for all }v\in V,w\in W, 
\end{align*}
we say $T^\dagger $ is an  \textbf{adjoint} of $T$. If both $V$ and $W$ are finite-dimensional, then $T^\dagger $ exists uniquely, and its matrix representation is always the complex conjugate of that of $T$, regardless of the choices of bases. It is obvious that the double adjoint of a linear map between two finite-dimensional inner product spaces over $\F$ is itself.\\

If $T$ is \textbf{normal}, then by direct computation, we see that its Schur's form must be diagonal. This argument is called \textbf{Spectral Theorem for complex finite-dimensional vector space}.\\








\end{mdframed}
\section{Matrix Exponential}
\begin{mdframed}
Given some square matrix $A$ over  $\F$, we define its  \textbf{matrix exponential} by 
\begin{align*}
e^A \triangleq \sum_{n=0}^{\infty} \frac{A^n}{n!}
\end{align*}
It 
\end{mdframed}
\chapter{NTU Math M.A. Program Entrance Exam} 
\section{Year 113}
\begin{question}{}{}
Let 
\begin{align*}
A\triangleq  \begin{bmatrix}
  -2 & -1 & 1\\
  1 & 0 & 1\\
  0 & 0 & 1
\end{bmatrix}
\end{align*}
Find the Jordan-Chevalley decomposition of $A$ and compute  $e^A$. 
\end{question}
\begin{proof}
Routine computation give us 
\begin{align*}
J=\begin{bmatrix}
  1 & 0 & 0 \\
  0 & -1 & 1\\
  0 & 0 & -1
\end{bmatrix}\text{ and }P=\begin{bmatrix}
  0 & 1 & -1 \\
  1 & -1 & 0 \\
  1 & 0 & 0
\end{bmatrix}\text{ and }A=PJP^{-1}
\end{align*}
Therefore, the Jordan-Chevalley decomposition of $A$ is 
\begin{align*}
A= D+ N \text{ where }D=P\begin{bmatrix}
  1 & 0 & 0 \\
  0 & -1 & 0 \\
  0 & 0 & -1 
\end{bmatrix}P^{-1}\text{ and }N=P\begin{bmatrix}
0 & 0 & 0 \\
0 & 0 & 1 \\
0 & 0 & 0
\end{bmatrix}P ^{-1}
\end{align*}
And 
\begin{align*}
e^A= P\begin{bmatrix}
  e & 0 & 0\\
  0 & e^{-1} & e^{-1} \\
  0 & 0 & e^{-1}
\end{bmatrix} P^{-1}
\end{align*}
\end{proof}
\begin{question}{}{}
Let $V$ be the space of polynomial in $x$ over  $\R$ of degree not higher than  $2$. Define an inner product for $V$ by 
\begin{align*}
\langle f,g\rangle \triangleq \int_{-1}^{1} fg
\end{align*}
Find a polynomial $k(x,t)$ such that 
\begin{align}
\label{fx=}
f(x)= \int_{-1}^{1}k(x,t)f(t)dt \quad\text{for all }f \in V
\end{align}
Define $T \in \operatorname{End}(V)$ by 
\begin{align*}
T(a_2x^2+a_1x+a_0) \triangleq 2a_2x+a_1
\end{align*}
That is, $Tf\triangleq f'$. Find the adjoint of $T$. 
\end{question}
\begin{proof}
Because of the linearity, for $k$ to satisfy \myref{Equation}{fx=} for all $f\in V$, it only has to satisfies 
\begin{align}
\label{solve}
1=\int_{-1}^{1}kdt\text{ and }x=\int_{-1}^{1}ktdt \text{ and }x^2 =\int_{-1}^{1}kt^2dt
\end{align}
If we write $k$ in the form 
\begin{align*}
k= f_0(x)+ f_1(x)t+ f_2(x)t^2 + \cdots 
\end{align*}
Then \myref{Equation}{solve} becomes 
\begin{align*}
1&= \frac{2f_0(x)}{1}+ \frac{2f_2(x)}{3}+ \frac{2f_4(x)}{5}+ \cdots  \\
x&=  \frac{2f_1(x)}{3}+ \frac{2f_3(x)}{5}+ \frac{2f_5(x)}{7}+ \cdots \\
x^2&= \frac{2f_0(x)}{3} + \frac{2f_2(x)}{5}+ \frac{2f_4(x)}{7}+ \cdots 
\end{align*}
Thus, $k$ can be 
\begin{align*}
k(x,t)= \Big(\frac{9}{8}+ \frac{-15}{8}x^2\Big)+ \Big(\frac{3x}{2}\Big)t+ \Big(\frac{-15}{8}+ \frac{45}{8}x^2\Big)t^2 
\end{align*}
Routine computation give us an orthonormal basis 
\begin{align*}
\bset{\sqrt{\frac{1}{2}}, \sqrt{\frac{3}{2}}x, \sqrt{\frac{45}{8}}(x^2- \frac{1}{3})}\text{ for $V$ }
\end{align*}
With respect to this ordered basis, the matrix representation of $T$ is 
\begin{align*}
\begin{bmatrix}
  0 & \sqrt{3}  & 0 \\
  0 & 0 & \sqrt{15} \\
  0 & 0 & 0 
\end{bmatrix}
\end{align*}
Because the matrix representation of $T^\dagger $ with respect to the same basis is the conjugate transpose of that of  $T$, we may now compute 
 \begin{align*}
T^\dagger (a_2x^2+a_1x+a_0)= \frac{15a_1}{2}(x^2- \frac{1}{3})+ (3a_0+a_2)x
\end{align*}
\end{proof}

\begin{question}{}{}
Let $V$ be the vector space of all $n$-by-$n$ square matrices over $\R$, and let $F:V\rightarrow \R$ be linear. Suppose  
\begin{align*}
F(AB)=F(BA)\text{ and }F(I)=n
\end{align*}
for all $A,B \in V$. Show that $F$ is the trace function. 
\end{question}
\begin{proof}
Let $E_{i,j}\in V$ be the matrix whose only nonzero entry is $1$, located in the $i$-th row and $j$-th column. If $i \neq j$, then  
\begin{align*}
E_{i,i}E_{i,j}=E_{i,j}\text{ and }E_{i,j}E_{i,i}=0.
\end{align*}
This, together with the linearity of $F$, shows that $F$ depends only on the diagonal entries. For each permutation $\sigma \in S_n$, there exists a \textbf{permutation matrix} $C$ such that  
\begin{align*}
\text{The $\sigma(i)$-th row of }CA\text{ is identical to the $i$-th row of }A.
\end{align*}
And  
\begin{align*}
\text{The $\sigma(i)$-th column of }AC^{-1}\text{ is identical to the $i$-th column of }A.
\end{align*}
Therefore,  
\begin{align*}
  A_{i,i}=(CAC^{-1})_{\sigma (i),\sigma (i)}.
\end{align*}
Observing that  
\begin{align*}
F(CAC^{-1})=F(A(CC^{-1}))=F(A),
\end{align*}
we have shown that $F$ is stable under the permutation of diagonal entries. Define $\sigma \in S_n$ by  
\begin{align*}
\sigma(i)\triangleq \begin{cases}
  i+1& \text{ if } 1\leq i<n, \\
  1& \text{ if } i=n.
\end{cases}
\end{align*}
Let $C_k$ be the permutation matrix corresponding to $\sigma^k$ for all $1\leq k\leq n$. The proof now follows from computing  
\begin{align*}
nF(A)= \sum_{k=1}^{n}F(C_kAC_k^{-1})=F\Big(\sum_{k=1}^n C_kAC_k^{-1}\Big)= F((\operatorname{tr}A)I)=n\operatorname{tr}A.
\end{align*}
\end{proof}
\begin{question}{}{}
  Let $U,V$ be two finite dimensional space over the same field. Let  $T:U\rightarrow V$ be a linear map, and let $T^*:U^*\rightarrow V^*$ be its dual map. Prove 
\begin{align*}
T\text{ is injective }\iff T^*\text{ is surjective }   
\end{align*}
And 
\begin{align*}
T\text{ is surjective }\iff T^*\text{ is injective }
\end{align*}
\end{question}
\begin{proof}
Let $\set{T(u_1),\dots ,T(u_n)}$ be a basis for the image of $T$. Extend this to a basis $\set{T(u_1),\dots ,T(u_n),v_1,\dots ,v_m}$ for $V$. Let $\set{\xi_1,\dots ,\xi_{n+m}}$ be its dual basis. It is clear that $\set{\xi_{n+1},\dots ,\xi_{n+m}}$ belongs to the kernel of $T^*$. Observe  
\begin{align*}
T^*\xi_1v_1 = 1
\end{align*}
to conclude that $\xi_i \not\in \operatorname{Ker}T^*$ for all $1\leq i\leq n$. We have shown $\set{\xi_{n+1},\dots ,\xi_{n+m}}$ is a basis for the kernel of $T^*$. In other words,  
\begin{align*}
   \operatorname{rank}T+\operatorname{Dim}(\operatorname{Ker}T^*)= \operatorname{dim}V.
\end{align*}
This, together with Rank-Nullity Theorem, proves both propositions.
\end{proof}
\begin{question}{}{}
Let $V$ be some finite-dimensional vector space over some field $\F$ and let $T \in \operatorname{End}(V)$. Let $f,g$ be two relatively prime polynomials. Prove 
\begin{align*}
\operatorname{Ker}(f(T)g(T))= \operatorname{Ker}f(T)\oplus \operatorname{Ker}g(T).
\end{align*}
\end{question}

\begin{proof}
To show that the two kernels form a direct sum, assume for contradiction that there exists a nonzero vector $v \in V$ such that $v$ belongs to both kernels. Since $f$ is a polynomial satisfying $f(T)v=0$, there must exist a polynomial $p$ of minimal degree such that $p(T)v=0$. By polynomial division, we can write 
\begin{align*}
f = pq + r, \quad \text{where } q, r \text{ are polynomials and } \deg r < \deg p.
\end{align*}
Since $r$ has a strictly smaller degree than $p$, the minimality of $p$ implies that $r = 0$. Thus, $f$ is divisible by $p$. WLOG, we may assume that $g$ is also divisible by $p$. This contradicts the assumption that $f$ and $g$ are relatively prime polynomials. We have shown the two kernels indeed form a direct sum. It is clear that 
\begin{align*}
\operatorname{Ker}f(T)\oplus  \operatorname{Ker}g(T)\subseteq \operatorname{Ker}(f(T)g(T)) 
\end{align*}
We now prove the opposite. Let $v \in \operatorname{Ker}(f(T)g(T))$. Because $f,g$ are relatively prime, by \textbf{Bezout's identity}, there exists some polynomials  $a,b$ such that 
\begin{align*}
af+bg=1
\end{align*}
The proof then follows from noting 
\begin{align*}
  (af)(T)v \in \operatorname{Ker}g(T) \text{ and }(bg)(T)v \in \operatorname{Ker}f(T)
\end{align*}
\end{proof}
\section{Year 112}
\begin{question}{}{}
Let 
\begin{align*}
\textbf{v}_1= (1,2,0,4),\textbf{v}_2=(-1,1,3,-3),\textbf{v}_3= (0,1,-5,-2),\textbf{v}_4=(-1,-9,-1,-4)
\end{align*}
be vectors in $\R^4$. Let $W_1$ be the subspace spanned by  $\textbf{v}_1$ and $\textbf{v}_2$, and let $W_2$ be the subspace spanned by $\textbf{v}_3$ and $\textbf{v}_4$. Find a basis for $W_1 \cap W_2$. 
\end{question}
\begin{proof}
Use Gauss elimination to show 
\begin{align*}
\begin{cases}
  \set{\textbf{v}_1,\textbf{v}_2,\textbf{v}_3}\text{ is linearly independent. }\\
  \set{\textbf{v}_1,\textbf{v}_2,\textbf{v}_4}\text{ is linearly independent. }
\end{cases}
\end{align*}
Use Gauss elimination to show 
\begin{align*}
3\textbf{v}_1+ 2\textbf{v}_2 +\textbf{v}_3+ \textbf{v}_4=0
\end{align*}
And conclude 
\begin{align*}
W_1 \cap W_2= \operatorname{span} \set{\textbf{v}_3+\textbf{v}_4}
\end{align*}
\end{proof}
\begin{question}{}{}
Let 
\begin{align*}
A= \begin{bmatrix}
  0 & 1 & -1 \\
  3 & -4 & 1 \\
  3 & -8 & 5 
\end{bmatrix}
\end{align*}
Find an invertible matrix $Q\in M_3(\C)$ such that 
\begin{align*}
QAQ^{-1}= \begin{bmatrix}
  0 & 0 & 0 \\
  1 & 0 & 12 \\
  0 & 1 & 1
\end{bmatrix}
\end{align*}
Find an invertible matrix $P \in M_3  (\C)$ such that $PAP^{-1}$ is diagonal. \emph{Note that my wording differs slightly with the original}.  
\end{question}
\begin{proof}
Noting that characteristic polynomial is defined uniquely up to change of basis, 
\begin{align*}
\operatorname{det}(tI- QAQ^{-1})=\operatorname{det}(Q(tI-A)Q^{-1})= \operatorname{det}(tI-A)
\end{align*}
We may quickly compute that the characteristic polynomial of $A$ is $t(t-4)(t+3)$. Routine computation now give us 
\begin{align*}
P=\begin{bmatrix}
  1 & 7 & 0 \\
  1 & -1 & 1 \\
  1 & -29 & 1
\end{bmatrix} \text{ and }PAP^{-1}=\begin{bmatrix}
  0 & 0 & 0 \\
  0 & 4 & 0 \\
  0 & 0 & -3 
\end{bmatrix}
\end{align*}
Note that one may find $Q$ by solving a  9-by-9 linear equation. Here we give a smarter approach.  By looking at $QAQ^{-1}$, we may reduce the problem into finding $v\inc^3$ such that 
\begin{align*}
A^3v= 12Av+A^2v \text{ and }\set{v,Av,A^2v}\text{ is linear independent }
\end{align*}
So that 
\begin{align*}
Q\triangleq \begin{bmatrix}
  v & Av & A^2v
\end{bmatrix}\text{ suffices. }
\end{align*}
Because the characteristic polynomial of $A$ is  $t(t-4)(t+3)$, by Cayley-Hamilton Theorem, we know all $v\inc^3$ satisfy the first condition.  To satisfy the linear independence, we  write 
\begin{align*}
v= c_1e_1+c_2e_2+c_3e_3
\end{align*}
where $e_1,e_2,e_3$ are the eigenvectors
 \begin{align*}
e_1\triangleq \begin{bmatrix}
  1 \\
  1 \\
  1 
\end{bmatrix} \text{ and } e_2 \triangleq  \begin{bmatrix}
7 \\
-1 \\
-29
\end{bmatrix} \text{ and }e_3\triangleq \begin{bmatrix}
0 \\
1 \\
1
\end{bmatrix}
\end{align*}
So that we may just looks for $c_1,c_2,c_3\inc$ that makes 
\begin{align*}
\begin{bmatrix}
  c_1 & c_2 & c_3 \\
  0 & 4c_2 & -3c_3 \\
  0 & 16c_3 & 9c_3
\end{bmatrix}\text{ invertible. }
\end{align*}
By computing the determinant, we see that an obvious choice is $c_1=c_2=c_3=1$. That is, 
\begin{align*}
v=\begin{bmatrix}
8 \\
1 \\
-27
\end{bmatrix}\text{ and }Q=\begin{bmatrix}
8 & 28 & 112 \\
1 & -7 & -7 \\
-27 & -119 & -455
\end{bmatrix}
\end{align*}
\end{proof}
\begin{question}{}{}
Define \textbf{matrix trigonometry} by 
\begin{align*}
\sin A \triangleq \sum_{n=0}^{\infty} \frac{(-1)^n}{(2n+1)!}A^{2n+1}
\end{align*}
Compute 
\begin{align*}
\sin \begin{pmatrix}
  1 & 3 \\
  0 & 1
\end{pmatrix}
\end{align*}
Show that there exist no matrix $A \in M_{2}(\R)$ such that 
\begin{align*}
\sin A = \begin{pmatrix}
 1 & 2022 \\
 0 & 1
\end{pmatrix}
\end{align*}
\end{question}
\begin{proof}
Using the identity 
\begin{align*}
\sin A= \frac{e^{iA}-e^{-iA}}{2i}
\end{align*}
we may compute 
\begin{align*}
\sin \begin{pmatrix}
  1 & 3 \\
  0 & 1
\end{pmatrix}= \begin{pmatrix}
  \sin (1) & 3 \cos (1) \\
  0 & \sin (1)
\end{pmatrix}
\end{align*}
Because $\sin (PBP^{-1})=P (\sin B)P^{-1}$ for all $B$ and all invertible  $P$, and because 
\begin{align*}
  \sin \begin{pmatrix} 
    \ld_1 & 0 \\
    0 & \ld_2
  \end{pmatrix}= \begin{pmatrix} 
    \sin \ld_1 & 0 \\
    0 & \sin \ld_2
  \end{pmatrix} \text{ and }\sin \begin{pmatrix} 
    \ld & 1 \\
    0 & \ld 
  \end{pmatrix}= \begin{pmatrix} 
    \sin \ld  & \cos \ld  \\
    0 & \sin \ld 
  \end{pmatrix}
\end{align*}
We may finish the proof by computing the Jordan form
\begin{align*}
\begin{pmatrix} 
  1 & 2022 \\
  0 & 1
\end{pmatrix}= \begin{pmatrix} 
  1 & 1 \\
  0 & \frac{1}{2022}
\end{pmatrix}\begin{pmatrix} 
  1 & 1 \\
  0 & 1
\end{pmatrix} \begin{pmatrix} 
  1 & 1 \\
  0 & \frac{1}{2022}
\end{pmatrix}^{-1}
\end{align*}
that there exists no matrix $A\in M_2(\R)$ such that 
\begin{align*}
\sin A = \begin{pmatrix}
 1 & 2022 \\
 0 & 1
\end{pmatrix}
\end{align*}
\end{proof}
\begin{question}{}{}
Let $A= (a_{ij})\in M_n(\C)$, and let $\ld_1,\dots ,\ld _n$ be roots of characteristic polynomial of $A$ counted with multiplicity. Show that 
 \begin{align*}
A\text{ is normal }\iff  \norm{A}_F= \sum_{k=1}^n \abso{\ld_k}^2
\end{align*}
where $\norm{A}_F$ is the Frobenius norm of $A$. 
\end{question}
\begin{proof}
Because the underlying field is $\C$, we may apply  \textbf{Schur's decomposition} to get an upper triangular matrix $D$ and an unitary matrix  $Q$ such that 
 \begin{align*}
A= QDQ^{-1}
\end{align*}
Because $A$ and  $D$ has the same characteristic polynomial, we know the eigenvalues of $A$ lie on the diagonal line of $D$ counted with multiplicity. The proof now follows from computing
\begin{align*}
\norm{A}_F= \operatorname{tr}(A^*A)=\operatorname{tr}(D^*D)= \norm{D}_F
\end{align*}
and \textbf{Spectral Theorem for complex finite-dimensional vector space}.
\end{proof}
\begin{question}{}{}
Let $A,B \in M_n(\C)$. Suppose that all of the eigenvalues of $A$ and  $B$ are positive real numbers. Prove 
 \begin{align*}
A^4=B^4 \implies  A=B
\end{align*}
\end{question}
\begin{proof}
\end{proof}
\section{Year 111}
\begin{question}{}{}
Let $V$ be a finite-dimensional complex inner product space. Let $d\in \operatorname{End}(V)$ satisfy 
\begin{align*}
d^2=0
\end{align*}
Let $\delta$ be the adjoint of $d$. Define $\Delta \in \operatorname{End}(V)$ by 
\begin{align*}
\Delta \triangleq d\delta + \delta d
\end{align*}
Prove 
\begin{enumerate}[label=(\alph*)]
  \item $\operatorname{Ker}d\delta\subseteq \operatorname{Ker}\delta$ and $\operatorname{Ker}\delta d\subseteq \operatorname{Ker}d$.
  \item  $\operatorname{Ker}\Delta = \operatorname{Ker}d\cap \operatorname{Ker}\delta$. 
  \item $V= \operatorname{Ker}\Delta \oplus  \operatorname{Im}d \oplus  \operatorname{Im}\delta$ is an orthogonal decomposition. 
  \item $\operatorname{Ker}d=\operatorname{Ker}\Delta \oplus  \operatorname{Im}d$ is an orthogonal decomposition. 
\end{enumerate}
\end{question}
\begin{proof}
  Because the underlying field is $\C$, we may express $d$ in its \textbf{Jordan form}. Because $d^2=0$, we know its Jordan blocks all have eigenvalue  $0$ and have size no greater than  $2$. Knowing that the matrix representation of $\delta$ with respect to the same Jordan basis is just the conjugate transpose of $d$, we may easily prove all four propositions with computation, except 
\begin{align*}
  \operatorname{Ker}\Delta \perp \operatorname{Im}d\quad\text{ and }\quad \operatorname{Im}d \perp \operatorname{Im}\delta \quad\text{ and }\quad \operatorname{Im}\delta \perp \operatorname{Ker}\Delta 
\end{align*}
Direct computation with the hint $\delta= d^{\dagger }$ and $d^2=0$ shows  $\operatorname{Im}d\perp \operatorname{Im}\delta$. Direct computation with hint $\operatorname{Ker}\Delta =\operatorname{Ker}d\cap \operatorname{Ker}\delta$ shows the other two orthogonal relationships. 
\end{proof}
\begin{question}{}{}
Let $V=\R^n$ be the space of column vectors, and  $M$ a positive definite symmetric  $n\times n$ real matrix. Suppose $A\in M_{n}(\R)$ satisfies 
\begin{align*}
MAM^{-1}=A^t
\end{align*}
Show that there exists some $P \in M_n(\R)$ such that 
\begin{align*}
P^t M P = I \text{ and }P^{-1}AP\text{ is diagonal }
\end{align*}
\end{question}
\chapter{Archived}
\section{Tensor Algebra}
\begin{abstract}
In this section, by the term \textbf{ring}, we mean a ring with a multiplication identity, and by the term \textbf{real algebra}, we mean a real vector space equipped with a vector multiplication compatible with both scalar multiplication and addition. In this definition, for a real algebra $A$ to be a ring, $A$ must be associative.  By the term \textbf{ideal}, we mean a 2-sided ideal. If we say a multi-linear map $M:V^k\rightarrow Z$ is \textbf{alternating}, we mean that $M$ maps $(v_1,\dots ,v_n)$ to $0$ if two arguments coincide.
\end{abstract}
\begin{mdframed}
Given a finite collection $(V_1,\dots ,V_n)$ of finite dimensional real vector space, by the term \textbf{tensor product of $V_1,\dots ,V_n$}, we mean a real vector space  usually denoted by $V_1 \otimes \cdots \otimes  V_n$ and a multilinear map  $\otimes  : V_1 \times \cdots \times V_n \rightarrow V_1 \otimes  \cdots \otimes  V_n$ satisfying the universal property: If $B:V_1 \times \cdots \times V_n\rightarrow Z$ is a multilinear map, then there exists a unique linear map $\beta :V_1\otimes \cdots \otimes  V_n$ such that 
\begin{align*}
B(v_1,\dots ,v_n)=\beta (v_1\otimes \cdots \otimes  v_n)
\end{align*}
In other words, we have the commutative diagram
% https://q.uiver.app/#q=WzAsMyxbMCwwLCJWXFx0aW1lcyBXIl0sWzAsMiwiViBcXG90aW1lcyBXIl0sWzIsMCwiVSJdLFswLDIsIkIiXSxbMSwyLCJcXGV4aXN0cyEgXFxiZXRhIiwyLHsic3R5bGUiOnsiYm9keSI6eyJuYW1lIjoiZGFzaGVkIn19fV0sWzAsMSwiKHYsdylcXG1hcHN0byB2XFxvdGltZXMgdyIsMl1d
\[\begin{tikzcd}
	{V_1\times \cdots \times V_n} && U \\
	\\
	{V_1 \otimes \cdots \otimes   V_n}
	\arrow["B", from=1-1, to=1-3] 
	\arrow["{\otimes }"', from=1-1, to=3-1]
	\arrow["{\exists! \beta}"', dashed, from=3-1, to=1-3]
\end{tikzcd}\]
This approach indeed define a pair of vector space and multilinear map uniquely up to isomorphism, in the sense of \myref{Theorem}{UoT}, where we define the isomorphism between tensor product. 
\end{mdframed}
\begin{theorem}
\label{UoT}
\textbf{(Uniqueness of Tensor product)} Given a finite collection $(V_1,\dots, V_n)$ of finite dimensional real vector space, if $V_1 \otimes \cdots \otimes  V_n ,V_1\otimes  '\cdots \otimes  ' V_n$ both satisfy the universal property, then there exists an linear isomorphism $T:V_1 \otimes  \cdots \otimes  V_n \rightarrow V_1 \otimes' \cdots \otimes  ' V_n $ such that 
\begin{align*}
T(v_1\otimes  \cdots \otimes  v_n)=v_1 \otimes  '\cdots \otimes  'v_n
\end{align*}
\end{theorem}
\begin{proof}
Because $V_1 \otimes  \cdots \otimes  V_n$ satisfies the universal property, there exists a linear map $T:V_1 \otimes  \cdots \otimes  V_n\rightarrow V_1 \otimes'  \cdots \otimes' V_n  $ such that  
\begin{align*}
\otimes' =T \circ \otimes  
\end{align*}
It remains to show $T$ is bijective. Similarly, because $V_1 \otimes ' \cdots \otimes'  V_n$ satisfies the universal property, there exists a linear map $T':V_1 \otimes'  \cdots \otimes'  V_n\rightarrow V_1 \otimes  \cdots \otimes V_n  $ such that 
\begin{align*}
\otimes = T' \circ \otimes  '
\end{align*}
Composing the two equations, we have 
\begin{align*}
\otimes '=T\circ T' \circ \otimes  '
\end{align*}
It then follows from uniqueness of the induced linear map in universal property that $T \circ T'=\textbf{id}:V_1 \otimes  ' \cdots \otimes  'V_n \rightarrow V_1 \otimes  ' \cdots \otimes  'V_n$. This implies $T$ is indeed bijective. 
\end{proof}
\begin{mdframed}
We have shown that tensor products is unique up to isomorphism. A construction further shows that if $B_i$ are bases for $V_i$, then 
 \begin{align*}
\set{v_1 \otimes  \cdots \otimes v_n: v_i \in B_i\text{ for all }1\leq i\leq n  }\text{ form a basis for }V_1 \otimes  \cdots \otimes  V_n
\end{align*}
\end{mdframed}
\begin{theorem}
\textbf{(Associativity of the Tensor product)} Given three finite-dimensional real vector spaces $X,Y,Z$,  there exists a unique linear isomorphism $F:X\otimes  Y\otimes  Z\rightarrow (X\otimes  Y)\otimes  Z$ that satisfy 
\begin{align*}
F(x\otimes  y \otimes  z)=(x\otimes  y)\otimes  z
\end{align*}
\end{theorem}
\begin{proof}
Define $f:X\times Y \times Z\rightarrow (X\otimes  Y)\otimes Z$ by 
\begin{align*}
f(x,y,z)\triangleq (x \otimes  y)\otimes z
\end{align*}
It follows from the universal property that there exists a unique linear map $F:X\otimes  Y\otimes Z\rightarrow (X \otimes  Y)\otimes  Z $ such that 
\begin{align*}
F(x\otimes  y\otimes  z)=f(x,y,z)= (x \otimes y) \otimes z 
\end{align*}
It remains to show \vi{$F$ is bijective}.  For all $z\in Z$, define $h_z:X\times Y\rightarrow X \otimes  Y\otimes Z$ by 
\begin{align*}
h_z(x,y)\triangleq x\otimes  y\otimes  z
\end{align*}
If follows from the universal property that there exists a unique linear map $H_z:X\otimes  Y \rightarrow X \otimes  Y\otimes  Z$ such that 
\begin{align*}
H_z(x \otimes  y)=h_z(x,y)=x \otimes  y \otimes  z
\end{align*}
Define $h:(X\otimes  Y)\times Z\rightarrow X\otimes Y \otimes Z$ by 
\begin{align*}
h(v,z)\triangleq H_z(v)
\end{align*}
It is clear that $h$ in linear in  $(X\otimes  Y)$. We now show $h$ is linear in  $Z$, that is 
 \begin{align*}
   \olive{H_{c_1z_1+z_2}=c_1H_{z_1}+H_{z_2}}
\end{align*}
By definition, 
\begin{align*}
  (c_1H_{z_1}+H_{z_2})(x\otimes  y)&=c_1 x\otimes  y\otimes  z_1+ x\otimes  y \otimes  z_2\\
  &= x\otimes  y\otimes  (c_1z_1+z_2)=h_{c_1z_1+z_2}(x,y)
\end{align*}
It then follows from the uniqueness part of the universal property that $H_{c_1z_1+z_2}=c_1H_{z_1}+H_{z_2}$. $\odone$  \\


We have shown $h$ is indeed bilinear. It follows from the universal property that there exists a unique linear map $H:(X\otimes  Y)\otimes  Z\rightarrow X\otimes Y\otimes Z$ such that 
\begin{align*}
H((x\otimes y)\otimes z)=h(x\otimes  y,z)=H_z(x\otimes  y)=x\otimes  y\otimes  z
\end{align*}
Let $\otimes:X\times Y\times Z\rightarrow X\otimes  Y\otimes  Z$ denotes the tensor product, we now have 
\begin{align*}
\otimes = H \circ F \circ \otimes 
\end{align*}
It then follows from universal property that $H \circ F=\textbf{id}:X\otimes  Y\otimes  Z\rightarrow X\otimes  Y\otimes  Z$. This implies $F$ is indeed bijective. $\vdone$
\end{proof}
\begin{mdframed}
Let $V$ be a finite-dimensional real vector space. By its  \textbf{tensor algebra}, we mean any real associative algebra $T(V)$ with an injective linear map $\diota:V\rightarrow T(V)$ that satisfies the universal property: If $A$ is a real associative algebra and $f:V\rightarrow A$ is a linear map, then there exists a unique algebra homomorphism $F:T(V)\rightarrow A$ such that the diagram  
% https://q.uiver.app/#q=WzAsMyxbMiwwLCJUKFYpIl0sWzIsMiwiQSJdLFswLDAsIlYiXSxbMiwwLCJcXGRvdHtcXGlvdGF9Il0sWzIsMSwiZiIsMl0sWzAsMSwiRiIsMCx7InN0eWxlIjp7ImJvZHkiOnsibmFtZSI6ImRhc2hlZCJ9fX1dXQ==
\[\begin{tikzcd}
	V && {T(V)} \\
	\\
	&& A
	\arrow["{\dot{\iota}}", from=1-1, to=1-3]
	\arrow["f"', from=1-1, to=3-3]
	\arrow["F", dashed, from=1-3, to=3-3]
\end{tikzcd}\]
commutes. The proof that such definition is indeed unique up to isomorphism is similar to that of \myref{Theorem}{UoT} and thus omitted. We now give the most useful construction. \\

Let $V$ be finite-dimensional real vector space. We use the notation 
 \begin{align*}
T^n(V)\triangleq  \overbrace{V \otimes  \cdots \otimes  V}^{n\text{ copies }}
\end{align*}
and call $T^{n}(V)$ the \textbf{$n$-th tensor power of $V$} or the \textbf{$n$-fold tensor product of  $V$}. Define  
\begin{align*}
  T(V)&\triangleq  \bigoplus_{n=0}^{\infty} T^n(V)\\
  &=\R \oplus V \oplus (V\otimes V) \oplus (V\otimes V \otimes V)  \oplus \cdots 
\end{align*}
and define for all $f,g \in T(V)$ the multiplication 
\begin{align*}
  (fg)(n)\triangleq   \sum_{k=0}^{n}f(k)g(n-k)
\end{align*}
where 
\begin{align*}
  &\Big(\sum_{I} a_I v_{I(1)}\otimes \cdots \otimes  v_{I(k)}\Big)\Big(\sum_J b_J v_{J(1)}\otimes  \cdots \otimes  v_{J(l)}\Big)\\
  &\triangleq \sum_{I,J} a_I b_J v_{I(1)}\otimes \cdots \otimes  v_{I(k)}\otimes  v_{J(1)}\otimes  \cdots \otimes  v_{J(l)}
\end{align*}
where $\set{v_1,\dots ,v_m}$ is some basis for $V$, $I$ run through the set of function that maps $\set{1,\dots ,k}$ into $\set{1,\dots ,m}$ and $J$ run through the set of function that maps  $\set{1,\dots ,l}$ into $\set{1,\dots ,m}$. For example, given two elements  
\begin{align*}
  (5,0,v_1\otimes v_2,0,0,\dots )\text{ and }(7,v_3,0,0,\dots)
\end{align*}
of $T(V)$, their product is defined to be 
\begin{align*}
  (35,5v_3,7v_1\otimes v_2, v_1 \otimes v_2 \otimes v_3 ,0,0,\dots)
\end{align*}
Tedious effort shows that our multiplication is consistent with abuse of notation in the sense that if $f,g\in T(V)$ is defined by  
\begin{align*}
 f(k)\triangleq \begin{cases}
   w_1\otimes \cdots \otimes  w_n & \text{ if $k=n$ }\\
   0& \text{ if otherwise }
 \end{cases} \text{ and } g(k)\triangleq \begin{cases}
   w_{n+1}\otimes  \cdots \otimes  w_{n+l}& \text{ if $k=l$ }\\
   0& \text{ if otherwise }
 \end{cases} 
\end{align*}
then 
\begin{align*}
  (fg)(k)=\begin{cases}
    w_1\otimes  \cdots \otimes  w_{n+l}& \text{ if $n=k+l$ }\\
    0& \text{ if otherwise }
  \end{cases}
\end{align*}


does form an associative algebra with multiplication identity $1\inr$. Thus, $T(V)$ is in fact a ring. Let $I(V)\subseteq T(V)$ be the ideal generated by $\set{v\otimes v:v \in V}$. By definition, ideal $I(V)$ is a subgroup of $T(V)$. To see that $I(V)$ is closed under scalar multiplication, observe that for all $t\inr\text{ and }x\in T(V)$, the scalar multiplication $tx$ is identical to $tx$ where $t$ is treated as an element of $T(V)$, so it follows from definition of ideal that $I(V)$ is also a vector subspace of $T(V)$. Let $\set{v_1,\dots ,v_n}$ be a basis for $V$, and let  $S$ be the set of function that maps  $\set{1,\dots ,n}$ into $\set{1,\dots ,k}$. We know for a fact that 
\begin{align*}
T^k(V) = \operatorname{span} \set{v_{I(1)}\otimes  \cdots \otimes  v_{I(k)}: I \in S}
\end{align*}
If we define $I^k(V)\triangleq I(V)\cap T^k(V)$, one then have 
\begin{align}
\label{ikv} I^k(V)=\operatorname{span}\set{v_{I(1)}\otimes \cdots \otimes  v_{I(k)}:I(j)=I(j+1)\text{ for some }j}
\end{align}
This is proved by showing $I^0(V)\oplus  I^1(V)\oplus  I^2(V)\oplus  \cdots $ is indeed the smallest ideal containing $\set{v\otimes  v:v \in V}$. Define an equivalence class on $T(V)$ by 
\begin{align*}
x\sim  y\overset{\triangle}{\iff }x-y \in I (V)
\end{align*}
Because ideal form a subgroup, we see that our definition indeed give an equivalence relation. We then can define on the set of equivalence class $T(V)\setminus I(V)$ addition, scalar multiplication and vector multiplication 
\begin{align*}
[x]+[y]\triangleq [x+y] \text{ and }[x]\wedge [y]\triangleq [xy]\text{ and }c[x]\triangleq [cx]
\end{align*}
which is well defined and form an algebra as one can check. We call this algebra $T(V)\setminus I(V)$ the \textbf{exterior algebra} $\wedge ^* (V)$. Note that if we refer to $v,w \in T^k(V)$ as elements of $\wedge ^*(V) $, we mean $[v],[w]$. Immediately, we see that the wedge product is \textbf{alternating} in the sense that if $v\in V$, then 
\begin{align*}
v\wedge v=0 
\end{align*}
and is \textbf{anti-symmetric} in the sense that if $v,w \in V$, then 
\begin{align*}
v\wedge w= -w\wedge  v
\end{align*}
We use the notation 
\begin{align*}
\wedge^k(V) \triangleq \bset{ [x] \in \wedge ^*(V): x \in \overbrace{V \otimes  \cdots \otimes  V}^{k\text{ copies }}} 
\end{align*}
Immediately from \myref{Equation}{ikv}, we see that $\wedge ^k (V) $ is the vector space 
\begin{align*}
\operatorname{span}\set{v_{I(1)}\wedge  \cdots \wedge  v_{I(k)}:I \in S  }
\end{align*}
where $\set{v_1,\dots ,v_n}$ is a basis for $V$ and $S$ is the set of function that maps  $\set{1,\dots ,k}$ into $\set{1,\dots ,n}$. If we define the vector subspace $I^k(V)\triangleq T^k(V)\cap I(V)$, there exists a natural vector space isomorphism
\begin{align*}
\wedge  ^k(V)\underset{\text{ v.s. }}{\cong }T^k(V)\setminus I^k(V);[x]\leftrightarrow [x]
\end{align*}
where $T^k(V)\setminus I^k(V)$ is the quotient vector space.
\end{mdframed}
\begin{theorem}
\label{Universal mapping property for alternating $k$-linear map}
\textbf{(Universal mapping property for alternating $k$-linear map)} For any vector space $Z$ over  $\R$ and any alternating  $k$-linear map  $f:V^k\rightarrow Z$, there is a unique linear map $F:\bigwedge ^k V\rightarrow Z $ such that the diagram 
% https://q.uiver.app/#q=WzAsMyxbMCwwLCJcXGJpZ3dlZGdlXmsgViAiXSxbMCwyLCJWXmsiXSxbMiwyLCJaIl0sWzEsMCwiXFx3ZWRnZSAiXSxbMCwyLCJcXHRpbGRle2Z9Il0sWzEsMiwiZiIsMl1d
\[\begin{tikzcd}
	{\bigwedge^k V } \\
	\\
	{V^k} && Z
	\arrow["F", from=1-1, to=3-3]
	\arrow["{\wedge }", from=3-1, to=1-1]
	\arrow["f"', from=3-1, to=3-3]
\end{tikzcd}\]
commute, i.e., 
\begin{align*}
F(v_1\wedge \cdots \wedge  v_k  )=f(v_1,\dots ,v_k) \text{ for all }v_1,\dots,v_k \in V
\end{align*}
\end{theorem}
\begin{proof}
By \customref{Universal Property of Tensor Product}{universal property of tensor product}, there exists unique linear map $h:T^k(V)\rightarrow Z$ such that 
 \begin{align*}
h(v_1 \otimes  \cdots \otimes  v_k)=f(v_1,\cdots ,v_k)
\end{align*}
Because $f$ is alternating, we see from the characterization of $I^k(V)$ given in \myref{Equation}{ikv} that $h$ vanishes on $I^k(V)$. We then can induce a linear map 
\begin{align*}
F:\wedge^k (V)\cong \frac{T^k(V)}{I^k(V)} \rightarrow Z 
\end{align*}
by $F([x])\triangleq h(x)$. This then give us the desired  
\begin{align*}
F(v_1 \wedge  \cdots \wedge  v_k )=h(v_1\otimes  \cdots \otimes  v_k)=f(v_1,\dots ,v_k)
\end{align*}
Note that $F$ is unique because all such linear map take the same values on  $\set{v_{I(1)}\wedge  \cdots \wedge  v_{I(k)}:I \in S }$ which spans $\wedge ^k(V) $. 
\end{proof}
\begin{mdframed}
Let $\set{w_1,\dots ,w_l}\subseteq V$ be linear independent. An immediate consequence of the \customref{Universal mapping property for alternating $k$-linear map}{universal mapping property for alternating $k$-linear map} is that one may define alternating multilinear $f:V^l\rightarrow \R$ by 
\begin{align*}
B(v_1,\dots ,v_l)\triangleq \operatorname{det}(M)\text{ where }v_i= \sum_j M_{i,j}w_j
\end{align*}
and see that $F:\wedge^l (V)\rightarrow \R $ take $w_1\wedge  \cdots \wedge  w_l$ to $1$. This implies that  
\begin{align*}
w_1\wedge  \cdots \wedge  w_l \neq 0  
\end{align*}
\end{mdframed}
\begin{theorem}
\label{ASo}
\textbf{(Anti-symmetry of wedge product)} If $\alpha \in \wedge ^k (V),\beta \in \wedge ^l (V)  $, then $\alpha \wedge  \beta =(-1)^{kl}(\beta  \wedge  \alpha  )  $. 
\end{theorem}
\begin{proof}
Let $v_1,\dots ,v_n$ be a basis of $V$. Let $S_k$ be the space of function that maps $\set{1,\dots ,k}$into  $\set{1,\dots,n}$ , $S_l$ be the space of function that maps  $\set{1,\dots ,l}$ into  $\set{1,\dots ,n}$. We may then write 
\begin{align*}
\alpha = \sum_{I \in S_k} a_I (v_{I(1)}\wedge \cdots \wedge  v_{I(k)})\text{ and }\beta =\sum_{J \in S_l} b_J (v_{J(1)}\wedge \cdots \wedge v_{J(l)}  )
\end{align*}
and compute 
\begin{align*}
\alpha \wedge \beta &= \sum_{I \in S_k,J\in S_l}a_Ib_J (v_{I(1)}\wedge \cdots \wedge v_{I(k)} \wedge  v_{J(1)} \wedge \cdots  \wedge v_{J(l)}   )\\
&=   \sum_{I \in S_k,J\in S_l}(-1)a_Ib_J (v_{I(1)}\wedge \cdots \wedge   v_{J(1)} \wedge  v_{I(k)}\cdots\wedge \cdots  \wedge v_{J(l)}   )   \\
&=  \sum_{I \in S_k,J\in S_l}(-1)^ka_Ib_J (v_{J(1)}\wedge v_{I(1)}\wedge  \cdots \wedge    v_{I(k)} \wedge  v_{J(2)} \wedge \cdots  \wedge v_{J(l)}   )   \\
  &=\sum_{I \in S_k,J\in S_l}(-1)^{kl}a_Ib_J (v_{J(1)}\wedge \cdots \wedge  v_{J(l)}\wedge v_{I(1)} \wedge   \cdots \wedge   v_{I(k)})= (-1)^{kl}\beta \wedge  \alpha  
\end{align*}
\end{proof}
\begin{mdframed}
Following from \myref{Theorem}{ASo}, \myref{Equation}{ikv} and tedious effort, one can see that if $\set{v_1,\dots ,v_n}$ is a basis for $V$, then 
 \begin{align*}
\set{v_{i_1}\wedge  \cdots \wedge  v_{i_k}:1\leq i_1< \cdots < i_k\leq n  }
\end{align*}
form a basis for $\wedge^k(V)$. If $A:V\rightarrow W$ is a linear map, we define linear map $\wedge^k A:\wedge^k (V)\rightarrow \wedge^k (W) $ by linear extension of 
\begin{align*}
\wedge^k(A)(v_1\wedge  \cdots \wedge   v_n)=Av_1 \wedge  \cdots \wedge  Av_n  
\end{align*}
Note that if  $A:V\rightarrow V$ and $\operatorname{dim}(V)=n$, then $\wedge ^nA: \wedge^n (V)\rightarrow \wedge ^n(V)$ is given by the determinant since given basis  $\set{v_1,\dots ,v_n}$, we have 
\begin{align*}
\wedge^nA(v_1 \wedge  \cdots \wedge  v_n   )&=\Big(\sum_j A_{j,1}v_j\Big)\wedge  \cdots \wedge \Big(\sum_j A_{j,n}v_j\Big)    \\
&=\sum_{\sigma \in S_n} A_{\sigma (1),1}\cdots A_{\sigma (n),n}v_{\sigma (1)}\wedge  \cdots \wedge  v_{\sigma(n)}   \\
&=\sum_{\sigma \in S_n}\operatorname{sgn}(\sigma)A_{\sigma(1),1}\cdots A_{\sigma(n),n}v_1\wedge  \cdots \wedge  v_n
\end{align*}
\end{mdframed}
\section{Operator Norm}
\begin{abstract}
This section introduces the concept of the operator norm and proves some fundamental results related operator norm and finite-dimensional normed spaces. For example, we establish results such as \customref{LOB}{a linear operator being bounded if and only if it is continuous} and \customref{ANoF}{the equivalence of all norms on finite-dimensional vector spaces}.
\end{abstract}
\begin{mdframed}
In this section, and particularly in functional analysis, we say a function $T$ between two metric space is a  \textbf{bounded operator} if $T$ always map bounded set to bounded set. In particular, if $T$ is a linear transformation between two normed space, we say $T$ is a \textbf{bounded linear operator}. Now, suppose $\mathcal{X},\mathcal{Y}$ are two normed space over $\R\text{ or }\C$. In space $L(\mathcal{X},\mathcal{Y})$, alternatively, we can define 
\begin{align*}
T\text{ is bounded } \overset{\triangle}{\iff} \exists M\inr,\forall x\in \mathcal{X}, \norm{Tx}\leq M\norm{x}
\end{align*}
The proof of equivalency is simple. For $(\longrightarrow )$, observe 
\begin{align*}
\norm{Tx}= \norm{x}\cdot\norm{T \frac{x}{\norm{x}}}\leq \Big(\sup \set{\norm{Ty}:\norm{y}=1} \Big)\norm{x}
\end{align*}
For $(\longleftarrow)$, observe 
\begin{align*}
\norm{Tx-Ty}=\norm{T(x-y)}\leq M \norm{x-y}
\end{align*}
We first show that \customref{LOB}{a linear transformation is continuous if and only if it is bounded}. 
\end{mdframed}
\begin{theorem}
\label{LOB}
\textbf{(Liner Operator is Bounded if and only if it is Continuous)} Given two normed space $\mathcal{X},\mathcal{Y}$ over $\R\text{ or }\C$ and  $T\in L(\mathcal{X},\mathcal{Y})$, we have 
\begin{align*}
T\text{ is a bounded operator }\iff T\text{ is continuous on $\mathcal{X}$}
\end{align*}
\end{theorem}
\begin{proof}
If $T$ is bounded, we see that $T$ is Lipschitz. 
\begin{align*}
\norm{Tx-Ty}\leq M \norm{x-y}
\end{align*}
Now, suppose $T$ is linear and continuous at $0$. Let $\epsilon $ satisfy 
\begin{align*}
\sup_{\norm{y}\leq \epsilon } \norm{Ty} \leq 1
\end{align*}
Observe that for all $x \in \mathcal{X}$, we have
\begin{align*}
\norm{Tx}= \frac{\norm{x}}{\epsilon } \bnorm{T \frac{\epsilon x}{\norm{x}}}\leq \frac{\norm{x}}{\epsilon}
\end{align*}
\end{proof}
\begin{mdframed}
Here, we introduce a new terminology, which shall later show its value. Given a set $X$, we say two metrics $d_1,d_2$ on $X$ are \textbf{equivalent}, and write $d_1\sim d_2$, if we have 
\begin{align*}
\exists m,M \inr^+, \forall x ,y\in X, md_1(x,y)\leq d_2(x,y) \leq Md_1(x,y)
\end{align*}
Now, given a fixed vector space $V$, naturally, we say two norms $\norm{\cdot}_1,\norm{\cdot}_2$ on $V$ are \textbf{equivalent} if 
\begin{align*}
\exists m,M \inr^+, \forall x\in X, m \norm{x}_1\leq \norm{x}_2 \leq M\norm{x}_1
\end{align*}
We say two metric $d_1,d_2$ on  $X$ are  \textbf{topologically equivalent} if the topology they induce on $X$ are identical.\\

A few properties can be immediately spotted.  
\begin{enumerate}[label=(\alph*)]
  \item Our definition of $\sim$ between metrics of a fixed $X$ is an equivalence relation.
  \item Our definition of $\sim$ between norms on a fixed $V$ is an equivalence relation.
  \item Equivalent norms induce equivalent metrics.
  \item Equivalent metrics are topologically equivalent. 
\end{enumerate}

We now prove \customref{ANoF}{if $V$ is finite-dimensional, then all norms on  $V$ are equivalent}. This property will later show its value, as used to prove \customref{LmoF}{linear map of finite-dimensional domain is always continuous} 
\end{mdframed}
\begin{theorem}
\label{ANoF}
\textbf{(All Norms on Finite-dimensional space are Equivalent)} Suppose $V$ is a finite-dimensional vector space over $\R\text{ or }\C$. Then 
\begin{align*}
\text{ all norms on $V$ are equivalent }
\end{align*}
\end{theorem}
\begin{proof}
Let $\set{e_1,\dots ,e_n}$ be a basis of $V$. Define $\infty$-norm $\norm{\cdot}_\infty$ on $V$ by 
\begin{align*}
\bnorm{\sum \alpha_i e_i}_{\infty}\triangleq  \max \abso{\alpha_i} 
\end{align*}
It is easily checked that $\norm{\cdot}_\infty$ is indeed a norm. Fix a norm $\norm{\cdot}$ on $V$. We reduce the problem into  
\begin{align*}
  \vi{\text{ finding $m,M\inr^+$ such that }m\norm{x}_\infty \leq \norm{x}\leq M\norm{x}_\infty}
\end{align*}
We first claim 
\begin{align*}
\blue{M=\sum \norm{e_i}\text{ suffices }}
\end{align*}
Compute 
\begin{align*}
\norm{x}= \bnorm{\sum \alpha_ie_i} \leq \sum \abso{\alpha _i} \norm{e_i} \leq \norm{x}_\infty \sum \norm{e_i}= M \norm{x}_\infty \bdone
\end{align*}
Note that reverse triangle inequality give us 
\begin{align}
\label{Lip1}
\Big|\norm{x}-\norm{y}\Big|\leq \norm{x-y} \leq M \norm{x-y}_\infty
\end{align}
Then we can check that 
\begin{enumerate}[label=(\alph*)]
  \item  $\norm{\cdot}:\Big(V,\norm{\cdot}_\infty \Big)\rightarrow \R$ is Lipschitz continuous because of \myref{Equation}{Lip1}.
  \item $S\triangleq \set{y\in V:\norm{y}_\infty=1}$ is sequentially compact in $\norm{\cdot}$ and non-empty. 
\end{enumerate}
Now, by EVT, we know $\min _{y \in S}\norm{y}$ exists. Note that $\min_{y \in S}\norm{y}>0$, since $0 \not\in S$. We claim 
\begin{align*}
  \olive{m= \min_{y \in S}\norm{y}\text{ suffices }}
\end{align*}
Fix $x \in V$ and compute 
\begin{align*}
m\norm{x}_\infty = \norm{x}_\infty (\min_{y \in S} \norm{y})\leq \norm{x}_\infty \cdot  \bnorm{ \frac{x}{\norm{x}_\infty}}=\norm{x}\odone \vdone
\end{align*}






\end{proof}
\begin{theorem}
\label{LmoF}
\textbf{(Linear map of Finite-dimensional Domain is always Continuous)} Given a finite-dimensional normed space $\mathcal{X}$ over $\R\text{ or }\C$, an arbitrary normed space  $\mathcal{Y}$ over $\R\text{ or }\C$ and a linear transformation  $T:\mathcal{X}\rightarrow \mathcal{Y}$, we have 
\begin{align*}
T\text{ is continuous }
\end{align*}
\end{theorem}
\begin{proof}
Fix $x\in \mathcal{X},\epsilon $. We wish 
\begin{align*}
\vi{\text{ to find $\delta$ such that }\forall h\in \mathcal{X}: \norm{h}\leq \delta, \norm{T(x+h)-Tx}\leq \epsilon }
\end{align*}
Let $\set{e_1,\dots ,e_n}$ be a basis of $\mathcal{X}$. Note that $ \norm{ \sum \alpha_i e_i }_1\triangleq  \sum \abso{\alpha _i}$ is a norm. \customref{ANoF}{Because $\mathcal{X}$ is finite-dimensional, we know $\norm{\cdot}$ and $\norm{\cdot}_1$ are equivalent}. Then, we can fix $M\inr^+$ such that 
\begin{align*}
\hspace{2cm}\norm{x}_1 \leq M\norm{x}\hspace{0.5cm}(x \in V)
\end{align*}
We claim 
\begin{align*}
\vi{\delta= \frac{\epsilon }{M(\max \norm{Te_i} )}\text{ suffices }} 
\end{align*}
Fix $\norm{h}\leq \delta$ and express $h=\sum \alpha_i e_i$. Compute using linearity of $T$
\begin{align*}
  \norm{T(x+h)-Tx}&=\norm{\sum \alpha_i T e_i}\\
  &\leq \sum \abso{\alpha _i} \norm{Te_i}\\
  &\leq  \norm{h}_1 (\max \norm{Te_i} )\\
  &\leq M \norm{h}(\max \norm{Te_i})=\epsilon \vdone
\end{align*}
\end{proof}
\begin{mdframed}
We now see that, because \customref{LOB}{Linear transformation is bounded if and only if it is continuous} and \customref{LmoF}{Linear map of finite-dimensional domain is always continuous}, if $\mathcal{X}$ is finite-dimensional, then all linear map of domain $\mathcal{X}$ are bounded. A counter example to the generalization of this statement is followed. 
\begin{Example}{\textbf{(Differentiation is an Unbounded Linear Operator)}}{}
\begin{align*}
\mathcal{X}=\Big(\R[x]|_{[0,1]}, \norm{\cdot}_\infty \Big), D(P)\triangleq P'
\end{align*}
Note that $\set{x^n}_{n\inn}$ is bounded in $\mathcal{X}$ and $\set{D(x^n)}_{n\inn}$ is not.   
\end{Example}
Now, suppose $\mathcal{X},\mathcal{Y}$ are two fixed normed spaces over $\R$ or $\C$. We can easily check that the set $BL(\mathcal{X},\mathcal{Y})$ of bounded linear operators from $\mathcal{X}$ to $\mathcal{Y}$ form a vector space over whichever field $\mathcal{Y}$ is over.\\

Naturally, our definition of boundedness of linear operator derive us a norm on $BL(\mathcal{X},\mathcal{Y})$, as followed 
\begin{align}
\label{tnop}
\norm{T}_{\text{op}}\triangleq \inf \set{M\inr^+ :\forall x \in \mathcal{X}, \norm{Tx}\leq M\norm{x}}
\end{align}
Before we show that our definition is indeed a norm, we first give some equivalent definitions and prove their equivalency. 
\end{mdframed}
\begin{theorem}
\textbf{(Equivalent Definitions of Operator Norm)} Given two fixed normed space $\mathcal{X},\mathcal{Y}$ over $\R$ or  $\C$, a bounded linear operator  $T:\mathcal{X}\rightarrow \mathcal{Y}$, and define $\norm{T}_{\text{op}}$ as in \myref{Equation}{tnop}, we have 
\begin{align*}
\norm{T}_{\text{op}}=\sup_{x\in \mathcal{X},x\neq 0} \frac{\norm{Tx}}{\norm{x}}
\end{align*}
\end{theorem}
\begin{proof}
Define $J\triangleq \set{M\inr^+:\forall x \in \mathcal{X},\norm{Tx}\leq M\norm{x}}$ and observe 
\begin{align*}
J&=\set{M\inr^+:M\geq \frac{\norm{Tx}}{\norm{x}},\forall x\neq 0\in \mathcal{X}}
\end{align*}
This let us conclude 
\begin{align*}
\sup_{x \in \mathcal{X},x\neq 0} \frac{\norm{Tx}}{\norm{x}}=\min  J= \norm{T}_{\text{op}}
\end{align*}
\end{proof}
\begin{mdframed}
It is now easy to see 
\begin{align}
  \norm{T}_{\text{op}}&=\sup_{x \in \mathcal{X},x\neq 0} \frac{\norm{Tx}}{\norm{x}}\label{equivdefopnorm1}\\
&=\sup_{x\in \mathcal{X},\norm{x}=1} \norm{Tx}\label{equivdefopnorm}
\end{align}
It is not all in vain to introduce the equivalent definitions. See that the verification of  $\norm{\cdot}_{\text{op}}$ being a norm on $BL(\mathcal{X},\mathcal{Y})$ become simple by utilizing the equivalent definitions. 
\begin{enumerate}[label=(\alph*)]
  \item For positive-definiteness, fix non-trivial $T$ and fix $x\in \mathcal{X}\setminus N(T)$. Use \myref{Equation}{equivdefopnorm1} to show $\norm{T}_{\text{op}}\geq \frac{\norm{Tx}}{\norm{x}}>0$. 
  \item For absolute homogenity, use \myref{Equation}{equivdefopnorm} and $\norm{Tcx}=\abso{c}\cdot \norm{Tx}$.
  \item For triangle inequality, use \myref{Equation}{equivdefopnorm} and $\norm{(T_1+T_2)x}\leq \norm{T_1x}+\norm{T_2x}$. 
\end{enumerate}
Naturally, and very very importantly, \myref{Equation}{equivdefopnorm1} give us 
\begin{align*}
\hspace{3cm}\norm{Tx}\leq \norm{T}_\text{op}\cdot \norm{x}\hspace{1.5cm}(x\in \mathcal{X})
\end{align*}
This inequality will later be the best tool to help analyze the derivatives of functions between Euclidean spaces, and perhaps better, it immediately give us 
\begin{align*}
 \frac{\norm{T_1T_2x}}{\norm{x}}\leq \frac{\norm{T_1}_\text{op}\cdot \norm{T_2}_\text{op}\cdot \norm{x}}{\norm{x}}=\norm{T_1}_\text{op}\cdot \norm{T_2}_{\text{op}}
\end{align*}
Then \myref{Equation}{equivdefopnorm1} give us  
\begin{align*}
\norm{T_1T_2}_\text{op}\leq \norm{T_1}_\text{op}\cdot \norm{T_2}_\text{op}
\end{align*}
\end{mdframed}
\section{Cauchy-Schwarz for Positive semi-definite Hermitian form}
\begin{mdframed}
Sometimes, we do not require $\langle \cdot,\cdot\rangle :V^2\rightarrow \F$ to be positive-definite, but only \textbf{positive semi-definite}, i.e. $\langle v,v\rangle \geq 0$ for all $v\in V$. In this case, we say $\langle \cdot, \cdot\rangle :V^2\rightarrow \F$ is a \textbf{positive semi-definite Hermitian form}. If we again induce a functional $\norm{\cdot}:V\rightarrow \R$ from some positive semi-definite Hermitian form using \myref{Equation}{indn}, then the functional $\norm{\cdot}:V\rightarrow \R$ in general is not positive-definite, and is called a \textbf{semi norm}, since, as we later show, it still satisfies triangle inequality and absolute homogeneity.
\end{mdframed}
\begin{theorem}
\label{BPoP}
\textbf{(Basic Property of Positive semi-definite Hermitian form)} Given a positive semi-definite Hermitian form $\langle \cdot,\cdot\rangle:V^2 \rightarrow \F$ and $x,y \in V$, we have 
\begin{align*}
\langle x,x\rangle =0 \implies \langle x,y\rangle =0
\end{align*}
\end{theorem}
\begin{proof}
\As{$\langle x,y\rangle\neq 0 $}. Fix $t> \frac{\norm{y}^2 }{2\abso{\langle x,y\rangle }^2}$. Compute 
\begin{align*}
\norm{y-t\langle y,x\rangle x}^2 &= \norm{y}^2 + \norm{(-t)\langle y,x\rangle x }^2  + \langle -t\langle y,x\rangle x,y \rangle + \langle y,-t\langle y,x\rangle x\rangle \\
&=\norm{y}^2 + t^2 \abso{\langle x,y\rangle }^2 \norm{x}^2 -t \langle y,x\rangle \langle x,y\rangle - t \langle x,y\rangle \langle y,x\rangle  \\
&= \norm{y}^2- 2t \abso{\langle x,y\rangle }^2 <0 \tCaC
\end{align*}
\end{proof}
\begin{theorem}
\label{CSI}
\textbf{(Cauchy-Schwarz Inequality)} Given a positive semi-definite Hermitian form $\langle \cdot,\cdot\rangle :V^2\rightarrow \C$ on vector space $V$ over $\C$, we have
\begin{align*}
\abso{\langle x,y\rangle }\leq \norm{x}\cdot \norm{y}
\end{align*}
Moreover, the equality hold true if $x,y$ are linearly independent. In addition, if $\langle \cdot,\cdot\rangle :V^2\rightarrow \F$ is an inner product, then the equality hold true only if $x,y$ are linearly independent.   
\end{theorem}
\begin{proof}
We first prove 
\begin{align*}
  \vi{\hspace{3cm}\abso{\langle x,y\rangle }\leq \norm{x}\cdot\norm{y}\hspace{1.5cm}(x,y \in V)}
\end{align*}
Fix $x,y \in V$. \myref{Theorem}{BPoP} tell us $\norm{x}=0 \implies  \langle x,y\rangle =0$. Then we can reduce the problem into proving 
\begin{align*}
\vi{\frac{\abso{\langle x,y\rangle }^2}{\norm{x}^2}\leq\norm{y}^2  }
\end{align*}
Set $z\triangleq y-\frac{\langle y,x\rangle }{\norm{x}^2}x$. We then have 
\begin{align*}
\langle z,x\rangle =\langle y-\frac{\langle y,x\rangle }{\norm{x}^2}x,x \rangle =\langle y,x\rangle - \frac{\langle y,x\rangle }{\norm{x}^2}\langle x,x\rangle =0
\end{align*}
Then from $y=z+\frac{\langle y,x\rangle }{\norm{x}^2}x$, we can now deduce
\begin{align*}
\langle y,y\rangle &= \langle z+\frac{\langle y,x\rangle }{\norm{x}^2}x,z+ \frac{\langle y,x\rangle }{\norm{x}^2}x\rangle  \\
&=\langle z,z\rangle + \abso{\frac{\langle y,x\rangle }{\langle x,x\rangle }}^2 \langle x,x\rangle \\
&=\langle z,z\rangle + \frac{\abso{\langle x,y\rangle }^2}{\langle x,x\rangle }
\end{align*}
Because $\langle z,z\rangle\geq 0 $, we now have
\begin{align*}
\langle y,y\rangle = \langle z,z\rangle + \frac{\abso{\langle x,y\rangle }^2}{\langle x,x\rangle }\geq  \frac{\abso{\langle x,y\rangle }^2}{\langle x,x\rangle }\vdone
\end{align*}
The equality hold true if and only if $\langle z,z\rangle =0$. This explains the other two statements regarding the equality. 
\end{proof}
\begin{mdframed}
Now, with Cauchy-Schwarz Inequality, we can check the triangle inequality 
\begin{align*}
  \norm{x+y}^2&=\langle x+y,x+y\rangle \\
&=\langle x,x\rangle +\langle x,y\rangle +\langle y,x\rangle +\langle y,y\rangle \\
&= \langle x,x\rangle + \langle y,y\rangle +2\text{ Re }\langle x,y\rangle \\
&\leq \norm{x}^2 + \norm{y}^2 + 2\abso{\langle x,y\rangle }\\
&\leq \norm{x}^2 + \norm{y}^2 + 2\norm{x}\cdot \norm{y}=\Big(\norm{x}+\norm{y}\Big)^2
\end{align*}
indeed holds true for space equipped with only positive semi-definite Hermitian form. 
\end{mdframed}
\end{document}
