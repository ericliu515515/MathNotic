\documentclass{report}
%%%%%%%%%%%%%% preamble.tex %%%%%%%%%%%%%%
\usepackage[T1]{fontenc}
\usepackage{etoolbox}
% Page Setup
\usepackage[letterpaper, tmargin=2cm, rmargin=0.5in, lmargin=0.5in, bmargin=80pt, footskip=.2in]{geometry}
\usepackage{adjustbox}
\usepackage{graphicx}
\usepackage{tikz}
\usepackage{mathrsfs}
\usepackage{mdframed}

% Create a new toggle
\newtoggle{firstsection}

% Redefine the \chapter command to reset the toggle for each new chapter
\let\oldchapter\chapter
\renewcommand{\chapter}{\toggletrue{firstsection}\oldchapter}

% Redefine the \section command to check the toggle
\let\oldsection\section
\renewcommand{\section}{
    \iftoggle{firstsection}
    {\togglefalse{firstsection}} % If it's the first section, just switch off the toggle for next sections
    {\clearpage} % If it's not the first section, start a new page
    \oldsection
}

% Abstract Design

\usepackage{lipsum}

\renewenvironment{abstract}
 {% Start of environment
  \quotation
  \small
  \noindent
  \rule{\linewidth}{.5pt} % Draw the rule to match the linewidth
  \par\smallskip
  {\centering\bfseries\abstractname\par}\medskip
 }
 {% End of environment
  \par\noindent
  \rule{\linewidth}{.5pt} % Ensure the closing rule also matches
  \endquotation
 }

% Mathematics
\usepackage{amsmath,amsfonts,amsthm,amssymb,mathtools}
\usepackage{xfrac}
\usepackage[makeroom]{cancel}
\usepackage{enumitem}
\usepackage{nameref}
\usepackage{multicol,array}
\usepackage{tikz-cd}
\usepackage{array}
\usepackage{multirow}% http://ctan.org/pkg/multirow
\usepackage{graphicx}

% Colors
\usepackage[dvipsnames]{xcolor}
\definecolor{myg}{RGB}{56, 140, 70}
\definecolor{myb}{RGB}{45, 111, 177}
\definecolor{myr}{RGB}{199, 68, 64}
% Define more colors here...
\definecolor{olive}{HTML}{6B8E23}
\definecolor{orange}{HTML}{CC5500}
\definecolor{brown}{HTML}{8B4513}
% Hyperlinks
\usepackage{bookmark}
\usepackage[colorlinks=true,linkcolor=blue,urlcolor=blue,citecolor=blue,anchorcolor=blue]{hyperref}
\usepackage{xcolor}
\hypersetup{
    colorlinks,
    linkcolor={red!50!black},
    citecolor={blue!50!black},
    urlcolor={blue!80!black}
}

% Text-related
\usepackage{blindtext}
\usepackage{fontsize}
\changefontsize[14]{14}
\setlength{\parindent}{0pt}
\linespread{1.2}

% Theorems and Definitions
\usepackage{amsthm}
\renewcommand\qedsymbol{$\blacksquare$}

% Define a new theorem style
\newtheoremstyle{mytheoremstyle}% name
  {}% Space above
  {}% Space below
  {}% Body font
  {}% Indent amount
  {\bfseries}% Theorem head font
  {.}% Punctuation after theorem head
  {.5em}% Space after theorem head
  {}% Theorem head spec (can be left empty, meaning ‘normal’)

% Apply the new theorem style to theorem-like environments
\theoremstyle{mytheoremstyle}

\newtheorem{theorem}{Theorem}[section]  
\newtheorem{definition}[theorem]{Definition} 
\newtheorem{lemma}[theorem]{Lemma}  
\newtheorem{corollary}[theorem]{Corollary}
\newtheorem{axiom}[theorem]{Axiom}
\newtheorem{example}[theorem]{Example}
\newtheorem{equiv_def}[theorem]{Equivalent Definition}

% tcolorbox Setup
\usepackage[most,many,breakable]{tcolorbox}
\tcbuselibrary{theorems}

% Define custom tcolorbox environments here...

%================================
% EXAMPLE BOX
%================================
% After you have defined the style and other theorem environments
\definecolor{myexamplebg}{RGB}{245, 245, 245} % Very light grey for background
\definecolor{myexamplefr}{RGB}{120, 120, 120} % Medium grey for frame
\definecolor{myexampleti}{RGB}{60, 60, 60}    % Darker grey for title

\newtcbtheorem[]{Example}{Example}{
    colback=myexamplebg,
    breakable,
    colframe=myexamplefr,
    coltitle=myexampleti,
    boxrule=1pt,
    sharp corners,
    detach title,
    before upper=\tcbtitle\par\vspace{-20pt}, % Reduced the space after the title
    fonttitle=\bfseries,
    description font=\mdseries,
    separator sign none,
    description delimiters={}{}, % No delimiters around the title
}{ex}
%================================
% Solution BOX
%================================
\makeatletter
\newtcolorbox{solution}{enhanced,
	breakable,
	colback=white,
	colframe=myg!80!black,
	attach boxed title to top left={yshift*=-\tcboxedtitleheight},
	title=Solution,
	boxed title size=title,
	boxed title style={%
			sharp corners,
			rounded corners=northwest,
			colback=tcbcolframe,
			boxrule=0pt,
		},
	underlay boxed title={%
			\path[fill=tcbcolframe] (title.south west)--(title.south east)
			to[out=0, in=180] ([xshift=5mm]title.east)--
			(title.center-|frame.east)
			[rounded corners=\kvtcb@arc] |-
			(frame.north) -| cycle;
		},
}
\makeatother

% %================================
% % Question BOX
% %================================
\makeatletter
\newtcbtheorem{question}{Question}{enhanced,
	breakable,
	colback=white,
	colframe=myb!80!black,
	attach boxed title to top left={yshift*=-\tcboxedtitleheight},
	fonttitle=\bfseries,
	title={#2},
	boxed title size=title,
	boxed title style={%
			sharp corners,
			rounded corners=northwest,
			colback=tcbcolframe,
			boxrule=0pt,
		},
	underlay boxed title={%
			\path[fill=tcbcolframe] (title.south west)--(title.south east)
			to[out=0, in=180] ([xshift=5mm]title.east)--
			(title.center-|frame.east)
			[rounded corners=\kvtcb@arc] |-
			(frame.north) -| cycle;
		},
	#1
}{question}
\makeatother

%%%%%%%%%%%%%%%%%%%%%%%%%%%%%%%%%%%%%%%%%%%
% TABLE OF CONTENTS
%%%%%%%%%%%%%%%%%%%%%%%%%%%%%%%%%%%%%%%%%%%


\usepackage{tikz}
\definecolor{doc}{RGB}{0,60,110}
\usepackage{titletoc}
\contentsmargin{0cm}
\titlecontents{chapter}[14pc]
{\addvspace{30pt}%
	\begin{tikzpicture}[remember picture, overlay]%
		\draw[fill=doc!60,draw=doc!60] (-7,-.1) rectangle (-0.9,.5);%
		\pgftext[left,x=-5.5cm,y=0.2cm]{\color{white}\Large\sc\bfseries Chapter\ \thecontentslabel};%
	\end{tikzpicture}\color{doc!60}\large\sc\bfseries}%
{}
{}
{\;\titlerule\;\large\sc\bfseries Page \thecontentspage
	\begin{tikzpicture}[remember picture, overlay]
		\draw[fill=doc!60,draw=doc!60] (2pt,0) rectangle (4,0.1pt);
	\end{tikzpicture}}%
\titlecontents{section}[3.7pc]
{\addvspace{2pt}}
{\contentslabel[\thecontentslabel]{3pc}}
{}
{\hfill\small \thecontentspage}
[]
\titlecontents*{subsection}[3.7pc]
{\addvspace{-1pt}\small}
{}
{}
{\ --- \small\thecontentspage}
[ \textbullet\ ][]

\makeatletter
\renewcommand{\tableofcontents}{
	\chapter*{%
	  \vspace*{-20\p@}%
	  \begin{tikzpicture}[remember picture, overlay]%
		  \pgftext[right,x=15cm,y=0.2cm]{\color{doc!60}\Huge\sc\bfseries \contentsname};%
		  \draw[fill=doc!60,draw=doc!60] (13,-.75) rectangle (20,1);%
		  \clip (13,-.75) rectangle (20,1);
		  \pgftext[right,x=15cm,y=0.2cm]{\color{white}\Huge\sc\bfseries \contentsname};%
	  \end{tikzpicture}}%
	\@starttoc{toc}}
\makeatother

\newcommand{\liff}{\llap{$\iff$}}
\newcommand{\rap}[1]{\rrap{\text{ (#1)}}}
\newcommand{\red}[1]{\textcolor{red}{#1}}
\newcommand{\blue}[1]{\textcolor{blue}{#1}}
\newcommand{\vi}[1]{\textcolor{violet}{#1}}
\newcommand{\olive}[1]{\textcolor{olive}{#1}}
\newcommand{\teal}[1]{\textcolor{teal}{#1}}
\newcommand{\brown}[1]{\textcolor{brown}{#1}}
\newcommand{\orange}[1]{\textcolor{orange}{#1}}
\newcommand{\tCaC}{\text{ \CaC }}
\newcommand{\CaC}{\red{CaC} }
\newcommand{\As}[1]{Assume \red{#1}}
\newcommand{\vdone}{\vi{\text{ (done) }}}
\newcommand{\bdone}{\blue{\text{ (done) }}}
\newcommand{\tdone}{\teal{\text{ (done) }}}
\newcommand{\odone}{\olive{\text{ (done) }}}
\newcommand{\bodone}{\brown{\text{ (done) }}}
\newcommand{\ordone}{\orange{\text{ (done) }}}
\newcommand{\ld}{\lambda}
\newcommand{\vecta}[1]{\textbf{#1}}
\newcommand{\set}[1]{\left\{ #1 \right\}}
\newcommand{\bset}[1]{\Big\{ #1 \Big\}}
\newcommand{\inR}{\in\R}
\newcommand{\inn}{\in\N}
\newcommand{\inz}{\in\Z}
\newcommand{\inr}{\in\R}
\newcommand{\inc}{\in\C}
\newcommand{\inq}{\in\Q}
\newcommand{\norm}[1]{\| #1 \|}
\newcommand{\bnorm}[1]{\Big\| #1 \Big\|}
\newcommand{\gen}[1]{\langle #1 \rangle}
\newcommand{\abso}[1]{\left|#1\right|}
\newcommand{\myref}[2]{\hyperref[#2]{#1\ \ref*{#2}}}
\newcommand{\customref}[2]{\hyperref[#1]{#2}}
\newcommand{\power}[1]{\mathcal{P}(#1)}
\newcommand{\dcup}{\mathbin{\dot{\cup}}}
\newcommand{\diam}[1]{\text{diam}\, #1}
\newcommand{\at}{\Big|}
\newcommand{\quotient}{\diagup}
\let\originalphi\phi % Store the original \phi in \originalphi
\renewcommand{\phi}{\varphi} % Redefine \phi to \varphi
\newcommand{\pfi}{\originalphi} % Define \pfi to display the original \phi
\newcommand{\diota}{\dot{\iota}}
\newcommand{\Log}{\operatorname{Log}}
\newcommand{\id}{\text{\textbf{id}}}
\usepackage{amsmath}

\makeatletter
\NewDocumentCommand{\extp}{e{^}}{%
  \mathop{\mathpalette\extp@{#1}}\nolimits
}
\NewDocumentCommand{\extp@}{mm}{%
  \bigwedge\nolimits\IfValueT{#2}{^{\extp@@{#1}#2}}%
  \IfValueT{#1}{\kern-2\scriptspace\nonscript\kern2\scriptspace}%
}
\newcommand{\extp@@}[1]{%
  \mkern
    \ifx#1\displaystyle-1.8\else
    \ifx#1\textstyle-1\else
    \ifx#1\scriptstyle-1\else
    -0.5\fi\fi\fi
  \thinmuskip
}
\makeatletter
\usepackage{pifont}
\makeatletter
\newcommand\Pimathsymbol[3][\mathord]{%
  #1{\@Pimathsymbol{#2}{#3}}}
\def\@Pimathsymbol#1#2{\mathchoice
  {\@Pim@thsymbol{#1}{#2}\tf@size}
  {\@Pim@thsymbol{#1}{#2}\tf@size}
  {\@Pim@thsymbol{#1}{#2}\sf@size}
  {\@Pim@thsymbol{#1}{#2}\ssf@size}}
\def\@Pim@thsymbol#1#2#3{%
  \mbox{\fontsize{#3}{#3}\Pisymbol{#1}{#2}}}
\makeatother
% the next two lines are needed to avoid LaTeX substituting upright from another font
\input{utxmia.fd}
\DeclareFontShape{U}{txmia}{m}{n}{<->ssub * txmia/m/it}{}
% you may also want
\DeclareFontShape{U}{txmia}{bx}{n}{<->ssub * txmia/bx/it}{}
% just in case
%\DeclareFontShape{U}{txmia}{l}{n}{<->ssub * txmia/l/it}{}
%\DeclareFontShape{U}{txmia}{b}{n}{<->ssub * txmia/b/it}{}
% plus info from Alan Munn at https://tex.stackexchange.com/questions/290165/how-do-i-get-a-nicer-lambda?noredirect=1#comment702120_290165
\newcommand{\pilambdaup}{\Pimathsymbol[\mathord]{txmia}{21}}
\renewcommand{\lambda}{\pilambdaup}
\renewcommand{\tilde}{\widetilde}
\DeclareMathOperator*{\esssup}{ess\,sup}
\newcommand{\bluecheck}{}%
\DeclareRobustCommand{\bluecheck}{%
  \tikz\fill[scale=0.4, color=blue]
  (0,.35) -- (.25,0) -- (1,.7) -- (.25,.15) -- cycle;%
}


\usepackage{tikz}
\newcommand*{\DashedArrow}[1][]{\mathbin{\tikz [baseline=-0.25ex,-latex, dashed,#1] \draw [#1] (0pt,0.5ex) -- (1.3em,0.5ex);}}

\newcommand{\C}{\mathbb{C}}	
\newcommand{\F}{\mathbb{F}}
\newcommand{\N}{\mathbb{N}}
\newcommand{\Q}{\mathbb{Q}}
\newcommand{\R}{\mathbb{R}}
\newcommand{\Z}{\mathbb{Z}}



\title{\raggedright\LARGE{Carathéodory's Extension Theorem}\\
       \large{Online note: new learning experience}}
\author{Authour: Eric Liu}
\date{}
\begin{document}
\maketitle
\newpage% or \cleardoublepage
% \pdfbookmark[<level>]{<title>}{<dest>}
\pdfbookmark[section]{\contentsname}{toc}

\tableofcontents
\pagebreak

\chapter{Text}
\section{Introduction}
\begin{mdframed}
This short note serves as an initial demonstration of a new learning format designed to improve how mathematics is studied, particularly by today’s college students, who increasingly rely on electronic devices rather than traditional textbooks. While the content is currently limited, it showcases the direction I’m aiming to take: a more interactive and accessible approach to learning mathematics.\\

Traditional mathematics textbooks often face the challenge of balancing generality, systematic structure, and economic considerations with providing clear motivation for the subject matter. In trying to strike this balance, authors are inevitably forced to compromise on some aspect. Regardless of the approach they take, something must be sacrificed due to the way readers naturally engage with such books.\\

This interactive PDF note addresses the challenges traditional mathematics textbooks often face. Instead of requiring readers to follow a linear progression, starting from a specific chapter, it allows them to choose their own path. Readers can jump directly to sections of interest without concern for unfamiliar notations or missing prerequisites, as these are clearly mentioned. When a prerequisite is referenced, a hyperlink is provided, guiding readers to concise explanations of the necessary theorems and propositions. This encourages a more flexible, self-directed learning experience, where readers can easily revisit foundational concepts whenever needed.\\

I will now explain further about the use of hyperlinks. The display format of the hyperlink typically appears as follows:
\begin{align*}
\customref{RGbS}{\text{This}}
\end{align*}
By clicking the brown text above, you will be directed to Theorem 1.2.1.\\

Another notable feature of this note is its emphasis on the logical connections between theorems, propositions, and the steps in proofs. We use color to highlight each claim we aim to prove, as well as the steps taken to reduce the claim. This is a feature that traditional textbooks are unable to offer.\\

 Though brief for now, this note offers a glimpse into the development of a more student-friendly resource that makes learning mathematics clearer, more engaging, and free from the financial burden of expensive textbooks.
\end{mdframed}
\section{Sigma-Algebra} 
\begin{abstract}
In this section, we first discuss properties of $\sigma$-algebra and some of its substructure for better understanding of a slightly generalized version of \customref{caratheodory_extension_theorem}{Carathéodory's extension theorem}. Note that in this section, the terms 'ring', 'field', or 'algebra' do not refer to algebraic structures like the integer ring. 
\end{abstract}
\begin{mdframed}
Given a set $X$ and an non-empty set $R$ of subsets of $X$, we say  $R$ is a \textbf{semi-ring}, if for each $A,B \in R$, we have 
\begin{enumerate}[label=(\alph*)]
  \item $A\cap B \in R$ (closed under finite intersection)
  \item  $A\setminus B=\bigsqcup_{i=1}^n K_i$ for some disjoint $K_1,\dots ,K_n \in R$. (relative complements can be written as finite disjoint union)
\end{enumerate}
and we say $R$ is a \textbf{ring}, if for each $A,B \in R$, we have 
\begin{enumerate}[label=(\alph*)]
  \item $A\cup B \in R$ (closed under finite union)
  \item $A\setminus  B \in R$ (closed under relative complement)
\end{enumerate}
One should check 
\begin{enumerate}[label=(\alph*)]
\label{c_ring}
  \item Semi-ring always contain the empty set.
\item Since $A\cap B=A\setminus (A\setminus B)$, \underline{closure under relative complement} implies \underline{closure under} \underline{finite intersection}. Thus, a ring, or any collection closed under relative complement, is always a semi-ring.
\item Note that $A\cup B=(A\setminus B)\sqcup (A\cap B) \sqcup  (B\setminus A)$. This implies we can replace  \underline{closure under finite union} with \underline{closure under finite disjoint union} as definition for ring. 
  \item Given a family $S$ of subsets of $X$, there exists smallest ring  $R(S)$ containing $S$. Such ring $R(S)$ is called \textbf{the ring generated by $S$}. 
\end{enumerate}
\end{mdframed}
\begin{theorem}
\label{RGbS}
\textbf{(Ring Generated by Semi-Ring)} If $S$ is a semi-ring, then 
\begin{align*}
R(S)&=\set{A:A\text{ is the union of some finite pair-wise disjoint sub-family $S'$ of $S $}}\\
&=\set{A:A\text{ is the union of some finite sub-family $S'$ of $S$ }}
\end{align*}
\end{theorem}
\begin{proof}
  Let $R\triangleq \set{A:A\text{ is the union of some finite pair-wise disjoint sub-family $S'$ of $S$}}$. $R'\triangleq \set{A:A\text{ is the union of some finite sub-family $S'$ of $S$ }}$. We first prove  \vi{$R=R(S)$}.\\


  Clearly, the problem can be reduced into proving \vi{$R$ is a ring}. Because $R$ is clearly closed under finite disjoint union, by \customref{c_ring}{property (c) of the ring}, we can reduce the problem into proving \vi{$R$ is closed under relative complement}.\\

We first show \olive{$R$ is closed under finite intersection}. Given $\bigsqcup E_i,\bigsqcup F_j \in R$, we see 
\begin{align*}
  \Big(\bigsqcup_i E_i\Big)\cap \Big(\bigsqcup_j F_j\Big)=\bigsqcup_{i,j} E_i\cap F_j \in R \odone
\end{align*}

Now observe
\begin{align*}
\Big(\bigsqcup_i E_i \Big)\setminus \Big(\bigsqcup_j F_j \Big)&=\bigcap_j \Big(\bigsqcup_i (E_i\setminus F_j) \Big)\\
&=\bigcap_j A_j  \text{ for some $A_j \in R$ }\vdone
\end{align*}
We now prove \blue{$R=R'$}. It is clear that $R\subseteq R'$. We only have to prove  \blue{$R'\subseteq R$}. This is trivially true, since any finite sub-family $S'$ of $S$ is a finite sub-family of $R$ and  $R$ is closed under finite union. $\bdone$
\end{proof}
\begin{mdframed}
We now give definition to the most important structure in this section. Given a family $\Sigma$ of subsets of $X$,  we say $\Sigma$ is a \textbf{$\sigma$-algebra} (or sometimes \textbf{$\sigma$-field}) on $X$ and say $(X,\Sigma)$ is a \textbf{measurable space}, if $\Sigma$ is ring and for each sequence $(A_n)_{n\inn}$ of elements of $\Sigma$ we have
\begin{enumerate}[label=(\alph*)]
  \item $X \in \Sigma$
  \item  $\bigcup_{n\inn} A_n \in \Sigma$ (Closed under countable union)
\end{enumerate}
Similarly, one should check 
\begin{enumerate}[label=(\alph*)]
\label{pop_sig}
\item Because $\bigcap_{n=1}^{\infty}A_n = A_1 \setminus (\bigcup_{n=2}^{\infty}(A_1\setminus A_n))$, we see  $\sigma$-algebra is \underline{closed under countable} \underline{intersection}. 
\item Because $\bigcup_{n=1}^{\infty}A_n = \bigsqcup_{n=1}^{\infty} (A_n \setminus (\bigcup_{k=n+1}^{\infty} A_k))$, we can replace \underline{closure under countable} \underline{union} with \underline{closure under countable disjoint union}.  
\item Because $\bigcup_{n\inn}A_n = \bigcup_{n\inn} (\bigcup_{k=1}^n A_k)$, we can replace \underline{closure under countable union} with the condition \underline{for all  $(A_n)\subseteq \Sigma$ such that $\forall n,A_n \subseteq A_{n+1}$, we have $\bigcup_n A_n \in \Sigma$}.
\item Given a family $S$ of subsets of $X$, there exists smallest sigma-algebra $\sigma(S)$ containing $S$. Such $\sigma$-algebra $\sigma(S)$ is called \textbf{the sigma-algebra generated by $S$}.
\end{enumerate}
Now, given a family $S$ of subsets of $X$, there in fact exists an explicit expression of $\sigma (S)$, albeit infamous. Let $\omega_1$ be the smallest uncountable ordinal, and let $\mathbf{\Sigma}^0_1\triangleq S$. For each ordinal $\alpha <\omega_1$, we recursively define $\mathbf{\Pi}^0_\alpha \triangleq  \set{X\setminus A: A \in \mathbf{\Sigma}^0_\alpha}$ and $\mathbf{\Sigma}^0_\alpha \triangleq \set{\bigcup_{n\inn}A_n: (A_n)\subseteq \bigcup_{1\leq \gamma <\alpha }\mathbf{\Pi}^0_\gamma }$. One can use transfinite induction to check that $\sigma (S)= \bigcup_{\alpha <\omega_1}\mathbf{\Sigma}^0_\alpha$.
\end{mdframed}
\section{Carathéodory's Extension Theorem}
\label{caratheodory_extension_theorem}
\begin{abstract}
In this section, we introduce a general process to construct measure on $X$. The process involve first inducing an outer measure from a pre-measure on some weaker structure $S$, and then restricting the outer measure onto a subfamily that contain exactly all the subset that "sharply cut" all the other subsets of $X$. The subfamily, as we shall prove, is a sigma-algebra. This extending-restricting process is known mostly by the name of Carathéodory Extension Theorem, and the rigorous  definition of "sharply cut" is known by the name of Carathéodory criterion. The selected 'some weaker structure' $S$, which we begin out extension from, is a semi-ring. Although as long as $S$ contain the empty set, the process works, in the sense that one can generate a measure with pre-measure $\mu:S\rightarrow [0,\infty]$, to have the generated measure agree with  $\mu$ on $S$, some necessary condition needs to be satisfied, and the axioms of semi-ring, not equivalent to some other popular choices that also suffice, e.g., ring or quasi-semi-ring, suffices to be a set of necessary conditions.\\

Note that in this section, if we write $\mu (A)$ without specifying whether $A$ is in the domain of  $\mu$, we mean that the statement always hold true as long as $A$ is in the domain of $\mu$, and note that the difference between the term \textbf{measure space} and measurable space lies in that the latter is not equipped with a measure yet. 
\end{abstract}
\begin{mdframed}
Given a collection $S$ of subsets of  $X$ containing the empty set, we say  $\mu:S\rightarrow [0,\infty]$ is a \textbf{pre-measure} (or \textbf{content}) on $(X,S)$ if 
\begin{enumerate}[label=(\alph*)] 
  \item $\mu(\varnothing)=0$ (null empty set) 
  \item $A \subseteq B \implies \mu (A) \leq \mu (B)$ (monotone)
  \item $\mu\big(\bigsqcup_{n\inn} E_n \big)=\sum_{n\inn} \mu (E_n)$ (countably additive, or $\sigma$-additive)
\end{enumerate}
and say $\mu$ is a \textbf{measure} if $S$ is a $\sigma$-algebra on $X$. Note that if $S$ is closed under relative complement, e.g., $S$ is a ring or any stronger structure, then monotone is implied by countable additive. Now, with the hints given below, one can check straightforward that if $S$ is a semi-ring, we have 
\begin{enumerate}[label=(\alph*)]
\label{pop_meas}
  \item $\mu (A_1 \sqcup   \cdots \sqcup   A_n)= \mu (A_1) + \cdots + \mu (A_n)$ (finitely additive)
  \item $\mu \big(\bigcup_{n \in J} A_n\big)\leq \sum_{n\in J} \mu (A_n)$, for each finite or countable $J$.   \item $A_n \nearrow A \implies \mu (A_n)\nearrow \mu \big(A\big)$.  
  \item $A_n\searrow A\text{ and }\mu (A_1)<\infty \text{ and }S\text{ is a $\sigma$-algebra }\implies \mu (A_n)\searrow \mu \big(A\big)$  
\end{enumerate}
Hints: Properties (b) and (c) are proved by letting $B_n\triangleq A_n \setminus (A_{n-1}\cup \cdots \cup  A_1)$; and property (d) is proved by letting $B_n\triangleq A_1 \setminus A_n,\forall n\inn$ and the equation 
\begin{align*}
\mu (A_1)-\lim_{n\to \infty} \mu (A_n)&=\lim_{n\to \infty}\mu (A_1)-\mu (A_n)\\
&=\lim_{n\to \infty} \mu (B_n)\\
(\text{property (c) is used here})\hspace{0.5cm}&=\mu (\bigcup_{n=1}^{\infty}B_n)=\mu (A_1 \setminus \bigcap_{n=1}^{\infty}A_n)=\mu (A_1)- \mu (\bigcap_{n=1}^{\infty}A_n)
\end{align*}
Note that our proof for property (d) require $A_1$ to be of finite measure in last step. 
\end{mdframed}
\begin{mdframed}
Now, suppose $S$ is a semi-ring and $\mu:S\rightarrow [0,\infty]$ is a pre-measure on $(X,S)$. This section address the question: Is there a unique extension of $\mu$ onto $\sigma (S)$? The answer is indeed affirmative: The extension always exists, and if $\mu$ is $\sigma$-finite, the extension is unique. 
While extensions of pre-measures can be constructed from structures weaker than a semi-ring, we focus on semi-rings here, as they are the starting points in common applications, e.g., Lebesgue-Stieltjes measure.\\

We first extend $\mu$ from $S$ onto $R(S)$. Since each element of $R(S)$ is a union of some finite pair-wise disjoint sub-family of $S$, as \customref{RGbS}{we proved before}, for each $A=\bigsqcup_{j=1}^{n_a}A_j \in R(S)$, we can assign $\mu(A)\triangleq \sum_{j=1}^{n_a}\mu (A_j)$. Such assignment is well-defined, as one can check using  
\begin{align*}
\bigsqcup_{j=1}^{n_a}A_j = \bigsqcup _{k=1}^{n_b}B_k \implies A=\bigsqcup_{j,k} A_j \cap B_k \text{ where }A_j\cap B_k \in S 
\end{align*}
At this point, one should check $\mu$ remains countably additive after the extension. This can be proved using the same trick and the fact \customref{ACSUC}{absolutely convergent series unconditionally converge}.\\

We now give definition to \textbf{outer measure}, and shows that \customref{Piom}{given any pre-measure $\mu$ on some semi-ring $S$ of subsets of $X$, there exists outer measure  $\mu^*$ on $X$ such that  $\mu^*\text{ and }\mu$ agrees on $S$}.
\end{mdframed}
\begin{definition}
\textbf{(Definition of outer measure)} Given a set $X$, by an  \textbf{outer measure}, we mean a function $\nu :2^X\rightarrow [0,\infty]$ such that 
\begin{enumerate}[label=(\alph*)]
  \item $ \nu  (\varnothing)=0$ (null empty set)
  \item $A\subseteq \bigcup_{n\inn} A_n\implies \nu  (A)\leq  \sum \nu (A_n)$ (countably subadditive)
\end{enumerate}
\end{definition}
\begin{mdframed}
Equivalently, one can replace countably subadditive with the following two axioms   
\begin{enumerate}[label=(\alph*)]
  \item $A \subseteq B \implies \nu (A) \leq \nu  (B)$  (monotone)
  \item $\nu  (\bigcup_{n\inn} A_n )\leq \sum \nu (A_n)$  
\end{enumerate}

\end{mdframed}
\begin{theorem}
\label{Piom}
\textbf{(Pre-measure on semi-ring induces outer measure)} Given a pre-measure $\mu$ on some semi-ring $S$ of subsets of $X$, if we define $\mu^*:2^X\rightarrow [0,\infty]$ by
\begin{align*}
\mu^*(E)\triangleq \inf \bset{\sum_n \mu (T_n):E \subseteq \bigcup _n T_n\text{ and }T_1,T_2,\dots \in S}\text{ where $\inf \varnothing= \infty$ }
\end{align*}
Then 
\begin{align*}
\mu^*\text{ is an outer measure agreeing with $\mu$ on $S$}
\end{align*}
\end{theorem}
\begin{proof}
  It is clear $\mu^*(\varnothing)=0\text{ and }A\subseteq B \implies \mu^*(A)\leq \mu^*(B)$. It remains to prove that for arbitrary $B_n$ we have 
\begin{align*}
  \vi{\mu^* \Big(\bigcup_n B_n \Big)  \leq \sum_n \mu^* (B_n)}
\end{align*}
If  $\sum_n \mu^*(B_n)=\infty$, the proof is trivial. We from now suppose $\sum_n \mu^*(B_n)< \infty$. Fix $\epsilon $. We prove 
\begin{align*}
  \vi{\mu^*\Big(\bigcup _n B_n \Big)\leq \sum_n \mu^* (B_n)+\epsilon }
\end{align*}
Because $\mu^*(B_n)<\infty$ for each $n\inn$, we know for each $n\inn$ there exists an countable cover $(S_{n,k})_{k\inn}\subseteq R$ of $B_n$ such that 
\begin{align*}
\sum_k \mu (S_{n,k}) \leq \mu^*(B_n)+\epsilon (2^{-n}) 
\end{align*}
It is clear that  $\set{T_{n,k}:n,k \inn}$ is a countable cover of $\bigcup_n B_n$, we now see 
\begin{align*}
  \mu^*\Big(\bigcup_n B_n \Big)\leq \sum_{n,k} \mu (T_{n,k})&\leq \sum_n \sum_k \mu(T_{n,k})\\
&\leq \sum_n \mu^*(B_n)+ \epsilon (2^{-n})= \sum_n \mu^*(B_n)+ \epsilon \vdone
\end{align*}
Note that the expression $\sum_{n,k} \mu (T_{n,k})$ make sense because of the \customref{RRT}{Riemann Rearrangement Theorem}, and $\sum_{n,k}\mu (T_{n,k})\leq \sum_{n} \sum_{k} \mu (T_{n,k})$ should be proved using a limit argument with the diagonal enumeration on $(n,k)$. The fact $\mu^*(T)=\mu (T),\forall T \in S$ follows from \customref{pop_meas}{property (c) of measure} with the trick that, provided $(T_n)$ is a cover of $T$, $(T\cap \bigcup_{n=1}^N T_n)\nearrow T$ as $N\to \infty$ in the sense of set. Note that it is only here we use the hypothesis that $\mu$ is a pre-measure and $S$ is a semi-ring.   
\end{proof}
\begin{mdframed}
  So far, we have proved that given a semi-ring $S$ and a pre-measure $\mu:S\rightarrow [0,\infty]$, there exist some outer measure agree with $\mu$ on $S$. One may wish to ask if such outer measure, an extension of the pre-measure $\mu$, is unique? The answer is negative even in the most trivial case, as the \customref{Nou}{example below} shows. In fact, the outer measure induced in \myref{Theorem}{Piom} is called the \textbf{maximal outer extension}, in the sense that if $\nu$ is an outer measure agreeing with  $\mu$ on $S$, then  $\nu(E)\leq \mu^*(E),\forall E\subseteq X$, as one can check straightforwardly. Also, one can check that it make no difference if we first extend $\mu$ from $S$ onto  $R(S)$ or not, before we extend $\mu$ to the maximal outer measure $\mu^*$. 
\end{mdframed}
\begin{Example}{\textbf{(Non-uniqueness of outer extension)}}{}
\label{Nou}
\begin{align*}
X\triangleq \set{1,2}\text{ and }R\triangleq \set{\varnothing,X}
\end{align*}
Define the pre-measure $\mu:R\rightarrow [0,\infty]$ and an outer measure $\nu:\power{X}\rightarrow [0,\infty]$ agreeing with $\mu$ on $S$ by 
\begin{align*}
\mu (A)\triangleq \begin{cases}
  0& \text{ if $A=\varnothing$ }\\
  1& \text{ if $A=X$ }
\end{cases}\text{ and }\nu(A)\triangleq \begin{cases}
  \mu (A)& \text{ if $A\in R$ }\\
  \frac{1}{2}& \text{ if $A\not\in R$ }
\end{cases}
\end{align*}
One can check that the maximal outer extension $\mu^*$ disagree with $\nu$ on $\power{X}\setminus R$.
\end{Example}
\begin{mdframed}
It is important to note that the final step, \myref{Theorem}{Omi}, inducing measure from outer measure, is a general Theorem and operate independently of \myref{Theorem}{Piom}. This distinction is crucial because many constructions of measures, such as the Hausdorff measure, begin by defining an outer measure explicitly, rather than inducing it from a weaker structure like a semi-ring.
\end{mdframed}
\begin{theorem}
\label{Omi}
\textbf{(Outer measure induce measure)} Given an outer measure $\mu^*$ on $X$, if we let 
\begin{align*}
  \mathcal{A}\triangleq \set{A \subseteq X:\mu^*(E)=\mu^*(E\cap A)+ \mu^*(E\setminus A)\text{ for all $E\subseteq X$ }} 
\end{align*}
then $\mathcal{A}$ is a sigma-algebra on $X$ and  $\mu^*|_{\mathcal{A}}:\mathcal{A}\rightarrow [0,\infty]$ is a measure. 
\end{theorem}
\begin{proof}
Because of the following facts 
\begin{enumerate}[label=(\alph*)]
  \item $A\setminus B=A\cap B^c$
  \item $A\cup B=(A^c \cap B^c)^c$ 
  \item \customref{pop_sig}{property (c) of sigma-algebra}
\end{enumerate}
we can reduce the problem into proving the following propositions 
\begin{enumerate}[label=(\roman*)]
  \item $\mathcal{A}$ is closed under complement. 
  \item $X\in \mathcal{A}$.
  \item $\mathcal{A}$ is closed under finite intersection. 
  \item $\mu^*|_{\mathcal{A}}$ is countably additive. (Thus at least form a pre-measure)
  \item $\mathcal{A}$ is closed under countable disjoint union. 
\end{enumerate}
We will prove the propositions sequentially, as the proof of each subsequent proposition may rely on the proofs of the preceding ones. The first two are straightforward to check. We now prove $\vi{\mathcal{A}\text{ is closed under finite intersection}}$. Fix $A,B \in \mathcal{A}\text{ and }E\subseteq X$, we wish to show 
\begin{align*}
\vi{\mu^*(E)=\mu^*(E\cap A\cap B)+\mu^*\big(E\setminus (A\cap B)\big)}
\end{align*}
Because $B \in \mathcal{A}$, we can "sharply cut $E\setminus (A\cap B)$ by $B$"; that is 
 \begin{align}
\label{sharp}
\mu^*(E\setminus (A\cap B))=\mu^*\big((E\cap B)\setminus A\big)+\mu^*(E\setminus B)
\end{align}
\myref{Equation}{sharp} together with $A,B \in \mathcal{A}$ then give us 
\begin{align*}
\mu^*(E\cap A\cap B)+\mu^*\big(E\setminus (A\cap B)\big)&=\mu^*(E\cap A\cap B)+\mu^*\big((E\cap B)\setminus A \big)+ \mu^*(E\setminus B)\\
&=\mu^*(E\cap B)+\mu^*(E\setminus B)=\mu^*(E)\vdone
\end{align*}
We now prove the \olive{claim: For each pairwise disjoint sequence $(A_n)\subseteq \mathcal{A}$ and $E\subseteq X$, we have the equality} 
\begin{align*}
\olive{\mu^*\Big(E\cap \bigsqcup_n A_n \Big)=\sum_n \mu^*(E\cap A_n)}
\end{align*}
The countably subadditivty of $\mu^*$ trivially implies the inequality  
\begin{align*}
\mu^*\Big(E\cap \bigsqcup_n A_n\Big)\leq \sum_n \mu^*(E\cap A_n)
\end{align*}
Using induction and the fact $(A_n)\subseteq \mathcal{A}$, we see that 
\begin{align*}
\mu^*(E\cap \bigsqcup_n A_n)= \sum_{n=1}^N \mu^*(E\cap A_n) + \mu^*\Big(E\cap \bigsqcup_{n=N+1}^{\infty}A_n \Big)\text{ for all $N \inn$ }
\end{align*}
Then since $\mu^*$ has codomain $[0,\infty]$, we see 
\begin{align*}
\mu^*(E\cap \bigsqcup_n A_n)\geq \sum_{n=1}^N \mu^*(E\cap A_n)\text{ for all $N \inn$ }
\end{align*}
This implies the desired inequality $\mu^*(E\cap \bigsqcup_n A_n)\geq \sum_{n=1}^{\infty}\mu^*(E\cap A_n)$. $\odone$\\

Using $E\triangleq \bigsqcup_n A_n$, one see our \olive{claim} implies that $\mu^*|_{\mathcal{A}}$ is countably additive. Lastly, we prove $\blue{\mathcal{A}\text{ is closed under countable disjoint union}}$. Fix a pairwise disjoint sequence $(A_n)\subseteq \mathcal{A}$ and $E\subseteq X$. We wish to prove
\begin{align*}
\blue{\mu^*(E)\geq \mu^*(E\cap \bigsqcup_n A_n)+\mu^*(E\setminus \bigsqcup_n A_n)}
\end{align*}
 Using induction and the fact $(A_n)\subseteq \mathcal{A}$, we see that 
\begin{align*}
\mu^*(E\cap \bigsqcup_{n=1}^N A_n)=\sum_{n=1}^N \mu^*(E\cap A_n)\text{ for all $N\inn$ }
\end{align*}
Then our \olive{claim} give us 
\begin{align}
\label{mue}
\mu^*(E\cap \bigsqcup_{n=1}^NA_n) \to \mu^*(E\cap \bigsqcup_n A_n)\text{ as $N \to \infty$ }
\end{align}
Now, because of the identity $F\cup G=(F^c\cap G^c)^c$, proposition (i) and (iii) have shown $\mathcal{A}$ is closed under finite union. This implies $\bigsqcup_{n=1}^N A_n\in \mathcal{A}$ for all $N\inn$, which, together with monotone of $\mu^*$, give 
\begin{align}
  \mu^*(E\cap \bigsqcup_{n=1}^N A_n)+\mu^*(E\setminus \bigsqcup_n A_n)&\leq \mu^*(E\cap \bigsqcup_{n=1}^{N}A_n)+ \mu^*(E\setminus \bigsqcup_{n=1}^N A_n)\notag\\
&\leq \mu^*(E) \text{ for all $N\inn$ }\label{mue2}
\end{align}
\myref{Equation}{mue} and \myref{Equation}{mue2} gives the desired inequality. $\bdone$
\end{proof}
\begin{mdframed}
\myref{Theorem}{Piom} together with \myref{Theorem}{Omi} shows that for each pre-measure $\mu$ on semi-ring $S$, we can induce a measure $\mu^*|_{\mathcal{A}}$ agreeing with $\mu$ on $S$. Although this result is correct, it doesn't show $S \subseteq \mathcal{A}$, which is necessary to refer to $\mu^*|_{\mathcal{A}}$ as an extension. However, it is straightforward to verify that $S \subseteq \mathcal{A}$ using the definition of a semi-ring and the property that $\mu^*(T) = \mu(T)$ for all $T \in S$.\\

Moving forward, here are some additional concepts we will utilize in subsequent sections: Given a measurable space $(X,\Sigma,\mu)$, we define a measurable set $N \in \Sigma$ as a \textbf{null set} if $\mu(N) = 0$. Moreover, we say that $\mu$ is a \textbf{complete measure} if every subset of a null set is also measurable. It is important to note that every measure induced by an outer measure is complete, as one can readily verify.
\end{mdframed}
\begin{mdframed}
Lastly, we wish to ask: Given a sigma-algebra $\Sigma$ containing $S$ and contained by $\mathcal{A}$, under what condition, is Carathéodory the only extension of $\mu$ onto $\Sigma$ ? \\

This question turns out to have direct connection with the notion named '$\sigma$-finite'. Given a pre-measure space $(X,S,\mu)$, we say $\mu$ is \textbf{$\sigma$-finite} if there exists a countable cover $(A_n)\subseteq S$ of $E$ such that  $\mu (A_n)<\infty$ for all $n\inn$. It is clear that   \begin{enumerate}[label=(\alph*)]  
  \item $\mu$ is $\sigma$-finite only if $S$ form a cover of $X$. 
  \item If $\mu:S\rightarrow [0,\infty]$ is $\sigma$-finite and $\nu $ is a pre-measure defined on a class larger than $S$, such that $\nu $ agree with $\mu$ on $S$, then  $\nu $ is also $\sigma$-finite.
\end{enumerate}
\end{mdframed}
\begin{theorem}
\textbf{(Uniqueness of Extension)} Suppose 
\begin{enumerate}[label=(\alph*)]
  \item $(X,S,\mu)$ is a pre-measure space, and $S$ is a semi-ring.
  \item $\mathcal{A}$ is the induced sigma-algebra in \myref{Theorem}{Omi}  
  \item $\Sigma\subseteq \mathcal{A}$ is a sigma-algebra containing $S$  
  \item $\nu :\Sigma \rightarrow [0,\infty]$ is a measure agreeing with $\mu$ on $S$
\end{enumerate}
We have 
\begin{align}
\label{vmu1}
\nu (A)\leq \mu^*(A)\text{ for all $A\in \Sigma$ }
\end{align}
and, if $\mu$ is $\sigma$-finite, we have
\begin{align}
\label{vmu2}
\nu (A)=\mu^*(A)\text{ for all $A \in \Sigma$  }
\end{align}
\end{theorem}
\begin{proof}
The \myref{inequality}{vmu1} follows from the greatest-lower-bound definition of induced outer-measure,  \customref{pop_meas}{property (b) of measure} and monotone of measure.\\

From now on, we suppose $\mu$ is $\sigma$-finite. Before we prove \myref{Equation}{vmu2}, we first prove the \olive{claim: for each  $A\in \Sigma$, there exists a pairwise disjoint sequence $(D_n)\subseteq R(S)\subseteq \Sigma$ such that $A\subseteq \bigsqcup_n D_n$ and $\nu (D_n)=\mu^*(D_n)<\infty$ for all $n\inn$}.\\

Because $\mu$ is $\sigma$-finite, there exists a sequence $(A_n)\subseteq S$ such that $A\subseteq \bigcup_n A_n$ and $\mu (A_n)<\infty,\forall n\inn$. Define $D_1\triangleq A_1$ and  $D_n\triangleq A_n \setminus (A_1 \cup  \cdots \cup A_{n-1})$ for all  $n>2$. Noting the
\customref{RGbS}{structure of $R(S)$}, it is clear that $D_n$ is a pairwise disjoint sequence  in $R(S)$. It is also clear that $A\subseteq \bigcup A_n = \bigsqcup  D_n$. Fix $n\inn$. It remains to prove 
\begin{align*}
\olive{\nu  (D_n)=\mu^*(D_n)<\infty}
\end{align*}
The inequality $\mu^*(D_n)<\infty$ follows from $\mu^*(D_n)\leq \mu (A_n)<\infty$, and the equation $\nu (D_n)=\mu^*(D_n)$ follows from $R(S)\subseteq \Sigma$ and $R(S)\subseteq \mathcal{A}$. $\odone$\\  

Note that for all $n\inn$, $A\cap D_n \in \Sigma \subseteq \mathcal{A}$. Now, since 
\begin{align*}
\nu  (A)=\sum_{n=1}^{\infty} \nu (A\cap D_n)\text{ and }\mu^*(A)=\sum_{n=1}^{\infty} \mu^*(A\cap D_n)
\end{align*}
To prove \myref{Equation}{vmu2}, it only remains to prove $\vi{\nu  (A\cap D_n)=\mu^*(A\cap D_n),\forall n\inn}$.\\

Because $\nu $ is a measure and $A\cap D_n \in \mathcal{A}$, we have the equations set 
\begin{align}
\begin{cases}
  \nu  (A\cap D_n)=\nu  (D_n)-\nu  (D_n\setminus A)\\  
\mu^*(A\cap D_n)=\mu^*(D_n)-\mu^*(D_n\setminus A)
\end{cases}
\end{align}
The proof then follows from the facts 
\begin{enumerate}[label=(\alph*)]
  \item $\nu (D_n)=\mu^*(D_n)<\infty$ 
  \item $\nu  (A\cap D_n)\leq \mu^*(A\cap D_n)\text{ and }\nu (D_n\setminus A)\leq \mu^*(D_n\setminus A)$ $\vdone$
\end{enumerate}
Note that fact (b) can be checked straightforwardly. 
\end{proof}
\end{document}
