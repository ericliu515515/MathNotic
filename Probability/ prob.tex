\documentclass{report}

%%%%%%%%%%%%%% preamble.tex %%%%%%%%%%%%%%
\usepackage[T1]{fontenc}
\usepackage{etoolbox}
% Page Setup
\usepackage[letterpaper, tmargin=2cm, rmargin=0.5in, lmargin=0.5in, bmargin=80pt, footskip=.2in]{geometry}
\usepackage{adjustbox}
\usepackage{graphicx}
\usepackage{tikz}
\usepackage{mathrsfs}
\usepackage{mdframed}

% Create a new toggle
\newtoggle{firstsection}

% Redefine the \chapter command to reset the toggle for each new chapter
\let\oldchapter\chapter
\renewcommand{\chapter}{\toggletrue{firstsection}\oldchapter}

% Redefine the \section command to check the toggle
\let\oldsection\section
\renewcommand{\section}{
    \iftoggle{firstsection}
    {\togglefalse{firstsection}} % If it's the first section, just switch off the toggle for next sections
    {\clearpage} % If it's not the first section, start a new page
    \oldsection
}

% Abstract Design

\usepackage{lipsum}

\renewenvironment{abstract}
 {% Start of environment
  \quotation
  \small
  \noindent
  \rule{\linewidth}{.5pt} % Draw the rule to match the linewidth
  \par\smallskip
  {\centering\bfseries\abstractname\par}\medskip
 }
 {% End of environment
  \par\noindent
  \rule{\linewidth}{.5pt} % Ensure the closing rule also matches
  \endquotation
 }

% Mathematics
\usepackage{amsmath,amsfonts,amsthm,amssymb,mathtools}
\usepackage{xfrac}
\usepackage[makeroom]{cancel}
\usepackage{enumitem}
\usepackage{nameref}
\usepackage{multicol,array}
\usepackage{tikz-cd}
\usepackage{array}
\usepackage{multirow}% http://ctan.org/pkg/multirow
\usepackage{graphicx}

% Colors
\usepackage[dvipsnames]{xcolor}
\definecolor{myg}{RGB}{56, 140, 70}
\definecolor{myb}{RGB}{45, 111, 177}
\definecolor{myr}{RGB}{199, 68, 64}
% Define more colors here...
\definecolor{olive}{HTML}{6B8E23}
\definecolor{orange}{HTML}{CC5500}
\definecolor{brown}{HTML}{8B4513}
% Hyperlinks
\usepackage{bookmark}
\usepackage[colorlinks=true,linkcolor=blue,urlcolor=blue,citecolor=blue,anchorcolor=blue]{hyperref}
\usepackage{xcolor}
\hypersetup{
    colorlinks,
    linkcolor={red!50!black},
    citecolor={blue!50!black},
    urlcolor={blue!80!black}
}

% Text-related
\usepackage{blindtext}
\usepackage{fontsize}
\changefontsize[14]{14}
\setlength{\parindent}{0pt}
\linespread{1.2}

% Theorems and Definitions
\usepackage{amsthm}
\renewcommand\qedsymbol{$\blacksquare$}

% Define a new theorem style
\newtheoremstyle{mytheoremstyle}% name
  {}% Space above
  {}% Space below
  {}% Body font
  {}% Indent amount
  {\bfseries}% Theorem head font
  {.}% Punctuation after theorem head
  {.5em}% Space after theorem head
  {}% Theorem head spec (can be left empty, meaning ‘normal’)

% Apply the new theorem style to theorem-like environments
\theoremstyle{mytheoremstyle}

\newtheorem{theorem}{Theorem}[section]  
\newtheorem{definition}[theorem]{Definition} 
\newtheorem{lemma}[theorem]{Lemma}  
\newtheorem{corollary}[theorem]{Corollary}
\newtheorem{axiom}[theorem]{Axiom}
\newtheorem{example}[theorem]{Example}
\newtheorem{equiv_def}[theorem]{Equivalent Definition}

% tcolorbox Setup
\usepackage[most,many,breakable]{tcolorbox}
\tcbuselibrary{theorems}

% Define custom tcolorbox environments here...

%================================
% EXAMPLE BOX
%================================
% After you have defined the style and other theorem environments
\definecolor{myexamplebg}{RGB}{245, 245, 245} % Very light grey for background
\definecolor{myexamplefr}{RGB}{120, 120, 120} % Medium grey for frame
\definecolor{myexampleti}{RGB}{60, 60, 60}    % Darker grey for title

\newtcbtheorem[]{Example}{Example}{
    colback=myexamplebg,
    breakable,
    colframe=myexamplefr,
    coltitle=myexampleti,
    boxrule=1pt,
    sharp corners,
    detach title,
    before upper=\tcbtitle\par\vspace{-20pt}, % Reduced the space after the title
    fonttitle=\bfseries,
    description font=\mdseries,
    separator sign none,
    description delimiters={}{}, % No delimiters around the title
}{ex}
%================================
% Solution BOX
%================================
\makeatletter
\newtcolorbox{solution}{enhanced,
	breakable,
	colback=white,
	colframe=myg!80!black,
	attach boxed title to top left={yshift*=-\tcboxedtitleheight},
	title=Solution,
	boxed title size=title,
	boxed title style={%
			sharp corners,
			rounded corners=northwest,
			colback=tcbcolframe,
			boxrule=0pt,
		},
	underlay boxed title={%
			\path[fill=tcbcolframe] (title.south west)--(title.south east)
			to[out=0, in=180] ([xshift=5mm]title.east)--
			(title.center-|frame.east)
			[rounded corners=\kvtcb@arc] |-
			(frame.north) -| cycle;
		},
}
\makeatother

% %================================
% % Question BOX
% %================================
\makeatletter
\newtcbtheorem{question}{Question}{enhanced,
	breakable,
	colback=white,
	colframe=myb!80!black,
	attach boxed title to top left={yshift*=-\tcboxedtitleheight},
	fonttitle=\bfseries,
	title={#2},
	boxed title size=title,
	boxed title style={%
			sharp corners,
			rounded corners=northwest,
			colback=tcbcolframe,
			boxrule=0pt,
		},
	underlay boxed title={%
			\path[fill=tcbcolframe] (title.south west)--(title.south east)
			to[out=0, in=180] ([xshift=5mm]title.east)--
			(title.center-|frame.east)
			[rounded corners=\kvtcb@arc] |-
			(frame.north) -| cycle;
		},
	#1
}{question}
\makeatother

%%%%%%%%%%%%%%%%%%%%%%%%%%%%%%%%%%%%%%%%%%%
% TABLE OF CONTENTS
%%%%%%%%%%%%%%%%%%%%%%%%%%%%%%%%%%%%%%%%%%%


\usepackage{tikz}
\definecolor{doc}{RGB}{0,60,110}
\usepackage{titletoc}
\contentsmargin{0cm}
\titlecontents{chapter}[14pc]
{\addvspace{30pt}%
	\begin{tikzpicture}[remember picture, overlay]%
		\draw[fill=doc!60,draw=doc!60] (-7,-.1) rectangle (-0.9,.5);%
		\pgftext[left,x=-5.5cm,y=0.2cm]{\color{white}\Large\sc\bfseries Chapter\ \thecontentslabel};%
	\end{tikzpicture}\color{doc!60}\large\sc\bfseries}%
{}
{}
{\;\titlerule\;\large\sc\bfseries Page \thecontentspage
	\begin{tikzpicture}[remember picture, overlay]
		\draw[fill=doc!60,draw=doc!60] (2pt,0) rectangle (4,0.1pt);
	\end{tikzpicture}}%
\titlecontents{section}[3.7pc]
{\addvspace{2pt}}
{\contentslabel[\thecontentslabel]{3pc}}
{}
{\hfill\small \thecontentspage}
[]
\titlecontents*{subsection}[3.7pc]
{\addvspace{-1pt}\small}
{}
{}
{\ --- \small\thecontentspage}
[ \textbullet\ ][]

\makeatletter
\renewcommand{\tableofcontents}{
	\chapter*{%
	  \vspace*{-20\p@}%
	  \begin{tikzpicture}[remember picture, overlay]%
		  \pgftext[right,x=15cm,y=0.2cm]{\color{doc!60}\Huge\sc\bfseries \contentsname};%
		  \draw[fill=doc!60,draw=doc!60] (13,-.75) rectangle (20,1);%
		  \clip (13,-.75) rectangle (20,1);
		  \pgftext[right,x=15cm,y=0.2cm]{\color{white}\Huge\sc\bfseries \contentsname};%
	  \end{tikzpicture}}%
	\@starttoc{toc}}
\makeatother

\newcommand{\liff}{\llap{$\iff$}}
\newcommand{\rap}[1]{\rrap{\text{ (#1)}}}
\newcommand{\red}[1]{\textcolor{red}{#1}}
\newcommand{\blue}[1]{\textcolor{blue}{#1}}
\newcommand{\vi}[1]{\textcolor{violet}{#1}}
\newcommand{\olive}[1]{\textcolor{olive}{#1}}
\newcommand{\teal}[1]{\textcolor{teal}{#1}}
\newcommand{\brown}[1]{\textcolor{brown}{#1}}
\newcommand{\orange}[1]{\textcolor{orange}{#1}}
\newcommand{\tCaC}{\text{ \CaC }}
\newcommand{\CaC}{\red{CaC} }
\newcommand{\As}[1]{Assume \red{#1}}
\newcommand{\vdone}{\vi{\text{ (done) }}}
\newcommand{\bdone}{\blue{\text{ (done) }}}
\newcommand{\tdone}{\teal{\text{ (done) }}}
\newcommand{\odone}{\olive{\text{ (done) }}}
\newcommand{\bodone}{\brown{\text{ (done) }}}
\newcommand{\ordone}{\orange{\text{ (done) }}}
\newcommand{\ld}{\lambda}
\newcommand{\vecta}[1]{\textbf{#1}}
\newcommand{\set}[1]{\left\{ #1 \right\}}
\newcommand{\bset}[1]{\Big\{ #1 \Big\}}
\newcommand{\inR}{\in\R}
\newcommand{\inn}{\in\N}
\newcommand{\inz}{\in\Z}
\newcommand{\inr}{\in\R}
\newcommand{\inc}{\in\C}
\newcommand{\inq}{\in\Q}
\newcommand{\norm}[1]{\| #1 \|}
\newcommand{\bnorm}[1]{\Big\| #1 \Big\|}
\newcommand{\gen}[1]{\langle #1 \rangle}
\newcommand{\abso}[1]{\left|#1\right|}
\newcommand{\myref}[2]{\hyperref[#2]{#1\ \ref*{#2}}}
\newcommand{\customref}[2]{\hyperref[#1]{#2}}
\newcommand{\power}[1]{\mathcal{P}(#1)}
\newcommand{\dcup}{\mathbin{\dot{\cup}}}
\newcommand{\diam}[1]{\text{diam}\, #1}
\newcommand{\at}{\Big|}
\newcommand{\quotient}{\diagup}
\let\originalphi\phi % Store the original \phi in \originalphi
\renewcommand{\phi}{\varphi} % Redefine \phi to \varphi
\newcommand{\pfi}{\originalphi} % Define \pfi to display the original \phi
\newcommand{\diota}{\dot{\iota}}
\newcommand{\Log}{\operatorname{Log}}
\newcommand{\id}{\text{\textbf{id}}}
\usepackage{amsmath}

\makeatletter
\NewDocumentCommand{\extp}{e{^}}{%
  \mathop{\mathpalette\extp@{#1}}\nolimits
}
\NewDocumentCommand{\extp@}{mm}{%
  \bigwedge\nolimits\IfValueT{#2}{^{\extp@@{#1}#2}}%
  \IfValueT{#1}{\kern-2\scriptspace\nonscript\kern2\scriptspace}%
}
\newcommand{\extp@@}[1]{%
  \mkern
    \ifx#1\displaystyle-1.8\else
    \ifx#1\textstyle-1\else
    \ifx#1\scriptstyle-1\else
    -0.5\fi\fi\fi
  \thinmuskip
}
\makeatletter
\usepackage{pifont}
\makeatletter
\newcommand\Pimathsymbol[3][\mathord]{%
  #1{\@Pimathsymbol{#2}{#3}}}
\def\@Pimathsymbol#1#2{\mathchoice
  {\@Pim@thsymbol{#1}{#2}\tf@size}
  {\@Pim@thsymbol{#1}{#2}\tf@size}
  {\@Pim@thsymbol{#1}{#2}\sf@size}
  {\@Pim@thsymbol{#1}{#2}\ssf@size}}
\def\@Pim@thsymbol#1#2#3{%
  \mbox{\fontsize{#3}{#3}\Pisymbol{#1}{#2}}}
\makeatother
% the next two lines are needed to avoid LaTeX substituting upright from another font
\input{utxmia.fd}
\DeclareFontShape{U}{txmia}{m}{n}{<->ssub * txmia/m/it}{}
% you may also want
\DeclareFontShape{U}{txmia}{bx}{n}{<->ssub * txmia/bx/it}{}
% just in case
%\DeclareFontShape{U}{txmia}{l}{n}{<->ssub * txmia/l/it}{}
%\DeclareFontShape{U}{txmia}{b}{n}{<->ssub * txmia/b/it}{}
% plus info from Alan Munn at https://tex.stackexchange.com/questions/290165/how-do-i-get-a-nicer-lambda?noredirect=1#comment702120_290165
\newcommand{\pilambdaup}{\Pimathsymbol[\mathord]{txmia}{21}}
\renewcommand{\lambda}{\pilambdaup}
\renewcommand{\tilde}{\widetilde}
\DeclareMathOperator*{\esssup}{ess\,sup}
\newcommand{\bluecheck}{}%
\DeclareRobustCommand{\bluecheck}{%
  \tikz\fill[scale=0.4, color=blue]
  (0,.35) -- (.25,0) -- (1,.7) -- (.25,.15) -- cycle;%
}


\usepackage{tikz}
\newcommand*{\DashedArrow}[1][]{\mathbin{\tikz [baseline=-0.25ex,-latex, dashed,#1] \draw [#1] (0pt,0.5ex) -- (1.3em,0.5ex);}}

\newcommand{\C}{\mathbb{C}}	
\newcommand{\F}{\mathbb{F}}
\newcommand{\N}{\mathbb{N}}
\newcommand{\Q}{\mathbb{Q}}
\newcommand{\R}{\mathbb{R}}
\newcommand{\Z}{\mathbb{Z}}



\title{\Huge{NCKU 112.1}\\Note for Probability Theory}
\author{\huge{Eric Liu}}
\date{}
\begin{document}
\maketitle
\newpage% or \cleardoublepage
% \pdfbookmark[<level>]{<title>}{<dest>}
\pdfbookmark[section]{\contentsname}{toc}
\tableofcontents
\pagebreak
\chapter{$\sigma$-Algebra}
\section{Basic Definition for Background Space}
\begin{definition}
\label{1.1.1}
\textbf{(Definition of $\sigma$-Algebra)} We say  $(\Omega,\mathcal{G}\subseteq\power{\Omega})$ is a $\sigma$-algebra if 
\begin{equation}
\Omega \in\mathcal{G}
\end{equation}
\begin{equation}
X \in \mathcal{G}\implies \Omega \setminus X\in \mathcal{G} \text{ (Closed under complement) }
\end{equation}
\begin{equation}
\mathcal{A} \subseteq  \mathcal{G}\text{ and }\abso{\mathcal{A}}\leq \abso{\N}\implies \bigcup \mathcal{A}\in \mathcal{G}\text{ (Closed under countable union) }
\end{equation}
From now, we denote $\Omega\setminus X$ by $X^c$. We say $\mathcal{G}=\set{\varnothing,\Omega}$ is the trivial $\sigma$-algebra on $\Omega$. 
\end{definition}
\begin{theorem}
\label{1.1.2}
\textbf{(Basic Property of $\sigma$-Algebra)} Let $(\Omega,\mathcal{G})$ be a $\sigma$-algebra. Then we have
\begin{equation}
\varnothing \in \mathcal{G}
\end{equation}
\begin{equation}
\mathcal{A}\subseteq \mathcal{G}\implies \bigcap \mathcal{A} \in \mathcal{G}
\end{equation}
\begin{equation}
A,B\in \mathcal{G}\implies A\setminus B \in \mathcal{G}
\end{equation}
\end{theorem}
\begin{proof}
Observe $\varnothing=\Omega^c$, and observe $\bigcap \mathcal{A}=(\bigcup_{X \in \mathcal{A}}X^c)^c$, and observe $A\setminus B=A\cup B^c$
\end{proof}
\begin{lemma}
\label{1.1.3}
\textbf{(Intersection of $\sigma$-Algebras is a $\sigma$-Algebra)} Let $S$ be a set of $\sigma$-algebra over $\Omega$, then $\bigcap S$ is a $\sigma$-algebra. 
\end{lemma}
\begin{proof}
  $\Omega\in \bigcap S$ is trivial. Observe $A\in \bigcap S\implies  \forall \mathcal{G}\in S, A\in \mathcal{G}\implies \forall \mathcal{G}\in S, A^c\in \mathcal{G}\implies A^c \in\bigcap S$. Observe $\mathcal{A}\subseteq \bigcap S\text{ and }\abso{\mathcal{A}}\leq \abso{\N}\implies \forall\mathcal{G}\in S, \mathcal{A}\subseteq \mathcal{G}\implies \forall \mathcal{G}\in S, \bigcup \mathcal{A} \in\mathcal{G}\implies \bigcup A\in \bigcap S$.
\end{proof}
\begin{definition}
\label{1.1.4}
\textbf{(Definition of Generating $\sigma$-Algebra)} Let $\mathcal{F}\subseteq \power{\Omega}$. The $\sigma$-algebra generated by $\mathcal{F}$ is defined to be the smallest  $\sigma$-algebra that contain $\mathcal{F}$
\end{definition}
\fbox{\begin{minipage}{39em}
We have now gave enough tools to generate $\sigma$-algebra. 
\end{minipage}}
\fbox{\begin{minipage}{39em}
The following concern a class of measure space, called probability space.
\end{minipage}}
\begin{definition}
\label{1.1.8}
  \textbf{(Definition of Measure)} Let $\mathcal{G}$ be a $\sigma$-algebra over $\Omega$. Function $\mu:\mathcal{G}\rightarrow \R$ is called a measure if
\begin{equation}
\forall E\in \mathcal{G}, \mu(E)\geq 0\text{ (Nonnegative) }
\end{equation}  
\begin{equation}
\mu(\varnothing)=0
\end{equation}
\begin{equation}
\mathcal{F}\subseteq \mathcal{G}\text{ and }\abso{\mathcal{F}}\leq \abso{\N}\text{ and }\mathcal{F}\text{ is disjoint }\implies \mu(\bigcup  \mathcal{F})=\sum_{X\in \mathcal{F}} \mu(X)\text{ ($\sigma$ additivity) }
\end{equation}
\end{definition}
\begin{theorem}
\textbf{(Intended Property of Measure)} We have
\begin{equation}
A\subseteq B\in \Omega\implies \mu(A)\leq  \mu(B)
\end{equation}
\end{theorem}
\begin{proof}
Observe $\mu(B)=\mu(A\cup (B\setminus A))=\mu(A)+\mu (B\setminus A)\implies \mu(B)-\mu(A)=\mu(B\setminus A)\geq 0$
\end{proof}
\begin{definition}
\textbf{(Definition of Probability Measure)} A probability measure $P$ is a measure that satisfy $P(\Omega)=1$ 
\end{definition}
\begin{definition}
\label{1.1.9}
\textbf{(Definition of Probability Space)} A probability space is a triple $(\Omega,\mathcal{\mathcal{G}},P)$ where 
\begin{equation}
\Omega \text{ is a set called \textit{sample space} }
\end{equation}
\begin{equation}
\mathcal{G} \text{ is a $\sigma$-algebra over $\Omega$ called event space}
\end{equation}
\begin{equation}
P:\mathcal{G}\rightarrow [0,1]\text{ is a measure called probability measure }
\end{equation}
\end{definition}
\fbox{\begin{minipage}{39em}
Basically, probability measure "measure" the probability an event will happen. Normally we don't measure the probability a sample will happen, since it is undoable in some cases, for instance, shooting a dart to an infinite set. We only measure the probability that the outcome sample is in a set of certain samples, where the assignment of measure is outside of the system.   
\end{minipage}}


\fbox{\begin{minipage}{39em}
A simple example of a $\sigma$-algebra is
\begin{equation}
\Omega_2=\set{HH,HT,TH,TT},\mathcal{G}=\set{\varnothing,X,\set{HT,HH},\set{TH,TT}}
\end{equation}

Notice in this example, $\Omega$ is ought to be interpreted as tossing two coins and $\mathcal{G}$ is to observe the first coin is head or tail.\\

To expand the first example, we have another simple example:
\begin{equation}
\Omega_3=\set{HHH,HHT,HTH,HTT,THH,THT,TTH,TTT}
\end{equation}
Define
\begin{equation}
A_H:=\set{HHH,HHT,HTH,HTT}\text{ and }A_T:=\set{THH,THT,TTH,TTT}
\end{equation}
which is the information of tossing head or tail on first try.\\

Notice $A_T=A_H^c$. Define
\begin{equation}
A_{HH}:=\set{HHH,HHT}\text{ and }A_{HT}:=\set{HTH,HTT}
\end{equation}
\begin{equation}
A_{TH}:=\set{THH,THT}\text{ and }A_{TT}:=\set{TTH,TTT}
\end{equation}
so we have
\begin{equation}
A_H=A_{HH}\cup A_{HT}\text{ and }A_T=A_{TH}\cup A_{TT}
\end{equation}
Then we can define 
\begin{equation}
\mathcal{G}:= \set{\bigcup N: N\in \power{M}}
\end{equation}
where  $M=\set{A_{HH},A_{HT},A_{TH},A_{TT}}$\\

Notice we can define four $\sigma$-algebras by
\begin{equation}
\mathcal{F}_0=\set{\varnothing,\Omega},\mathcal{F}_1=\set{\varnothing,\Omega,A_T,A_H},\mathcal{F}_2=\mathcal{G},\mathcal{F}_3=\power{\Omega}
\end{equation}
then we have
\begin{equation}
\mathcal{F}_0\subset \mathcal{F}_1\subset \mathcal{F}_2 \subset \mathcal{F}_3
\end{equation}

The following concern Borel $\sigma$-algebra
\end{minipage}}
\begin{definition}
\label{1.1.12}
\textbf{(Definition of a Random Variable)} We say the function from $\Omega$ to $\R$ is a random variable, and often a random variable is defined on a fixed probability space, since the usage of random variable are mostly associated with a probability measure. 
\end{definition}
\fbox{\begin{minipage}{39em}
We now define 3 random variable for example from the last example of $\sigma$-algebra.\\

Let $S_0,u,d\inr^+$ and let $d<1<u$. We define three random variables $S_1,S_2,S_3$ on $\Omega_3$
 \begin{equation}
S_1(\omega)= \begin{cases}
  uS_0& \text{ if $\omega \in A_H$ }\\
  dS_0& \text{ if $\omega\inA_T$ }
\end{cases}S_2(\omega)=\begin{cases}
  u^2S_0& \text{ if $\omega\in A_{HH}$ }\\
  udS_0& \text{ if $\omega\in A_{HT}\cup A_{TH}$ }\\
  d^2S_0& \text{ if $\omega\in A_{TT}$ }
\end{cases}
\end{equation}
\begin{equation}
S_3(\omega)=\begin{cases}
  u^3S_0& \text{ if $\omega\in \set{HHH}$ }\\
  u^2dS_0& \text{ if $\omega\in \set{HHT,HTH,THH}$ }\\
  ud^2S_0& \text{ if $\omega\in \set{HTT,THT,TTH}$ }\\
  d^3S_0& \text{ if $\omega\in \set{TTT}$ }
\end{cases}
\end{equation}
Often, we just use $S$ to denote  $S(\omega)$. 
\end{minipage}}
\begin{definition}
\label{1.1.11}
\textbf{(Definition of Borel-Algebra)} The Borel-Algebra on $\R$, which we denote  $\mathcal{B}(\R)$ is the $\sigma$-algebra generated by all open interval of $\R$.\\

Some members of $\mathcal{B}(\R):$ 
 \begin{equation}
   (b,a),(a,\infty),\R
\end{equation}
\begin{equation}
  (b,a]=(b,\infty)\setminus (a,\infty)
\end{equation}
\begin{equation}
[a,\infty)=\R\setminus (-\infty, a) 
\end{equation}
\begin{equation}
[a,b]=[a,\infty)\setminus (b,\infty) 
\end{equation}
\begin{equation}
\set{a}=\R \setminus (-\infty, a)\cup (a,\infty)
\end{equation}
\end{definition}
\begin{theorem}
\label{1.1.13}
\textbf{(Construct  $\sigma$-Algebra with Random Variable)} Let $X$ be a random variable on  $\Omega$. We define
\begin{equation}
X^{-1}[B]=\set{\omega \in \Omega : X(\omega)\in B}
\end{equation}
and define the $\sigma$-algebra $\sigma(X)$ by
\begin{equation}
\sigma(X)=\set{X^{-1}[B]:B\in\mathcal{B}(\R)}
\end{equation}
We can verify $\sigma(X)$ is a $\sigma$-algebra.
\end{theorem}
\begin{proof}
Notice $\Omega= X^{-1}[\R]\in \sigma(X)$. Observe $(X^{-1}[B])^c=X^{-1}[B^c]\in \sigma (X)$. Let $C=\set{X^{-1}[B]:B\in \cc}\subseteq \sigma$ be countable. Observe  $\bigcup_{B\in C}X^{-1}[B]=X^{-1}[\bigcup_{B\in\cc }B]\in \sigma(X)$ 
\end{proof}
\fbox{\begin{minipage}{39em}
Notice $\sigma(S_1)=\mathcal{F}_1,\sigma(S_2)\neq \mathcal{F}_2,\sigma(S_3)\neq \mathcal{F}_3$
\end{minipage}}
\begin{theorem}
\textbf{(Induce Measure on Borel Algebra by a Random Variable and a Probability Space)} Let $(\Omega,\mathcal{F},P:\mathcal{F}\rightarrow \R)$ be a probability space and $X:\Omega\rightarrow \R$ be a random variable. We can induce a measure on  $\mathcal{B}(\R)$ (we say this measure is induced by $X$ and  $P$)
\begin{equation}
  \mathcal{L}_X:\mathcal{B}(\R)\rightarrow \R, B\mapsto P(X^{-1}[B])
\end{equation}
\end{theorem}
\begin{proof}
Because $P$ is nonnegative, we know  $\mathcal{L}_X$ is nonnegative. Notice that $X^{-1}[\varnothing]=\varnothing$, so we know $\mathcal{L}_X(\varnothing)=P(\varnothing)=0$. Let $\mathcal{A}\subseteq \mathcal{B}(\R)$ and $\mathcal{A}$ be countable and disjoint. We have
\begin{align}
\mathcal{L}_X(\bigcup \mathcal{A})&=P(X^{-1}[\bigcup \mathcal{A}])\\
&=P(\bigcup \mathcal{C})\text{ where $\cc=\set{X^{-1}[a]:a \in \mathcal{A}}$ }\\
&=\sum_{c \in \cc} P(c)\\
&=\sum_{a \in \mathcal{A}}P(X^{-1}[a])=\sum_{a\in\mathcal{A}}\mathcal{L}_X(a)
\end{align} 
Notice that $\cc$ is countable because $\mathcal{A}$ is countable, and $\cc$ is disjoint because $x\in X^{-1}[a]\cap X^{-1}[b]\implies X(x)\in a\cap b$.
\end{proof}
\begin{definition}
\textbf{(Definition of Cumulative Distribution Function)} We say $F_X:\R\rightarrow \R$ is the cumulative distribution function of  $X$ if
 \begin{equation}
   F_X(x)=\mathcal{L}_X((-\infty, x])=P(X^{-1}[(-\infty,x]])
\end{equation}
\end{definition}
\fbox{\begin{minipage}{39em}
We now use the random variable $S_3:\Omega \rightarrow \R $ on $(\Omega,\mathcal{F},P:\power{\Omega}\rightarrow \R)$ to serve as an example of induced measure and cumulative distribution function.\\

Notice that the range of $S_3$ is $\set{d^3S_0<d^2uS_0<du^2S_0<u^3S_0}$. We have 
\begin{equation} F_{S_3}(d^3S_0)=\mathcal{L}_{S_3}((-\infty,d^3S_0])=P(S_3^{-1}[(-\infty,d^3S_0]])=P(\set{TTT})
\end{equation}
\begin{equation} F_{S_3}(d^2uS_0)=\mathcal{L}_{S_3}((-\infty,d^2uS_0])=P(S_3^{-1}[(-\infty,d^2uS_0]])=P(\set{TTT,TTH,THT,HTT})
\end{equation}
\begin{equation} F_{S_3}(du^2S_0)=\mathcal{L}_{S_3}((-\infty,du^2S_0])=P(S_3^{-1}[(-\infty,du^2S_0]])=P(\Omega \setminus \set{HHH} )
\end{equation}
\begin{equation} F_{S_3}(u^3S_0)=\mathcal{L}_{S_3}((-\infty, u^3S_0])=P(S_3^{-1}[(-\infty,u^3S_0]])=P(\Omega)
\end{equation}
With above deduction, we can precisely deduce
\begin{equation}
F_{S_3}(x)=\begin{cases}
  0& \text{ if  }x<d^3S_0\\
  P(\set{TTT})& \text{ if  }d^3S_0\leq x<d^2uS_0\\
  P(\set{TTT,TTH,THT,HTT})& \text{ if  }d^2uS_0\leq x<du^2S_0\\
  P(\Omega \setminus \set{HHH})& \text{ if  }du^2S_0\leq x<u^3S_0\\
  P(\Omega)& \text{ if  }u^3S_0\leq x
\end{cases}
\end{equation}
Notice that we can define measure $P$ in multiple ways, and then we will have different induced measures $\mathcal{L}_{S_3}$ and different cumulative distribution function $F_{S_3}(x)$. For instance, one can check that $P:\power{\Omega}=\mathcal{F}_3\rightarrow \R,X\mapsto \frac{\abso{X}}{8}$ is a measure, and $P:\mathcal{F}_3\rightarrow \R,X\mapsto \sum_{x\in X}f(x)$ where $f$ is defined by $\omega_1\omega_2\omega_3\mapsto \frac{2^i}{3^3}: i=\abso{\set{\omega_j:\omega_j=T,j=1,2,3}}$ are all measures, but the former is 
\begin{equation}
F_{S_3}(x)=\begin{cases}
  0& \text{ if  }x<d^3S_0\\
  \frac{1}{8}& \text{ if  }d^3S_0\leq x<d^2uS_0\\
  \frac{1}{2}& \text{ if  }d^2uS_0\leq x<du^2S_0\\
  \frac{7}{8}& \text{ if  }du^2S_0\leq x<u^3S_0\\
  1& \text{ if  }u^3S_0\leq x
\end{cases}
\end{equation}
and the latter is
\begin{equation}
F_{S_3}(x)=\begin{cases}
  0& \text{ if  }x<d^3S_0\\
  \frac{8}{27}& \text{ if  }d^3S_0\leq x<d^2uS_0\\
  \frac{8}{27}+\binom{3}{1}\frac{4}{27}& \text{ if  }d^2uS_0\leq x<du^2S_0\\
  \frac{8}{27}+\binom{3}{1}\frac{4}{27}+\binom{3}{2}\frac{2}{27}& \text{ if  }du^2S_0\leq x<u^3S_0\\
  1& \text{ if  }u^3S_0\leq x
\end{cases}
\end{equation}
\end{minipage}}
\section{Binomial Asset Pricing Model}
\begin{definition}
\textbf{(Definition of Stochastic Process)} Given a common probability space $(\Omega,\mathcal{F},P:\mathcal{F}\rightarrow \R)$, we say an indexed family of random variables defined on our given probability space is a stochastic process, denoted
\begin{equation}
\set{X_t:t\in \Lambda,\omega \in \Omega}
\end{equation}
where the index set $\Lambda$ is called time horizon. So, if someone say $X$ is a stochastic process, we can say $X_t$ is a random variable, and we can say  $X(t,\omega)=X_t(\omega)\in \R$.\\

Additionally, we say $\Omega$ is "background space" and $\Lambda \times \R$, where we can draw a trajectory to record the sequence of outcome, is "foreground space" or "configuration space", and we say $\Theta=\set{X(t,\omega):t \in \Lambda, \omega \in \Omega}$ is the "state space" that contains "state of process"\\

The "frequency" of a state of the process $X(t,\omega)$ is ?
\end{definition}
\fbox{\begin{minipage}{39em}
We now give an example of stochastic process. Let $\Omega=\set{H,T}$ and $\Lambda = [0,2]$. The stochastic process if the indexed set $\set{X_t:t\in \Lambda}$ of random variables specified by
\begin{equation}
X_t(H)=\sin(2\pi t)\text{ and }X_t(T)=\sin(4\pi t)
\end{equation}
Then 
\begin{equation}
\Theta=[-1,1]
\end{equation}
Another example is: Let $\Lambda=\set{1,2,3,4,5}$ and let $\Omega=\set{(\omega_1,\omega_2,\omega_3,\omega_4,\omega_5):\forall i\in \Lambda, \omega_i\in \set{H,T}}$. The configuration space is then $\set{1,2,3,4,5}\times \R$, and we can specify the stochastic process $\set{S_t:t \in \Lambda}$ by
\begin{equation}
S_t((\omega_1,\omega_2,\omega_3,\omega_4,\omega_5))=u^id^{t-i}S_0: i=\abso{\set{\omega_j: 1\leq j\leq t,\omega_j= H}}
\end{equation}
Then 
\begin{equation}
\Theta= \set{u^id^jS_0:i,j\in \N\cup \set{0},1\leq i+j\leq 5} 
\end{equation}
\end{minipage}}
\begin{definition}
\textbf{()}
\end{definition}
\fbox{\begin{minipage}{39em}
To buy an option from someone means acquiring the right, but not the obligation, to purchase a stock at a predetermined price, $K$. This comes with an associated cost called the ``risk premium'' or $V_0$, which is paid at the time of the transaction.\\

Notice when we say net earn or profit, we are comparing his money with and without making the deal.\\

If the stock ends up at price $S_1(\omega)$  lower than $K$, the buyer won't buy the stock at the price $K$, so for seller, he profit $V_0$.\\

If the stock ends up at price $S_1(\omega)$ higher than  $K$, then the seller net earn:  $-(S_1(\omega)-K)+V_0$\\

Now, we consider the profit we can make for depositing money into bank and get interest. The interest rate is $r$.\\


If the stock ends up at price $S_1(\omega)$ lower than $K$, the buyer won't buy the stock at the price $K$, so for seller, he profit $(1+r)V_0$.\\

If the stock ends up at price $S_1(\omega)$ higher than  $K$, then the seller net earn:  $-(S_1(\omega)-K)+(1+r)V_0$\\

Now, we wish to lower the risk so we buy $\Delta_0$ share of the stock we short at price $S_0$ to form a portfolio.\\

If the stock ends up at price $S_1(\omega)$ lower than $K$, the buyer won't buy the stock at the price $K$, so for seller, he profit $(1+r)V_0+\Delta_0(S_1(\omega)-(1+r)S_0)$.\\

If the stock ends up at price $S_1(\omega)$ higher than  $K$, then the seller net earn:  $-(S_1(\omega)-K)+(1+r)V_0+\Delta_0(S_1(\omega)-(1+r)S_0)$\\

\end{minipage}}

\fbox{\begin{minipage}{39em}
From now, we denote the value of the option at time $t$
\begin{equation}
V_t(\omega):=\max \set{S_t(\omega)-K,0}
\end{equation}
So we can say the value of our portfolio $X_t(\omega)$, our action of selling the option and buy the stock, at time $1$ is 
\begin{equation}
X_1(\omega)=-V_1(\omega)+(1+r)V_0+\Delta_0(S_1(\omega)-(1+r)S_0)
\end{equation}
Now we wish, for all $\omega$, we have $X_1(\omega)\geq 0$, by negotiating with the buyer on $V_0$ and deciding how many share  $\Delta_0$ we want to buy.\\

Notice that if we negotiate $V_0$ at
 \begin{equation}
V_0:=\frac{1}{1+r}[\frac{1+r-d}{u-d}V_1(H)+\frac{u-(1+r)}{u-d}V_1(T)]
\end{equation}
and buy share $\Delta_0$
\begin{equation}
\Delta_0:= \frac{V_1(H)-V_1(T)}{S_1(H)-S_1(T)}
\end{equation}
We see 
\begin{multline}
X_1(H)=-V_1(H)+(1+r)\frac{1}{1+r}[\frac{1+r-d}{u-d}V_1(H)+\frac{u-(1+r)}{u-d}V_1(T)]\\ 
+\frac{V_1(H)-V_1(T)}{S_1(H)-S_1(T)}(S_1(H)-(1+r)S_0)
\end{multline}
equals $0$ by substituting  $S_1(H)\text{ and }S_1(T)$ with $S_0u\text{ and }S_0d$
\end{minipage}}
\end{document}
