\documentclass{report}

%%%%%%%%%%%%%% macros.tex %%%%%%%%%%%%%%
% Place your custom macros here, if any.

%%%%%%%%%%%%%% letterfonts.tex %%%%%%%%%%%%%%
% Place your font setup here, if any.

%%%%%%%%%%%%%% preamble.tex %%%%%%%%%%%%%%
\usepackage[T1]{fontenc}
\usepackage{lmodern}
\usepackage{etoolbox}
\usepackage{pdfpages}
\usepackage{transparent}
\usepackage[utf8]{inputenc}
\usepackage[english]{babel}

% Page Setup
\usepackage[tmargin=2cm, rmargin=0.5in, lmargin=0.5in, bmargin=80pt, footskip=.2in]{geometry}

% Mathematics
\usepackage{amsmath,amsfonts,amsthm,amssymb,mathtools}
\usepackage{xfrac}
\usepackage[makeroom]{cancel}
\usepackage{enumitem}
\usepackage{nameref}
\usepackage{multicol,array}
\usepackage{tikz-cd}
\usepackage[ruled,vlined,linesnumbered]{algorithm2e}

% Colors
\usepackage[dvipsnames]{xcolor}
\definecolor{myg}{RGB}{56, 140, 70}
\definecolor{myb}{RGB}{45, 111, 177}
\definecolor{myr}{RGB}{199, 68, 64}
% Define more colors here...

% Hyperlinks
\usepackage{bookmark}
\usepackage{hyperref}
\hypersetup{
    pdftitle={Assignment},
    colorlinks=true, linkcolor=doc!90,
    bookmarksnumbered=true,
    bookmarksopen=true
}

% Figures and Graphics
\usepackage{import}
\usepackage{svg}
\newcommand{\incfig}[1]{%
    \def\svgwidth{\columnwidth}
    \import{./figures/}{#1.pdf_tex}
}

% Text-related
\usepackage{blindtext}
\usepackage{fontsize}
\changefontsize[14]{14}
\setlength{\parindent}{0pt}

% Theorems and Definitions
\usepackage{amsthm}
\renewcommand\qedsymbol{$\blacksquare$}

% Define a new theorem style
\newtheoremstyle{mytheoremstyle}% name
  {}% Space above
  {}% Space below
  {\sffamily}% Body font
  {}% Indent amount
  {\bfseries}% Theorem head font
  {.}% Punctuation after theorem head
  {.5em}% Space after theorem head
  {}% Theorem head spec (can be left empty, meaning ‘normal’)

% Apply the new theorem style to theorem-like environments
\theoremstyle{mytheoremstyle}
\newtheorem{theorem}{Theorem}[section]
\newtheorem{definition}{Definition}[section]
\newtheorem{corollary}{Corollary}[section]
\newtheorem{lemma}{Lemma}[section]
\newtheorem{axiom}{Axiom}[section]

% tcolorbox Setup
\usepackage[most,many,breakable]{tcolorbox}

% Define custom tcolorbox environments here...

%================================
% EXAMPLE BOX
%================================
\newtcbtheorem[definition]{Example}{Example}
{%
    colback = myexamplebg,
    breakable,
    colframe = myexamplefr,
    coltitle = myexampleti,
    boxrule = 1pt,
    sharp corners,
    detach title,
    before upper=\tcbtitle\par\smallskip,
    fonttitle = \bfseries,
    description font = \mdseries,
    separator sign none,
    description delimiters parenthesis,
}
{ex}

%================================
% Solution BOX
%================================
\makeatletter
\newtcolorbox{solution}{enhanced,
	breakable,
	colback=white,
	colframe=myg!80!black,
	attach boxed title to top left={yshift*=-\tcboxedtitleheight},
	title=Solution,
	boxed title size=title,
	boxed title style={%
			sharp corners,
			rounded corners=northwest,
			colback=tcbcolframe,
			boxrule=0pt,
		},
	underlay boxed title={%
			\path[fill=tcbcolframe] (title.south west)--(title.south east)
			to[out=0, in=180] ([xshift=5mm]title.east)--
			(title.center-|frame.east)
			[rounded corners=\kvtcb@arc] |-
			(frame.north) -| cycle;
		},
}
\makeatother

%================================
% Question BOX
%================================
\makeatletter
\newtcbtheorem{question}{Question}{enhanced,
	breakable,
	colback=white,
	colframe=myb!80!black,
	attach boxed title to top left={yshift*=-\tcboxedtitleheight},
	fonttitle=\bfseries,
	title={#2},
	boxed title size=title,
	boxed title style={%
			sharp corners,
			rounded corners=northwest,
			colback=tcbcolframe,
			boxrule=0pt,
		},
	underlay boxed title={%
			\path[fill=tcbcolframe] (title.south west)--(title.south east)
			to[out=0, in=180] ([xshift=5mm]title.east)--
			(title.center-|frame.east)
			[rounded corners=\kvtcb@arc] |-
			(frame.north) -| cycle;
		},
	#1
}{def}
\makeatother
\makeatletter
\newtcbtheorem{qstion}{Question}{enhanced,
    breakable,
    colback=white,
    colframe=mygr,
    attach boxed title to top left={yshift*=-\tcboxedtitleheight},
    fonttitle=\bfseries,
    title={#2},
    boxed title size=title,
    boxed title style={%
        sharp corners,
        rounded corners=northwest,
        colback=tcbcolframe,
        boxrule=0pt,
    },
    underlay boxed title={%
        \path[fill=tcbcolframe] (title.south west)--(title.south east)
        to[out=0, in=180] ([xshift=5mm]title.east)--
        (title.center-|frame.east)
        [rounded corners=\kvtcb@arc] |-
        (frame.north) -| cycle;
    },
    #1
}{def}
\makeatother

%%%%%%%%%%%%%%%%%%%%%%%%%%%%%%%%%%%%%%%%%%%
% TABLE OF CONTENTS
%%%%%%%%%%%%%%%%%%%%%%%%%%%%%%%%%%%%%%%%%%%
\usepackage{tikz}
\definecolor{doc}{RGB}{0,60,110}
\usepackage{titletoc}
\contentsmargin{0cm}
\titlecontents{chapter}[14pc]
{\addvspace{30pt}%
	\begin{tikzpicture}[remember picture, overlay]%
		\draw[fill=doc!60,draw=doc!60] (-7,-.1) rectangle (-0.9,.5);%
		\pgftext[left,x=-4.5cm,y=0.2cm]{\color{white}\Large\sc\bfseries Chapter\ \thecontentslabel};%
	\end{tikzpicture}\color{doc!60}\large\sc\bfseries}%
{}
{}
{\;\titlerule\;\large\sc\bfseries Page \thecontentspage
	\begin{tikzpicture}[remember picture, overlay]
		\draw[fill=doc!60,draw=doc!60] (2pt,0) rectangle (4,0.1pt);
	\end{tikzpicture}}%
\titlecontents{section}[3.7pc]
{\addvspace{2pt}}
{\contentslabel[\thecontentslabel]{2pc}}
{}
{\hfill\small \thecontentspage}
[]
\titlecontents*{subsection}[3.7pc]
{\addvspace{-1pt}\small}
{}
{}
{\ --- \small\thecontentspage}
[ \textbullet\ ][]

\makeatletter
\renewcommand{\tableofcontents}{
	\chapter*{%
	  \vspace*{-20\p@}%
	  \begin{tikzpicture}[remember picture, overlay]%
		  \pgftext[right,x=15cm,y=0.2cm]{\color{doc!60}\Huge\sc\bfseries \contentsname};%
		  \draw[fill=doc!60,draw=doc!60] (13,-.75) rectangle (20,1);%
		  \clip (13,-.75) rectangle (20,1);
		  \pgftext[right,x=15cm,y=0.2cm]{\color{white}\Huge\sc\bfseries \contentsname};%
	  \end{tikzpicture}}%
	\@starttoc{toc}}
\makeatother

\newcommand{\liff}{\llap{$\iff$}}
\newcommand{\rap}[1]{\rrap{\text{ (#1)}}}
\newcommand{\red}[1]{\textcolor{red}{#1}}
\newcommand{\blue}[1]{\textcolor{blue}{#1}}
\newcommand{\vi}[1]{\textcolor{violet}{#1}}
\newcommand{\teal}[1]{\textcolor{teal}{#1}}
\newcommand{\tCaC}{\text{ \CaC }}
\newcommand{\CaC}{\red{CaC} }
\newcommand{\As}[1]{Assume \red{#1}}
\newcommand{\vdone}{\vi{\text{ (done) }}}
\newcommand{\bdone}{\blue{\text{ (done) }}}
\newcommand{\tdone}{\teal{\text{ (done) }}}
\newcommand{\set}[1]{\{ #1 \}}
\newcommand{\inS}{\in S}
\newcommand{\inF}{\in\F}
\newcommand{\inE}{\in E}
\newcommand{\inA}{\in A}
\newcommand{\inB}{\in B}
\newcommand{\inC}{\in C}
\newcommand{\inU}{\in U}

\newcommand{\C}{\mathbb{C}}	
\renewcommand{\H}{\mathbb{H}}
\newcommand{\F}{\mathbb{F}}
\newcommand{\N}{\mathbb{N}}
\newcommand{\Q}{\mathbb{Q}}
\newcommand{\R}{\mathbb{R}}
\newcommand{\Z}{\mathbb{Z}}
\renewcommand{\P}{\mathbb{P}}
\renewcommand{\S}{\mathbb{S}}
\newcommand{\A}{\mathbb{A}}
\newcommand{\RP}{\R P}


\title{\Huge{NCKU 112.1}\\Note for Probability Theory}
\author{\huge{Eric Liu}}
\date{}
\begin{document}
\maketitle
\newpage% or \cleardoublepage
% \pdfbookmark[<level>]{<title>}{<dest>}
\pdfbookmark[section]{\contentsname}{toc}
\tableofcontents
\pagebreak
\chapter{$\sigma$-Algebra}
\section{}
\begin{definition}
\label{1.1.1}
\textbf{(Definition of Measure Space)} A measure space is a triple $(\Omega,\mathcal{G},\mu)$ where
\begin{equation}
\Omega \text{ is a set }
\end{equation}
\begin{equation}
\mathcal{G}\text{ is a $\sigma$-algebra over $\Omega$}
\end{equation}
\begin{equation}
\mu\text{ is a measure on $(\Omega,\mathcal{G})$ }
\end{equation}
\end{definition}
\begin{definition}
\label{1.1.2}
\textbf{(Definition of $\sigma$-Algebra)} We say  $\mathcal{G}\subseteq\power{\Omega}$ is a $\sigma$-algebra if 
\begin{equation}
\Omega \in\mathcal{G} 
\end{equation}
\begin{equation}
X \in \mathcal{G}\implies \Omega \setminus X\in \mathcal{G} \text{ (Closed under complement) }
\end{equation}
\begin{equation}
\mathcal{A} \subseteq  \mathcal{G}\text{ and }\abso{\mathcal{A}}\leq \abso{\N}\implies \bigcup \mathcal{A}\in \mathcal{G}\text{ (Closed under countable union) }
\end{equation}
From now, we denote $\Omega\setminus X$ by $X^c$. We say $\mathcal{G}=\set{\varnothing,\Omega}$ is the trivial $\sigma$-algebra on $\Omega$. 
\end{definition}
\begin{theorem}
\label{1.1.3}
\textbf{(Basic Property of $\sigma$-Algebra)} Let $(\Omega,\mathcal{G})$ be a $\sigma$-algebra. Then we have
\begin{equation}
\varnothing \in \mathcal{G}
\end{equation}
\begin{equation}
\mathcal{A}\subseteq \mathcal{G}\implies \bigcap \mathcal{A} \in \mathcal{G}
\end{equation}
\begin{equation}
A,B\in \mathcal{G}\implies A\setminus B \in \mathcal{G}
\end{equation}
\end{theorem}
\begin{proof}
Observe $\varnothing=\Omega^c$, and observe $\bigcap \mathcal{A}=(\bigcup_{X \in \mathcal{A}}X^c)^c$, and observe $A\setminus B=A\cup B^c$
\end{proof}
\begin{theorem}
\label{1.1.4}
\textbf{(Intersection of $\sigma$-Algebras is a $\sigma$-Algebra)} Let $S$ be a set of $\sigma$-algebra over $\Omega$, then $\bigcap S$ is a $\sigma$-algebra. 
\end{theorem}
\begin{proof}
  \red{missed}
\end{proof}
\fbox{\begin{minipage}{39em}
The following concerning measure
\end{minipage}}
\begin{definition}
\label{1.1.5}
  \textbf{(Definition of a Measure)} Let $\mathcal{G}$ be a $\sigma$-algebra over $\Omega$. Function $\mu:\mathcal{G}\rightarrow \R$ is called a measure if
\begin{equation}
\forall E\in \mathcal{G}, \mu(E)\geq 0\text{ (Nonnegative) }
\end{equation}  
\begin{equation}
\mu(\varnothing)=0
\end{equation}
\begin{equation}
F\subseteq \mathcal{G}\text{ and }\abso{F}\leq \abso{\N}\implies \mu(\bigcup  F)=\sum_{X\in F} \mu(X)\text{ (Countable additivity) }
\end{equation}
\end{definition}
\fbox{\begin{minipage}{39em}
The following concern generating a $\sigma$-Algebra from a set of subsets of sample space.
\end{minipage}}
\begin{theorem}
\label{1.1.6}
\textbf{(Representation of $\sigma$-Algebra)} Let $M$ be a countable partition of $\Omega$. Then the set
\begin{equation}
\set{\bigcup N: N \in\power{M}}
\end{equation}
is a $\sigma$-algebra
\end{theorem}
\begin{proof}
  \red{missed}
\end{proof}
\begin{definition}
\label{1.1.7}
\textbf{(Definition of Generating $\sigma$-Algebra)} Let $\mathcal{F}\subseteq \power{\Omega}$. The $\sigma$-algebra generated by $\mathcal{F}$ is defined to be the smallest  $\sigma$-algebra that contain $\mathcal{F}$
\end{definition}
\begin{theorem}
\label{1.1.8}
\textbf{(Definition of Generating $\sigma$-Algebra)} Let $\mathcal{F}\subseteq \power{\Omega}$. The smallest $\sigma$-algebra containing $\mathcal{F}$ consists precisely of set taking countable operation of complement countable operation.
\end{theorem}
\begin{proof}
  \red{need verified}
\end{proof}
\begin{theorem}
\textbf{(Representation of $\sigma$-Algebra)} Let $M$ be a countable partition of  $\Omega$. Then the $\sigma$-algebra 
\begin{equation}
\set{\bigcup N: N\in\power{M}}
\end{equation}
is the $\sigma$-algebra generate by $M$
\end{theorem}
\begin{proof}
  \red{need verified}
\end{proof}
\begin{theorem}
\textbf{(Representation of $\sigma$-Algebra)} Let $M$ be a countable partition of  $\Omega$. Then the $\sigma$-algebra 
\begin{equation}
\set{\bigcup N: N \in \power{M}}
\end{equation}
contain no proper subset of element of $M$
\end{theorem}
\begin{proof}
  \red{need verified}
\end{proof}
\fbox{\begin{minipage}{39em}
The following concern a class of measure space, called probability space.
\end{minipage}}
\begin{definition}
\label{1.1.9}
\textbf{(Definition of Probability Space)} A probability space is a triple $(\Omega,\mathcal{\mathcal{G}},P)$ where 
\begin{equation}
\Omega \text{ is a set called \textit{sample space} }
\end{equation}
\begin{equation}
\mathcal{G} \text{ is a $\sigma$-algebra over $\Omega$ called event space}
\end{equation}
\begin{equation}
P:\Omega\rightarrow [0,1]\text{ is a measure called probability measure }
\end{equation}
where $\Omega$ is a set, called sample space, $\mathcal{G}$ is a $\sigma$-algebra over $\Omega$, called event space and $P:\Omega\rightarrow [0,1]$ is called probability measure. 
\end{definition}

\fbox{\begin{minipage}{39em}
A simple example of a $\sigma$-algebra is
\begin{equation}
\Omega_2=\set{HH,HT,TH,TT},\mathcal{G}=\set{\varnothing,X,\set{HT,HH},\set{TH,TT}}
\end{equation}

Notice in this example, $\Omega$ is ought to be interpreted as tossing two coins and $\mathcal{G}$ is to observe the first coin is head or tail.\\

To expand the first example, we have another simple example:
\begin{equation}
\Omega_3=\set{HHH,HHT,HTH,HTT,THH,THT,TTH,TTT}
\end{equation}
Define
\begin{equation}
A_H:=\set{HHH,HHT,HTH,HTT}\text{ and }A_T:=\set{THH,THT,TTH,TTT}
\end{equation}
which is the information of tossing head or tail on first try.\\

Notice $A_T=A_H^c$. Define
\begin{equation}
A_{HH}:=\set{HHH,HHT}\text{ and }A_{HT}:=\set{HTH,HTT}
\end{equation}
\begin{equation}
A_{TH}:=\set{THH,THT}\text{ and }A_{TT}:=\set{TTH,TTT}
\end{equation}
so we have
\begin{equation}
A_H=A_{HH}\cup A_{HT}\text{ and }A_T=A_{TH}\cup A_{TT}
\end{equation}
Then we can define 
\begin{equation}
\mathcal{G}:= \set{\bigcup N: N\in \power{M}}
\end{equation}
where  $M=\set{A_{HH},A_{HT},A_{TH},A_{TT}}$\\

Notice we can define four $\sigma$-algebras by
\begin{equation}
\mathcal{F}_0=\set{\varnothing,\Omega},\mathcal{F}_1=\set{\varnothing,\Omega,A_T,A_H},\mathcal{F}_2=\mathcal{G},\mathcal{F}_3=\power{\Omega}
\end{equation}
then we have
\begin{equation}
\mathcal{F}_0\subset \mathcal{F}_1\subset \mathcal{F}_2 \subset \mathcal{F}_3
\end{equation}

The following concern Borel $\sigma$-algebra
\end{minipage}}
\begin{definition}
\label{1.1.10}
\textbf{(Definition of Borel-Algebra)} The Borel-Algebra on $\R$, which we denote  $\mathcal{B}(\R)$ is the $\sigma$-algebra generated by all open interval of $\R$.\\

Some members of $\mathcal{B}(\R):$ 
 \begin{equation}
   (b,a),(a,\infty),\R
\end{equation}
\begin{equation}
  (b,a]=(b,\infty)\setminus (a,\infty)
\end{equation}
\begin{equation}
[a,\infty)=\R\setminus (-\infty, a) 
\end{equation}
\begin{equation}
[a,b]=[a,\infty)\setminus (b,\infty) 
\end{equation}
\begin{equation}
\set{a}=\R \setminus (-\infty, a)\cup (a,\infty)
\end{equation}
\end{definition}
\fbox{\begin{minipage}{39em}
    Some members of $\mathcal{B}(\R)$ : $(b,a),(a,\infty),(b,a],(-\infty,a),[a,\infty),[a,b]$
\end{minipage}}
\begin{definition}
\label{1.1.11}
\textbf{(Definition of a Random Variable)} We say the function from $\Omega$ to $\R$ is a random variable.
\end{definition}
\fbox{\begin{minipage}{39em}
We now define 3 random variable for example from the last example of $\sigma$-algebra.\\

Let $S_0,u,d\inr^+$ and let $d<1<u$. We define three random variables $S_1,S_2,S_3$ on $\Omega_3$
 \begin{equation}
S_1(\omega)= \begin{cases}
  uS_0& \text{ if $\omega \in A_H$ }\\
  dS_0& \text{ if $\omega\inA_T$ }
\end{cases}S_2(\omega)=\begin{cases}
  u^2S_0& \text{ if $\omega\in A_{HH}$ }\\
  udS_0& \text{ if $\omega\in A_{HT}\cup A_{TH}$ }\\
  d^2S_0& \text{ if $\omega\in A_{TT}$ }
\end{cases}
\end{equation}
\begin{equation}
S_3(\omega)=\begin{cases}
  u^3S_0& \text{ if $\omega\in \set{HHH}$ }\\
  u^2dS_0& \text{ if $\omega\in \set{HHT,HTH,THH}$ }\\
  ud^2S_0& \text{ if $\omega\in \set{HTT,THT,TTH}$ }\\
  d^3S_0& \text{ if $\omega\in \set{TTT}$ }
\end{cases}
\end{equation}
Often, we just use $S$ to denote  $S(\omega)$. 
\end{minipage}}
\begin{theorem}
\textbf{(Construct  $\sigma$-Algebra with Random Variable)} Let $X$ be a random variable on  $\Omega$. We define
\begin{equation}
X^{-1}[B]=\set{\omega \in \Omega : X(\omega)\in B}
\end{equation}
and define the $\sigma$-algebra $\sigma(X)$ by
\begin{equation}
\sigma(X)=\set{X^{-1}[B]:B\in\mathcal{B}(\R)}
\end{equation}
We can verify $\sigma(X)$ is a $\sigma$-algebra.
\end{theorem}
\begin{proof}
  \red{missed}
\end{proof}
\fbox{\begin{minipage}{39em}
Notice $\sigma(S_1)=\mathcal{F}_1,\sigma(S_2)\neq \mathcal{F}_2,\sigma(S_3)\neq \mathcal{F}_3$
\end{minipage}}
\end{document}
