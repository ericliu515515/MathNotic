\documentclass{report}
%%%%%%%%%%%%%% preamble.tex %%%%%%%%%%%%%%
\usepackage[T1]{fontenc}
\usepackage{etoolbox}
% Page Setup
\usepackage[letterpaper, tmargin=2cm, rmargin=0.5in, lmargin=0.5in, bmargin=80pt, footskip=.2in]{geometry}
\usepackage{adjustbox}
\usepackage{graphicx}
\usepackage{tikz}
\usepackage{mathrsfs}
\usepackage{mdframed}

% Create a new toggle
\newtoggle{firstsection}

% Redefine the \chapter command to reset the toggle for each new chapter
\let\oldchapter\chapter
\renewcommand{\chapter}{\toggletrue{firstsection}\oldchapter}

% Redefine the \section command to check the toggle
\let\oldsection\section
\renewcommand{\section}{
    \iftoggle{firstsection}
    {\togglefalse{firstsection}} % If it's the first section, just switch off the toggle for next sections
    {\clearpage} % If it's not the first section, start a new page
    \oldsection
}

% Abstract Design

\usepackage{lipsum}

\renewenvironment{abstract}
 {% Start of environment
  \quotation
  \small
  \noindent
  \rule{\linewidth}{.5pt} % Draw the rule to match the linewidth
  \par\smallskip
  {\centering\bfseries\abstractname\par}\medskip
 }
 {% End of environment
  \par\noindent
  \rule{\linewidth}{.5pt} % Ensure the closing rule also matches
  \endquotation
 }

% Mathematics
\usepackage{amsmath,amsfonts,amsthm,amssymb,mathtools}
\usepackage{xfrac}
\usepackage[makeroom]{cancel}
\usepackage{enumitem}
\usepackage{nameref}
\usepackage{multicol,array}
\usepackage{tikz-cd}
\usepackage{array}
\usepackage{multirow}% http://ctan.org/pkg/multirow
\usepackage{graphicx}

% Colors
\usepackage[dvipsnames]{xcolor}
\definecolor{myg}{RGB}{56, 140, 70}
\definecolor{myb}{RGB}{45, 111, 177}
\definecolor{myr}{RGB}{199, 68, 64}
% Define more colors here...
\definecolor{olive}{HTML}{6B8E23}
\definecolor{orange}{HTML}{CC5500}
\definecolor{brown}{HTML}{8B4513}
% Hyperlinks
\usepackage{bookmark}
\usepackage[colorlinks=true,linkcolor=blue,urlcolor=blue,citecolor=blue,anchorcolor=blue]{hyperref}
\usepackage{xcolor}
\hypersetup{
    colorlinks,
    linkcolor={red!50!black},
    citecolor={blue!50!black},
    urlcolor={blue!80!black}
}

% Text-related
\usepackage{blindtext}
\usepackage{fontsize}
\changefontsize[14]{14}
\setlength{\parindent}{0pt}
\linespread{1.2}

% Theorems and Definitions
\usepackage{amsthm}
\renewcommand\qedsymbol{$\blacksquare$}

% Define a new theorem style
\newtheoremstyle{mytheoremstyle}% name
  {}% Space above
  {}% Space below
  {}% Body font
  {}% Indent amount
  {\bfseries}% Theorem head font
  {.}% Punctuation after theorem head
  {.5em}% Space after theorem head
  {}% Theorem head spec (can be left empty, meaning ‘normal’)

% Apply the new theorem style to theorem-like environments
\theoremstyle{mytheoremstyle}

\newtheorem{theorem}{Theorem}[section]  
\newtheorem{definition}[theorem]{Definition} 
\newtheorem{lemma}[theorem]{Lemma}  
\newtheorem{corollary}[theorem]{Corollary}
\newtheorem{axiom}[theorem]{Axiom}
\newtheorem{example}[theorem]{Example}
\newtheorem{equiv_def}[theorem]{Equivalent Definition}

% tcolorbox Setup
\usepackage[most,many,breakable]{tcolorbox}
\tcbuselibrary{theorems}

% Define custom tcolorbox environments here...

%================================
% EXAMPLE BOX
%================================
% After you have defined the style and other theorem environments
\definecolor{myexamplebg}{RGB}{245, 245, 245} % Very light grey for background
\definecolor{myexamplefr}{RGB}{120, 120, 120} % Medium grey for frame
\definecolor{myexampleti}{RGB}{60, 60, 60}    % Darker grey for title

\newtcbtheorem[]{Example}{Example}{
    colback=myexamplebg,
    breakable,
    colframe=myexamplefr,
    coltitle=myexampleti,
    boxrule=1pt,
    sharp corners,
    detach title,
    before upper=\tcbtitle\par\vspace{-20pt}, % Reduced the space after the title
    fonttitle=\bfseries,
    description font=\mdseries,
    separator sign none,
    description delimiters={}{}, % No delimiters around the title
}{ex}
%================================
% Solution BOX
%================================
\makeatletter
\newtcolorbox{solution}{enhanced,
	breakable,
	colback=white,
	colframe=myg!80!black,
	attach boxed title to top left={yshift*=-\tcboxedtitleheight},
	title=Solution,
	boxed title size=title,
	boxed title style={%
			sharp corners,
			rounded corners=northwest,
			colback=tcbcolframe,
			boxrule=0pt,
		},
	underlay boxed title={%
			\path[fill=tcbcolframe] (title.south west)--(title.south east)
			to[out=0, in=180] ([xshift=5mm]title.east)--
			(title.center-|frame.east)
			[rounded corners=\kvtcb@arc] |-
			(frame.north) -| cycle;
		},
}
\makeatother

% %================================
% % Question BOX
% %================================
\makeatletter
\newtcbtheorem{question}{Question}{enhanced,
	breakable,
	colback=white,
	colframe=myb!80!black,
	attach boxed title to top left={yshift*=-\tcboxedtitleheight},
	fonttitle=\bfseries,
	title={#2},
	boxed title size=title,
	boxed title style={%
			sharp corners,
			rounded corners=northwest,
			colback=tcbcolframe,
			boxrule=0pt,
		},
	underlay boxed title={%
			\path[fill=tcbcolframe] (title.south west)--(title.south east)
			to[out=0, in=180] ([xshift=5mm]title.east)--
			(title.center-|frame.east)
			[rounded corners=\kvtcb@arc] |-
			(frame.north) -| cycle;
		},
	#1
}{question}
\makeatother

%%%%%%%%%%%%%%%%%%%%%%%%%%%%%%%%%%%%%%%%%%%
% TABLE OF CONTENTS
%%%%%%%%%%%%%%%%%%%%%%%%%%%%%%%%%%%%%%%%%%%


\usepackage{tikz}
\definecolor{doc}{RGB}{0,60,110}
\usepackage{titletoc}
\contentsmargin{0cm}
\titlecontents{chapter}[14pc]
{\addvspace{30pt}%
	\begin{tikzpicture}[remember picture, overlay]%
		\draw[fill=doc!60,draw=doc!60] (-7,-.1) rectangle (-0.9,.5);%
		\pgftext[left,x=-5.5cm,y=0.2cm]{\color{white}\Large\sc\bfseries Chapter\ \thecontentslabel};%
	\end{tikzpicture}\color{doc!60}\large\sc\bfseries}%
{}
{}
{\;\titlerule\;\large\sc\bfseries Page \thecontentspage
	\begin{tikzpicture}[remember picture, overlay]
		\draw[fill=doc!60,draw=doc!60] (2pt,0) rectangle (4,0.1pt);
	\end{tikzpicture}}%
\titlecontents{section}[3.7pc]
{\addvspace{2pt}}
{\contentslabel[\thecontentslabel]{3pc}}
{}
{\hfill\small \thecontentspage}
[]
\titlecontents*{subsection}[3.7pc]
{\addvspace{-1pt}\small}
{}
{}
{\ --- \small\thecontentspage}
[ \textbullet\ ][]

\makeatletter
\renewcommand{\tableofcontents}{
	\chapter*{%
	  \vspace*{-20\p@}%
	  \begin{tikzpicture}[remember picture, overlay]%
		  \pgftext[right,x=15cm,y=0.2cm]{\color{doc!60}\Huge\sc\bfseries \contentsname};%
		  \draw[fill=doc!60,draw=doc!60] (13,-.75) rectangle (20,1);%
		  \clip (13,-.75) rectangle (20,1);
		  \pgftext[right,x=15cm,y=0.2cm]{\color{white}\Huge\sc\bfseries \contentsname};%
	  \end{tikzpicture}}%
	\@starttoc{toc}}
\makeatother

\newcommand{\liff}{\llap{$\iff$}}
\newcommand{\rap}[1]{\rrap{\text{ (#1)}}}
\newcommand{\red}[1]{\textcolor{red}{#1}}
\newcommand{\blue}[1]{\textcolor{blue}{#1}}
\newcommand{\vi}[1]{\textcolor{violet}{#1}}
\newcommand{\olive}[1]{\textcolor{olive}{#1}}
\newcommand{\teal}[1]{\textcolor{teal}{#1}}
\newcommand{\brown}[1]{\textcolor{brown}{#1}}
\newcommand{\orange}[1]{\textcolor{orange}{#1}}
\newcommand{\tCaC}{\text{ \CaC }}
\newcommand{\CaC}{\red{CaC} }
\newcommand{\As}[1]{Assume \red{#1}}
\newcommand{\vdone}{\vi{\text{ (done) }}}
\newcommand{\bdone}{\blue{\text{ (done) }}}
\newcommand{\tdone}{\teal{\text{ (done) }}}
\newcommand{\odone}{\olive{\text{ (done) }}}
\newcommand{\bodone}{\brown{\text{ (done) }}}
\newcommand{\ordone}{\orange{\text{ (done) }}}
\newcommand{\ld}{\lambda}
\newcommand{\vecta}[1]{\textbf{#1}}
\newcommand{\set}[1]{\left\{ #1 \right\}}
\newcommand{\bset}[1]{\Big\{ #1 \Big\}}
\newcommand{\inR}{\in\R}
\newcommand{\inn}{\in\N}
\newcommand{\inz}{\in\Z}
\newcommand{\inr}{\in\R}
\newcommand{\inc}{\in\C}
\newcommand{\inq}{\in\Q}
\newcommand{\norm}[1]{\| #1 \|}
\newcommand{\bnorm}[1]{\Big\| #1 \Big\|}
\newcommand{\gen}[1]{\langle #1 \rangle}
\newcommand{\abso}[1]{\left|#1\right|}
\newcommand{\myref}[2]{\hyperref[#2]{#1\ \ref*{#2}}}
\newcommand{\customref}[2]{\hyperref[#1]{#2}}
\newcommand{\power}[1]{\mathcal{P}(#1)}
\newcommand{\dcup}{\mathbin{\dot{\cup}}}
\newcommand{\diam}[1]{\text{diam}\, #1}
\newcommand{\at}{\Big|}
\newcommand{\quotient}{\diagup}
\let\originalphi\phi % Store the original \phi in \originalphi
\renewcommand{\phi}{\varphi} % Redefine \phi to \varphi
\newcommand{\pfi}{\originalphi} % Define \pfi to display the original \phi
\newcommand{\diota}{\dot{\iota}}
\newcommand{\Log}{\operatorname{Log}}
\newcommand{\id}{\text{\textbf{id}}}
\usepackage{amsmath}

\makeatletter
\NewDocumentCommand{\extp}{e{^}}{%
  \mathop{\mathpalette\extp@{#1}}\nolimits
}
\NewDocumentCommand{\extp@}{mm}{%
  \bigwedge\nolimits\IfValueT{#2}{^{\extp@@{#1}#2}}%
  \IfValueT{#1}{\kern-2\scriptspace\nonscript\kern2\scriptspace}%
}
\newcommand{\extp@@}[1]{%
  \mkern
    \ifx#1\displaystyle-1.8\else
    \ifx#1\textstyle-1\else
    \ifx#1\scriptstyle-1\else
    -0.5\fi\fi\fi
  \thinmuskip
}
\makeatletter
\usepackage{pifont}
\makeatletter
\newcommand\Pimathsymbol[3][\mathord]{%
  #1{\@Pimathsymbol{#2}{#3}}}
\def\@Pimathsymbol#1#2{\mathchoice
  {\@Pim@thsymbol{#1}{#2}\tf@size}
  {\@Pim@thsymbol{#1}{#2}\tf@size}
  {\@Pim@thsymbol{#1}{#2}\sf@size}
  {\@Pim@thsymbol{#1}{#2}\ssf@size}}
\def\@Pim@thsymbol#1#2#3{%
  \mbox{\fontsize{#3}{#3}\Pisymbol{#1}{#2}}}
\makeatother
% the next two lines are needed to avoid LaTeX substituting upright from another font
\input{utxmia.fd}
\DeclareFontShape{U}{txmia}{m}{n}{<->ssub * txmia/m/it}{}
% you may also want
\DeclareFontShape{U}{txmia}{bx}{n}{<->ssub * txmia/bx/it}{}
% just in case
%\DeclareFontShape{U}{txmia}{l}{n}{<->ssub * txmia/l/it}{}
%\DeclareFontShape{U}{txmia}{b}{n}{<->ssub * txmia/b/it}{}
% plus info from Alan Munn at https://tex.stackexchange.com/questions/290165/how-do-i-get-a-nicer-lambda?noredirect=1#comment702120_290165
\newcommand{\pilambdaup}{\Pimathsymbol[\mathord]{txmia}{21}}
\renewcommand{\lambda}{\pilambdaup}
\renewcommand{\tilde}{\widetilde}
\DeclareMathOperator*{\esssup}{ess\,sup}
\newcommand{\bluecheck}{}%
\DeclareRobustCommand{\bluecheck}{%
  \tikz\fill[scale=0.4, color=blue]
  (0,.35) -- (.25,0) -- (1,.7) -- (.25,.15) -- cycle;%
}


\usepackage{tikz}
\newcommand*{\DashedArrow}[1][]{\mathbin{\tikz [baseline=-0.25ex,-latex, dashed,#1] \draw [#1] (0pt,0.5ex) -- (1.3em,0.5ex);}}

\newcommand{\C}{\mathbb{C}}	
\newcommand{\F}{\mathbb{F}}
\newcommand{\N}{\mathbb{N}}
\newcommand{\Q}{\mathbb{Q}}
\newcommand{\R}{\mathbb{R}}
\newcommand{\Z}{\mathbb{Z}}



\title{\Huge{HWs}}
\author{\huge{Eric Liu}}
\date{}
\begin{document}
\maketitle
\newpage% or \cleardoublepage
% \pdfbookmark[<level>]{<title>}{<dest>}
\pdfbookmark[section]{\contentsname}{toc}
\pagebreak
\chapter{Differential Geometry HW} 
\section{HW 3}
\begin{question}{}{}
Let $V$ be a finite dimensional vector space over $\R$. Show that for 
 \begin{align*}
\operatorname{dim}(V)<4
\end{align*}
Every non-zero element of $\extp^2(V)$ can be expressed as a wedge product of two vectors in $V$.  Give an example to show that this is not true if $\operatorname{dim}(V)=4$. 
\end{question}
\begin{theorem}
\textbf{(Case of Zero and One Dimension)} If 
\begin{align*}
\operatorname{dim}(V)\leq 1
\end{align*}
Then every non-zero element of $\extp^2(V)$ can be expressed as a wedge product of two vectors in $V$. 
\end{theorem}
\begin{proof}
Recall
\begin{align*}
\operatorname{dim}\Big( \extp^2(V) \Big)= \binom{\operatorname{dim}(V)}{2}=0
\end{align*}
This implies $\extp^2(V)=0$. There does not exists non-zero element of $\extp^2(V)$, rendering the proposition viciously true. 
\end{proof}
\begin{theorem}
\textbf{(Case of Two Dimension)} If 
\begin{align*}
\operatorname{dim}(V)=2
\end{align*}
Then every non-zero element of $\extp^2(V)$ can be expressed as a wedge product of two vectors in $V$. 
\end{theorem}
\begin{proof}
Let  $\set{e_1,e_2}$ be a basis for $V$.  We have 
\begin{align*}
\extp^2(V)=\operatorname{span} \set{e_1 \wedge  e_2}
\end{align*}
Therefore, for all $\omega \in \extp^2(V)$, we have 
\begin{align*}
\omega  = c(e_1 \wedge  e_2)=(ce_1) \wedge  e_2 \text{ for some }c\inr  
\end{align*}
\end{proof}
\begin{theorem}
\textbf{(Case of Three Dimensions)} If 
\begin{align*}
  \set{e_1,e_2,e_3}\text{ is a basis for }V
\end{align*}
Then every non-zero element of $\extp^2(V)$ can be expressed as a wedge product of two vectors in $V$. 
\end{theorem}
\begin{proof}
We know $\extp^2(V)$ have the following basis  
\begin{align*}
\set{e_1\wedge  e_2, e_1 \wedge  e_3, e_2 \wedge  e_3} 
\end{align*}
Therefore, for arbitrary $\omega  \in \extp^2(V)$, we may express 
\begin{align*}
\omega = \omega_1 (e_1 \wedge e_2)+ \omega_2 (e_1 \wedge  e_3) + \omega_3 (e_2 \wedge  e_3)  \text{ for some }\omega_1,\omega_2,\omega_3 \inr
\end{align*}
Write $\textbf{x}=(\omega_3,-\omega_2,\omega_1)\inr^3$. By premise, $\textbf{x}\neq \textbf{0}$. Using Gram-Schmidt algorithm, we know there exists some $\textbf{a}=(a_1,a_2,a_3),\textbf{b}=(b_1,b_2,b_3)\inr^3$ such that 
\begin{align*}
\abso{\textbf{a}}=\abso{\textbf{b}}=1 \text{ and }\set{\textbf{x},\textbf{a},\textbf{b}}\text{ are orthogonal }
\end{align*}
The orthogonality of $\set{\textbf{x},\textbf{a},\textbf{b}}$ implies 
\begin{align*}
\textbf{x}= c \textbf{a}\times \textbf{b}\text{ for some }c\inr
\end{align*}
Explicitly, 
\begin{align*}
\begin{cases}
  \omega_1 = \textbf{x}_3 = c(a_1b_2-a_2b_1) \\
  \omega_2= - \textbf{x}_2= c(a_1b_3-a_3b_1) \\
  \omega_3= \textbf{x}_1= c(a_2b_3-a_3b_2)
\end{cases}
\end{align*}
We now see 
\begin{align*}
  &[c(a_1e_1+a_2e_2+a_3e_3)]\wedge  (b_1e_1+b_2e_2+b_3e_3)\\
  &= c (a_1b_2-a_2b_1) (e_1 \wedge  e_2) + c(a_1b_3 -a_3b_1)(e_1 \wedge  e_3 )+ c(a_2b_3 -a_3b_2)(e_2\wedge  e_3 ) \\
  &=\omega_1 (e_1 \wedge  e_2 )+ \omega_2 (e_1 \wedge  e_3 )+ \omega_3(e_2\wedge  e_3)=\omega
\end{align*}
We have shown 
\begin{align*}
\omega= (ca_1e_2+ca_2e_2+ca_3e_3)\wedge (b_1e_1+b_2e_2+b_3e_3) 
\end{align*}
That is, $\omega$ can indeed be expressed as a wedge product of two vectors in $V$. 

\end{proof}
\begin{theorem}
\textbf{(Case of Four Dimensions)} If 
\begin{align*}
\set{e_1,e_2,e_3,e_4}\text{ is a basis for }V
\end{align*}
Then $e_1 \wedge  e_2 +e_3 \wedge  e_4 $ can not be expressed as a wedge product of two vectors in $V$. 
\end{theorem}
\begin{proof}
\As{for a contradiction that for some $(a_1,a_2,a_3,a_4),(b_1,b_2,b_3,b_4)\inr^4$, we have} 
 \begin{align}
\label{gi}
 \red{e_1 \wedge  e_2 + e_3 \wedge  e_4  = (a_1e_1+a_2e_2+a_3e_3+a_4e_4) \wedge  (b_1e_1+b_2e_2+b_3e_3+b_4e_4)}
\end{align}
Equating the coefficients of $e_1 \wedge  e_2$, we have  
\begin{align*}
  a_1b_2-a_2b_1 =1
\end{align*}
This implies that one of $a_1,b_1$ is non-zero. WLOG, suppose  $a_1\neq 0$. Now, equating the coefficients of $e_1 \wedge  e_3$ and $e_1 \wedge  e_4$, we have 
\begin{align*}
a_1b_3-a_3b_1=0=a_1b_4-a_4b_1
\end{align*}
Dividing $a_1$, we may deduce 
 \begin{align*}
b_3= \frac{a_3b_1}{a_1} \text{ and }b_4= \frac{a_4b_1}{a_1}
\end{align*}
Therefore, the coefficients of $e_3 \wedge  e_4$ in the right side expression of \myref{Equation}{gi} is 
\begin{align*}
a_3b_4-a_4b_3= \frac{a_3a_4b_1}{a_1}- \frac{a_4a_3b_1}{a_1}=0
\end{align*}
which does not equals to $1$, the coefficient of $e_3 \wedge  e_4$ in the left side expression of \myref{Equation}{gi}. This cause a contradiction. 
\end{proof}
\begin{question}{}{}
Let $\alpha $ be the $1$-form  $dz+xdy$  on  $\R^3$. 
 \begin{enumerate}[label=(\alph*)]
  \item Find a basis for $\operatorname{Ker}\alpha $. 
  \item Compute $\alpha \wedge  d\alpha  $. 
  \item Find the vector field  $R$ that satisfies  $\alpha (R)=1$ and $\iota_R d\alpha =0$.
  \item Let $R$ be the same vector filed in  (c), and let  $\phi_t:\R^3 \rightarrow \R^3$ denote its flows. Compute $\mathcal{L}_R\alpha $ and $\phi_t^*\alpha  $ for all fixed $t$.  
\end{enumerate}
\end{question}
\begin{theorem}
  (a) For all $\textbf{x}=(x,y,z)\in \R^3$, the kernel of $\alpha _{\textbf{x}}:T_\textbf{x}\R^3 \rightarrow \R$ has the basis 
\begin{align*}
\bset{\frac{\partial }{\partial x}\Big|_{\textbf{x}}, \frac{\partial }{\partial y}\Big|_{\textbf{x}}-x \frac{\partial }{\partial z}\Big|_{\textbf{x}}}
\end{align*}
\end{theorem}
\begin{proof}
Let 
\begin{align*}
 c_1\frac{\partial }{\partial x}\Big|_{\textbf{x}}+c_2 \frac{\partial }{\partial y}\Big|_{\textbf{x}}+ c_3\frac{\partial }{\partial z}\Big|_{\textbf{x}} \in \operatorname{Ker}\alpha _\textbf{x}
\end{align*}
Compute 
\begin{align*}
0=\alpha_{\textbf{x}}\Big( c_1\frac{\partial }{\partial x}\Big|_{\textbf{x}}+c_2 \frac{\parial }{\partial y}\Big|_{\textbf{x}}+ c_3\frac{\partial }{\partial z}\Big|_{\textbf{x}}  \Big)&=  (dz+xdy) \Big( c_1\frac{\partial }{\partial x}\Big|_{\textbf{x}}+c_2 \frac{\partial }{\partial y}\Big|_{\textbf{x}}+ c_3\frac{\partial }{\partial z}\Big|_{\textbf{x}}  \Big) \\
&= c_3 + x c_2
\end{align*}
This implies 
\begin{align*}
\frac{\partial }{\partial x}\Big|_{\textbf{x}}, \frac{\partial }{\partial y}\Big|_{\textbf{x}}-x \frac{\partial }{\partial z}\Big|_{\textbf{x}} \in \operatorname{Ker}\alpha_{\textbf{x}}
\end{align*}
Because 
\begin{align*}
\alpha_{\textbf{x}} \frac{\partial }{\partial z}\Big|_{\textbf{x}} =1
\end{align*}
We know $\operatorname{Im}(\alpha_{\textbf{x}})=\R$. Therefore, 
\begin{align*}
\operatorname{dim}(\operatorname{Ker}\alpha_{\textbf{x}})=3- \operatorname{dim}(\operatorname{Im}\alpha _{\textbf{x}})= 2
\end{align*}
It is clear that 
\begin{align*}
  \bset{\frac{\partial }{\partial x}\Big|_{{\textbf{x}}}, \frac{\partial }{\partial y}\Big|_{\textbf{x}}-x \frac{\partial }{\partial z}\Big|_{\textbf{x}}}\subseteq \operatorname{Ker}\alpha _\textbf{x}\text{ is linearly independent }
\end{align*}
It then follows that 
\begin{align*}
  \bset{\frac{\partial }{\partial x}\Big|_{\textbf{x}}, \frac{\partial }{\partial y}\Big|_{\textbf{x}}-x \frac{\partial }{\partial z}\Big|_{\textbf{x}}} \text{ is indeed a basis for }\operatorname{Ker}\alpha _{\textbf{x}}
\end{align*}
\end{proof}
\begin{theorem}
  (b)
\begin{align*}
\alpha \wedge  d \alpha  = dx \wedge  dy \wedge  dz   
\end{align*}
\end{theorem}
\begin{proof}
Compute 
\begin{align*}
d\alpha &= d(dz+xdy)\\
&=d^2z+dx\wedge  dy+ xd^2y \\
&=dx\wedge  dy   
\end{align*}
Compute 
\begin{align*}
\alpha \wedge  d\alpha &= (dz+xdy)\wedge (dx \wedge  dy )\\
&= dz\wedge  dx \wedge  dy + x dy \wedge  dx\wedge  dy \\
&=dx\wedge  dy \wedge  dz       
\end{align*}
\end{proof}
\begin{theorem}
 (c)
\begin{align*}
R\triangleq  \frac{\partial }{\partial z}\text{ is the unique vector filed that satisfies }\alpha (R)=1 \text{ and }\iota_R d\alpha =0 
\end{align*}
\end{theorem}
\begin{proof}
Suppose 
\begin{align*}
R\triangleq R^1 \frac{\partial }{\partial x}+ R^2 \frac{\partial }{\partial y}+ R^3 \frac{\partial }{\partial z}\text{ satisfies }\alpha (R)=1 \text{ and }\iota_R d\alpha =0 
\end{align*}
For all $V \in \mathfrak{X}(\R^3)$, if we write 
\begin{align*}
V=V^1 \frac{\partial }{\partial x}+ V^2 \frac{\partial }{\partial y}+ V^3 \frac{\partial }{\partial z}
\end{align*}
Then 
\begin{align}
\label{sg}
\begin{vmatrix} 
  R^1 & V^1 \\
  R^2 & V^2
\end{vmatrix}=\begin{vmatrix} 
  dx R & dx V\\
  dy R & dy V
\end{vmatrix}&= (dx \wedge  dy)(R,V) \\
&=d\alpha (R,V)=\iota_R d\alpha (V)  =0 \notag 
\end{align}
If any of $R^1$ or $R^2$ is non-zero at some point $p \inr^3$, by setting $V^1=-R^2$ and $V_2=R^1$ at  $p$ we have   
\begin{align*}
\begin{vmatrix} 
  R^1 & V^1 \\
  R^2 & V^2 
\end{vmatrix}\text{ is non-zero at $p$ }
\end{align*}
which contradicts to \myref{Equation}{sg}. Therefore, we must have $R^1=R^2=0$ on  $\R^3$. We may now compute 
 \begin{align*}
1=\alpha (R)= (dz+xdy) (R^3 \frac{\partial }{\partial z})= R^3
\end{align*}
We may now conclude 
\begin{align*}
  R=  \frac{\partial }{\partial z}
\end{align*}

\end{proof}
\begin{theorem}
  (d) For all fixed $t$, 
\begin{align*}
\phi_t^* \alpha = \alpha 
\end{align*}
And 
\begin{align*}
\mathcal{L}_R \alpha =0
\end{align*}
\end{theorem}
\begin{proof}
Fix $t$. Obviously,  
\begin{align*}
\phi_t (x,y,z)= (x,y,z+t)
\end{align*}
Let $p=(x_0,y_0,z_0)\inr^3$ and 
\begin{align*}
v= v^1\frac{\partial }{\partial x}\Big|_{p}+ v^2\frac{\partial }{\partial y}\Big|_{p}+v^3 \frac{\partial }{\partial z}\Big|_{p} \in T_p\R^3
\end{align*}
Denote $(x_0,y_0,z_0+t)$ by $q$.  Compute  
\begin{align*}
  (\phi_t^*\alpha )_{p}(v)&=\alpha_{q}((\phi_t)_{*,p}v)  \\
  &=\alpha_{q} \Big( v^1\frac{\partial }{\partial x}\Big|_{q}+ v^2\frac{\partial }{\partial y}\Big|_{q}+v^3 \frac{\partial }{\partial z}\Big|_{q} \Big) \\
  &= (dz+x_0dy) \Big( v^1\frac{\partial }{\partial x}\Big|_{q}+ v^2\frac{\partial }{\partial y}\Big|_{q}+v^3 \frac{\partial }{\partial z}\Big|_{q} \Big)  \\
  &= v^3 + x_0v^2\\
  &=(dz+x_0dy)\Big( v^1\frac{\partial }{\partial x}\Big|_{p}+ v^2\frac{\partial }{\partial y}\Big|_{p}+v^3 \frac{\partial }{\partial z}\Big|_{p}  \Big) \\
  &=\alpha_p (v)
\end{align*}
We have shown $(\phi_t^* \alpha )_p=\alpha _p$. Because $p$ is arbitrary, this implies  $\phi_t^*\alpha =\alpha $. We may now compute 
\begin{align*}
\mathcal{L}_R \alpha =\lim_{t\to 0} \frac{(\phi_t^*\alpha )_p-\alpha_p}{t}= \lim_{t\to 0} \frac{0}{t}=0
\end{align*}
\end{proof}
\begin{question}{}{}
Orient $S^n$ in  $\R^{n+1}$ as the boundary of the unit closed ball. 
\begin{enumerate}[label=(\alph*)]
  \item Show that a volume form on $S^n$ is 
     \begin{align*}
    \omega = \sum_{i=1}^{n+1}(-1)^{i-1}\textbf{x}^i d\textbf{x}^1 \wedge \cdots \wedge  \widehat{d\textbf{x}^i} \wedge  \cdots \wedge d\textbf{x}^{n+1}       
    \end{align*}
  where the caret $\text{  }\widehat{}\text{  }$ over $d\textbf{x}^i$ indicates that $d\textbf{x}^i$ is to be omitted. 
  \item Show that on $S^2$ 
 \begin{align*}
 \omega= \begin{cases}
   \frac{dy \wedge  dz}{x}\text{ for }x\neq 0\\
   \frac{dz \wedge  dx}{y}\text{ for }y\neq 0\\
   \frac{dx \wedge  dy}{z}\text{ for }z\neq 0
 \end{cases}
 \end{align*}
 \item Calculate $\int_{S_2}\omega$
\end{enumerate}
\end{question}
\begin{theorem}
 (a) Show that a volume form on $S^n$ is 
     \begin{align*}
    \omega = \sum_{i=1}^{n+1}(-1)^{i-1}\textbf{x}^i d\textbf{x}^1 \wedge \cdots \wedge  \widehat{d\textbf{x}^i} \wedge  \cdots \wedge d\textbf{x}^{n+1}       
    \end{align*}
  where the caret $\text{  }\widehat{}\text{  }$ over $d\textbf{x}^i$ indicates that $d\textbf{x}^i$ is to be omitted. 

\end{theorem}
\begin{proof}
Let $\diota : S^{n}\rightarrow \R^{n+1}$ be the inclusion map and define $V \in \mathfrak{X}(\R^{n+1})$ by 
\begin{align*}
V_{\textbf{y}}\triangleq  \sum_{i=1}^{n+1} \textbf{y}^i \frac{\partial }{\partial \textbf{x}^i}\Big|_\textbf{y}
\end{align*}
So that $V$ is nowhere tangent to  $S^{n}$. By Proposition 15.21 of "Introduction to Smooth Manifold" by John Lee, we know  
\begin{align*}
\diota^* (\iota_V d\textbf{x}^1 \wedge  \cdots \wedge  d\textbf{x}^{n+1}   ) \text{ is a volume form on }S^n
\end{align*}
Compute
\begin{align*}
\diota^*   (\iota_V d\textbf{x}^1\wedge  \cdots \wedge  d\textbf{x}^{n+1}  )&=\diota^* \Big(\sum_{i=1}^{n+1}(-1)^{i-1}V^i d\textbf{x}^1 \wedge  \cdots \wedge   \widehat{d\textbf{x}^i} \wedge  \cdots \wedge  d\textbf{x}^{n+1}    \Big)   \\
&=\sum_{i=1}^{n+1}(-1)^{i-1} (V^i \circ \diota) d(\textbf{x}^1\circ \diota) \wedge  \cdots \wedge \widehat{d\textbf{x}^i} \wedge  \cdots \wedge  d(\textbf{x}^{n+1}\circ \diota)       \\
&= \sum_{i=1}^{n+1}(-1)^{i-1}V^i d\textbf{x}^1 \wedge  \cdots \wedge  \widehat{d\textbf{x}^i} \wedge  \cdots \wedge d\textbf{x}^{n+1}   \\
&= \sum_{i=1}^{n+1}(-1)^{i-1}\textbf{x}^i d\textbf{x}^1 \wedge  \cdots \wedge  \widehat{d\textbf{x}^i} \wedge  \cdots \wedge d\textbf{x}^{n+1}=\omega
\end{align*}
We have shown 
\begin{align*}
\omega = \diota^*(\iota_V d\textbf{x}^1 \wedge  \cdots \wedge  d\textbf{x}^{n+1}   )   
\end{align*}
This implies $\omega$ is indeed a volume form on $S^n$.  
\end{proof}
\begin{theorem}
 (b) Show that on $S^2$, 
 \begin{align*}
 \omega= \begin{cases}
   \frac{dy \wedge  dz}{x}\text{ for }x\neq 0\\
   \frac{dz \wedge  dx}{y}\text{ for }y\neq 0\\
   \frac{dx \wedge  dy}{z}\text{ for }z\neq 0
 \end{cases}
 \end{align*}
\end{theorem}
\begin{proof}
Define  $f\in \Omega^0(\R^3)$ by 
\begin{align*}
f(x,y,z)\triangleq  \sqrt{x^2+y^2+z^2} 
\end{align*}
So we have 
\begin{align*}
df= \frac{\partial f}{\partial x}dx+ \frac{\partial f}{\partial y}dy + \frac{\partial f}{\partial z}dz
\end{align*}
Let $\diota:S^2\rightarrow \R^3$ be the inclusion map. Because $f \circ \diota:S^2\rightarrow \R^3$ is constant  $1$, we may compute 
\begin{align*}
  0&=d(f\circ \diota  )=d(\diota^* f )=\diota^* (df)= \diota^*\Big( \frac{\partial f}{\partial x}dx+  \frac{\partial f}{\partial y}dy + \frac{\partial f}{\partial z}dz \Big)   \\
  &=\frac{\partial f}{\partial x}dx+ \frac{\partial f}{\partial y}dy + \frac{\partial f}{\partial z}dz \\
  &=\frac{xdx+ydy+zdz}{\sqrt{x^2+y^2+z^2} }= xdx+ydy+zdz
\end{align*}
This give us 
\begin{align*}
\begin{cases}
dx= \frac{ydy + zdz}{-x}\text{ for }x\neq 0 \\
dy= \frac{xdx+zdz}{-y}\text{ for }y\neq 0\\
dz= \frac{xdx+ydy}{-z}\text{ for }z\neq 0
\end{cases}
\end{align*}
Therefore, for $x\neq 0$ 
\begin{align*}
\omega&= xdy \wedge dz- ydx\wedge  dz + zdx \wedge  dy      \\
&= xdy \wedge  dz- y (\frac{ydy+zdz}{-x}) \wedge dz+ z (\frac{ydy+zdz}{-x})\wedge  dy  \\
&=(x+ \frac{y^2}{x}+ \frac{z^2}{x})dy \wedge  dz \\
&=\frac{(x^2+y^2+z^2)dy \wedge  dz }{x}= \frac{dy \wedge  dz}{x}
\end{align*}
Similarly, for $y\neq 0$ 
\begin{align*}
\omega&=  xdy \wedge dz- ydx\wedge  dz + zdx \wedge  dy  \\
&=x(\frac{xdx+zdz}{-y}) \wedge  dz -ydx\wedge  dz + z dx \wedge  (\frac{xdx+zdz}{-y})    \\
&= (\frac{x^2}{y} +y + \frac{z^2}{y}) dz \wedge  dx \\
&=\frac{(x^2+y^2+z^2)dz\wedge  dx }{y}= \frac{dz \wedge  dx}{y}
\end{align*}
Lastly, for $z\neq 0$ 
\begin{align*}
\omega&=   xdy \wedge dz- ydx\wedge  dz + zdx \wedge  dy      \\
&= xdy\wedge  (\frac{xdx+ydy}{-z}) -ydx \wedge   (\frac{xdx+ydy}{-z})  + z dx\wedge  dy  \\
&= (\frac{x^2}{z}+ \frac{y^2}{z}+ z)dx \wedge  dy \\
&=\frac{(x^2+y^2+z^2)dx\wedge  dy }{z}= \frac{dx \wedge  dy }{z} 
\end{align*}
\end{proof}
\begin{theorem}
  (c) If we orient $S^n$ in  $\R^{n+1}$ as the boundary of the unit closed ball, then 
\begin{align*}
\int_{S^2}\omega =4\pi  
\end{align*}
\end{theorem}
\begin{proof}
Because 
\begin{align*} 
\omega = \diota^*(\iota_V d\textbf{x}^1 \wedge  \cdots \wedge  d\textbf{x}^{n+1}   )   
\end{align*}
And because  
\begin{align*}
d\textbf{x}^1 \wedge  \cdots \wedge  d \textbf{x}^{n+1}   \text{ is a positively oriented volume form on the unit closed ball }
\end{align*}
We know $\omega$ as a volume form of $S^n$ is also positively oriented. Therefore, when we consider the chart  
\begin{align*}
U \triangleq \set{(x,y,z)\in S^2:z>0}\text{ and } \phi(x,y,z)\triangleq (x,y)
\end{align*}
And the chart 
\begin{align*}
V\triangleq \set{(x,y,z)\in S^2 :z<0}\text{ and }\psi (x,y,z)\triangleq (x,y)
\end{align*}
According to our computation in part 2, we may integrate  
\begin{align*}
\int_U \omega&= \int_{\phi (U)} \frac{1}{\sqrt{1-x^2-y^2} }dxdy \\
&=\int_{0}^1 \int_{0}^{2\pi } \frac{r}{\sqrt{1-r^2} }d\theta dr=2\pi  \int_{0}^1 \frac{r}{\sqrt{1-r^2} }dr= 2\pi  (-\sqrt{1-r^2} )\Big|_{r=0}^1=2\pi 
\end{align*}
And integrate 
\begin{align*}
\int_V \omega= -\int_{\psi (V)} \frac{1}{-\sqrt{1-x^2-y^2} }dxdy= 2\pi 
\end{align*}
Therefore, 
\begin{align*}
\int_{S^2}\omega= \int_U \omega + \int_V \omega = 4\pi 
\end{align*}
\end{proof}
\begin{question}{}{}
Let $M$ be a manifold of dimension  $n$, and  $\set{U_i}_{i \in I}$ be a countable open cover. Suppose that each $U_i$  is diffeomorphic to  $\R^n$ and all $U_{ij}\triangleq U_i \cap U_j$  and $U_{ijk}\triangleq U_i \cap U_j \cap U_k$ are either diffeomorphic to $\R^n$ or empty. Choose a total order $<$ on  $I$, and consider the following sequence of real vector space 
 \begin{align*}
\mathcal{W}_1 = \prod_{i \in I} \R \overset{\ld }{\longrightarrow} \mathcal{W}_2 = \prod_{i<j \in I;U_{ij}\neq \varnothing} \R \overset{\mu }{\longrightarrow} \mathcal{W}_3 = \prod_{i<j<k\in I;U_{ijk}\neq \varnothing} \R
\end{align*}
where the linear maps are defined by 
 \begin{align*}
&\ld :(c_i)_{i \in I}\mapsto  (c_i-c_j)_{i<j \in I;U_{ij\neq \varnothing}}  \\
&\mu: (c_{ij})_{i<j \in I;U_{ij}\neq \varnothing} \mapsto  (c_{ij}+c_{jk}-c_{ik})_{i<j<k \in I; I_{ijk}\neq \varnothing}
\end{align*}
which satisfies 
\begin{align*}
\mu \circ \ld = 0
\end{align*}
\begin{enumerate}[label=(\alph*)]
  \item Let $\alpha $ be a closed 1-form. Show that for each $i \in I$, we have $\alpha |_{U_i}=df_i$ for some smooth function $f_i:U_i\rightarrow \R$. Show that there exists a unique element $(c_{ij})$ in $\mathcal{W}_2$ with $f_i|_{U_{ij}}- f_j |_{U_{ij}}=c_{ij}$ for all $i<j,U_{ij}\neq \varnothing$. Show that $\mu ((c_{ij}))=0$. 
  \item  Show that in Part (a), the element $(c_{ij})+\operatorname{Im}\ld \in \operatorname{Ker} \mu \diagup \operatorname{Im}\ld $ is independent of the choice of $f_i$, and depend only on the cohomology class $[\alpha ]\in H^1(M)$.  \\

  From part (a) and (b), one define a linear map $\Phi:H^1(M)\rightarrow \operatorname{Ker} \mu \diagup \operatorname{Im}\ld $. \\

  \item Show that $\Phi$ is injective. 
  \item Suppose $(c_{ij})_{i<j\in I;U_{ij}\neq \varnothing}$ lies in $\operatorname{Ker}\mu \subseteq \mathcal{W}_2$. Choose a partition of unity $\set{\rho_i}_{i \in I}$ subordinate to $\set{U_i}_{i \in I}$. Define function $f_i:U_i\rightarrow \R$ by 
    \begin{align*}
    f_i \triangleq  \sum _{j\in I;i <j ,U_{ij}\neq \varnothing } c_{ij}\rho_j |_{U_i} - \sum_{j \in I, j< i, U_{ij}\neq \varnothing } c_{ji}\rho_j |_{U_i}
    \end{align*}
  Show that there exists a closed one-form $\alpha $ such that these $f_i$ and  $c_{ij}$ are possible choices in part (a). Deduce that $\Phi$ is surjective.  
  \item Show that if $M$ is compact, then  $H^1(M)$ is finite-dimensional. 
\end{enumerate}
\end{question}
\begin{mdframed}
For part (a), note that because 
\begin{enumerate}[label=(\roman*)]
  \item $U_i$ is diffeomorphic to  $\R^n$. 
  \item $\alpha $ is closed. 
  \item $H^1(\R^n)=0$ by Poincare Lemma.  
\end{enumerate}
There indeed exists smooth $f_i:U_i\rightarrow \R$ such that $\alpha |_{U_i}=df_i$. Fix such $(f_i)_{i \in I}$. Observe that for all fixed $i<j,U_{ij}\neq \varnothing$, we may compute 
\begin{align*}
d(f_i-f_j)=df_i - df_j = \alpha - \alpha =0\text{ on }U_{ij}
\end{align*}
This implies 
\begin{align*}
f_i|_{U_{ij}}-f_j|_{U_{ij}}\text{ is some unique constant }c_{ij}\text{ on }U_{ij}
\end{align*}
Fix such $(c_{ij})\in \mathcal{W}_2$. To see $\mu ((c_{ij}))=0$, fix $i<j<k, p \in U_{ijk}$ , and compute 
\begin{align*}
  c_{ij}+c_{jk}-c_{ik}&= (f_i-f_j)(p)+ (f_j-f_k)(p) - (f_i-f_k)(p)\\
  &=(f_i-f_j-f_k+f_k-f_i)(p)=0
\end{align*}
\end{mdframed}
\begin{theorem}
  (b) The map 
\begin{align*}
\alpha \mapsto (c_{ij})+ \operatorname{Im} \ld  \in \frac{\operatorname{Ker}\mu}{\operatorname{Im}\ld }
\end{align*}
is well-defined and sends closed one-forms within the same cohomology class to the same element.  
\end{theorem}
\begin{proof}
Let $\widehat{f_i} :U_i\rightarrow \R$ also satisfy $\alpha |_{U_i}=d \widehat{f_i} $, and again induce 
\begin{align*}
\widehat{c_{ij}}\triangleq \widehat{f_i}-\widehat{f_j}
\end{align*}
Because 
\begin{align*}
d(f_i-\widehat{f_i})= df_i -d \widehat{f_i}= \alpha - \alpha =0
\end{align*}
We know $f_i,\widehat{f_i}$ differ by some constant, which we denote 
\begin{align*}
c_i \triangleq  f_i - \widehat{f_i}
\end{align*}
Now, compute 
\begin{align*}
  \ld(c_i)&= (c_i-c_j)_{i<j \in I;U_{ij}\neq \varnothing}\\
  &=(f_i- \widehat{f_i}- f_j + \widehat{f_j})_{i<j \in I;U_{ij}\neq \varnothing} \\
  &=(c_{ij}- \widehat{c_{ij}})_{i<j \in I;U_{ij}\neq \varnothing} 
\end{align*}
We have shown $(\widehat{c_{ij}})_{i<j \in I; U_{ij}\neq \varnothing},(c_{ij})_{i<j \in I;U_{ij}\neq \varnothing}$ differ by $\ld  (c_i)$. That is, the map 
\begin{align*}
\alpha \mapsto (c_{ij})+ \operatorname{Im} \ld \in \frac{\operatorname{Ker}\mu}{\operatorname{Im}\ld }  \text{ is well-defined }
\end{align*}
Let $\gamma \in \Omega^1(M)$ be some exact one-form. It remains to show 
\begin{align*}
  (c_{ij})\in \operatorname{Im}\ld \text{ where }(c_{ij})\text{ is induced by }\gamma 
\end{align*}
Write $\gamma =dg$, where $g\in \Omega^0(M)$. Let $f_i:U_i\rightarrow \R$ satisfy 
\begin{align*}
\gamma|_{U_i} = df_i
\end{align*}
Because 
\begin{align*}
d(g-f_i)=dg-df_i=\gamma -\gamma =0\text{ on }U_i
\end{align*}
We know $g,f_i$ on $U_i$ differ by some constant, which we denote 
\begin{align*}
c_i \triangleq  f_i-g\text{ on }U_i
\end{align*}
Now, to close the proof, compute 
\begin{align*}
\ld ((c_i)_{i \in I})&= (c_i-c_j)_{i < j \in I;U_{ij}\neq \varnothing}  \\
&= (f_i-g - f_j+g)_{i<j \in I;U_{ij}\neq \varnothing}\\
&=(f_i-f_j)_{i<j \in I;U_{ij}\neq \varnothing}= (c_{ij})_{i<j\in I:U_{ij}\neq \varnothing}
\end{align*}
where the last inequality hold because the $(c_{ij})$ we are referring to is induced by $\gamma $. 
\end{proof}
\begin{theorem}
(c)  $\Phi$ is injective. 
\end{theorem}
\begin{proof}
Fix $[\alpha ]\in H^1(M)$, and let $(f_i:U_i\rightarrow \R),(c_{ij})\in \mathcal{W}_2$ be induced by $\alpha $. Suppose $(c_{ij})=\ld  ((c_i))$ for some $(c_i)\in \mathcal{W}_1$. We are required to show 
\begin{align*}
\alpha \text{ is exact }
\end{align*}
Define $g_i:U_i \rightarrow \R$ by 
\begin{align*}
g_i\triangleq  f_i -c_i
\end{align*}
So if $i<j$ satisfy  $U_{ij}\neq \varnothing$, we see 
\begin{align*}
g_i-g_j= (f_i-c_i)-(f_j-c_j)=(f_i-f_j)-c_{ij}=0\text{ on }U_{ij}
\end{align*}
Note that the second equality follows from $(c_{ij})= \ld  ((c_i))$ and the last  equality follows from definition of $(c_{ij})$. In summary, we have shown 
\begin{align*}
g_i=g_j\text{ on }U_{ij}
\end{align*}
Therefore, we may well define $g:M\rightarrow \R$ by 
\begin{align*}
g(p)\triangleq g_i (p)\text{ if }p \in U_i
\end{align*}
To close out the proof, observe on each $U_i$, 
\begin{align*}
\alpha = df_i = dg_i = dg
\end{align*}
where the second equality hold because $g_i,f_i$ differ by a constant. This implies 
\begin{align*}
\alpha = dg\text{ on }M
\end{align*}
We have shown $\alpha $ is exact. That is, $[\alpha ]=0$.  
\end{proof}
\begin{theorem}
\label{d}
(d) Let $(c_{ij})\in \operatorname{Ker}\mu \subseteq \mathcal{W}_2$ and $\set{\rho_i}_{i\in I}$ be a partition of unity subordinate to $\set{U_i}_{i \in I}$. If we define $f_i:U_i\rightarrow \R$ by 
\begin{align*}
    f_i \triangleq  \sum _{j\in I;i <j ,U_{ij}\neq \varnothing } c_{ij}\rho_j |_{U_i} - \sum_{j \in I, j< i, U_{ij}\neq \varnothing } c_{ji}\rho_j |_{U_i}
\end{align*}
there exists some closed one-form $\alpha $ such that $\alpha =df_i$ on each $U_i$ and 
\begin{align*}
f_i-f_j= c_{ij}\text{ on }U_{ij}\text{ for all }i<j,U_{ij}\neq \varnothing
\end{align*}
\end{theorem}
\begin{proof}
Fix $i<j,U_{ij}\neq \varnothing$. Because 
\begin{align*}
f_i&= \sum_{i <k}c_{ik}\rho_k - \sum_{k<i}c_{ki}\rho_k \\
f_j&=\sum_{j<k}c_{jk}\rho_k - \sum_{k<j} c_{kj}\rho_k 
\end{align*}
We may compute 
\begin{align}
\label{fij}
f_i - f_j = \sum_{j<k} (c_{ik}-c_{jk})\rho_k + \sum_{i<k<j}(c_{ik}+c_{kj})\rho_k + \sum_{k<i}(-c_{ki}+c_{kj})\rho_k +c_{ij}\rho_j + c_{ij}\rho_i 
\end{align}
Because  $(c_{ij})\in \operatorname{Ker}\mu$, for all $k>j$, we may deduce
 \begin{align*}
c_{ij}+c_{jk}-c_{ik}=0 \implies c_{ik}- c_{jk}= c_{ij}
\end{align*}
For all $k:i<k<j$, we may deduce 
 \begin{align*}
c_{ik}+c_{kj}-c_{ij}=0 \implies c_{ik}+c_{kj}=c_{ij}
\end{align*}
For all $k<i$, we may deduce 
\begin{align*}
c_{ki}+ c_{ij}-c_{kj}=0 \implies -c_{ki}+c_{kj}=c_{ij}
\end{align*}
Thus, we may continue the computation from \myref{Equation}{fij} and get 
\begin{align*}
f_i-f_j= \sum_{k \in I}c_{ij}\rho_k= c_{ij}\text{ on }U_{ij}
\end{align*}
where the last equality hold true because $\set{\rho_k}_{k\in I}$ is a partition of unity. We have established that for each $i<j,U_{ij}\neq \varnothing$, the functions  $f_i,f_j$ differ by a constant on where they overlap. Therefore, we may well define a closed one form  $\alpha $ on $M$ by 
 \begin{align*}
\alpha |_{U_i}\triangleq df_i\text{ for all }i \in I
\end{align*}
Note that $\alpha $ is indeed closed, since 
\begin{align*}
  (d\alpha) |_{U_i}= d(\alpha |_{U_i} )=d(df_i)=0 \text{ for all }i \in I \implies  d\alpha =0\text{ on }M
\end{align*}
\end{proof}
\begin{mdframed}
Now, for all element $X$ of $\operatorname{Ker} \mu \diagup \operatorname{Im}\ld $, when we pick a representative element $(c_{ij})\in X \subseteq \operatorname{Ker}\mu$, using \myref{Theorem}{d}, we may find some closed one-from $\alpha $ such that $\alpha $ can induce $(c_{ij})$, which give us 
\begin{align*}
\Phi ([\alpha ])=X
\end{align*}
In other words, $\Phi$ is surjective. For part (e), suppose $M$ is compact. Because $M$ is compact, we may let  $I$ be finite, which allow us to deduce 
\begin{align*}
\operatorname{Dim} (\mathcal{W}_2) \leq (\operatorname{card} I)^2  \inn
\end{align*}
and deduce 
\begin{align*}
\operatorname{Dim}(\operatorname{Ker}\mu\diagup \operatorname{Im} \ld ) \leq \operatorname{Dim}(\operatorname{Ker}\mu ) \leq \operatorname{Dim}(\mathcal{W}_2) \inz_0^+
\end{align*}
Lastly, because $\Phi:H^1(M)\rightarrow \operatorname{Ker}\mu \diagup \operatorname{Im}\ld $ is injective, we can moreover deduce 
\begin{align*}
\operatorname{Dim}(H^1(M)) \leq \operatorname{Dim}(\operatorname{Ker}\mu \diagup \operatorname{Im}\ld ) \inz_0^+
\end{align*}
That is, $H^1(M)$ is finite dimensional. 
\end{mdframed}
\end{document}
