\documentclass{report}
%%%%%%%%%%%%%% macros.tex %%%%%%%%%%%%%%
% Place your custom macros here, if any.

%%%%%%%%%%%%%% letterfonts.tex %%%%%%%%%%%%%%
% Place your font setup here, if any.

%%%%%%%%%%%%%% preamble.tex %%%%%%%%%%%%%%
\usepackage[T1]{fontenc}
\usepackage{lmodern}
\usepackage{etoolbox}
\usepackage{pdfpages}
\usepackage{transparent}
\usepackage[utf8]{inputenc}
\usepackage[english]{babel}

% Page Setup
\usepackage[tmargin=2cm, rmargin=0.5in, lmargin=0.5in, bmargin=80pt, footskip=.2in]{geometry}

% Mathematics
\usepackage{amsmath,amsfonts,amsthm,amssymb,mathtools}
\usepackage{xfrac}
\usepackage[makeroom]{cancel}
\usepackage{enumitem}
\usepackage{nameref}
\usepackage{multicol,array}
\usepackage{tikz-cd}
\usepackage[ruled,vlined,linesnumbered]{algorithm2e}

% Colors
\usepackage[dvipsnames]{xcolor}
\definecolor{myg}{RGB}{56, 140, 70}
\definecolor{myb}{RGB}{45, 111, 177}
\definecolor{myr}{RGB}{199, 68, 64}
% Define more colors here...

% Hyperlinks
\usepackage{bookmark}
\usepackage{hyperref}
\hypersetup{
    pdftitle={Assignment},
    colorlinks=true, linkcolor=doc!90,
    bookmarksnumbered=true,
    bookmarksopen=true
}

% Figures and Graphics
\usepackage{import}
\usepackage{svg}
\newcommand{\incfig}[1]{%
    \def\svgwidth{\columnwidth}
    \import{./figures/}{#1.pdf_tex}
}

% Text-related
\usepackage{blindtext}
\usepackage{fontsize}
\changefontsize[14]{14}
\setlength{\parindent}{0pt}

% Theorems and Definitions
\usepackage{amsthm}
\renewcommand\qedsymbol{$\blacksquare$}

% Define a new theorem style
\newtheoremstyle{mytheoremstyle}% name
  {}% Space above
  {}% Space below
  {\sffamily}% Body font
  {}% Indent amount
  {\bfseries}% Theorem head font
  {.}% Punctuation after theorem head
  {.5em}% Space after theorem head
  {}% Theorem head spec (can be left empty, meaning ‘normal’)

% Apply the new theorem style to theorem-like environments
\theoremstyle{mytheoremstyle}
\newtheorem{theorem}{Theorem}[section]
\newtheorem{definition}{Definition}[section]
\newtheorem{corollary}{Corollary}[section]
\newtheorem{lemma}{Lemma}[section]
\newtheorem{axiom}{Axiom}[section]

% tcolorbox Setup
\usepackage[most,many,breakable]{tcolorbox}

% Define custom tcolorbox environments here...

%================================
% EXAMPLE BOX
%================================
\newtcbtheorem[definition]{Example}{Example}
{%
    colback = myexamplebg,
    breakable,
    colframe = myexamplefr,
    coltitle = myexampleti,
    boxrule = 1pt,
    sharp corners,
    detach title,
    before upper=\tcbtitle\par\smallskip,
    fonttitle = \bfseries,
    description font = \mdseries,
    separator sign none,
    description delimiters parenthesis,
}
{ex}

%================================
% Solution BOX
%================================
\makeatletter
\newtcolorbox{solution}{enhanced,
	breakable,
	colback=white,
	colframe=myg!80!black,
	attach boxed title to top left={yshift*=-\tcboxedtitleheight},
	title=Solution,
	boxed title size=title,
	boxed title style={%
			sharp corners,
			rounded corners=northwest,
			colback=tcbcolframe,
			boxrule=0pt,
		},
	underlay boxed title={%
			\path[fill=tcbcolframe] (title.south west)--(title.south east)
			to[out=0, in=180] ([xshift=5mm]title.east)--
			(title.center-|frame.east)
			[rounded corners=\kvtcb@arc] |-
			(frame.north) -| cycle;
		},
}
\makeatother

%================================
% Question BOX
%================================
\makeatletter
\newtcbtheorem{question}{Question}{enhanced,
	breakable,
	colback=white,
	colframe=myb!80!black,
	attach boxed title to top left={yshift*=-\tcboxedtitleheight},
	fonttitle=\bfseries,
	title={#2},
	boxed title size=title,
	boxed title style={%
			sharp corners,
			rounded corners=northwest,
			colback=tcbcolframe,
			boxrule=0pt,
		},
	underlay boxed title={%
			\path[fill=tcbcolframe] (title.south west)--(title.south east)
			to[out=0, in=180] ([xshift=5mm]title.east)--
			(title.center-|frame.east)
			[rounded corners=\kvtcb@arc] |-
			(frame.north) -| cycle;
		},
	#1
}{def}
\makeatother
\makeatletter
\newtcbtheorem{qstion}{Question}{enhanced,
    breakable,
    colback=white,
    colframe=mygr,
    attach boxed title to top left={yshift*=-\tcboxedtitleheight},
    fonttitle=\bfseries,
    title={#2},
    boxed title size=title,
    boxed title style={%
        sharp corners,
        rounded corners=northwest,
        colback=tcbcolframe,
        boxrule=0pt,
    },
    underlay boxed title={%
        \path[fill=tcbcolframe] (title.south west)--(title.south east)
        to[out=0, in=180] ([xshift=5mm]title.east)--
        (title.center-|frame.east)
        [rounded corners=\kvtcb@arc] |-
        (frame.north) -| cycle;
    },
    #1
}{def}
\makeatother

%%%%%%%%%%%%%%%%%%%%%%%%%%%%%%%%%%%%%%%%%%%
% TABLE OF CONTENTS
%%%%%%%%%%%%%%%%%%%%%%%%%%%%%%%%%%%%%%%%%%%
\usepackage{tikz}
\definecolor{doc}{RGB}{0,60,110}
\usepackage{titletoc}
\contentsmargin{0cm}
\titlecontents{chapter}[14pc]
{\addvspace{30pt}%
	\begin{tikzpicture}[remember picture, overlay]%
		\draw[fill=doc!60,draw=doc!60] (-7,-.1) rectangle (-0.9,.5);%
		\pgftext[left,x=-4.5cm,y=0.2cm]{\color{white}\Large\sc\bfseries Chapter\ \thecontentslabel};%
	\end{tikzpicture}\color{doc!60}\large\sc\bfseries}%
{}
{}
{\;\titlerule\;\large\sc\bfseries Page \thecontentspage
	\begin{tikzpicture}[remember picture, overlay]
		\draw[fill=doc!60,draw=doc!60] (2pt,0) rectangle (4,0.1pt);
	\end{tikzpicture}}%
\titlecontents{section}[3.7pc]
{\addvspace{2pt}}
{\contentslabel[\thecontentslabel]{2pc}}
{}
{\hfill\small \thecontentspage}
[]
\titlecontents*{subsection}[3.7pc]
{\addvspace{-1pt}\small}
{}
{}
{\ --- \small\thecontentspage}
[ \textbullet\ ][]

\makeatletter
\renewcommand{\tableofcontents}{
	\chapter*{%
	  \vspace*{-20\p@}%
	  \begin{tikzpicture}[remember picture, overlay]%
		  \pgftext[right,x=15cm,y=0.2cm]{\color{doc!60}\Huge\sc\bfseries \contentsname};%
		  \draw[fill=doc!60,draw=doc!60] (13,-.75) rectangle (20,1);%
		  \clip (13,-.75) rectangle (20,1);
		  \pgftext[right,x=15cm,y=0.2cm]{\color{white}\Huge\sc\bfseries \contentsname};%
	  \end{tikzpicture}}%
	\@starttoc{toc}}
\makeatother

\newcommand{\liff}{\llap{$\iff$}}
\newcommand{\rap}[1]{\rrap{\text{ (#1)}}}
\newcommand{\red}[1]{\textcolor{red}{#1}}
\newcommand{\blue}[1]{\textcolor{blue}{#1}}
\newcommand{\vi}[1]{\textcolor{violet}{#1}}
\newcommand{\teal}[1]{\textcolor{teal}{#1}}
\newcommand{\tCaC}{\text{ \CaC }}
\newcommand{\CaC}{\red{CaC} }
\newcommand{\As}[1]{Assume \red{#1}}
\newcommand{\vdone}{\vi{\text{ (done) }}}
\newcommand{\bdone}{\blue{\text{ (done) }}}
\newcommand{\tdone}{\teal{\text{ (done) }}}
\newcommand{\set}[1]{\{ #1 \}}
\newcommand{\inS}{\in S}
\newcommand{\inF}{\in\F}
\newcommand{\inE}{\in E}
\newcommand{\inA}{\in A}
\newcommand{\inB}{\in B}
\newcommand{\inC}{\in C}
\newcommand{\inU}{\in U}

\newcommand{\C}{\mathbb{C}}	
\renewcommand{\H}{\mathbb{H}}
\newcommand{\F}{\mathbb{F}}
\newcommand{\N}{\mathbb{N}}
\newcommand{\Q}{\mathbb{Q}}
\newcommand{\R}{\mathbb{R}}
\newcommand{\Z}{\mathbb{Z}}
\renewcommand{\P}{\mathbb{P}}
\renewcommand{\S}{\mathbb{S}}
\newcommand{\A}{\mathbb{A}}
\newcommand{\RP}{\R P}


\title{\Huge{NCKU 112.2}\\
General Analysis}
\author{\huge{Eric Liu}}
\date{}
\begin{document}
\newpage% or \cleardoublepage
% \pdfbookmark[<level>]{<title>}{<dest>}
\pdfbookmark[section]{\contentsname}{toc}

\pagebreak
\chapter{Final} 
\section{Main Body}
\begin{abstract}
  Note that $\mathfrak{X}(M)$ always denote the space of smooth vector field on $M$, and $\Omega^k(M)$ always denote the space of smooth $k$-forms on  $M$. 
\end{abstract}
\begin{question}{}{}
Let $X$ be a vector field on a compact manifold $M$. Let  $\theta_t$ be the global flow generated by $X$. Let  $\alpha \in \Omega^k(M)$. Show 
\begin{align*}
\theta_t^* \alpha =\alpha \text{ for all }t \iff  \mathcal{L}_X \alpha =0
\end{align*}
\end{question}
\begin{proof}
Left to right follows from computing 
\begin{align*}
  (\mathcal{L}_X \alpha)_p = \lim_{t\to 0} \frac{(\theta_t^*\alpha )_p -\alpha _p}{t} = \lim_{t\to 0} \frac{\alpha_p -\alpha _p}{t}= 0\text{ for all }p \in M
\end{align*}
Suppose $\mathcal{L}_X \alpha =0$. Fix arbitrary $p \in M$ and arbitrary $V_1,\dots ,V_k \in \mathfrak{X}(M)$. Define $\beta :\R\rightarrow \R$ by 
\begin{align*}
\beta  (t)\triangleq  (\theta_t^* \alpha )_p (V_1,\dots ,V_k)
\end{align*}
Fix arbitrary $t\inr$. By definition, 
\begin{align}
\label{11}
  (\theta_t^* \alpha )_p (V_1,\dots ,V_k)= \alpha_{\theta_t (p)}\big( (\theta_t)_{*,p}V_1,\dots ,(\theta_t)_{*,p}V_k  \big) 
\end{align}
Because  
\begin{align*}
\theta_s= \theta_{s-t} \circ \theta_{t}
\end{align*}
We know 
\begin{align*}
\theta_s^*\alpha  = \theta_{t}^*  \theta_{s-t}^*\alpha 
\end{align*}
This give us 
\begin{align}
  (\theta_s ^* \alpha )_p (V_1,\dots ,V_k)&=  (\theta_{t}^*\theta_{s-t}^*\alpha )_p (V_1,\dots ,V_k) \notag   \\
  &=(\theta_{s-t}^*\alpha)_{\theta_{t}(p )} \big( (\theta_{t})_{*,p}V_1,\dots ,(\theta_{t})_{*,p}V_k  \big)  \label{12}
\end{align}
We may now use \myref{Equation}{11} and \myref{Equation}{12} to compute 
\begin{align*}
\beta  '(t)&=\lim_{s\to t} \frac{(\theta_s^* \alpha)_p - (\theta_t^* \alpha )_p }{s-t} (V_1,\dots ,V_k) \\
&=\lim_{s\to t} \frac{ (\theta_{s-t}^*\alpha )_{\theta_t(p)}-\alpha_{\theta_t (p)}}{s-t}  \big( (\theta_{t})_{*,p}V_1,\dots ,(\theta_{t})_{*,p}V_k  \big)  \\
&= \lim_{h\to 0} \frac{(\theta_h^* \alpha)_{\theta_t (p)}- \alpha_{\theta_t (p)} }{h}    \big( (\theta_{t})_{*,p}V_1,\dots ,(\theta_{t})_{*,p}V_k  \big)  \\
&= (\mathcal{L}_X \alpha)_{\theta_t (p)}     \big( (\theta_{t})_{*,p}V_1,\dots ,(\theta_{t})_{*,p}V_k  \big)=0 
\end{align*}
Because $t$ is arbitrary, we have shown $\beta '=0$ on $\R$. This implies  $\beta $ is a constant, i.e., 
\begin{align*}
  (\theta_t^* \alpha )_p (V_1,\dots ,V_k)= \alpha _p (V_1,\dots ,V_k)\text{ for all }t
\end{align*}
Because $V_1,\dots ,V_k$ are arbitrary, this implies 
\begin{align*}
  (\theta_t^* \alpha)_p = \alpha_p  \text{ for all }t
\end{align*}
Because $p$ is arbitrary, this implies  
\begin{align*}
  \theta_t^* \alpha =\alpha \text{ for all }t
\end{align*}
\end{proof}
\begin{question}{}{}
Show that the total space $T^*M$ of the cotangent bundle of any manifold  $M^n$ is orientable. 
\end{question}
\begin{proof}
  Let $\set{\phi_\alpha }$ be an atlas for $M$, and use $\set{\ld^i}$ to denote the dual basis of $\set{\frac{\partial }{\partial \textbf{x}^i}}$. The smooth structure  of $T^*M$ is by definition given by the atlas $\set{\Phi_\alpha }$ defined by
 \begin{align*}
\Phi_\alpha  (p, \sum_{j} \xi^j \ld ^j)\triangleq  \Big( \phi_\alpha  (p),\xi^1 ,\dots ,\xi^n \Big)
\end{align*}
We first show 
\begin{align}
\label{ldi}
\vi{\widetilde{\ld}^i = \sum_{j} \frac{\partial \tilde{ \textbf{x}}^i }{\partial \textbf{x}^j} {\ld }^j\text{ for all }i}
\end{align}
For all $\omega \in \Omega^1(M)$, we may write in local coordinate 
\begin{align}
\label{ldi2}
\omega = \sum_j \omega^j \ld^j = \sum_i \tilde{\omega}^i \tilde{\ld}^i   
\end{align}
Compute   
\begin{align}
\label{ldi3}
  \omega^j= \omega \Big( \frac{\partial }{\partial \textbf{x}^j}\Big|_p \Big)= \omega \Big( \sum_i \frac{\partial \tilde{\textbf{x}}^i }{\partial \textbf{x}^j}(p) \frac{\partial }{\partial \tilde{\textbf{x}}^i }\Big|_p \Big)= \sum_i \frac{\partial \tilde{\textbf{x}}^i }{\partial \textbf{x}^j} (p)\tilde{\omega}^i 
\end{align}
Because $\ld ^j$ is a basis, for all $i$, we may write 
\begin{align*}
  \tilde{ \ld}^i = \sum_{j} c_{i,j}\ld ^j
\end{align*}
It then follows from \myref{Equation}{ldi2} and \myref{Equation}{ldi3} that 
\begin{align*}
\sum_{i,j} \tilde{\omega}^i  c_{i,j} \ld ^j =\sum_{i} \tilde{\omega}^i \tilde{\ld }^i  = \sum_{j} \omega^j \ld ^j  = \sum_{i,j} \frac{\partial \tilde{\textbf{x}}^i }{\partial \textbf{x}^j} \tilde{\omega}^i  \ld ^j
\end{align*}
Because $\set{\ld ^j}$ is linearly independent, we may now deduce for all fixed $j$
\begin{align}
  \label{ldi4}
\sum_i \tilde{\omega}^i c_{i,j}= \sum_i \frac{\partial \tilde{\textbf{x}}^i }{\partial \textbf{x}^j} \tilde{\omega}^i 
\end{align}
Fix $i$. If we let $\tilde{\omega}^k=\delta^k_i$, \myref{Equation}{ldi4} becomes 
\begin{align*}
c_{i,j}= \frac{\partial \tilde{\textbf{x}}^i }{\partial \textbf{x}^j}
\end{align*}
Which implies 
\begin{align*}
\tilde{\ld }^i= \sum_{j} \frac{\partial \tilde{\textbf{x}}^i }{\partial \textbf{x}^j} \ld^j \vdone 
\end{align*}
We may now compute the transition function between  $\tilde{\Phi},\Phi$ by 
 \begin{align*}
\tilde{\Phi}\circ \Phi^{-1}(\textbf{x}^1,\dots ,\textbf{x}^n,\xi^1,\dots ,\xi^n)&= \tilde{\Phi}\Big(\phi^{-1}(\textbf{x}^1,\dots ,\textbf{x}^n), \sum_{i} \xi^i \ld^i \Big) \\
&= \tilde{\Phi}\Big(\phi^{-1}(\textbf{x}^1,\dots ,\textbf{x}^n), \sum_{i} \xi^i \sum_j  \frac{\partial \tilde{\textbf{x}}^j }{\partial \textbf{x}^i}\tilde{\ld}^j  \Big) \\
&= \tilde{\Phi}\Big(\phi^{-1}(\textbf{x}^1,\dots ,\textbf{x}^n), \sum_{j}\sum_i   \frac{\partial \tilde{\textbf{x}}^j }{\partial \textbf{x}^i}\xi^i\tilde{\ld}^j  \Big) \\
&=\Big(\tilde{\textbf{x}}^1,\dots, \tilde{\textbf{x}}^n, \sum_i \frac{\partial \tilde{\textbf{x}}^1 }{\partial \textbf{x}^i} (\textbf{x})\xi^i ,\dots , \sum_i  \frac{\partial \tilde{\textbf{x}}^n }{\partial \textbf{x}^i} (\textbf{x})\xi^i \Big)
\end{align*}
And compute the derivative of $\tilde{\Phi}\circ \Phi^{-1}$
\begin{align}
\label{dphi}
 [d(\tilde{\Phi}\circ \Phi^{-1}) ]= \begin{bmatrix}
   \frac{\partial \tilde{\textbf{x}}^1 }{\partial \textbf{x}^1} & \cdots  & \frac{\partial \tilde{\textbf{x}}^1 }{\partial \textbf{x}^n} & 0 & \cdots & 0 \\
   \vdots & \ddots & \vdots & \vdots & \ddots & \vdots \\
   \frac{\partial \tilde{\textbf{x}}^n }{\partial \textbf{x}^1} &  \cdots & \frac{\partial \tilde{\textbf{x}}^n }{\partial \textbf{x}^n} & 0 &\cdots & 0 \\
   A_{1,1} & \cdots & A_{1,n} & \frac{\partial \tilde{\textbf{x}}^1 }{\partial \textbf{x}^1} & \cdots  & \frac{\partial \tilde{\textbf{x}}^1 }{\partial \textbf{x}^n} \\
   \vdots & \ddots & \vdots & \vdots & \ddots &  \vdots \\
   A_{n,1} & \cdots & A_{n,n} &   \frac{\partial \tilde{\textbf{x}}^n }{\partial \textbf{x}^1} &  \cdots & \frac{\partial \tilde{\textbf{x}}^n }{\partial \textbf{x}^n} 
 \end{bmatrix}
\end{align}
Where 
\begin{align*}
A_{i,j}= \sum_{k} \frac{\partial \tilde{\textbf{x}}^i }{\partial \textbf{x}^k \partial \textbf{x}^j} \xi^k
\end{align*}
Using \myref{Equation}{dphi}, we may conclude
\begin{align*}
  \operatorname{det} d(\tilde{\Phi}\circ \Phi^{-1} )= [ \operatorname{det}d(\tilde{\phi}  \circ \phi^{-1}) ] ^2> 0
\end{align*}
Note that the last inequality hold true because the fact $\tilde{\phi}\circ \phi^{-1} $ is a diffeomorphism between open subsets of $\R^{n}$ implies $d(\tilde{\phi}\circ \phi^{-1} )$ is invertible, which implies $\operatorname{det}d(\tilde{\phi}\circ \phi^{-1} )$ is non-zero. We have shown $\set{\Phi_\alpha }$ is an orientable atlas, which implies $T^*M$ is orientable.  
\end{proof}
\begin{mdframed}
Note that one can give a quick proof for \myref{Equation}{ldi} by changing the notation 
\begin{align}
 \label{sup}
 \tilde{\ld }^i = d\tilde{\textbf{x}}^i = \sum_{j} \frac{\partial \tilde{\textbf{x}}^i }{\partial \textbf{x}^j}  d\textbf{x}^j= \sum_{j} \frac{\partial \tilde{\textbf{x}}^i }{\partial \textbf{x}^j} \ld^j
\end{align}
I submitted \myref{Equation}{sup} as one of my supplementary argument. Its core lies in the standard extension of functions $\textbf{x}^i$ from a single chart to the whole manifold using bump function.    
\end{mdframed}
\begin{question}{}{}
Show that the following are special cases of Stoke's Theorem for manifold with boundary. 
\begin{enumerate}[label=(\alph*)]
  \item Let $C$ be the image of a smooth embedding  $\textbf{r}:S^1 \rightarrow \R^2$ and let $D$ be the region in  $\R^2$ bounded by  $C$. If  $P,Q: \R^2 \rightarrow \R$ are smooth functions. 
    \begin{align*}
    \int_C Pdx+ Qdy = \int_D \Big( \frac{\partial Q}{\partial x}- \frac{\partial P}{\partial y} \Big)dxdy
    \end{align*} 
    \item Let $S$ be a compact oriented surface with smooth boundary $C$. Let  $\textbf{F}:\R^3 \rightarrow \R^3$ be smooth. 
      \begin{align*}
      \int_C \textbf{F}\cdot d\textbf{r} =\int_S  \nabla \times \textbf{F} \cdot d\textbf{S}
      \end{align*} 
      \item Let $E$ be the compact closure of some open subset of  $\R^3$ with smooth boundary  $S$. Let  $\textbf{F}:\R^3 \rightarrow \R^3$ be smooth.  
      \begin{align*}
        \int_S \textbf{F} \cdot d \textbf{S} = \int_E \nabla \cdot \textbf{F} dxdydz 
      \end{align*}
\end{enumerate}
\end{question}
\begin{proof}
Because $P,Q:\R^2 \rightarrow \R$ are smooth, we know 
\begin{align*}
\omega \triangleq  Pdx + Qdy \text{ is a smooth 1-form on }\R^2
\end{align*}
Note that all interpretations  (Riemann, Riemann-Stieltjes, Lebesgue or Lebesgue-Stieltjes integral) equal to 
\begin{align*}
\int_C P dx +Q dy \triangleq \int_{\partial D} \omega  \text{ and } \int_D \Big( \frac{\partial Q}{\partial x}- \frac{\partial P}{\partial y} \Big)dxdy \triangleq \int_D  \Big( \frac{\partial Q}{\partial x}- \frac{\partial P}{\partial y} \Big) dx \wedge  dy  
\end{align*}


Compute 
\begin{align*}
d\omega &= \Big(\frac{\partial P}{\partial x}dx+ \frac{\partial P}{\partial y}dy\Big)\wedge  dx + \Big( \frac{\partial Q}{\partial x}dx+ \frac{\partial Q}{\partial y}dy \Big) \wedge  dy  \\
&= \Big( \frac{\partial Q}{\partial x}- \frac{\partial P}{\partial y} \Big)dx\wedge  dy 
\end{align*}
Green's Theorem now follows from Stoke's Theorem, 
\begin{align*}
\int_C Pdx + Qdy= \int_{\partial D} \omega = \int_{D} d\omega = \int_D  \Big( \frac{\partial Q}{\partial x}- \frac{\partial P}{\partial y} \Big)dx\wedge  dy  = \int_D  \Big( \frac{\partial Q}{\partial x}- \frac{\partial P}{\partial y} \Big)dxdy 
\end{align*}
\end{proof}
\begin{proof}
Note that the correct interpretation of surface integral is 
\begin{align}
\label{cor1}
\int_S \textbf{G} \cdot d \textbf{S} \triangleq  \int_S \textbf{G}^1 dy\wedge  dz +  \textbf{G}^2 dz \wedge  dx + \textbf{G}^3 dx \wedge  dy   
\end{align}
And the correct interpretation of line integral is 
\begin{align}
\label{cor2}
\int_C \textbf{G} \cdot d \textbf{r}\triangleq  \int_C \textbf{G}^1 dx + \textbf{G}^2 dy + \textbf{G}^3 dz
\end{align}
Because $\textbf{F}$ is smooth, we know 
\begin{align*}
\omega \triangleq \textbf{F}^1 dx + \textbf{F}^2 dy + \textbf{F}^3 dz\text{ is a smooth 1-form on }\R^3
\end{align*}
Compute 
\begin{align*}
d\omega &= \Big( \frac{\partial \textbf{F}^1}{\partial x}dx + \frac{\partial \textbf{F}^1}{\partial y}dy + \frac{\partial \textbf{F}^1}{\partial z}dz \Big)\wedge  dx +   \Big( \frac{\partial \textbf{F}^2}{\partial x}dx + \frac{\partial \textbf{F}^2}{\partial y}dy + \frac{\partial \textbf{F}^2}{\partial z}dz \Big)\wedge  dy \\
& +  \Big( \frac{\partial \textbf{F}^3}{\partial x}dx + \frac{\partial \textbf{F}^3}{\partial y}dy + \frac{\partial \textbf{F}^3}{\partial z}dz \Big)\wedge  dz  \\
&=\Big( \frac{\partial \textbf{F}^3}{\partial y}- \frac{\partial \textbf{F}^2}{\partial z} \Big)dy \wedge  dz + \Big( \frac{\partial \textbf{F}^1}{\partial z}- \frac{\partial \textbf{F}^3}{\partial x}  \Big)dz \wedge  dx + \Big( \frac{\partial \textbf{F}^2}{\partial x}-\frac{\partial \textbf{F}^1}{\partial y}  \Big)dx \wedge  dy   \\
&= (\nabla \times \textbf{F})^1 dy \wedge  dz + (\nabla \times \textbf{F})^2 dz \wedge  dx +  (\nabla \times  \textbf{F})^3 dx \wedge  dy  
\end{align*}
Because $C= \partial S$, by \myref{Equation}{cor1}, \myref{Equation}{cor2} and Stoke's Theorem, we now have 
\begin{align*}
\int_{C} \textbf{F} \cdot d \textbf{r}= \int_C \omega = \int_{S} d\omega = \int_{S} (\nabla \times  \textbf{F}) \cdot d \textbf{S} 
\end{align*}
\end{proof}
\begin{proof}
Note that all interpretations  (Riemann, Riemann-Stieltjes, Lebesgue or Lebesgue-Stieltjes integral) equal to 
\begin{align}
\label{cor4}
\int_E f dxdydz= \int_E f dx \wedge  dy \wedge  dz  
\end{align}
Identifying $\textbf{F}\in \mathfrak{X}(\R^3)$ as the vector field 
\begin{align*}
\textbf{F} \simeq \textbf{F}^1 \frac{\partial }{\partial x} + \textbf{F}^2 \frac{\partial }{\partial y} + \textbf{F}^3 \frac{\partial }{\partial z}
\end{align*}
We may compute 
\begin{align*}
\iota_\textbf{F}(dx\wedge dy \wedge  dz )&=  \textbf{F}^1dy \wedge  dz - \textbf{F}^2 dx \wedge  dz + \textbf{F}^3 dx \wedge  dy  \\
&= \textbf{F}^1dy \wedge  dz + \textbf{F}^2 dz \wedge  dx + \textbf{F}^3 dx \wedge  dy  
\end{align*}
And compute 
\begin{align*}
d\iota_\textbf{F} (dx \wedge  dy \wedge  dz  )&=d (  \textbf{F}^1dy \wedge  dz - \textbf{F}^2 dx \wedge  dz + \textbf{F}^3 dx \wedge  dy )  \\
&= \Big(\frac{\partial \textbf{F}^1}{\partial x}+ \frac{\partial \textbf{F}^2}{\partial y}+\frac{\partial \textbf{F}^3}{\partial z}\Big) dx\wedge  dy \wedge  dz  \\
&= \nabla \cdot \textbf{F}   dx \wedge  dy \wedge  dz  
\end{align*}
Because $S= \partial E$, by \myref{Equation}{cor1}, \myref{Equation}{cor4} and Stoke's Theorem, we now have   
\begin{align*}
\int_S \textbf{F} \cdot d \textbf{S} = \int_{S} \iota_\textbf{F} (dx\wedge  dy \wedge  dz)    =\int_E d \iota_\textbf{F} (dx \wedge  dy \wedge  dz) = \int_E \nabla \cdot \textbf{F} dxdydz
\end{align*}

\end{proof}
\begin{mdframed}
Note that our interpretation of line integral in \myref{Equation}{cor2} is correct because for all parametrization $\gamma :[a,b]\rightarrow C$, we have 
\begin{align*}
\int_a^b \textbf{G}(\gamma (t)) \cdot \gamma '(t)dt &= \int_{\gamma^{-1}(C)}\gamma ^* (\textbf{G}^1 dx + \textbf{G}^2 dy + \textbf{G}^3 dz) \\
&=\int_C \textbf{G}^1 dx + \textbf{G}^2 dy + \textbf{G}^3 dz
\end{align*}
Also note that our interpretation of surface integral in \myref{Equation}{cor1} is correct because for all parametrization $\textbf{u}:D\subseteq\R^2 \rightarrow S$, we have 
\begin{align*}
\int_D \textbf{G} (\textbf{u}(s,t))\cdot \Big(\frac{\partial \textbf{u}}{\partial s}\times \frac{\partial \textbf{u}}{\partial t} \Big) d(s,t)&= \int_{\textbf{u}^{-1}(S)}\textbf{u}^* (\textbf{G}^1 dy \wedge  dz + \textbf{G}^2 dz \wedge  dx + \textbf{G}^3 dx \wedge  dy   ) \\
&=\int_{S} \textbf{G}^1 dy \wedge  dz + \textbf{G}^2  dz\wedge  dx + \textbf{G}^3 dx \wedge  dy  
\end{align*}
\end{mdframed}
\begin{mdframed}
Let $F:S^3 \rightarrow S^2$ be a smooth map, and $\alpha ,\beta  $ be two 2-forms on  $S^2$ that lie in some non-trivial De Rham cohomology class.  Let $\theta,\theta'$ be some 1-form on $S^3$ such that  $F^* \alpha = d\theta$ and $F^* \beta = d \theta'$. Show that $\theta \wedge  d \theta ,\theta '\wedge  d \theta ' $ lie in the same De Rham cohomology class. 
\end{mdframed}
\begin{question}{}{}
Consider a smooth map $F:S^3 \rightarrow S^2$. Let $\alpha \in \Omega^2 (S^2)$ be a form representing a non-trivial De Rham cohomology class $a\in H^2(S^2)$. Show that there exists a 1-form $\theta$ on $S^3$ such that $F^* \alpha = d\theta$. Moreover show that the De Rham cohomology class in $H^3 (S^3)$ of the 3-form $\theta \wedge  F^* \alpha  $ is independent of the choice of $\theta$ and of $\alpha $ representing $a$. 
\end{question}
\begin{proof}
Because $\alpha $ is a 2-form and $S^2$ has dimension  2, we may compute   
\begin{align*}
  d(F^* \alpha )= F^* (d\alpha )=F^*(0)=0
\end{align*}
Therefore, $F^* \alpha $ is a closed 2-form on $S^3$. It then follows from the fact $H^2(S^3)\cong 0$ that $F^*\alpha $ is exact. That is, there exists some $\theta \in \Omega^1(S^3)$ such that 
\begin{align*}
F^* \alpha = d\theta
\end{align*} 
To see that the cohomology class of the 3-form $\theta \wedge  F^* \alpha = \theta \wedge  d\theta  $ is independent of the choice of $\theta$, let $\theta' \in \Omega^1 (S^3)$ also satisfy 
\begin{align*}
d\theta ' = F^* \alpha = d \theta 
\end{align*}
Because 
\begin{align*}
d(\theta' -\theta )=0 \text{ and }H^1(S^3)\cong 0 
\end{align*}
We know $\theta'- \theta = d \delta $ for some $\delta \in \Omega^0 (S^3)$. Therefore, we may compute
\begin{align*}
\theta' \wedge  d\theta' - \theta \wedge  d\theta  &= (\theta ' - \theta ) \wedge  d \theta   \\
&=d\delta \wedge  d\theta = d(\delta \wedge  d \theta ) 
\end{align*}
Showing that $[\theta ' \wedge  d \theta' ]= [\theta \wedge  d\theta ]$. 
\end{proof}
\begin{mdframed}
The following is a failed attempt to prove the independence of choice of $\alpha $. 
\end{mdframed}
\begin{proof}
Fix arbitrary $\beta   \in a$, and let  $\gamma \in \Omega^1(S^2), \theta' \in \Omega^1 (S^3)$ satisfy 
\begin{align*}
\beta  = \alpha  + d\gamma  \text{ and }F^*\beta  = d\theta'
\end{align*}
So that  
\begin{align*}
d(\theta'- \theta - F^* \gamma )=0
\end{align*}
Because $H^1(S^3)\cong  0$, we now have 
\begin{align*}
\theta ' - \theta - F^* \gamma  = d \phi \text{ for some }\phi \in \Omega^0(S^3)
\end{align*}
Compute 
\begin{align*}
\theta' = \theta + F^* \gamma  + d\phi  \text{ and } d \theta ' = d\theta + dF^* \gamma 
\end{align*}
Compute 
\begin{align*}
  \theta ' \wedge  d\theta' &= (\theta + F^* \gamma  + d\phi) \wedge  (d\theta + dF^* \gamma )   \\
  &= \theta \wedge  d\theta + \theta  \wedge  dF^* \gamma  +  F^* \gamma  \wedge  d\theta + F^* \gamma  \wedge  dF^* \gamma  + d\phi \wedge  d\theta + d\phi \wedge  dF^* \gamma \\    
  &= \theta \wedge  d\theta + d(\phi \wedge  d\theta + dF^* \gamma  ) + \theta \wedge  dF^* \gamma  +  F^* \gamma  \wedge  d \theta + F^* \gamma  \wedge  dF^* \gamma   
\end{align*}
This implies 
\begin{align*}
  [\theta ' \wedge  d\theta ' - \theta \wedge  d\theta ]= [\theta \wedge  dF^* \gamma  + F^* \gamma  \wedge  d\theta   +F^* \gamma  \wedge  dF^* \gamma ] 
\end{align*}
\end{proof}
\begin{question}{}{}
Suppose $M,N,P$ are compact connected orientable manifolds without boundary of the same dimension $n$ and  $F:M\rightarrow N$ and $G:N\rightarrow P$ are smooth maps. Shows that 
\begin{align*}
\operatorname{deg}(G \circ F)= \operatorname{deg}G \cdot \operatorname{deg}F
\end{align*}
Then show that the antipodal map $\phi (p)\triangleq -p$ on the unit sphere in $\R^m$ has degree  $(-1)^m$. 
\end{question}
\begin{proof}
Because $P$ is compact connected orientable and without boundary, there exists a top degree smooth form $\omega$ on $P$ such that 
\begin{align*}
\int_{P} \omega= 1
\end{align*}
Compute  
\begin{align*}
\operatorname{deg}(G \circ F) =\int_{M} (G \circ F)^* \omega = \int_M F^* (G^* \omega)= \operatorname{deg}F\int_N G^* \omega =(\operatorname{deg}F)(\operatorname{deg}G) 
\end{align*}
For each $1\leq i \leq m$, define  $\phi_i:S^{m-1}\rightarrow S^{m-1}$ by 
\begin{align*}
\phi_i (\textbf{x}^1,\dots ,\textbf{x}^m)\triangleq  (\textbf{x}^1,\dots , - \textbf{x}^i , \dots , \textbf{x}^m)
\end{align*}
So that the antipodal map $\phi$ can be expressed as the product 
\begin{align}
\label{pfi}
\phi = \phi_1 \circ  \cdots  \circ \phi_{m}
\end{align}
Fix $i$. Define 
\begin{align*}
U_i\triangleq \set{\textbf{x}\in S^{m-1}: \textbf{x}^j>0 }\text{ for some }j\neq i
\end{align*}
And define $\psi_i:U_i \rightarrow \R^{m-1}$ by 
\begin{align*}
 \psi_i (\textbf{x})\triangleq (\textbf{x}^1,\dots ,\textbf{x}^{j-1},\textbf{x}^{j+1} , \dots ,\textbf{x}^{m-1})
\end{align*}
Compute 
\begin{align*}
\psi_i \circ \phi_i \circ \psi_i^{-1}(\textbf{x}^1, \dots ,\textbf{x}^{j-1},\textbf{x}^{j+1},\dots ,\textbf{x}^{m-1})=(\textbf{x}^1, \dots ,-\textbf{x}^i , \dots ,\textbf{x}^{j-1},\textbf{x}^{j+1},\dots ,\textbf{x}^{m-1}) 
\end{align*}
This implies the derivative matrix of $d(\psi_i\circ \phi_i\circ \psi_i^{-1})_{\textbf{x}}$ has only non-zero entry in diagonal line, and all the diagonal entries, except the $i$-th one being $-1$, are $1$. It then follows that $\operatorname{det}d(\psi_i\circ \phi_i\circ \psi^{-1}_i)_{\textbf{x}}=-1$. We have shown that all points in $U_i$ are regular points of $\phi_i$. Fix $\textbf{x}\in U_i$. Because $\phi_i$ is bijective, we may now conclude 
\begin{align*}
\operatorname{deg}\phi_i = \sum_{\textbf{y} \in \phi^{-1}(\phi(\textbf{x}))} \operatorname{sgn}(\operatorname{det}(\psi_i \circ \phi_i\circ \psi_i^{-1})_\textbf{y}) = -1 
\end{align*}
It then follows from \myref{Equation}{pfi} that 
\begin{align*}
\operatorname{deg}\phi= \prod_{i=1}^m \operatorname{det}(\phi_i)= (-1)^m
\end{align*}
\end{proof}
\end{document}
