\documentclass{report}
%%%%%%%%%%%%%% preamble.tex %%%%%%%%%%%%%%
\usepackage[T1]{fontenc}
\usepackage{etoolbox}
% Page Setup
\usepackage[letterpaper, tmargin=2cm, rmargin=0.5in, lmargin=0.5in, bmargin=80pt, footskip=.2in]{geometry}
\usepackage{adjustbox}
\usepackage{graphicx}
\usepackage{tikz}
\usepackage{mathrsfs}
\usepackage{mdframed}

% Create a new toggle
\newtoggle{firstsection}

% Redefine the \chapter command to reset the toggle for each new chapter
\let\oldchapter\chapter
\renewcommand{\chapter}{\toggletrue{firstsection}\oldchapter}

% Redefine the \section command to check the toggle
\let\oldsection\section
\renewcommand{\section}{
    \iftoggle{firstsection}
    {\togglefalse{firstsection}} % If it's the first section, just switch off the toggle for next sections
    {\clearpage} % If it's not the first section, start a new page
    \oldsection
}

% Abstract Design

\usepackage{lipsum}

\renewenvironment{abstract}
 {% Start of environment
  \quotation
  \small
  \noindent
  \rule{\linewidth}{.5pt} % Draw the rule to match the linewidth
  \par\smallskip
  {\centering\bfseries\abstractname\par}\medskip
 }
 {% End of environment
  \par\noindent
  \rule{\linewidth}{.5pt} % Ensure the closing rule also matches
  \endquotation
 }

% Mathematics
\usepackage{amsmath,amsfonts,amsthm,amssymb,mathtools}
\usepackage{xfrac}
\usepackage[makeroom]{cancel}
\usepackage{enumitem}
\usepackage{nameref}
\usepackage{multicol,array}
\usepackage{tikz-cd}
\usepackage{array}
\usepackage{multirow}% http://ctan.org/pkg/multirow
\usepackage{graphicx}

% Colors
\usepackage[dvipsnames]{xcolor}
\definecolor{myg}{RGB}{56, 140, 70}
\definecolor{myb}{RGB}{45, 111, 177}
\definecolor{myr}{RGB}{199, 68, 64}
% Define more colors here...
\definecolor{olive}{HTML}{6B8E23}
\definecolor{orange}{HTML}{CC5500}
\definecolor{brown}{HTML}{8B4513}
% Hyperlinks
\usepackage{bookmark}
\usepackage[colorlinks=true,linkcolor=blue,urlcolor=blue,citecolor=blue,anchorcolor=blue]{hyperref}
\usepackage{xcolor}
\hypersetup{
    colorlinks,
    linkcolor={red!50!black},
    citecolor={blue!50!black},
    urlcolor={blue!80!black}
}

% Text-related
\usepackage{blindtext}
\usepackage{fontsize}
\changefontsize[14]{14}
\setlength{\parindent}{0pt}
\linespread{1.2}

% Theorems and Definitions
\usepackage{amsthm}
\renewcommand\qedsymbol{$\blacksquare$}

% Define a new theorem style
\newtheoremstyle{mytheoremstyle}% name
  {}% Space above
  {}% Space below
  {}% Body font
  {}% Indent amount
  {\bfseries}% Theorem head font
  {.}% Punctuation after theorem head
  {.5em}% Space after theorem head
  {}% Theorem head spec (can be left empty, meaning ‘normal’)

% Apply the new theorem style to theorem-like environments
\theoremstyle{mytheoremstyle}

\newtheorem{theorem}{Theorem}[section]  
\newtheorem{definition}[theorem]{Definition} 
\newtheorem{lemma}[theorem]{Lemma}  
\newtheorem{corollary}[theorem]{Corollary}
\newtheorem{axiom}[theorem]{Axiom}
\newtheorem{example}[theorem]{Example}
\newtheorem{equiv_def}[theorem]{Equivalent Definition}

% tcolorbox Setup
\usepackage[most,many,breakable]{tcolorbox}
\tcbuselibrary{theorems}

% Define custom tcolorbox environments here...

%================================
% EXAMPLE BOX
%================================
% After you have defined the style and other theorem environments
\definecolor{myexamplebg}{RGB}{245, 245, 245} % Very light grey for background
\definecolor{myexamplefr}{RGB}{120, 120, 120} % Medium grey for frame
\definecolor{myexampleti}{RGB}{60, 60, 60}    % Darker grey for title

\newtcbtheorem[]{Example}{Example}{
    colback=myexamplebg,
    breakable,
    colframe=myexamplefr,
    coltitle=myexampleti,
    boxrule=1pt,
    sharp corners,
    detach title,
    before upper=\tcbtitle\par\vspace{-20pt}, % Reduced the space after the title
    fonttitle=\bfseries,
    description font=\mdseries,
    separator sign none,
    description delimiters={}{}, % No delimiters around the title
}{ex}
%================================
% Solution BOX
%================================
\makeatletter
\newtcolorbox{solution}{enhanced,
	breakable,
	colback=white,
	colframe=myg!80!black,
	attach boxed title to top left={yshift*=-\tcboxedtitleheight},
	title=Solution,
	boxed title size=title,
	boxed title style={%
			sharp corners,
			rounded corners=northwest,
			colback=tcbcolframe,
			boxrule=0pt,
		},
	underlay boxed title={%
			\path[fill=tcbcolframe] (title.south west)--(title.south east)
			to[out=0, in=180] ([xshift=5mm]title.east)--
			(title.center-|frame.east)
			[rounded corners=\kvtcb@arc] |-
			(frame.north) -| cycle;
		},
}
\makeatother

% %================================
% % Question BOX
% %================================
\makeatletter
\newtcbtheorem{question}{Question}{enhanced,
	breakable,
	colback=white,
	colframe=myb!80!black,
	attach boxed title to top left={yshift*=-\tcboxedtitleheight},
	fonttitle=\bfseries,
	title={#2},
	boxed title size=title,
	boxed title style={%
			sharp corners,
			rounded corners=northwest,
			colback=tcbcolframe,
			boxrule=0pt,
		},
	underlay boxed title={%
			\path[fill=tcbcolframe] (title.south west)--(title.south east)
			to[out=0, in=180] ([xshift=5mm]title.east)--
			(title.center-|frame.east)
			[rounded corners=\kvtcb@arc] |-
			(frame.north) -| cycle;
		},
	#1
}{question}
\makeatother

%%%%%%%%%%%%%%%%%%%%%%%%%%%%%%%%%%%%%%%%%%%
% TABLE OF CONTENTS
%%%%%%%%%%%%%%%%%%%%%%%%%%%%%%%%%%%%%%%%%%%


\usepackage{tikz}
\definecolor{doc}{RGB}{0,60,110}
\usepackage{titletoc}
\contentsmargin{0cm}
\titlecontents{chapter}[14pc]
{\addvspace{30pt}%
	\begin{tikzpicture}[remember picture, overlay]%
		\draw[fill=doc!60,draw=doc!60] (-7,-.1) rectangle (-0.9,.5);%
		\pgftext[left,x=-5.5cm,y=0.2cm]{\color{white}\Large\sc\bfseries Chapter\ \thecontentslabel};%
	\end{tikzpicture}\color{doc!60}\large\sc\bfseries}%
{}
{}
{\;\titlerule\;\large\sc\bfseries Page \thecontentspage
	\begin{tikzpicture}[remember picture, overlay]
		\draw[fill=doc!60,draw=doc!60] (2pt,0) rectangle (4,0.1pt);
	\end{tikzpicture}}%
\titlecontents{section}[3.7pc]
{\addvspace{2pt}}
{\contentslabel[\thecontentslabel]{3pc}}
{}
{\hfill\small \thecontentspage}
[]
\titlecontents*{subsection}[3.7pc]
{\addvspace{-1pt}\small}
{}
{}
{\ --- \small\thecontentspage}
[ \textbullet\ ][]

\makeatletter
\renewcommand{\tableofcontents}{
	\chapter*{%
	  \vspace*{-20\p@}%
	  \begin{tikzpicture}[remember picture, overlay]%
		  \pgftext[right,x=15cm,y=0.2cm]{\color{doc!60}\Huge\sc\bfseries \contentsname};%
		  \draw[fill=doc!60,draw=doc!60] (13,-.75) rectangle (20,1);%
		  \clip (13,-.75) rectangle (20,1);
		  \pgftext[right,x=15cm,y=0.2cm]{\color{white}\Huge\sc\bfseries \contentsname};%
	  \end{tikzpicture}}%
	\@starttoc{toc}}
\makeatother

\newcommand{\liff}{\llap{$\iff$}}
\newcommand{\rap}[1]{\rrap{\text{ (#1)}}}
\newcommand{\red}[1]{\textcolor{red}{#1}}
\newcommand{\blue}[1]{\textcolor{blue}{#1}}
\newcommand{\vi}[1]{\textcolor{violet}{#1}}
\newcommand{\olive}[1]{\textcolor{olive}{#1}}
\newcommand{\teal}[1]{\textcolor{teal}{#1}}
\newcommand{\brown}[1]{\textcolor{brown}{#1}}
\newcommand{\orange}[1]{\textcolor{orange}{#1}}
\newcommand{\tCaC}{\text{ \CaC }}
\newcommand{\CaC}{\red{CaC} }
\newcommand{\As}[1]{Assume \red{#1}}
\newcommand{\vdone}{\vi{\text{ (done) }}}
\newcommand{\bdone}{\blue{\text{ (done) }}}
\newcommand{\tdone}{\teal{\text{ (done) }}}
\newcommand{\odone}{\olive{\text{ (done) }}}
\newcommand{\bodone}{\brown{\text{ (done) }}}
\newcommand{\ordone}{\orange{\text{ (done) }}}
\newcommand{\ld}{\lambda}
\newcommand{\vecta}[1]{\textbf{#1}}
\newcommand{\set}[1]{\left\{ #1 \right\}}
\newcommand{\bset}[1]{\Big\{ #1 \Big\}}
\newcommand{\inR}{\in\R}
\newcommand{\inn}{\in\N}
\newcommand{\inz}{\in\Z}
\newcommand{\inr}{\in\R}
\newcommand{\inc}{\in\C}
\newcommand{\inq}{\in\Q}
\newcommand{\norm}[1]{\| #1 \|}
\newcommand{\bnorm}[1]{\Big\| #1 \Big\|}
\newcommand{\gen}[1]{\langle #1 \rangle}
\newcommand{\abso}[1]{\left|#1\right|}
\newcommand{\myref}[2]{\hyperref[#2]{#1\ \ref*{#2}}}
\newcommand{\customref}[2]{\hyperref[#1]{#2}}
\newcommand{\power}[1]{\mathcal{P}(#1)}
\newcommand{\dcup}{\mathbin{\dot{\cup}}}
\newcommand{\diam}[1]{\text{diam}\, #1}
\newcommand{\at}{\Big|}
\newcommand{\quotient}{\diagup}
\let\originalphi\phi % Store the original \phi in \originalphi
\renewcommand{\phi}{\varphi} % Redefine \phi to \varphi
\newcommand{\pfi}{\originalphi} % Define \pfi to display the original \phi
\newcommand{\diota}{\dot{\iota}}
\newcommand{\Log}{\operatorname{Log}}
\newcommand{\id}{\text{\textbf{id}}}
\usepackage{amsmath}

\makeatletter
\NewDocumentCommand{\extp}{e{^}}{%
  \mathop{\mathpalette\extp@{#1}}\nolimits
}
\NewDocumentCommand{\extp@}{mm}{%
  \bigwedge\nolimits\IfValueT{#2}{^{\extp@@{#1}#2}}%
  \IfValueT{#1}{\kern-2\scriptspace\nonscript\kern2\scriptspace}%
}
\newcommand{\extp@@}[1]{%
  \mkern
    \ifx#1\displaystyle-1.8\else
    \ifx#1\textstyle-1\else
    \ifx#1\scriptstyle-1\else
    -0.5\fi\fi\fi
  \thinmuskip
}
\makeatletter
\usepackage{pifont}
\makeatletter
\newcommand\Pimathsymbol[3][\mathord]{%
  #1{\@Pimathsymbol{#2}{#3}}}
\def\@Pimathsymbol#1#2{\mathchoice
  {\@Pim@thsymbol{#1}{#2}\tf@size}
  {\@Pim@thsymbol{#1}{#2}\tf@size}
  {\@Pim@thsymbol{#1}{#2}\sf@size}
  {\@Pim@thsymbol{#1}{#2}\ssf@size}}
\def\@Pim@thsymbol#1#2#3{%
  \mbox{\fontsize{#3}{#3}\Pisymbol{#1}{#2}}}
\makeatother
% the next two lines are needed to avoid LaTeX substituting upright from another font
\input{utxmia.fd}
\DeclareFontShape{U}{txmia}{m}{n}{<->ssub * txmia/m/it}{}
% you may also want
\DeclareFontShape{U}{txmia}{bx}{n}{<->ssub * txmia/bx/it}{}
% just in case
%\DeclareFontShape{U}{txmia}{l}{n}{<->ssub * txmia/l/it}{}
%\DeclareFontShape{U}{txmia}{b}{n}{<->ssub * txmia/b/it}{}
% plus info from Alan Munn at https://tex.stackexchange.com/questions/290165/how-do-i-get-a-nicer-lambda?noredirect=1#comment702120_290165
\newcommand{\pilambdaup}{\Pimathsymbol[\mathord]{txmia}{21}}
\renewcommand{\lambda}{\pilambdaup}
\renewcommand{\tilde}{\widetilde}
\DeclareMathOperator*{\esssup}{ess\,sup}
\newcommand{\bluecheck}{}%
\DeclareRobustCommand{\bluecheck}{%
  \tikz\fill[scale=0.4, color=blue]
  (0,.35) -- (.25,0) -- (1,.7) -- (.25,.15) -- cycle;%
}


\usepackage{tikz}
\newcommand*{\DashedArrow}[1][]{\mathbin{\tikz [baseline=-0.25ex,-latex, dashed,#1] \draw [#1] (0pt,0.5ex) -- (1.3em,0.5ex);}}

\newcommand{\C}{\mathbb{C}}	
\newcommand{\F}{\mathbb{F}}
\newcommand{\N}{\mathbb{N}}
\newcommand{\Q}{\mathbb{Q}}
\newcommand{\R}{\mathbb{R}}
\newcommand{\Z}{\mathbb{Z}}



\title{\Huge{HWs}}
\author{\huge{Eric Liu}}
\date{}
\begin{document}
\maketitle
\newpage% or \cleardoublepage
% \pdfbookmark[<level>]{<title>}{<dest>}
\pdfbookmark[section]{\contentsname}{toc}
\tableofcontents
\pagebreak

\chapter{HW for General Analysis}
\section{HW1}
\begin{theorem}
$\R^n$ is complete.
\end{theorem}
\begin{proof}
Let $\textbf{x}_k$ be an arbitrary Cauchy sequence in $\R^n$. We are required to show  $\textbf{x}_k$ converge in $\R^n$. For each $k$, denote  $\textbf{x}_k$ by $(x_{(1,k)},\dots ,x_{(n,k)})$. We claim that for each $i \in \set{1,\dots ,n}$
\begin{align*}
x_{(i,k)}\text{ is a Cauchy sequence }
\end{align*}
Fix $i$ and $\epsilon >0$. To show $x_{(i,k)}$ is a Cauchy sequence, we are required to find $N\inn$ such that for all $r,m\geq N$ we have 
\begin{align*}
\abso{x_{(i,r)}-x_{(i,m)}}\leq \epsilon 
\end{align*}
Because $\textbf{x}_k$ is a Cauchy sequence in $\R^n$, we know there exists  $N\inn$ such that for all $r,m\geq N$, we have 
\begin{align*}
\abso{\textbf{x}_r-\textbf{x}_m}< \epsilon 
\end{align*}
Fix such $N$ and arbitrary $r,m\geq M$. Observe 
\begin{align*}
\abso{x_{(i,r)}-x_{(i,m)}}\leq \sqrt{\sum_{j=1}^n \abso{x_{(j,r)}-x_{(j,m)}}^2}= \abso{\textbf{x}_r - \textbf{x}_m}< \epsilon 
\end{align*}
We have proved that for each $i\in \set{1,\dots ,n}$, the real sequence $x_{(i,k)}$ is Cauchy. We now claim that for each $i \in \set{1,\dots, n}$, we have 
\begin{align*}
  \limsup_{r\to\infty} x_{(i,r)}\inr \text{ and }\lim_{k\to \infty}x_{(i,k)}= \limsup_{r\to\infty} x_{(i,r)}
\end{align*}
Again fix $i$. Because $x_{(i,k)}$ is a Cauchy sequence, we know there exists some $N$ such that for all $r,m\geq N$, we have 
 \begin{align*}
\abso{x_{(i,r)}-x_{(i,m)}}<1
\end{align*}
This implies that for all $r\geq N$, we have 
\begin{align}
x_{(i,r)}< x_{(i,N)}+1 \label{ran1}
\end{align}
\myref{Equation}{ran1} then tell us 
\begin{align*}
x_{(i,N)}+1\text{ is an upper bound of }\set{x_{(i,r)}:r\geq N}
\end{align*}
Then by definition of $\sup $, we have
\begin{align*}
\sup \set{x_{(i,r)}:r\geq N} \leq x_{(i,N)}+1 \inr
\end{align*}
This then implies $\limsup_{r\to\infty} x_{(i,r)}\inr$. We now prove 
\begin{align}
\label{ran2}
\lim_{k\to \infty}x_{(i,k)}= \limsup_{r\to\infty} x_{(i,r)}
\end{align}
Fix $\epsilon >0$. We are required to find $N$ such that
\begin{align*}
\forall k\geq N, \abso{x_{(i,k)}-\limsup_{r\to\infty}x_{(i,r)}} \leq \epsilon 
\end{align*}
Because $\set{x_{(i,k)}}_{k\inn}$ is a Cauchy sequence, we can let $N_0$ satisfy 
\begin{align*}
\forall k,m\geq N_0, \abso{x_{(i,k)}-x_{(i,m)}}<\frac{\epsilon}{2}
\end{align*}
Because $\sup \set{x_{(i,k)}:k\geq N'}\searrow \limsup_{r\to\infty} x_{(i,r)}$ as $N'\to \infty$, we know there exists $N_1>N_0$ such that 
 \begin{align*}
\limsup_{r\to\infty} x_{(i,r)}-\frac{\epsilon}{2} <\sup \set{x_{(i,k)}:k\geq N_0}\leq \limsup_{r\to\infty}  x_{(i,r)} + \frac{\epsilon}{2} 
\end{align*}
Then because $\limsup_{r\to\infty} x_{(i,r)}-\frac{\epsilon}{2}$ is strictly smaller than the smallest upper bound of $\set{x_{(i,k)}:k\geq N_1}$, we see $\limsup_{n\to\infty} x_{(i,r)}-\frac{\epsilon}{2}$ is not an upper bound of $\set{x_{(i,k)}:k\geq N_1}$. This implies the existence of some $N$ such that  $N\geq N_1$ and
\begin{align*}
\limsup_{r\to\infty} x_{(i,r)}-\frac{\epsilon}{2}< x_{(i,N)}\leq \limsup_{r\to\infty}x_{(i,r)} + \frac{\epsilon}{2}
\end{align*}
Now observe that for all $k\geq N$, because $N\geq N_1\geq N_0$ 
\begin{align*}
  \limsup_{r\to\infty} x_{(i,r)} - \epsilon <x_{(i,N)}-\frac{\epsilon}{2}<x_{(i,k)}< x_{(i,N)}+\frac{\epsilon}{2} \leq \limsup_{r\to\infty} x_{(i,r)} + \epsilon   
\end{align*}
This implies for all $k\geq N$, we have 
\begin{align*}
\abso{x_{(i,k)}-\limsup_{r\to\infty} x_{(i,r)}}\leq \epsilon 
\end{align*}
We have just proved \myref{Equation}{ran2}. Lastly, to close out the proof, we show 
\begin{align}
\label{ran3}
\lim_{k\to \infty}\textbf{x}_k = \Big(\lim_{k\to \infty}x_{(1,k)},\dots ,\lim_{k\to \infty}x_{(n,k)}\Big)
\end{align}
Fix $\epsilon >0$. For each $i \in \set{1,\dots ,n}$, let $N_i$ satisfy 
 \begin{align*}
  \forall r\geq N_i ,\abso{x_{(i,r)}- \lim_{k\to \infty}x_{(i,k)}}\leq  \frac{\epsilon }{\sqrt{n} }
\end{align*}
Observe that for all $r\geq \max_{i\in \set{1,\dots,n}} N_i$, we have 
\begin{align*}
\abso{\textbf{x}_r - \Big(\lim_{k\to \infty}x_{(1,k)},\dots ,\lim_{k\to \infty}x_{(n,k)} \Big)}&=\sqrt{\sum_{i=1}^n \abso{x_{(i,r)}- \lim_{k\to \infty}x_{(i,k)}}^2} \\
&\leq \sqrt{\sum_{i=1}^n \frac{\epsilon^2}{n}}=\epsilon 
\end{align*}
We have proved \myref{Equation}{ran2}.











\end{proof}
\begin{theorem}
\textbf{($\Q$ is dense in $\R$)}
\end{theorem}
\chapter{Complex Analysis HW}
\section{HW1}
\begin{theorem}
\begin{align*}
  (1+i)^n, \frac{(1+i)^n}{n},\frac{n!}{(1+i)^n}\text{ all diverge as }n\to \infty
\end{align*}
\begin{proof}
Note that 
\begin{align*}
\abso{(1+i)^n}=2^{\frac{n}{2}}\to \infty\text{ as }n\to \infty
\end{align*}
This implies $(1+i)$ is unbounded, thus diverge.\\

Note that 
\begin{align*}
\abso{\frac{(1+i)^n}{n}}=\frac{(\sqrt{2})^n}{n}
\end{align*}
Observe 
\begin{align*}
\frac{(\sqrt{2})^n}{n}= \frac{[(\sqrt{2}-1)+1]^n}{n}&=\frac{\sum_{k=0}^n \binom{n}{k}(\sqrt{2}-1)^k}{n}\\
&\geq \frac{\binom{n}{2}(\sqrt{2}-1 )^2}{n}=(n-1) [\frac{(\sqrt{2}-1 )^2}{2}]\to \infty\text{ as }n\to \infty
\end{align*}
This implies $\frac{(1+i)^n}{n}$ is unbounded, thus diverge. \\

Note that 
\begin{align*}
\abso{\frac{n!}{(1+i)^n}}= \frac{n!}{(\sqrt{2})^n}
\end{align*}
Note that for all $k\geq 8$, we have 
\begin{align*}
\frac{k}{\sqrt{2}}\geq \frac{\sqrt{8}}{\sqrt{2}}=2
\end{align*}
This implies 
\begin{align*}
  \frac{n!}{(\sqrt{2})^n}=\prod_{k=1}^n \frac{k}{\sqrt{2}}= \frac{7!}{(\sqrt{2})^7}\prod_{k=8}^n \frac{k}{\sqrt{2}}\geq \frac{7!}{(\sqrt{2})^7}\prod_{k=8}^n 2\geq \frac{7!2^{n-8+1}}{(\sqrt{2})^7}\to \infty
\end{align*}
which implies $\frac{n!}{(1+i)^n}$ is unbounded, thus diverge.
\end{proof}
\end{theorem}
\begin{theorem}
  \begin{align*}
  n!z^n\text{ converge }\iff z=0
  \end{align*}
\end{theorem}
\begin{proof}
If $z=0$, then  $n!z^n=0$ for all  $n$, which implies  $n!z^n\to 0$. Now, suppose $z\neq 0$. Let $M\inn$ satisfy $\abso{z}>\frac{1}{M}$. Observe 
\begin{align*}
\abso{n!z^n}=\abso{\prod_{k=1}^n kz}=\abso{\prod_{k=1}^{2M-1} kz } \abso{\prod_{k=2M}^{n} kz }\geq \abso{\prod_{k=1}^{2M-1}kz}\abso{\prod_{k=2M}^n 2Mz}\geq \abso{\prod_{k=1}^{2M-1}kz}2^{n-2M+1}\to \infty
\end{align*}
This implies $n!z^n$ is unbounded, thus diverge.
\end{proof}
\begin{theorem}
  \begin{align*}
  u_n\to u\implies v_n\triangleq \sum_{k=1}^n \frac{u_k}{n}\to u
  \end{align*}
\end{theorem}
\begin{proof}
Because 
\begin{align*}
\sum_{k=1}^n \frac{u_k}{n}=\sum_{k\leq \sqrt{n}}\frac{u_k}{n}+ \sum_{\sqrt{n}<k\leq n}\frac{u_k}{n}
\end{align*}
It suffices to prove 
\begin{align*}
\vi{\sum_{k\leq \sqrt{n} }\frac{u_k}{n}\to 0}\text{ and }\blue{\sum_{\sqrt{n}<k\leq n}\frac{u_k}{n}\to u}\text{ as }n\to \infty
\end{align*}
Because $u_n$ converge, we can let  $M$ bound  $\abso{u_n}$. Observe 
\begin{align*}
\abso{\sum_{k\leq \sqrt{n} } \frac{u_k}{n}}\leq \sum_{k\leq \sqrt{n}} \abso{\frac{u_k}{n}}\leq \sum_{k\leq \sqrt{n}} \frac{M}{n}\leq \frac{M\sqrt{n} }{n}=\frac{M}{\sqrt{n}}\to 0\text{ as }n\to 0\vdone
\end{align*}
Because 
\begin{align*}
\sum_{\sqrt{n}<k\leq n} \frac{u_k}{n}= \frac{n-\lceil \sqrt{n}\rceil +1 }{n}\sum_{\sqrt{n}<k\leq n} \frac{u_k}{n-\lceil \sqrt{n}  \rceil+1}
\end{align*}
and 
\begin{align*}
  \lim_{n\to \infty} \frac{n- \lceil \sqrt{n}\rceil +1 }{n}=1
\end{align*}
We can reduce the problem into proving 
\begin{align*}
  \blue{\lim_{n\to \infty}\sum_{\sqrt{n}<k\leq n} \frac{u_k}{n- \lceil \sqrt{n}\rceil +1 }=u}
\end{align*}
Fix $\epsilon $. Let $N$ satisfy that for all  $n\geq N$, we have $\abso{u_n-u}<\epsilon $. Then for all $n \geq N^2$, we have 
\begin{align*}
  \abso{\Big(\sum_{\sqrt{n}<k \leq n } \frac{u_k}{n-\lceil \sqrt{n}\rceil +1 } \Big)- u}  &=\abso{\sum_{\sqrt{n}<k\leq n } \frac{u_k-u}{n-\lceil \sqrt{n}\rceil +1 } }\\
  &\leq \sum_{\sqrt{n}<k\leq n} \frac{\abso{u_k-u}}{n-\lceil \sqrt{n}\rceil +1  }\\
  &\leq \sum_{\sqrt{n}<k\leq n} \frac{\epsilon }{n-\lceil \sqrt{n}\rceil +1  }=\epsilon \bdone
\end{align*}
\end{proof}
\end{document}
