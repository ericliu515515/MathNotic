\documentclass{report}
%%%%%%%%%%%%%% macros.tex %%%%%%%%%%%%%%
% Place your custom macros here, if any.

%%%%%%%%%%%%%% letterfonts.tex %%%%%%%%%%%%%%
% Place your font setup here, if any.

%%%%%%%%%%%%%% preamble.tex %%%%%%%%%%%%%%
\usepackage[T1]{fontenc}
\usepackage{lmodern}
\usepackage{etoolbox}
\usepackage{pdfpages}
\usepackage{transparent}
\usepackage[utf8]{inputenc}
\usepackage[english]{babel}

% Page Setup
\usepackage[tmargin=2cm, rmargin=0.5in, lmargin=0.5in, bmargin=80pt, footskip=.2in]{geometry}

% Mathematics
\usepackage{amsmath,amsfonts,amsthm,amssymb,mathtools}
\usepackage{xfrac}
\usepackage[makeroom]{cancel}
\usepackage{enumitem}
\usepackage{nameref}
\usepackage{multicol,array}
\usepackage{tikz-cd}
\usepackage[ruled,vlined,linesnumbered]{algorithm2e}

% Colors
\usepackage[dvipsnames]{xcolor}
\definecolor{myg}{RGB}{56, 140, 70}
\definecolor{myb}{RGB}{45, 111, 177}
\definecolor{myr}{RGB}{199, 68, 64}
% Define more colors here...

% Hyperlinks
\usepackage{bookmark}
\usepackage{hyperref}
\hypersetup{
    pdftitle={Assignment},
    colorlinks=true, linkcolor=doc!90,
    bookmarksnumbered=true,
    bookmarksopen=true
}

% Figures and Graphics
\usepackage{import}
\usepackage{svg}
\newcommand{\incfig}[1]{%
    \def\svgwidth{\columnwidth}
    \import{./figures/}{#1.pdf_tex}
}

% Text-related
\usepackage{blindtext}
\usepackage{fontsize}
\changefontsize[14]{14}
\setlength{\parindent}{0pt}

% Theorems and Definitions
\usepackage{amsthm}
\renewcommand\qedsymbol{$\blacksquare$}

% Define a new theorem style
\newtheoremstyle{mytheoremstyle}% name
  {}% Space above
  {}% Space below
  {\sffamily}% Body font
  {}% Indent amount
  {\bfseries}% Theorem head font
  {.}% Punctuation after theorem head
  {.5em}% Space after theorem head
  {}% Theorem head spec (can be left empty, meaning ‘normal’)

% Apply the new theorem style to theorem-like environments
\theoremstyle{mytheoremstyle}
\newtheorem{theorem}{Theorem}[section]
\newtheorem{definition}{Definition}[section]
\newtheorem{corollary}{Corollary}[section]
\newtheorem{lemma}{Lemma}[section]
\newtheorem{axiom}{Axiom}[section]

% tcolorbox Setup
\usepackage[most,many,breakable]{tcolorbox}

% Define custom tcolorbox environments here...

%================================
% EXAMPLE BOX
%================================
\newtcbtheorem[definition]{Example}{Example}
{%
    colback = myexamplebg,
    breakable,
    colframe = myexamplefr,
    coltitle = myexampleti,
    boxrule = 1pt,
    sharp corners,
    detach title,
    before upper=\tcbtitle\par\smallskip,
    fonttitle = \bfseries,
    description font = \mdseries,
    separator sign none,
    description delimiters parenthesis,
}
{ex}

%================================
% Solution BOX
%================================
\makeatletter
\newtcolorbox{solution}{enhanced,
	breakable,
	colback=white,
	colframe=myg!80!black,
	attach boxed title to top left={yshift*=-\tcboxedtitleheight},
	title=Solution,
	boxed title size=title,
	boxed title style={%
			sharp corners,
			rounded corners=northwest,
			colback=tcbcolframe,
			boxrule=0pt,
		},
	underlay boxed title={%
			\path[fill=tcbcolframe] (title.south west)--(title.south east)
			to[out=0, in=180] ([xshift=5mm]title.east)--
			(title.center-|frame.east)
			[rounded corners=\kvtcb@arc] |-
			(frame.north) -| cycle;
		},
}
\makeatother

%================================
% Question BOX
%================================
\makeatletter
\newtcbtheorem{question}{Question}{enhanced,
	breakable,
	colback=white,
	colframe=myb!80!black,
	attach boxed title to top left={yshift*=-\tcboxedtitleheight},
	fonttitle=\bfseries,
	title={#2},
	boxed title size=title,
	boxed title style={%
			sharp corners,
			rounded corners=northwest,
			colback=tcbcolframe,
			boxrule=0pt,
		},
	underlay boxed title={%
			\path[fill=tcbcolframe] (title.south west)--(title.south east)
			to[out=0, in=180] ([xshift=5mm]title.east)--
			(title.center-|frame.east)
			[rounded corners=\kvtcb@arc] |-
			(frame.north) -| cycle;
		},
	#1
}{def}
\makeatother
\makeatletter
\newtcbtheorem{qstion}{Question}{enhanced,
    breakable,
    colback=white,
    colframe=mygr,
    attach boxed title to top left={yshift*=-\tcboxedtitleheight},
    fonttitle=\bfseries,
    title={#2},
    boxed title size=title,
    boxed title style={%
        sharp corners,
        rounded corners=northwest,
        colback=tcbcolframe,
        boxrule=0pt,
    },
    underlay boxed title={%
        \path[fill=tcbcolframe] (title.south west)--(title.south east)
        to[out=0, in=180] ([xshift=5mm]title.east)--
        (title.center-|frame.east)
        [rounded corners=\kvtcb@arc] |-
        (frame.north) -| cycle;
    },
    #1
}{def}
\makeatother

%%%%%%%%%%%%%%%%%%%%%%%%%%%%%%%%%%%%%%%%%%%
% TABLE OF CONTENTS
%%%%%%%%%%%%%%%%%%%%%%%%%%%%%%%%%%%%%%%%%%%
\usepackage{tikz}
\definecolor{doc}{RGB}{0,60,110}
\usepackage{titletoc}
\contentsmargin{0cm}
\titlecontents{chapter}[14pc]
{\addvspace{30pt}%
	\begin{tikzpicture}[remember picture, overlay]%
		\draw[fill=doc!60,draw=doc!60] (-7,-.1) rectangle (-0.9,.5);%
		\pgftext[left,x=-4.5cm,y=0.2cm]{\color{white}\Large\sc\bfseries Chapter\ \thecontentslabel};%
	\end{tikzpicture}\color{doc!60}\large\sc\bfseries}%
{}
{}
{\;\titlerule\;\large\sc\bfseries Page \thecontentspage
	\begin{tikzpicture}[remember picture, overlay]
		\draw[fill=doc!60,draw=doc!60] (2pt,0) rectangle (4,0.1pt);
	\end{tikzpicture}}%
\titlecontents{section}[3.7pc]
{\addvspace{2pt}}
{\contentslabel[\thecontentslabel]{2pc}}
{}
{\hfill\small \thecontentspage}
[]
\titlecontents*{subsection}[3.7pc]
{\addvspace{-1pt}\small}
{}
{}
{\ --- \small\thecontentspage}
[ \textbullet\ ][]

\makeatletter
\renewcommand{\tableofcontents}{
	\chapter*{%
	  \vspace*{-20\p@}%
	  \begin{tikzpicture}[remember picture, overlay]%
		  \pgftext[right,x=15cm,y=0.2cm]{\color{doc!60}\Huge\sc\bfseries \contentsname};%
		  \draw[fill=doc!60,draw=doc!60] (13,-.75) rectangle (20,1);%
		  \clip (13,-.75) rectangle (20,1);
		  \pgftext[right,x=15cm,y=0.2cm]{\color{white}\Huge\sc\bfseries \contentsname};%
	  \end{tikzpicture}}%
	\@starttoc{toc}}
\makeatother

\newcommand{\liff}{\llap{$\iff$}}
\newcommand{\rap}[1]{\rrap{\text{ (#1)}}}
\newcommand{\red}[1]{\textcolor{red}{#1}}
\newcommand{\blue}[1]{\textcolor{blue}{#1}}
\newcommand{\vi}[1]{\textcolor{violet}{#1}}
\newcommand{\teal}[1]{\textcolor{teal}{#1}}
\newcommand{\tCaC}{\text{ \CaC }}
\newcommand{\CaC}{\red{CaC} }
\newcommand{\As}[1]{Assume \red{#1}}
\newcommand{\vdone}{\vi{\text{ (done) }}}
\newcommand{\bdone}{\blue{\text{ (done) }}}
\newcommand{\tdone}{\teal{\text{ (done) }}}
\newcommand{\set}[1]{\{ #1 \}}
\newcommand{\inS}{\in S}
\newcommand{\inF}{\in\F}
\newcommand{\inE}{\in E}
\newcommand{\inA}{\in A}
\newcommand{\inB}{\in B}
\newcommand{\inC}{\in C}
\newcommand{\inU}{\in U}

\newcommand{\C}{\mathbb{C}}	
\renewcommand{\H}{\mathbb{H}}
\newcommand{\F}{\mathbb{F}}
\newcommand{\N}{\mathbb{N}}
\newcommand{\Q}{\mathbb{Q}}
\newcommand{\R}{\mathbb{R}}
\newcommand{\Z}{\mathbb{Z}}
\renewcommand{\P}{\mathbb{P}}
\renewcommand{\S}{\mathbb{S}}
\newcommand{\A}{\mathbb{A}}
\newcommand{\RP}{\R P}


\title{\Huge{HWs}}
\author{\huge{Eric Liu}}
\date{}
\begin{document}
\maketitle
\newpage% or \cleardoublepage
% \pdfbookmark[<level>]{<title>}{<dest>}
\pdfbookmark[section]{\contentsname}{toc}
\tableofcontents
\pagebreak

\chapter{HW for General Analysis}
\section{HW1}
\begin{theorem}
$\R^n$ is complete.
\end{theorem}
\begin{proof}
Let $\textbf{x}_k$ be an arbitrary Cauchy sequence in $\R^n$. We are required to show  $\textbf{x}_k$ converge in $\R^n$. For each $k$, denote  $\textbf{x}_k$ by $(x_{(1,k)},\dots ,x_{(n,k)})$. We claim that for each $i \in \set{1,\dots ,n}$
\begin{align*}
x_{(i,k)}\text{ is a Cauchy sequence }
\end{align*}
Fix $i$ and $\epsilon >0$. To show $x_{(i,k)}$ is a Cauchy sequence, we are required to find $N\inn$ such that for all $r,m\geq N$ we have 
\begin{align*}
\abso{x_{(i,r)}-x_{(i,m)}}\leq \epsilon 
\end{align*}
Because $\textbf{x}_k$ is a Cauchy sequence in $\R^n$, we know there exists  $N\inn$ such that for all $r,m\geq N$, we have 
\begin{align*}
\abso{\textbf{x}_r-\textbf{x}_m}< \epsilon 
\end{align*}
Fix such $N$ and arbitrary $r,m\geq M$. Observe 
\begin{align*}
\abso{x_{(i,r)}-x_{(i,m)}}\leq \sqrt{\sum_{j=1}^n \abso{x_{(j,r)}-x_{(j,m)}}^2}= \abso{\textbf{x}_r - \textbf{x}_m}< \epsilon 
\end{align*}
We have proved that for each $i\in \set{1,\dots ,n}$, the real sequence $x_{(i,k)}$ is Cauchy. We now claim that for each $i \in \set{1,\dots, n}$, we have 
\begin{align*}
  \limsup_{r\to\infty} x_{(i,r)}\inr \text{ and }\lim_{k\to \infty}x_{(i,k)}= \limsup_{r\to\infty} x_{(i,r)}
\end{align*}
Again fix $i$. Because $x_{(i,k)}$ is a Cauchy sequence, we know there exists some $N$ such that for all $r,m\geq N$, we have 
 \begin{align*}
\abso{x_{(i,r)}-x_{(i,m)}}<1
\end{align*}
This implies that for all $r\geq N$, we have 
\begin{align}
x_{(i,r)}< x_{(i,N)}+1 \label{ran1}
\end{align}
\myref{Equation}{ran1} then tell us 
\begin{align*}
x_{(i,N)}+1\text{ is an upper bound of }\set{x_{(i,r)}:r\geq N}
\end{align*}
Then by definition of $\sup $, we have
\begin{align*}
\sup \set{x_{(i,r)}:r\geq N} \leq x_{(i,N)}+1 \inr
\end{align*}
This then implies $\limsup_{r\to\infty} x_{(i,r)}\inr$. We now prove 
\begin{align}
\label{ran2}
\lim_{k\to \infty}x_{(i,k)}= \limsup_{r\to\infty} x_{(i,r)}
\end{align}
Fix $\epsilon >0$. We are required to find $N$ such that
\begin{align*}
\forall k\geq N, \abso{x_{(i,k)}-\limsup_{r\to\infty}x_{(i,r)}} \leq \epsilon 
\end{align*}
Because $\set{x_{(i,k)}}_{k\inn}$ is a Cauchy sequence, we can let $N_0$ satisfy 
\begin{align*}
\forall k,m\geq N_0, \abso{x_{(i,k)}-x_{(i,m)}}<\frac{\epsilon}{2}
\end{align*}
Because $\sup \set{x_{(i,k)}:k\geq N'}\searrow \limsup_{r\to\infty} x_{(i,r)}$ as $N'\to \infty$, we know there exists $N_1>N_0$ such that 
 \begin{align*}
\limsup_{r\to\infty} x_{(i,r)}-\frac{\epsilon}{2} <\sup \set{x_{(i,k)}:k\geq N_0}\leq \limsup_{r\to\infty}  x_{(i,r)} + \frac{\epsilon}{2} 
\end{align*}
Then because $\limsup_{r\to\infty} x_{(i,r)}-\frac{\epsilon}{2}$ is strictly smaller than the smallest upper bound of $\set{x_{(i,k)}:k\geq N_1}$, we see $\limsup_{n\to\infty} x_{(i,r)}-\frac{\epsilon}{2}$ is not an upper bound of $\set{x_{(i,k)}:k\geq N_1}$. This implies the existence of some $N$ such that  $N\geq N_1$ and
\begin{align*}
\limsup_{r\to\infty} x_{(i,r)}-\frac{\epsilon}{2}< x_{(i,N)}\leq \limsup_{r\to\infty}x_{(i,r)} + \frac{\epsilon}{2}
\end{align*}
Now observe that for all $k\geq N$, because $N\geq N_1\geq N_0$ 
\begin{align*}
  \limsup_{r\to\infty} x_{(i,r)} - \epsilon <x_{(i,N)}-\frac{\epsilon}{2}<x_{(i,k)}< x_{(i,N)}+\frac{\epsilon}{2} \leq \limsup_{r\to\infty} x_{(i,r)} + \epsilon   
\end{align*}
This implies for all $k\geq N$, we have 
\begin{align*}
\abso{x_{(i,k)}-\limsup_{r\to\infty} x_{(i,r)}}\leq \epsilon 
\end{align*}
We have just proved \myref{Equation}{ran2}. Lastly, to close out the proof, we show 
\begin{align}
\label{ran3}
\lim_{k\to \infty}\textbf{x}_k = \Big(\lim_{k\to \infty}x_{(1,k)},\dots ,\lim_{k\to \infty}x_{(n,k)}\Big)
\end{align}
Fix $\epsilon >0$. For each $i \in \set{1,\dots ,n}$, let $N_i$ satisfy 
 \begin{align*}
  \forall r\geq N_i ,\abso{x_{(i,r)}- \lim_{k\to \infty}x_{(i,k)}}\leq  \frac{\epsilon }{\sqrt{n} }
\end{align*}
Observe that for all $r\geq \max_{i\in \set{1,\dots,n}} N_i$, we have 
\begin{align*}
\abso{\textbf{x}_r - \Big(\lim_{k\to \infty}x_{(1,k)},\dots ,\lim_{k\to \infty}x_{(n,k)} \Big)}&=\sqrt{\sum_{i=1}^n \abso{x_{(i,r)}- \lim_{k\to \infty}x_{(i,k)}}^2} \\
&\leq \sqrt{\sum_{i=1}^n \frac{\epsilon^2}{n}}=\epsilon 
\end{align*}
We have proved \myref{Equation}{ran2}.











\end{proof}
\begin{theorem}
\textbf{($\Q$ is dense in $\R$)}
\end{theorem}
\chapter{Complex Analysis HW}
\section{HW1}
\begin{theorem}
\begin{align*}
  (1+i)^n, \frac{(1+i)^n}{n},\frac{n!}{(1+i)^n}\text{ all diverge as }n\to \infty
\end{align*}
\begin{proof}
Note that 
\begin{align*}
\abso{(1+i)^n}=2^{\frac{n}{2}}\to \infty\text{ as }n\to \infty
\end{align*}
This implies $(1+i)$ is unbounded, thus diverge.\\

Note that 
\begin{align*}
\abso{\frac{(1+i)^n}{n}}=\frac{(\sqrt{2})^n}{n}
\end{align*}
Observe 
\begin{align*}
\frac{(\sqrt{2})^n}{n}= \frac{[(\sqrt{2}-1)+1]^n}{n}&=\frac{\sum_{k=0}^n \binom{n}{k}(\sqrt{2}-1)^k}{n}\\
&\geq \frac{\binom{n}{2}(\sqrt{2}-1 )^2}{n}=(n-1) [\frac{(\sqrt{2}-1 )^2}{2}]\to \infty\text{ as }n\to \infty
\end{align*}
This implies $\frac{(1+i)^n}{n}$ is unbounded, thus diverge. \\

Note that 
\begin{align*}
\abso{\frac{n!}{(1+i)^n}}= \frac{n!}{(\sqrt{2})^n}
\end{align*}
Note that for all $k\geq 8$, we have 
\begin{align*}
\frac{k}{\sqrt{2}}\geq \frac{\sqrt{8}}{\sqrt{2}}=2
\end{align*}
This implies 
\begin{align*}
  \frac{n!}{(\sqrt{2})^n}=\prod_{k=1}^n \frac{k}{\sqrt{2}}= \frac{7!}{(\sqrt{2})^7}\prod_{k=8}^n \frac{k}{\sqrt{2}}\geq \frac{7!}{(\sqrt{2})^7}\prod_{k=8}^n 2\geq \frac{7!2^{n-8+1}}{(\sqrt{2})^7}\to \infty
\end{align*}
which implies $\frac{n!}{(1+i)^n}$ is unbounded, thus diverge.
\end{proof}
\end{theorem}
\begin{theorem}
  \begin{align*}
  n!z^n\text{ converge }\iff z=0
  \end{align*}
\end{theorem}
\begin{proof}
If $z=0$, then  $n!z^n=0$ for all  $n$, which implies  $n!z^n\to 0$. Now, suppose $z\neq 0$. Let $M\inn$ satisfy $\abso{z}>\frac{1}{M}$. Observe 
\begin{align*}
\abso{n!z^n}=\abso{\prod_{k=1}^n kz}=\abso{\prod_{k=1}^{2M-1} kz } \abso{\prod_{k=2M}^{n} kz }\geq \abso{\prod_{k=1}^{2M-1}kz}\abso{\prod_{k=2M}^n 2Mz}\geq \abso{\prod_{k=1}^{2M-1}kz}2^{n-2M+1}\to \infty
\end{align*}
This implies $n!z^n$ is unbounded, thus diverge.
\end{proof}
\begin{theorem}
  \begin{align*}
  u_n\to u\implies v_n\triangleq \sum_{k=1}^n \frac{u_k}{n}\to u
  \end{align*}
\end{theorem}
\begin{proof}
Because 
\begin{align*}
\sum_{k=1}^n \frac{u_k}{n}=\sum_{k\leq \sqrt{n}}\frac{u_k}{n}+ \sum_{\sqrt{n}<k\leq n}\frac{u_k}{n}
\end{align*}
It suffices to prove 
\begin{align*}
\vi{\sum_{k\leq \sqrt{n} }\frac{u_k}{n}\to 0}\text{ and }\blue{\sum_{\sqrt{n}<k\leq n}\frac{u_k}{n}\to u}\text{ as }n\to \infty
\end{align*}
Because $u_n$ converge, we can let  $M$ bound  $\abso{u_n}$. Observe 
\begin{align*}
\abso{\sum_{k\leq \sqrt{n} } \frac{u_k}{n}}\leq \sum_{k\leq \sqrt{n}} \abso{\frac{u_k}{n}}\leq \sum_{k\leq \sqrt{n}} \frac{M}{n}\leq \frac{M\sqrt{n} }{n}=\frac{M}{\sqrt{n}}\to 0\text{ as }n\to 0\vdone
\end{align*}
Because 
\begin{align*}
\sum_{\sqrt{n}<k\leq n} \frac{u_k}{n}= \frac{n-\lceil \sqrt{n}\rceil +1 }{n}\sum_{\sqrt{n}<k\leq n} \frac{u_k}{n-\lceil \sqrt{n}  \rceil+1}
\end{align*}
and 
\begin{align*}
  \lim_{n\to \infty} \frac{n- \lceil \sqrt{n}\rceil +1 }{n}=1
\end{align*}
We can reduce the problem into proving 
\begin{align*}
  \blue{\lim_{n\to \infty}\sum_{\sqrt{n}<k\leq n} \frac{u_k}{n- \lceil \sqrt{n}\rceil +1 }=u}
\end{align*}
Fix $\epsilon $. Let $N$ satisfy that for all  $n\geq N$, we have $\abso{u_n-u}<\epsilon $. Then for all $n \geq N^2$, we have 
\begin{align*}
  \abso{\Big(\sum_{\sqrt{n}<k \leq n } \frac{u_k}{n-\lceil \sqrt{n}\rceil +1 } \Big)- u}  &=\abso{\sum_{\sqrt{n}<k\leq n } \frac{u_k-u}{n-\lceil \sqrt{n}\rceil +1 } }\\
  &\leq \sum_{\sqrt{n}<k\leq n} \frac{\abso{u_k-u}}{n-\lceil \sqrt{n}\rceil +1  }\\
  &\leq \sum_{\sqrt{n}<k\leq n} \frac{\epsilon }{n-\lceil \sqrt{n}\rceil +1  }=\epsilon \bdone
\end{align*}
\end{proof}
\end{document}
