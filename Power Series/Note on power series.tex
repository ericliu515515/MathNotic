\documentclass{report}

%%%%%%%%%%%%%% macros.tex %%%%%%%%%%%%%%
% Place your custom macros here, if any.

%%%%%%%%%%%%%% letterfonts.tex %%%%%%%%%%%%%%
% Place your font setup here, if any.

%%%%%%%%%%%%%% preamble.tex %%%%%%%%%%%%%%
\usepackage[T1]{fontenc}
\usepackage{lmodern}
\usepackage{etoolbox}
\usepackage{pdfpages}
\usepackage{transparent}
\usepackage[utf8]{inputenc}
\usepackage[english]{babel}

% Page Setup
\usepackage[tmargin=2cm, rmargin=0.5in, lmargin=0.5in, bmargin=80pt, footskip=.2in]{geometry}

% Mathematics
\usepackage{amsmath,amsfonts,amsthm,amssymb,mathtools}
\usepackage{xfrac}
\usepackage[makeroom]{cancel}
\usepackage{enumitem}
\usepackage{nameref}
\usepackage{multicol,array}
\usepackage{tikz-cd}
\usepackage[ruled,vlined,linesnumbered]{algorithm2e}

% Colors
\usepackage[dvipsnames]{xcolor}
\definecolor{myg}{RGB}{56, 140, 70}
\definecolor{myb}{RGB}{45, 111, 177}
\definecolor{myr}{RGB}{199, 68, 64}
% Define more colors here...

% Hyperlinks
\usepackage{bookmark}
\usepackage{hyperref}
\hypersetup{
    pdftitle={Assignment},
    colorlinks=true, linkcolor=doc!90,
    bookmarksnumbered=true,
    bookmarksopen=true
}

% Figures and Graphics
\usepackage{import}
\usepackage{svg}
\newcommand{\incfig}[1]{%
    \def\svgwidth{\columnwidth}
    \import{./figures/}{#1.pdf_tex}
}

% Text-related
\usepackage{blindtext}
\usepackage{fontsize}
\changefontsize[14]{14}
\setlength{\parindent}{0pt}

% Theorems and Definitions
\usepackage{amsthm}
\renewcommand\qedsymbol{$\blacksquare$}

% Define a new theorem style
\newtheoremstyle{mytheoremstyle}% name
  {}% Space above
  {}% Space below
  {\sffamily}% Body font
  {}% Indent amount
  {\bfseries}% Theorem head font
  {.}% Punctuation after theorem head
  {.5em}% Space after theorem head
  {}% Theorem head spec (can be left empty, meaning ‘normal’)

% Apply the new theorem style to theorem-like environments
\theoremstyle{mytheoremstyle}
\newtheorem{theorem}{Theorem}[section]
\newtheorem{definition}{Definition}[section]
\newtheorem{corollary}{Corollary}[section]
\newtheorem{lemma}{Lemma}[section]
\newtheorem{axiom}{Axiom}[section]

% tcolorbox Setup
\usepackage[most,many,breakable]{tcolorbox}

% Define custom tcolorbox environments here...

%================================
% EXAMPLE BOX
%================================
\newtcbtheorem[definition]{Example}{Example}
{%
    colback = myexamplebg,
    breakable,
    colframe = myexamplefr,
    coltitle = myexampleti,
    boxrule = 1pt,
    sharp corners,
    detach title,
    before upper=\tcbtitle\par\smallskip,
    fonttitle = \bfseries,
    description font = \mdseries,
    separator sign none,
    description delimiters parenthesis,
}
{ex}

%================================
% Solution BOX
%================================
\makeatletter
\newtcolorbox{solution}{enhanced,
	breakable,
	colback=white,
	colframe=myg!80!black,
	attach boxed title to top left={yshift*=-\tcboxedtitleheight},
	title=Solution,
	boxed title size=title,
	boxed title style={%
			sharp corners,
			rounded corners=northwest,
			colback=tcbcolframe,
			boxrule=0pt,
		},
	underlay boxed title={%
			\path[fill=tcbcolframe] (title.south west)--(title.south east)
			to[out=0, in=180] ([xshift=5mm]title.east)--
			(title.center-|frame.east)
			[rounded corners=\kvtcb@arc] |-
			(frame.north) -| cycle;
		},
}
\makeatother

%================================
% Question BOX
%================================
\makeatletter
\newtcbtheorem{question}{Question}{enhanced,
	breakable,
	colback=white,
	colframe=myb!80!black,
	attach boxed title to top left={yshift*=-\tcboxedtitleheight},
	fonttitle=\bfseries,
	title={#2},
	boxed title size=title,
	boxed title style={%
			sharp corners,
			rounded corners=northwest,
			colback=tcbcolframe,
			boxrule=0pt,
		},
	underlay boxed title={%
			\path[fill=tcbcolframe] (title.south west)--(title.south east)
			to[out=0, in=180] ([xshift=5mm]title.east)--
			(title.center-|frame.east)
			[rounded corners=\kvtcb@arc] |-
			(frame.north) -| cycle;
		},
	#1
}{def}
\makeatother
\makeatletter
\newtcbtheorem{qstion}{Question}{enhanced,
    breakable,
    colback=white,
    colframe=mygr,
    attach boxed title to top left={yshift*=-\tcboxedtitleheight},
    fonttitle=\bfseries,
    title={#2},
    boxed title size=title,
    boxed title style={%
        sharp corners,
        rounded corners=northwest,
        colback=tcbcolframe,
        boxrule=0pt,
    },
    underlay boxed title={%
        \path[fill=tcbcolframe] (title.south west)--(title.south east)
        to[out=0, in=180] ([xshift=5mm]title.east)--
        (title.center-|frame.east)
        [rounded corners=\kvtcb@arc] |-
        (frame.north) -| cycle;
    },
    #1
}{def}
\makeatother

%%%%%%%%%%%%%%%%%%%%%%%%%%%%%%%%%%%%%%%%%%%
% TABLE OF CONTENTS
%%%%%%%%%%%%%%%%%%%%%%%%%%%%%%%%%%%%%%%%%%%
\usepackage{tikz}
\definecolor{doc}{RGB}{0,60,110}
\usepackage{titletoc}
\contentsmargin{0cm}
\titlecontents{chapter}[14pc]
{\addvspace{30pt}%
	\begin{tikzpicture}[remember picture, overlay]%
		\draw[fill=doc!60,draw=doc!60] (-7,-.1) rectangle (-0.9,.5);%
		\pgftext[left,x=-4.5cm,y=0.2cm]{\color{white}\Large\sc\bfseries Chapter\ \thecontentslabel};%
	\end{tikzpicture}\color{doc!60}\large\sc\bfseries}%
{}
{}
{\;\titlerule\;\large\sc\bfseries Page \thecontentspage
	\begin{tikzpicture}[remember picture, overlay]
		\draw[fill=doc!60,draw=doc!60] (2pt,0) rectangle (4,0.1pt);
	\end{tikzpicture}}%
\titlecontents{section}[3.7pc]
{\addvspace{2pt}}
{\contentslabel[\thecontentslabel]{2pc}}
{}
{\hfill\small \thecontentspage}
[]
\titlecontents*{subsection}[3.7pc]
{\addvspace{-1pt}\small}
{}
{}
{\ --- \small\thecontentspage}
[ \textbullet\ ][]

\makeatletter
\renewcommand{\tableofcontents}{
	\chapter*{%
	  \vspace*{-20\p@}%
	  \begin{tikzpicture}[remember picture, overlay]%
		  \pgftext[right,x=15cm,y=0.2cm]{\color{doc!60}\Huge\sc\bfseries \contentsname};%
		  \draw[fill=doc!60,draw=doc!60] (13,-.75) rectangle (20,1);%
		  \clip (13,-.75) rectangle (20,1);
		  \pgftext[right,x=15cm,y=0.2cm]{\color{white}\Huge\sc\bfseries \contentsname};%
	  \end{tikzpicture}}%
	\@starttoc{toc}}
\makeatother

\newcommand{\liff}{\llap{$\iff$}}
\newcommand{\rap}[1]{\rrap{\text{ (#1)}}}
\newcommand{\red}[1]{\textcolor{red}{#1}}
\newcommand{\blue}[1]{\textcolor{blue}{#1}}
\newcommand{\vi}[1]{\textcolor{violet}{#1}}
\newcommand{\teal}[1]{\textcolor{teal}{#1}}
\newcommand{\tCaC}{\text{ \CaC }}
\newcommand{\CaC}{\red{CaC} }
\newcommand{\As}[1]{Assume \red{#1}}
\newcommand{\vdone}{\vi{\text{ (done) }}}
\newcommand{\bdone}{\blue{\text{ (done) }}}
\newcommand{\tdone}{\teal{\text{ (done) }}}
\newcommand{\set}[1]{\{ #1 \}}
\newcommand{\inS}{\in S}
\newcommand{\inF}{\in\F}
\newcommand{\inE}{\in E}
\newcommand{\inA}{\in A}
\newcommand{\inB}{\in B}
\newcommand{\inC}{\in C}
\newcommand{\inU}{\in U}

\newcommand{\C}{\mathbb{C}}	
\renewcommand{\H}{\mathbb{H}}
\newcommand{\F}{\mathbb{F}}
\newcommand{\N}{\mathbb{N}}
\newcommand{\Q}{\mathbb{Q}}
\newcommand{\R}{\mathbb{R}}
\newcommand{\Z}{\mathbb{Z}}
\renewcommand{\P}{\mathbb{P}}
\renewcommand{\S}{\mathbb{S}}
\newcommand{\A}{\mathbb{A}}
\newcommand{\RP}{\R P}


\title{\Huge{NCKU 112.1}\\Sum of k-th Power}
\author{\huge{Eric Liu}}
\date{}
\begin{document}

\maketitle
\newpage% or \cleardoublepage
% \pdfbookmark[<level>]{<title>}{<dest>}
\pdfbookmark[section]{\contentsname}{toc}
\tableofcontents
\pagebreak
\chapter{Sum of 1,2,3th power}
\section{k=1}
\begin{theorem}
  Given a natural number $m$, We have the identity 
\begin{equation}
1+\cdots +m=\frac{m^2+m}{2}
\end{equation} 
\end{theorem} 
\begin{proof}
Observe that given any natural number $n$
\begin{align}
  n^2-( n-1 ) ^2 &= n^2-( n^2-2n+1 ) \\ &= 2n-1
\end{align} 
Then we can deduce an identity
\begin{align}
  n&=\frac{1}{2}[ n^2-( n-1 ) ^2+1 ]  
\end{align} 
Then
\begin{align}
  \sum_{n=1}^{m} n &= \sum_{n=1}^{m}\frac{1}{2}[n^2-( n-1 ) ^2+1] \\&= \frac{1}{2}[\sum_{n=1}^{m} n^2-\sum_{n=1}^{m}(n-1)^2+\sum_{n=1}^{m}1] \\&=\frac{1}{2}[ \sum_{n=1}^{m} n^2- \sum_{n=1}^{m-1} n^2+m ] \\&= \frac{m^2+m}{2}
\end{align} 
\end{proof} 
\section{k=2}%
\label{sec:power to 2}
\begin{theorem}
  Given a natural number $m$, We have the identity 
\begin{equation}
 1^2+\cdots +m^2= \frac{m^3}{3}+\frac{m^2}{2}+\frac{m}{6}
\end{equation} 
\end{theorem} 
\begin{proof}
Observe that given any natural number $n$
\begin{align}
  n^3-( n-1 ) ^3&=n^3-( n^3-3n^2+3n-1 ) \\&= 3n^2-3n+1
\end{align} 
Then we can deduce an identity
\begin{equation}
n^2=\frac{1}{3}[ n^3-( n-1 ) ^3+3n-1 ] 
\end{equation} 
Then 
\begin{align}
  \sum_{n=1}^{m} n^2&=\sum_{n=1}^{m}\frac{1}{3}[  n^3-( n-1 ) ^3+3n-1 ]\\&= \frac{m^3}{3}+\sum_{n=1}^{m} n-\frac{m}{3}               \\&=\frac{m^3}{3}+\frac{m^2}{2}+\frac{m}{6}
\end{align} 
\end{proof} 
\section{k=3}%
\begin{theorem}
  Given a natural number $m$, We have the identity 
\begin{equation}
1^3+\cdots +m^3=\frac{m^4}{4}+ \frac{m^3}{2}+\frac{m^2}{4}
\end{equation} 
\end{theorem} 
\begin{proof}
Observe that given any natural number $n$
\begin{align}
  n^{4}-( n-1 ) ^{4}&= n^{4}- ( n^{4}-4n^3+6n^2-4n+1 )\\&=4n^3-6n^2+4n-1 
\end{align} 
Then we can deduce an identity
\begin{equation}
 n^3=\frac{1}{4} [ n^4-( n-1 ) ^4+6n^2-4n+1 ]
\end{equation} 
Then
\begin{align}
  \sum_{n=1}^{m} n^3&=\frac{m^4}{4}+\frac{3}{2}\sum_{n=1}^{m} n^2-\sum_{n=1}^{m} n+\frac{m}{4} 
  \\&= \frac{m^4}{4}+\frac{3}{2}( \frac{m^3}{3}+\frac{m^2}{2}+\frac{m}{6} ) - (  \frac{m^2+m}{2}	) +\frac{m}{4}
  \\&=\frac{m^4}{4}+ \frac{m^3}{2}+\frac{m^2}{4}
\end{align} 
\end{proof} 
\chapter{Generalization}
\section{Summary}
So far, we have collected the following using the same method. It isn't difficult to show for all natural number $k$, the sum $\sum_{n=1}^{m} n^k$ can be expressed by a polynomial of $m$ of $k+1$ degree.
\begin{align}
  \sum_{n=1}^{m} n^0 &= m\\
  \sum_{n=1}^{m} n^1 &= \frac{m^2}{2}+\frac{m}{2}\\
  \sum_{n=1}^{m} n^2 &= \frac{m^3}{3}+\frac{m^2}{2}+\frac{m}{6}\\
  \sum_{n=1}^{m} n^3 &= \frac{m^4}{4}+\frac{m^3}{2}+\frac{m^2}{4}
\end{align} 
Now, we use the same method to give an inductive formula of sum of $k$-th power. Notice that this formula is very inefficient if $k$ is large.
\begin{theorem}
  Given a natural number $m$, We have the identity 
\begin{equation}
\sum_{n=1}^{m} n^k = \frac{1}{k+1}[m^{k+1}+\sum_{i=0}^{k-1} \binom{k+1}{i}(-1)^{k-i+1}\sum_{n=1}^{m} n^i ]
\end{equation}
\end{theorem}
\begin{proof}
\begin{align}
  n^{k+1}-(n-1)^{k+1}&= -\sum_{i=0}^{k} \binom{k+1}{i}n^i(-1)^{k+1-i}\\
                     &=\sum_{i=0}^{k} \binom{k+1}{i}n^i(-1)^{k-i}\\
                     &= \binom{k+1}{k}n^k(-1)^0+\sum_{i-0}^{k-1} \binom{k+1}{i}n^i(-1)^{k-i}\\
                     &= (k+1)n^k+\sum_{i=0}^{k-1} \binom{k+1}{i}n^i(-1)^{k-i}\\
  (k+1)n^k &= n^{k+1}-(n-1)^{k+1}+\sum_{i=0}^{k-1} \binom{k+1}{i}n^i(-1)^{k-i+1}\\
  n^k &=\frac{1}{k+1}[n^{k+1}-(n-1)^{k+1}+\sum_{i=0}^{k-1} \binom{k+1}{i} n^i(-1)^{k-i+1}]\\
  \sum_{n=1}^{m} n^k&=\frac{1}{k+1}[m^{k+1}+\sum_{n=1}^{m} \sum_{i=0}^{k-1} \binom{k+1}{i}n^i(-1)^{k-i+1}]\\
                    &= \frac{1}{k+1}[m^{k+1}+\sum_{i=0}^{k-1} \binom{k+1}{i}(-1)^{k-i+1}\sum_{n=1}^{m} n^i]
\end{align} 
\end{proof}

Now we show an intreseting property of the sum of $k$-th power.
\begin{theorem}
If $m,k$ are natural numbers, then $\sum_{n=1}^{m} n^k$ is a polynomial of $m$, where the sum of coefficient is $1$
\end{theorem}
\begin{proof}
Let $f$ be a function that maps a polynomial of $m$ to the sum of coefficients of the polynomial.
We prove our theorem by induction.\\

We know that $\sum_{n=1}^{m}n^0=m$, which finish the proof for base case. Given a natural number $r$, assume that $\forall u\leq r\in \N, f(\sum_{n=1}^{m}n^u)=1$.\\

Notice
\begin{align}
  0=(1-1)^{r+2} &= \sum_{i=0}^{r+2}\binom{r+2}{i}(1)^i(-1)^{r+2-i}\\
                &= [\sum_{i=0}^{r}\binom{r+2}{i}(-1)^{r-i}]-(r+2)+1\\
  r+1 &= \sum_{i=0}^{r}\binom{r+2}{i}(-1)^{r-i} 
\end{align} 
Notice that $f$ is a linear function. We use this fact to deduce
\begin{align}
  \sum_{n=1}^{m}n^{r+1} &= \frac{1}{r+2}[m^{r+2}+\sum_{i=0}^{r}\binom{r+2}{i}(-1)^{r-i}\sum_{n=1}^{m}n^i]\\
  f(\sum_{n=1}^{m}n^{r+1}) &=f(\frac{1}{r+2}[m^{r+2}+\sum_{i=0}^{r}\binom{r+2}{i}(-1)^{r-i}\sum_{n=1}^{m}n^i])\\
                           &= \frac{1}{r+2}[f(m^{r+2})+\sum_{i=0}^{r}\binom{r+2}{i}(-1)^{r-i}f(\sum_{n=1}^{m}n^i)]\\
                           &= \frac{1}{r+2}(1+\sum_{i=0}^{r}\binom{r+2}{i}(-1)^{r-i})\\
                           &= \frac{1}{r+2}(1+r+1)\\
                           &= 1
\end{align} 
\end{proof}
\end{document}
