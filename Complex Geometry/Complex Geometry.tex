\documentclass{report}
%%%%%%%%%%%%%% macros.tex %%%%%%%%%%%%%%
% Place your custom macros here, if any.

%%%%%%%%%%%%%% letterfonts.tex %%%%%%%%%%%%%%
% Place your font setup here, if any.

%%%%%%%%%%%%%% preamble.tex %%%%%%%%%%%%%%
\usepackage[T1]{fontenc}
\usepackage{lmodern}
\usepackage{etoolbox}
\usepackage{pdfpages}
\usepackage{transparent}
\usepackage[utf8]{inputenc}
\usepackage[english]{babel}

% Page Setup
\usepackage[tmargin=2cm, rmargin=0.5in, lmargin=0.5in, bmargin=80pt, footskip=.2in]{geometry}

% Mathematics
\usepackage{amsmath,amsfonts,amsthm,amssymb,mathtools}
\usepackage{xfrac}
\usepackage[makeroom]{cancel}
\usepackage{enumitem}
\usepackage{nameref}
\usepackage{multicol,array}
\usepackage{tikz-cd}
\usepackage[ruled,vlined,linesnumbered]{algorithm2e}

% Colors
\usepackage[dvipsnames]{xcolor}
\definecolor{myg}{RGB}{56, 140, 70}
\definecolor{myb}{RGB}{45, 111, 177}
\definecolor{myr}{RGB}{199, 68, 64}
% Define more colors here...

% Hyperlinks
\usepackage{bookmark}
\usepackage{hyperref}
\hypersetup{
    pdftitle={Assignment},
    colorlinks=true, linkcolor=doc!90,
    bookmarksnumbered=true,
    bookmarksopen=true
}

% Figures and Graphics
\usepackage{import}
\usepackage{svg}
\newcommand{\incfig}[1]{%
    \def\svgwidth{\columnwidth}
    \import{./figures/}{#1.pdf_tex}
}

% Text-related
\usepackage{blindtext}
\usepackage{fontsize}
\changefontsize[14]{14}
\setlength{\parindent}{0pt}

% Theorems and Definitions
\usepackage{amsthm}
\renewcommand\qedsymbol{$\blacksquare$}

% Define a new theorem style
\newtheoremstyle{mytheoremstyle}% name
  {}% Space above
  {}% Space below
  {\sffamily}% Body font
  {}% Indent amount
  {\bfseries}% Theorem head font
  {.}% Punctuation after theorem head
  {.5em}% Space after theorem head
  {}% Theorem head spec (can be left empty, meaning ‘normal’)

% Apply the new theorem style to theorem-like environments
\theoremstyle{mytheoremstyle}
\newtheorem{theorem}{Theorem}[section]
\newtheorem{definition}{Definition}[section]
\newtheorem{corollary}{Corollary}[section]
\newtheorem{lemma}{Lemma}[section]
\newtheorem{axiom}{Axiom}[section]

% tcolorbox Setup
\usepackage[most,many,breakable]{tcolorbox}

% Define custom tcolorbox environments here...

%================================
% EXAMPLE BOX
%================================
\newtcbtheorem[definition]{Example}{Example}
{%
    colback = myexamplebg,
    breakable,
    colframe = myexamplefr,
    coltitle = myexampleti,
    boxrule = 1pt,
    sharp corners,
    detach title,
    before upper=\tcbtitle\par\smallskip,
    fonttitle = \bfseries,
    description font = \mdseries,
    separator sign none,
    description delimiters parenthesis,
}
{ex}

%================================
% Solution BOX
%================================
\makeatletter
\newtcolorbox{solution}{enhanced,
	breakable,
	colback=white,
	colframe=myg!80!black,
	attach boxed title to top left={yshift*=-\tcboxedtitleheight},
	title=Solution,
	boxed title size=title,
	boxed title style={%
			sharp corners,
			rounded corners=northwest,
			colback=tcbcolframe,
			boxrule=0pt,
		},
	underlay boxed title={%
			\path[fill=tcbcolframe] (title.south west)--(title.south east)
			to[out=0, in=180] ([xshift=5mm]title.east)--
			(title.center-|frame.east)
			[rounded corners=\kvtcb@arc] |-
			(frame.north) -| cycle;
		},
}
\makeatother

%================================
% Question BOX
%================================
\makeatletter
\newtcbtheorem{question}{Question}{enhanced,
	breakable,
	colback=white,
	colframe=myb!80!black,
	attach boxed title to top left={yshift*=-\tcboxedtitleheight},
	fonttitle=\bfseries,
	title={#2},
	boxed title size=title,
	boxed title style={%
			sharp corners,
			rounded corners=northwest,
			colback=tcbcolframe,
			boxrule=0pt,
		},
	underlay boxed title={%
			\path[fill=tcbcolframe] (title.south west)--(title.south east)
			to[out=0, in=180] ([xshift=5mm]title.east)--
			(title.center-|frame.east)
			[rounded corners=\kvtcb@arc] |-
			(frame.north) -| cycle;
		},
	#1
}{def}
\makeatother
\makeatletter
\newtcbtheorem{qstion}{Question}{enhanced,
    breakable,
    colback=white,
    colframe=mygr,
    attach boxed title to top left={yshift*=-\tcboxedtitleheight},
    fonttitle=\bfseries,
    title={#2},
    boxed title size=title,
    boxed title style={%
        sharp corners,
        rounded corners=northwest,
        colback=tcbcolframe,
        boxrule=0pt,
    },
    underlay boxed title={%
        \path[fill=tcbcolframe] (title.south west)--(title.south east)
        to[out=0, in=180] ([xshift=5mm]title.east)--
        (title.center-|frame.east)
        [rounded corners=\kvtcb@arc] |-
        (frame.north) -| cycle;
    },
    #1
}{def}
\makeatother

%%%%%%%%%%%%%%%%%%%%%%%%%%%%%%%%%%%%%%%%%%%
% TABLE OF CONTENTS
%%%%%%%%%%%%%%%%%%%%%%%%%%%%%%%%%%%%%%%%%%%
\usepackage{tikz}
\definecolor{doc}{RGB}{0,60,110}
\usepackage{titletoc}
\contentsmargin{0cm}
\titlecontents{chapter}[14pc]
{\addvspace{30pt}%
	\begin{tikzpicture}[remember picture, overlay]%
		\draw[fill=doc!60,draw=doc!60] (-7,-.1) rectangle (-0.9,.5);%
		\pgftext[left,x=-4.5cm,y=0.2cm]{\color{white}\Large\sc\bfseries Chapter\ \thecontentslabel};%
	\end{tikzpicture}\color{doc!60}\large\sc\bfseries}%
{}
{}
{\;\titlerule\;\large\sc\bfseries Page \thecontentspage
	\begin{tikzpicture}[remember picture, overlay]
		\draw[fill=doc!60,draw=doc!60] (2pt,0) rectangle (4,0.1pt);
	\end{tikzpicture}}%
\titlecontents{section}[3.7pc]
{\addvspace{2pt}}
{\contentslabel[\thecontentslabel]{2pc}}
{}
{\hfill\small \thecontentspage}
[]
\titlecontents*{subsection}[3.7pc]
{\addvspace{-1pt}\small}
{}
{}
{\ --- \small\thecontentspage}
[ \textbullet\ ][]

\makeatletter
\renewcommand{\tableofcontents}{
	\chapter*{%
	  \vspace*{-20\p@}%
	  \begin{tikzpicture}[remember picture, overlay]%
		  \pgftext[right,x=15cm,y=0.2cm]{\color{doc!60}\Huge\sc\bfseries \contentsname};%
		  \draw[fill=doc!60,draw=doc!60] (13,-.75) rectangle (20,1);%
		  \clip (13,-.75) rectangle (20,1);
		  \pgftext[right,x=15cm,y=0.2cm]{\color{white}\Huge\sc\bfseries \contentsname};%
	  \end{tikzpicture}}%
	\@starttoc{toc}}
\makeatother

\newcommand{\liff}{\llap{$\iff$}}
\newcommand{\rap}[1]{\rrap{\text{ (#1)}}}
\newcommand{\red}[1]{\textcolor{red}{#1}}
\newcommand{\blue}[1]{\textcolor{blue}{#1}}
\newcommand{\vi}[1]{\textcolor{violet}{#1}}
\newcommand{\teal}[1]{\textcolor{teal}{#1}}
\newcommand{\tCaC}{\text{ \CaC }}
\newcommand{\CaC}{\red{CaC} }
\newcommand{\As}[1]{Assume \red{#1}}
\newcommand{\vdone}{\vi{\text{ (done) }}}
\newcommand{\bdone}{\blue{\text{ (done) }}}
\newcommand{\tdone}{\teal{\text{ (done) }}}
\newcommand{\set}[1]{\{ #1 \}}
\newcommand{\inS}{\in S}
\newcommand{\inF}{\in\F}
\newcommand{\inE}{\in E}
\newcommand{\inA}{\in A}
\newcommand{\inB}{\in B}
\newcommand{\inC}{\in C}
\newcommand{\inU}{\in U}

\newcommand{\C}{\mathbb{C}}	
\renewcommand{\H}{\mathbb{H}}
\newcommand{\F}{\mathbb{F}}
\newcommand{\N}{\mathbb{N}}
\newcommand{\Q}{\mathbb{Q}}
\newcommand{\R}{\mathbb{R}}
\newcommand{\Z}{\mathbb{Z}}
\renewcommand{\P}{\mathbb{P}}
\renewcommand{\S}{\mathbb{S}}
\newcommand{\A}{\mathbb{A}}
\newcommand{\RP}{\R P}


\title{Eric's note on Complex Geometry}
\author{Eric Liu}
\date{}
\begin{document}
\maketitle
\newpage% or \cleardoublepage
% \pdfbookmark[<level>]{<title>}{<dest>}
\pdfbookmark[section]{\contentsname}{toc}

\tableofcontents
\pagebreak
\chapter{Single Variable Complex Analysis}
\section{Quick Recap}
Given a complex-valued function $u+iv=f(x+iy)$ defined on some neighborhood of $z\inc$, we say $f$ is \textbf{complex-differentiable} at $z$ if there exists some complex number denoted by $f'(x)$ such that 
\begin{align*}
\frac{f(z+h)-f(z)-f'(z)h}{h}\to 0 \text{ as }h\to 0;h \inc
\end{align*}
If $f$ is complex-differentiable at $z$, then obviously $f$ satisfies the \textbf{Cauchy-Riemann equation}
\begin{align*}
\frac{\partial u}{\partial x}= \frac{\partial v}{\partial y} \text{ and }\frac{\partial u}{\partial y}= -\frac{\partial v}{\partial x}\text{ at }z
\end{align*}
The converse hold true under an extra condition. If $u,v$ satisfy the Cauchy-Riemann equation at $z$ and are real-differentiable at $z$, then from a direct estimation, $f$ is also complex-differentiable at $z$. Given open $U\subseteq \C$ and some complex-valued function $f:U\rightarrow \C$, we say $f$ is \textbf{holomorphic} if $f$ is complex-differentiable on all $z \in U$. In this note, if we say $\gamma :[a,b]\rightarrow \C$ is $C^1$, we mean there exists some  $C^1$ map $\tilde{\gamma }:(a-\epsilon ,b+\epsilon )\rightarrow \C $ such that $\gamma =\tilde{\gamma }|_{[a,b]} $. By a \textbf{contour}, precisely, we mean a map $\gamma :[ a,b]\rightarrow \C$ such that for a finite set of points  $\set{a=x_0<x_1<\cdots <x_n < b=x_{n+1}}$, the maps $\gamma  |_{[x_i,x_{i+1}]}$ are $C^1$ and 
\begin{align*}
\gamma'(t)\neq 0,\quad\text{for all }t \in [x_{i},x_{i+1}]
\end{align*}
Given some contour $\gamma :[a,b]\rightarrow \C$ and some $z$ that does not lie in the image of $\gamma $, we define the \textbf{winding number} of $z$ with respect to  $\gamma $ to be 
\begin{align*}
\operatorname{Ind}_\gamma (z)\triangleq  \frac{1}{2\pi i}\int_\gamma \frac{d\xi}{\xi -z}
\end{align*}
If the contour $\gamma :[a,b]\rightarrow \C$ is \textbf{closed}\footnote{$\gamma  (a)=\gamma (b)$.}, by setting 
\begin{align*}
f(t)\triangleq  \frac{1}{2\pi i}\int_a^{a+t} \frac{\gamma '(s)}{\gamma (s)-z}ds
\end{align*}
and by noting that  $\frac{d}{dt}[e^{-2\pi  if(t)}(\gamma (t)-z)]$ is zero everywhere, we see that the winding number $\operatorname{Ind}_\gamma (z)$ is indeed an integer as expected. Moreover, because $\operatorname{Ind}_\gamma $ is continuous on $\C \setminus \set{\gamma (t):t \in [a,b]}$ \footnote{One may prove this continuity by direct estimation.}, we see $\operatorname{Ind}_\gamma $ is constant on each connected component of $\C \setminus \set{\gamma (t):t \in [a,b]}$. Now, by a \textbf{domain}, we mean an nonempty open connected subset of $\C$. Finally, we may state our version of \textbf{Cauchy's Integral Theorem}.  
\begin{theorem}
\textbf{(Cauchy's Integral Theorem)} Given some domain $D$, some holomorphic function $f:D\rightarrow \C$, and some closed contour $\gamma :[a,b]\rightarrow D$ that does not wind around any point in $D \setminus \set{\gamma (t):t \in [a,b]}$, we have 
\begin{align*}
\int_\gamma f=0
\end{align*}
\end{theorem}
Cauchy's Integral Theorem is the cornerstone of complex analysis. Its proof fundamentally relies on triangulation and its special case for triangles. For brevity, the proof is presented \customref{CITA}{here}. Note that when integrating along the boundary of a disk, the orientation matters unless the integral equals $0$. To simplify matters, we adopt the universal convention that integration is always performed counterclockwise. Now, by a geometric arguments using 'cuts', we have \textbf{Cauchy's Integral Formula}, stating that if $f$ is holomorphic on  $\abso{z-z_0}<r$, then  
\begin{align*}
f(z)= \frac{1}{2\pi  i}\int_{\partial B_\epsilon (z_0) }\frac{f(\xi)}{\xi - z}d\xi,\quad\text{ for all }\epsilon <r
\end{align*}
This with the decomposition 
\begin{align}
\label{dexiz}
\frac{1}{\xi-z}= \frac{1}{\xi-z_0} + \frac{z-z_0}{(\xi -z_0)^2} + \cdots + \frac{(z-z_0)^n}{(\xi-z_0)^{n+1}}+  \frac{(z-z_0)^{n+1}}{(\xi-z)(\xi-z_0)^{n+1}}
\end{align}
and an estimation using $\abso{z-z_0} < \abso{\xi-z_0}$ shows that holomorphic functions are locally power series 
\begin{align*}
f(z)= \sum_{n=0}^{\infty} \frac{1}{2\pi i}\Big(\int_{\partial B_\epsilon (z_0)} \frac{f(\xi)}{(\xi- z_0)^{n+1}}d\xi \Big) (z-z_0)^{n},\quad\text{for all }\epsilon <r \text{ and }z\in B_\epsilon (z_0)
\end{align*}


Because all power series converge uniformly on disk with radius strictly smaller than its convergence radius, we may differentiate term by term and have \textbf{Taylor's Theorem for power series} 
\begin{align*}
f(z)= \sum_{n=0}^{\infty} \frac{f^{(n)}(z_0)}{n!}(z-z_0)^n
\end{align*}
If $D\subseteq \C$ is a domain and $f:D\rightarrow \C$ is holomorphic, Taylor's Theorem for power series tell us that $\set{z\in D: f^{(n)}(z)=0\text{ for all }n\geq 0}$ is not only closed in $D$ but also open, and thus equals to $D$ if proved nonempty. One particularly weak condition for $T$ to be nonempty  is that $f\equiv 0$ on some $S \subseteq D$, and $S$ has a limit point in  $D$. This result is commonly referred to as \textbf{Identity Theorem}. By an \textbf{entire} function, we mean a holomorphic function $f:\C\rightarrow \C$ defined on the whole complex plane. Obviously, for all $r>0$ and $z \in B_r(0)$, we have 
\begin{align*}
f(z)= \sum_{n=0}^{\infty}c_nz^n,\quad\text{where }c_n=\frac{1}{2\pi i} \int_{\partial B_r(0)} \frac{f(\xi)}{\xi^{n+1}}d\xi
\end{align*}
If $f$ is bounded, then direct estimations show that $c_n=0$ for all $n>0$. This result is commonly referred to as \textbf{Liouville's Theorem}. Suppose $f$ is holomorphic on some annulus $r<\abso{z-z_0}<R$. Cauchy's integral theorem, Cauchy's integral formula and a geometric argument using 'cuts' give us 
\begin{align*}
f(z)= \frac{1}{2\pi  i} \Big(\int_{\partial B_{R- \epsilon }(z_0)} \frac{f(\xi)}{\xi - z}d\xi - \int_{\partial B_{r+\epsilon }(z_0)} \frac{f(\xi)}{\xi -z}d\xi  \Big)
\end{align*}
This with \myref{decomposition}{dexiz} and the following decomposition: 
\begin{align*}
\frac{1}{z-\xi}= \frac{1}{z-z_0}+ \frac{\xi - z_0}{(z-z_0)^2}+ \cdots + \frac{(\xi -z_0)^{n-1}}{(z-z_0)^n} - \frac{(\xi - z_0)^n}{(z-z_0)^n (\xi -z)}
\end{align*}
and two estimations that use 
\begin{align*}
\abso{z-z_0}< \abso{\xi - z_0}\text{ for }\xi \in \partial B_{R-\epsilon} (z_0)\text{ and } \abso{\xi - z_0} < \abso{z-z_0} \text{ for }\xi \in \partial B_{r+\epsilon } (z_0)
\end{align*}
shows that 
\begin{align*}
f(z)&= \sum_{n=0}^{\infty} \frac{1}{2\pi  i}\Big( \int_{\partial  B_{R- \epsilon }(z_0)} \frac{f(\xi)}{(\xi - z_0)^{n+1}}d\xi \Big) (z-z_0)^n \\
&+ \sum_{n=1}^{\infty} \frac{1}{2\pi i}\Big( \int_{\partial B_{r+\epsilon }(z_0)} (\xi- z_0)^{n-1}f(\xi)d \xi \Big)(z-z_0)^{-n}
\end{align*}
Because the integrands $(\xi -z_0)^kf(\xi)$ have no singularities on the annulus, again, we may apply a geometric argument using 'cuts' to simplify the expression into its \textbf{Laurent series}:
\begin{align*}
  f(z)= \sum_{n=-\infty}^{\infty}c_n (z-z_0)^n
\end{align*}
where
\begin{align}
\label{cnp}
c_n= \frac{1}{2\pi i} \int_{\partial B_\epsilon (z_0)} \frac{f(\xi)}{(\xi - z_0)^{n+1}}d\xi \text{ for all } r<\epsilon  <R \text{ and }n\inz
\end{align}
If $f$ is defined and holomorphic on some deleted neighborhood $B_\epsilon (z_0)\setminus \set{z_0}$ but not defined on $\set{z_0}$, we say $z_0$ is an  \textbf{isolated singularity} of $f$, and we write 
\begin{align*}
\operatorname{Res}(f,z_0)\triangleq c_{-1}
\end{align*}
to denote the \textbf{residue} of $f$ at $z_0$. Let $D\subseteq \C$ be a simply connected domain, and suppose holomorphic $f$ is defined on $D$ except at some finite numbers of singularities. Let $\gamma :[a,b]\rightarrow D$ be some \textbf{simple}\footnote{By simple, we mean  $\gamma (t)=\gamma (s)$ if and only if $\abso{t-s}=\abso{a-b}$} closed contour, so that the image of $\gamma $ is a Jordan curve. Under this condition, we may apply the \textbf{Jordan Curve Theorem} to distinguish between the interior and the exterior of $\gamma $. A simple closed contour $\gamma $ is \textbf{positively oriented} if the winding number is positive in the region enclosed by $\gamma $. We may now state our version of \textbf{Cauchy's Residue Theorem}. 
\begin{theorem}
\textbf{(Cauchy's Residue Theorem)}  Let $D\subseteq \C$ be a simply connected domain, and let $\gamma :[a,b]\rightarrow D$ be some positively oriented simple closed contour. If $f$ is is defined and holomorphic on  $D$ except at a finite set of points $\set{z_1,\dots ,z_n}$ that are all enclosed by $\gamma $, then 
\begin{align*}
\int_\gamma f(\xi)d\xi = 2\pi  i\sum_{j=1}^n \operatorname{Res}(f,z_j) 
\end{align*}
\end{theorem}
There are many distinct rigorous proofs for Cauchy's residue Theorem. None of them are trivial by some geometric argument. Again, for brevity, we present a proof \customref{CITA}{here} using Green's Theorem. Suppose $z_0$ is an isolated singularity of  $f$, and  $c_n$ are the coefficients of the Laurent series of  $f$ about $z_0$. There are three types of singularities depending on $c_n$. If $c_n=0$ for all  $n<0$, then  $z_0$ is  said to be a \textbf{removable singularity}. By direct estimation\footnote{For each $n<0$, let $\epsilon \to 0$ in \myref{Equation}{cnp}.}, we see that if  $f$ is bounded on some deleted neighborhood of $z_0$, then  $z_0$ is removable. This recognition is called  \textbf{Riemann's removable singularity Theorem}. If there exists some sequence  $n_k$ of integers that converges to  $-\infty$ such that $c_{n_k}\neq 0$ for all $k$, then we say  $z_0$  is an  \textbf{essential singularity}. The last type of singularities is perhaps the most interesting. If there exists some $m<0$ such that  $c_m\neq 0$ and $c_n=0$ for all  $n<m$, we say  $z_0$ is a \textbf{pole} of $f$ with multiplicity $m$. In such case, obviously we may define some $g$ by 
\begin{align*}
g(z)\triangleq (z-z_0)^mf(z)\text{ for all }z\neq z_0
\end{align*}
so that $z_0$ is merely a removable singularity of $g$, and $g(z_0)\neq 0$ after the removal. Now, because $g$ is continuous at  $z_0$, we see  $f$ is nonzero on some neighborhood around $z_0$, and we may compute on that neighborhood: 
\begin{align*}
\frac{f'(z)}{f(z)}= \frac{g'(z)}{g(z)}- \frac{m}{z-z_0} \text{ for all }z\neq z_0
\end{align*}
This implies 
\begin{align*}
\operatorname{Res}\Big(\frac{f'}{f},z_0\Big)=-m
\end{align*}
Similarly, if $z_0$ is a  \textbf{zero}\footnote{By $z_0$ being a zero of  $f$ with multiplicity  $k$, we mean  $f$ is holomorphically defined on some neighborhood of $z_0$, $f(z_0)=0$, and $k$ is the smallest integer such that  $c_k\neq 0$ where $f(z)=\sum_{n=0}^{\infty}c_n(z-z_0)^n$. Note that zeros and poles are dual to each other.}  of $f$ with multiplicity $k$, we may define  $g$ by 
\begin{align*}
g(z)\triangleq (z-z_0)^{-k}f(z)
\end{align*}
and compute 
\begin{align*}
\frac{f'(z)}{f(z)}= \frac{g'(z)}{g(z)}+ \frac{k}{z-z_0}\text{ for all }z\neq z_0 \text{ in some neighborhood of }z_0
\end{align*}
to deduce 
\begin{align*}
\operatorname{Res}\Big(\frac{f'}{f},z_0 \Big)=k
\end{align*}
These observations together with Cauchy's Residue Theorem now give us the \textbf{Argument Principle}. Given simply connected domain $D\subseteq \C$, positively oriented simple closed contour $\gamma :[a,b]\rightarrow D$ and some $f$ \textbf{meromorphic}\footnote{By $f$ being meromorphic on open $U$, we mean that $f$ is holomorphic on $U$ except on a finite set of poles.} on $D$, if $f$ has no zeros and has no poles on the image of $\gamma $, then   
\begin{align*}
\int_{\gamma } \frac{f'(\xi)}{f(\xi)}d\xi = 2\pi  i (Z-P)
\end{align*}
where $Z$ and $P$ are respectively the numbers of zeros and poles enclosed by $\gamma $ counted with multiplicity.   \\

Now, let $D$ be some simply connected domain,  let $\gamma :[a,b]\rightarrow D$ be some positively oriented simple closed contour, and let $f,g:D\rightarrow \C$ be two holomorphic function. If we require that $\abso{g}<\abso{f}$ on the image of $\gamma $, then obviously neither $f$, $f+g$ nor $1+\frac{g}{f}$ can have a zero on the image of $\gamma $, so after we compute 
\begin{align*}
\frac{(1+\frac{g}{f})'}{1+ \frac{g}{f}}=\frac{(f+g)'}{f+g}- \frac{f'}{f} 
\end{align*}
we may apply the argument principle to $f$ and  $g$ to conclude 
\begin{align*}
Z_{f+g}-Z_f= \int_{\gamma } \frac{(1+\frac{g}{f})'}{1+\frac{g}{f}}d\xi
\end{align*}
where $Z_{f+g}$ and $Z_f$ are the numbers of zeros of  $f+g$ and  $f$ enclosed by $\gamma $ counted with multiplicity. Moreover, if we define $\tilde{\gamma }:[a,b]\rightarrow \C $ by 
\begin{align*}
\tilde{\gamma }(t)\triangleq  1 + \frac{g(t)}{f(t)} 
\end{align*}
we see 
\begin{align*}
\int_{\tilde{\gamma }} \frac{d\xi}{\xi}= \int_{\gamma } \frac{(1+ \frac{g}{f})'}{1+\frac{g}{f}}d\xi = Z_{f+g}-Z_f 
\end{align*}
Finally, noting that $\tilde{\gamma }$ as a simple closed contour lies in $\operatorname{Re}(z)>0$ \footnote{This is because  $\abso{\frac{g}{f}}<1$ on $\gamma $.}, and that $\log(\xi)$ is well defined on $\operatorname{Re}(z)>0$ with derivative $\frac{1}{\xi}$\footnote{By real inverse function theorem.}, we can finally conclude the \textbf{Rouché's theorem}: 
\begin{align*}
Z_{f+g}-Z_f=0
\end{align*}
\section{Riemann Mapping Theorem}
\section{Cauchy's Integral and Residue Theorem}
\label{CITA}
\end{document}
