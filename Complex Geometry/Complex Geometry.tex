\documentclass{report}
%%%%%%%%%%%%%% preamble.tex %%%%%%%%%%%%%%
\usepackage[T1]{fontenc}
\usepackage{etoolbox}
% Page Setup
\usepackage[letterpaper, tmargin=2cm, rmargin=0.5in, lmargin=0.5in, bmargin=80pt, footskip=.2in]{geometry}
\usepackage{adjustbox}
\usepackage{graphicx}
\usepackage{tikz}
\usepackage{mathrsfs}
\usepackage{mdframed}

% Create a new toggle
\newtoggle{firstsection}

% Redefine the \chapter command to reset the toggle for each new chapter
\let\oldchapter\chapter
\renewcommand{\chapter}{\toggletrue{firstsection}\oldchapter}

% Redefine the \section command to check the toggle
\let\oldsection\section
\renewcommand{\section}{
    \iftoggle{firstsection}
    {\togglefalse{firstsection}} % If it's the first section, just switch off the toggle for next sections
    {\clearpage} % If it's not the first section, start a new page
    \oldsection
}

% Abstract Design

\usepackage{lipsum}

\renewenvironment{abstract}
 {% Start of environment
  \quotation
  \small
  \noindent
  \rule{\linewidth}{.5pt} % Draw the rule to match the linewidth
  \par\smallskip
  {\centering\bfseries\abstractname\par}\medskip
 }
 {% End of environment
  \par\noindent
  \rule{\linewidth}{.5pt} % Ensure the closing rule also matches
  \endquotation
 }

% Mathematics
\usepackage{amsmath,amsfonts,amsthm,amssymb,mathtools}
\usepackage{xfrac}
\usepackage[makeroom]{cancel}
\usepackage{enumitem}
\usepackage{nameref}
\usepackage{multicol,array}
\usepackage{tikz-cd}
\usepackage{array}
\usepackage{multirow}% http://ctan.org/pkg/multirow
\usepackage{graphicx}

% Colors
\usepackage[dvipsnames]{xcolor}
\definecolor{myg}{RGB}{56, 140, 70}
\definecolor{myb}{RGB}{45, 111, 177}
\definecolor{myr}{RGB}{199, 68, 64}
% Define more colors here...
\definecolor{olive}{HTML}{6B8E23}
\definecolor{orange}{HTML}{CC5500}
\definecolor{brown}{HTML}{8B4513}
% Hyperlinks
\usepackage{bookmark}
\usepackage[colorlinks=true,linkcolor=blue,urlcolor=blue,citecolor=blue,anchorcolor=blue]{hyperref}
\usepackage{xcolor}
\hypersetup{
    colorlinks,
    linkcolor={red!50!black},
    citecolor={blue!50!black},
    urlcolor={blue!80!black}
}

% Text-related
\usepackage{blindtext}
\usepackage{fontsize}
\changefontsize[14]{14}
\setlength{\parindent}{0pt}
\linespread{1.2}

% Theorems and Definitions
\usepackage{amsthm}
\renewcommand\qedsymbol{$\blacksquare$}

% Define a new theorem style
\newtheoremstyle{mytheoremstyle}% name
  {}% Space above
  {}% Space below
  {}% Body font
  {}% Indent amount
  {\bfseries}% Theorem head font
  {.}% Punctuation after theorem head
  {.5em}% Space after theorem head
  {}% Theorem head spec (can be left empty, meaning ‘normal’)

% Apply the new theorem style to theorem-like environments
\theoremstyle{mytheoremstyle}

\newtheorem{theorem}{Theorem}[section]  
\newtheorem{definition}[theorem]{Definition} 
\newtheorem{lemma}[theorem]{Lemma}  
\newtheorem{corollary}[theorem]{Corollary}
\newtheorem{axiom}[theorem]{Axiom}
\newtheorem{example}[theorem]{Example}
\newtheorem{equiv_def}[theorem]{Equivalent Definition}

% tcolorbox Setup
\usepackage[most,many,breakable]{tcolorbox}
\tcbuselibrary{theorems}

% Define custom tcolorbox environments here...

%================================
% EXAMPLE BOX
%================================
% After you have defined the style and other theorem environments
\definecolor{myexamplebg}{RGB}{245, 245, 245} % Very light grey for background
\definecolor{myexamplefr}{RGB}{120, 120, 120} % Medium grey for frame
\definecolor{myexampleti}{RGB}{60, 60, 60}    % Darker grey for title

\newtcbtheorem[]{Example}{Example}{
    colback=myexamplebg,
    breakable,
    colframe=myexamplefr,
    coltitle=myexampleti,
    boxrule=1pt,
    sharp corners,
    detach title,
    before upper=\tcbtitle\par\vspace{-20pt}, % Reduced the space after the title
    fonttitle=\bfseries,
    description font=\mdseries,
    separator sign none,
    description delimiters={}{}, % No delimiters around the title
}{ex}
%================================
% Solution BOX
%================================
\makeatletter
\newtcolorbox{solution}{enhanced,
	breakable,
	colback=white,
	colframe=myg!80!black,
	attach boxed title to top left={yshift*=-\tcboxedtitleheight},
	title=Solution,
	boxed title size=title,
	boxed title style={%
			sharp corners,
			rounded corners=northwest,
			colback=tcbcolframe,
			boxrule=0pt,
		},
	underlay boxed title={%
			\path[fill=tcbcolframe] (title.south west)--(title.south east)
			to[out=0, in=180] ([xshift=5mm]title.east)--
			(title.center-|frame.east)
			[rounded corners=\kvtcb@arc] |-
			(frame.north) -| cycle;
		},
}
\makeatother

% %================================
% % Question BOX
% %================================
\makeatletter
\newtcbtheorem{question}{Question}{enhanced,
	breakable,
	colback=white,
	colframe=myb!80!black,
	attach boxed title to top left={yshift*=-\tcboxedtitleheight},
	fonttitle=\bfseries,
	title={#2},
	boxed title size=title,
	boxed title style={%
			sharp corners,
			rounded corners=northwest,
			colback=tcbcolframe,
			boxrule=0pt,
		},
	underlay boxed title={%
			\path[fill=tcbcolframe] (title.south west)--(title.south east)
			to[out=0, in=180] ([xshift=5mm]title.east)--
			(title.center-|frame.east)
			[rounded corners=\kvtcb@arc] |-
			(frame.north) -| cycle;
		},
	#1
}{question}
\makeatother

%%%%%%%%%%%%%%%%%%%%%%%%%%%%%%%%%%%%%%%%%%%
% TABLE OF CONTENTS
%%%%%%%%%%%%%%%%%%%%%%%%%%%%%%%%%%%%%%%%%%%


\usepackage{tikz}
\definecolor{doc}{RGB}{0,60,110}
\usepackage{titletoc}
\contentsmargin{0cm}
\titlecontents{chapter}[14pc]
{\addvspace{30pt}%
	\begin{tikzpicture}[remember picture, overlay]%
		\draw[fill=doc!60,draw=doc!60] (-7,-.1) rectangle (-0.9,.5);%
		\pgftext[left,x=-5.5cm,y=0.2cm]{\color{white}\Large\sc\bfseries Chapter\ \thecontentslabel};%
	\end{tikzpicture}\color{doc!60}\large\sc\bfseries}%
{}
{}
{\;\titlerule\;\large\sc\bfseries Page \thecontentspage
	\begin{tikzpicture}[remember picture, overlay]
		\draw[fill=doc!60,draw=doc!60] (2pt,0) rectangle (4,0.1pt);
	\end{tikzpicture}}%
\titlecontents{section}[3.7pc]
{\addvspace{2pt}}
{\contentslabel[\thecontentslabel]{3pc}}
{}
{\hfill\small \thecontentspage}
[]
\titlecontents*{subsection}[3.7pc]
{\addvspace{-1pt}\small}
{}
{}
{\ --- \small\thecontentspage}
[ \textbullet\ ][]

\makeatletter
\renewcommand{\tableofcontents}{
	\chapter*{%
	  \vspace*{-20\p@}%
	  \begin{tikzpicture}[remember picture, overlay]%
		  \pgftext[right,x=15cm,y=0.2cm]{\color{doc!60}\Huge\sc\bfseries \contentsname};%
		  \draw[fill=doc!60,draw=doc!60] (13,-.75) rectangle (20,1);%
		  \clip (13,-.75) rectangle (20,1);
		  \pgftext[right,x=15cm,y=0.2cm]{\color{white}\Huge\sc\bfseries \contentsname};%
	  \end{tikzpicture}}%
	\@starttoc{toc}}
\makeatother

\newcommand{\liff}{\llap{$\iff$}}
\newcommand{\rap}[1]{\rrap{\text{ (#1)}}}
\newcommand{\red}[1]{\textcolor{red}{#1}}
\newcommand{\blue}[1]{\textcolor{blue}{#1}}
\newcommand{\vi}[1]{\textcolor{violet}{#1}}
\newcommand{\olive}[1]{\textcolor{olive}{#1}}
\newcommand{\teal}[1]{\textcolor{teal}{#1}}
\newcommand{\brown}[1]{\textcolor{brown}{#1}}
\newcommand{\orange}[1]{\textcolor{orange}{#1}}
\newcommand{\tCaC}{\text{ \CaC }}
\newcommand{\CaC}{\red{CaC} }
\newcommand{\As}[1]{Assume \red{#1}}
\newcommand{\vdone}{\vi{\text{ (done) }}}
\newcommand{\bdone}{\blue{\text{ (done) }}}
\newcommand{\tdone}{\teal{\text{ (done) }}}
\newcommand{\odone}{\olive{\text{ (done) }}}
\newcommand{\bodone}{\brown{\text{ (done) }}}
\newcommand{\ordone}{\orange{\text{ (done) }}}
\newcommand{\ld}{\lambda}
\newcommand{\vecta}[1]{\textbf{#1}}
\newcommand{\set}[1]{\left\{ #1 \right\}}
\newcommand{\bset}[1]{\Big\{ #1 \Big\}}
\newcommand{\inR}{\in\R}
\newcommand{\inn}{\in\N}
\newcommand{\inz}{\in\Z}
\newcommand{\inr}{\in\R}
\newcommand{\inc}{\in\C}
\newcommand{\inq}{\in\Q}
\newcommand{\norm}[1]{\| #1 \|}
\newcommand{\bnorm}[1]{\Big\| #1 \Big\|}
\newcommand{\gen}[1]{\langle #1 \rangle}
\newcommand{\abso}[1]{\left|#1\right|}
\newcommand{\myref}[2]{\hyperref[#2]{#1\ \ref*{#2}}}
\newcommand{\customref}[2]{\hyperref[#1]{#2}}
\newcommand{\power}[1]{\mathcal{P}(#1)}
\newcommand{\dcup}{\mathbin{\dot{\cup}}}
\newcommand{\diam}[1]{\text{diam}\, #1}
\newcommand{\at}{\Big|}
\newcommand{\quotient}{\diagup}
\let\originalphi\phi % Store the original \phi in \originalphi
\renewcommand{\phi}{\varphi} % Redefine \phi to \varphi
\newcommand{\pfi}{\originalphi} % Define \pfi to display the original \phi
\newcommand{\diota}{\dot{\iota}}
\newcommand{\Log}{\operatorname{Log}}
\newcommand{\id}{\text{\textbf{id}}}
\usepackage{amsmath}

\makeatletter
\NewDocumentCommand{\extp}{e{^}}{%
  \mathop{\mathpalette\extp@{#1}}\nolimits
}
\NewDocumentCommand{\extp@}{mm}{%
  \bigwedge\nolimits\IfValueT{#2}{^{\extp@@{#1}#2}}%
  \IfValueT{#1}{\kern-2\scriptspace\nonscript\kern2\scriptspace}%
}
\newcommand{\extp@@}[1]{%
  \mkern
    \ifx#1\displaystyle-1.8\else
    \ifx#1\textstyle-1\else
    \ifx#1\scriptstyle-1\else
    -0.5\fi\fi\fi
  \thinmuskip
}
\makeatletter
\usepackage{pifont}
\makeatletter
\newcommand\Pimathsymbol[3][\mathord]{%
  #1{\@Pimathsymbol{#2}{#3}}}
\def\@Pimathsymbol#1#2{\mathchoice
  {\@Pim@thsymbol{#1}{#2}\tf@size}
  {\@Pim@thsymbol{#1}{#2}\tf@size}
  {\@Pim@thsymbol{#1}{#2}\sf@size}
  {\@Pim@thsymbol{#1}{#2}\ssf@size}}
\def\@Pim@thsymbol#1#2#3{%
  \mbox{\fontsize{#3}{#3}\Pisymbol{#1}{#2}}}
\makeatother
% the next two lines are needed to avoid LaTeX substituting upright from another font
\input{utxmia.fd}
\DeclareFontShape{U}{txmia}{m}{n}{<->ssub * txmia/m/it}{}
% you may also want
\DeclareFontShape{U}{txmia}{bx}{n}{<->ssub * txmia/bx/it}{}
% just in case
%\DeclareFontShape{U}{txmia}{l}{n}{<->ssub * txmia/l/it}{}
%\DeclareFontShape{U}{txmia}{b}{n}{<->ssub * txmia/b/it}{}
% plus info from Alan Munn at https://tex.stackexchange.com/questions/290165/how-do-i-get-a-nicer-lambda?noredirect=1#comment702120_290165
\newcommand{\pilambdaup}{\Pimathsymbol[\mathord]{txmia}{21}}
\renewcommand{\lambda}{\pilambdaup}
\renewcommand{\tilde}{\widetilde}
\DeclareMathOperator*{\esssup}{ess\,sup}
\newcommand{\bluecheck}{}%
\DeclareRobustCommand{\bluecheck}{%
  \tikz\fill[scale=0.4, color=blue]
  (0,.35) -- (.25,0) -- (1,.7) -- (.25,.15) -- cycle;%
}


\usepackage{tikz}
\newcommand*{\DashedArrow}[1][]{\mathbin{\tikz [baseline=-0.25ex,-latex, dashed,#1] \draw [#1] (0pt,0.5ex) -- (1.3em,0.5ex);}}

\newcommand{\C}{\mathbb{C}}	
\newcommand{\F}{\mathbb{F}}
\newcommand{\N}{\mathbb{N}}
\newcommand{\Q}{\mathbb{Q}}
\newcommand{\R}{\mathbb{R}}
\newcommand{\Z}{\mathbb{Z}}



\title{Eric's note on Complex Geometry}
\author{Eric Liu}
\date{}
\begin{document}
\maketitle
\newpage% or \cleardoublepage
% \pdfbookmark[<level>]{<title>}{<dest>}
\pdfbookmark[section]{\contentsname}{toc}

\tableofcontents
\pagebreak
\chapter{Single Variable Complex Analysis}
\section{Quick Recap}
Given a complex-valued function $u+iv=f(x+iy)$ defined on some neighborhood of $z\inc$, we say $f$ is \textbf{complex-differentiable} at $z$ if there exists some complex number denoted by $f'(x)$ such that 
\begin{align*}
\frac{f(z+h)-f(z)-f'(z)h}{h}\to 0 \text{ as }h\to 0;h \inc
\end{align*}
If $f$ is complex-differentiable at $z$, then obviously $f$ satisfies the \textbf{Cauchy-Riemann equation}
\begin{align*}
\frac{\partial u}{\partial x}= \frac{\partial v}{\partial y} \text{ and }\frac{\partial u}{\partial y}= -\frac{\partial v}{\partial x}\text{ at }z
\end{align*}
The converse hold true under an extra condition. If $u,v$ satisfy the Cauchy-Riemann equation at $z$ and are real-differentiable at $z$, then from a direct estimation, $f$ is also complex-differentiable at $z$. Given open $U\subseteq \C$ and some complex-valued function $f:U\rightarrow \C$, we say $f$ is \textbf{holomorphic} if $f$ is complex-differentiable on all $z \in U$. In this note, if we say $\gamma :[a,b]\rightarrow \C$ is $C^1$, we mean there exists some  $C^1$ map $\tilde{\gamma }:(a-\epsilon ,b+\epsilon )\rightarrow \C $ such that $\gamma =\tilde{\gamma }|_{[a,b]} $. By a \textbf{contour}, precisely, we mean a map $\gamma :[ a,b]\rightarrow \C$ such that for a finite set of points  $\set{a=x_0<x_1<\cdots <x_n < b=x_{n+1}}$, the maps $\gamma  |_{[x_i,x_{i+1}]}$ are $C^1$ and 
\begin{align*}
\gamma'(t)\neq 0,\quad\text{for all }t \in [x_{i},x_{i+1}]
\end{align*}
Given some contour $\gamma :[a,b]\rightarrow \C$ and some $z$ that does not lie in the image of $\gamma $, we define the \textbf{winding number} of $z$ with respect to  $\gamma $ to be 
\begin{align*}
\operatorname{Ind}_\gamma (z)\triangleq  \frac{1}{2\pi i}\int_\gamma \frac{d\xi}{\xi -z}
\end{align*}
If the contour $\gamma :[a,b]\rightarrow \C$ is \textbf{closed}\footnote{$\gamma  (a)=\gamma (b)$.}, by setting 
\begin{align*}
f(t)\triangleq  \frac{1}{2\pi i}\int_a^{a+t} \frac{\gamma '(s)}{\gamma (s)-z}ds
\end{align*}
and by noting that  $\frac{d}{dt}[e^{-2\pi  if(t)}(\gamma (t)-z)]$ is zero everywhere, we see that the winding number $\operatorname{Ind}_\gamma (z)$ is indeed an integer as expected. Moreover, because $\operatorname{Ind}_\gamma $ is continuous on $\C \setminus \set{\gamma (t):t \in [a,b]}$ \footnote{One may prove this continuity by direct estimation.}, we see $\operatorname{Ind}_\gamma $ is constant on each connected component of $\C \setminus \set{\gamma (t):t \in [a,b]}$. Now, by a \textbf{domain}, we mean an nonempty open connected subset of $\C$. Finally, we may state our version of \textbf{Cauchy's Integral Theorem}.  
\begin{theorem}
\textbf{(Cauchy's Integral Theorem)} Given some domain $D$, some holomorphic function $f:D\rightarrow \C$, and some closed contour $\gamma :[a,b]\rightarrow D$ that does not wind around any point in $D \setminus \set{\gamma (t):t \in [a,b]}$, we have 
\begin{align*}
\int_\gamma f=0
\end{align*}
\end{theorem}
Cauchy's Integral Theorem is the cornerstone of complex analysis. Its proof fundamentally relies on triangulation and its special case for triangles. For brevity, the proof is presented \customref{CITA}{here}. Note that when integrating along the boundary of a disk, the orientation matters unless the integral equals $0$. To simplify matters, we adopt the universal convention that integration is always performed counterclockwise. Now, by a geometric arguments using 'cuts', we have \textbf{Cauchy's Integral Formula}, stating that if $f$ is holomorphic on  $\abso{z-z_0}<r$, then  
\begin{align*}
f(z)= \frac{1}{2\pi  i}\int_{\partial B_\epsilon (z_0) }\frac{f(\xi)}{\xi - z}d\xi,\quad\text{ for all }\epsilon <r
\end{align*}
This with the uniform convergence of 
\begin{align*}
\frac{f(\xi)}{\xi -z} = \sum_{n=0}^{\infty} \frac{f(\xi) (z-z_0)^n }{(\xi -z_0)^{n+1}}\text{ on }\partial B_\epsilon (z_0)
\end{align*}
shows that holomorphic functions are locally power series 
\begin{align*}
f(z)= \sum_{n=0}^{\infty} \frac{1}{2\pi i}\Big(\int_{\partial B_\epsilon (z_0)} \frac{f(\xi)}{(\xi- z_0)^{n+1}}d\xi \Big) (z-z_0)^{n},\quad\text{for all }\epsilon <r \text{ and }z\in B_\epsilon (z_0)
\end{align*}
Because all power series converge uniformly on disk with radius strictly smaller than its convergence radius, we may differentiate term by term and have \textbf{Taylor's Theorem for power series} 
\begin{align*}
f(z)= \sum_{n=0}^{\infty} \frac{f^{(n)}(z_0)}{n!}(z-z_0)^n
\end{align*}
If $D\subseteq \C$ is a domain and $f:D\rightarrow \C$ is holomorphic, Taylor's Theorem for power series tell us that $\set{z\in D: f^{(n)}(z)=0\text{ for all }n\geq 0}$ is not only closed in $D$ but also open, and thus equals to $D$ if proved nonempty. One particularly weak condition for $T$ to be nonempty  is that $f\equiv 0$ on some $S \subseteq D$, and $S$ has a limit point in  $D$. This result is commonly referred to as \textbf{Identity Theorem}. By an \textbf{entire} function, we mean a holomorphic function $f:\C\rightarrow \C$ defined on the whole complex plane. Obviously, for all $r>0$ and $z \in B_r(0)$, we have 
\begin{align*}
f(z)= \sum_{n=0}^{\infty}c_nz^n,\quad\text{where }c_n=\frac{1}{2\pi i} \int_{\partial B_r(0)} \frac{f(\xi)}{\xi^{n+1}}d\xi
\end{align*}
If $f$ is bounded, then direct estimations show that $c_n=0$ for all $n>0$. This result is commonly referred to as \textbf{Liouville's Theorem}. Suppose $f$ is holomorphic on some annulus $r<\abso{z-z_0}<R$. Cauchy's integral theorem, Cauchy's integral formula and a geometric argument using 'cuts' give us 
\begin{align*}
f(z)= \frac{1}{2\pi  i} \Big(\int_{\partial B_{R- \epsilon }(z_0)} \frac{f(\xi)}{\xi - z}d\xi - \int_{\partial B_{r+\epsilon }(z_0)} \frac{f(\xi)}{\xi -z}d\xi  \Big)
\end{align*}
This with the uniform convergence
\begin{align*}
\frac{f(\xi)}{\xi -z} = \sum_{n=0}^{\infty} \frac{f(\xi) (z-z_0)^n }{(\xi -z_0)^{n+1}}\text{ on }\partial B_{R-\epsilon } (z_0)
\end{align*}
and the uniform convergence 
\begin{align*}
\frac{f(\xi)}{z-\xi}= \sum_{n=0}^{\infty} \frac{f(\xi)(\xi-z_0)^n}{(z-z_0)^{n+1}}\text{ on }\partial B_{r+\epsilon }(z_0)
\end{align*}
shows that 
\begin{align*}
f(z)&= \sum_{n=0}^{\infty} \frac{1}{2\pi  i}\Big( \int_{\partial  B_{R- \epsilon }(z_0)} \frac{f(\xi)}{(\xi - z_0)^{n+1}}d\xi \Big) (z-z_0)^n \\
&+ \sum_{n=1}^{\infty} \frac{1}{2\pi i}\Big( \int_{\partial B_{r+\epsilon }(z_0)} (\xi- z_0)^{n-1}f(\xi)d \xi \Big)(z-z_0)^{-n}
\end{align*}
Because the integrands $(\xi -z_0)^kf(\xi)$ have no singularities on the annulus, again, we may apply a geometric argument using 'cuts' to simplify the expression into its \textbf{Laurent series}:
\begin{align*}
  f(z)= \sum_{n=-\infty}^{\infty}c_n (z-z_0)^n
\end{align*}
where
\begin{align}
\label{cnp}
c_n= \frac{1}{2\pi i} \int_{\partial B_\epsilon (z_0)} \frac{f(\xi)}{(\xi - z_0)^{n+1}}d\xi \text{ for all } r<\epsilon  <R \text{ and }n\inz
\end{align}
If $f$ is defined and holomorphic on some deleted neighborhood $B_\epsilon (z_0)\setminus \set{z_0}$ but not defined on $\set{z_0}$, we say $z_0$ is an  \textbf{isolated singularity} of $f$, and we write 
\begin{align*}
\operatorname{Res}(f,z_0)\triangleq c_{-1}
\end{align*}
to denote the \textbf{residue} of $f$ at $z_0$. Let $D\subseteq \C$ be a simply connected domain, and suppose holomorphic $f$ is defined on $D$ except at some finite numbers of singularities. Let $\gamma :[a,b]\rightarrow D$ be some \textbf{simple}\footnote{By simple, we mean  $\gamma (t)=\gamma (s)$ if and only if $\abso{t-s}=\abso{a-b}$} closed contour, so that the image of $\gamma $ is a Jordan curve, and we may apply the \textbf{Jordan Curve Theorem} to distinguish between the interior and the exterior of $\gamma $. A simple closed contour $\gamma $ is \textbf{positively oriented} if the winding number is positive in the region enclosed by $\gamma $. We may now state our version of \textbf{Cauchy's Residue Theorem}. 
\begin{theorem}
\textbf{(Cauchy's Residue Theorem)}  Let $D\subseteq \C$ be a simply connected domain, and let $\gamma :[a,b]\rightarrow D$ be some positively oriented simple closed contour. If $f$ is is defined and holomorphic on  $D$ except at a finite set of points $\set{z_1,\dots ,z_n}$ that are all enclosed by $\gamma $, then 
\begin{align*}
\int_\gamma f(\xi)d\xi = 2\pi  i\sum_{j=1}^n \operatorname{Res}(f,z_j) 
\end{align*}
\end{theorem}
There are many distinct rigorous proofs for Cauchy's residue Theorem. None of them are trivial by some geometric argument. Again, for brevity, we present a proof \customref{CITA}{here} using Green's Theorem. Suppose $z_0$ is an isolated singularity of  $f$, and  $c_n$ are the coefficients of the Laurent series of  $f$ about $z_0$. There are three types of singularities depending on $c_n$. If $c_n=0$ for all  $n<0$, then  $z_0$ is  said to be a \textbf{removable singularity}. By direct estimation\footnote{For each $n<0$, let $\epsilon \to 0$ in \myref{Equation}{cnp}.}, we see that if  $f$ is bounded on some deleted neighborhood of $z_0$, then  $z_0$ is removable. This recognition is called  \textbf{Riemann's removable singularity Theorem}. If there exists some sequence  $n_k$ of integers that converges to  $-\infty$ such that $c_{n_k}\neq 0$ for all $k$, then we say  $z_0$  is an  \textbf{essential singularity}. The last type of singularities is perhaps the most interesting. If there exists some $m<0$ such that  $c_m\neq 0$ and $c_n=0$ for all  $n<m$, we say  $z_0$ is a \textbf{pole} of $f$ with multiplicity $m$. In such case, obviously we may define some $g$ by 
\begin{align*}
g(z)\triangleq (z-z_0)^mf(z)\text{ for all }z\neq z_0
\end{align*}
so that $z_0$ is merely a removable singularity of $g$, and $g(z_0)\neq 0$ after the removal. Now, because $g$ is continuous at  $z_0$, we see  $f$ is nonzero on some neighborhood around $z_0$, and we may compute on that neighborhood: 
\begin{align*}
\frac{f'(z)}{f(z)}= \frac{g'(z)}{g(z)}- \frac{m}{z-z_0} \text{ for all }z\neq z_0
\end{align*}
This implies 
\begin{align*}
\operatorname{Res}\Big(\frac{f'}{f},z_0\Big)=-m
\end{align*}
Similarly, if $z_0$ is a  \textbf{zero}\footnote{By $z_0$ being a zero of  $f$ with multiplicity  $k$, we mean  $f$ is holomorphically defined on some neighborhood of $z_0$, $f(z_0)=0$, and $k$ is the smallest integer such that  $c_k\neq 0$ where $f(z)=\sum_{n=0}^{\infty}c_n(z-z_0)^n$. Note that zeros and poles are dual to each other.}  of $f$ with multiplicity $k$, we may define  $g$ by 
\begin{align*}
g(z)\triangleq (z-z_0)^{-k}f(z)
\end{align*}
and compute 
\begin{align*}
\frac{f'(z)}{f(z)}= \frac{g'(z)}{g(z)}+ \frac{k}{z-z_0}\text{ for all }z\neq z_0 \text{ in some neighborhood of }z_0
\end{align*}
to deduce 
\begin{align*}
\operatorname{Res}\Big(\frac{f'}{f},z_0 \Big)=k
\end{align*}
These observations together with Cauchy's Residue Theorem now give us the \textbf{Argument Principle}. Given simply connected domain $D\subseteq \C$, positively oriented simple closed contour $\gamma :[a,b]\rightarrow D$ and some $f$ \textbf{meromorphic}\footnote{By $f$ being meromorphic on open $U$, we mean that $f$ is holomorphic on $U$ except on a finite set of poles.} on $D$, if $f$ has neither zeros nor poles on the image of $\gamma $, then   
\begin{align*}
\int_{\gamma } \frac{f'(\xi)}{f(\xi)}d\xi = 2\pi  i (Z-P)
\end{align*}
where $Z$ and $P$ are respectively the numbers of zeros and poles enclosed by $\gamma $ counted with multiplicity.   \\

Now, let $D$ be some simply connected domain,  let $\gamma :[a,b]\rightarrow D$ be some positively oriented simple closed contour, and let $f,g:D\rightarrow \C$ be two holomorphic function. If we require that $\abso{g}<\abso{f}$ on the image of $\gamma $, then obviously neither $f$, $f+g$ nor $1+\frac{g}{f}$ can have a zero on the image of $\gamma $, so after we compute 
\begin{align*}
\frac{(1+\frac{g}{f})'}{1+ \frac{g}{f}}=\frac{(f+g)'}{f+g}- \frac{f'}{f} 
\end{align*}
we may apply the argument principle to $f$ and  $g$ to conclude 
\begin{align*}
Z_{f+g}-Z_f= \int_{\gamma } \frac{(1+\frac{g}{f})'}{1+\frac{g}{f}}d\xi
\end{align*}
where $Z_{f+g}$ and $Z_f$ are the numbers of zeros of  $f+g$ and  $f$ enclosed by $\gamma $ counted with multiplicity. Moreover, if we define $\tilde{\gamma }:[a,b]\rightarrow \C $ by 
\begin{align*}
\tilde{\gamma }(t)\triangleq  1 + \frac{g(t)}{f(t)} 
\end{align*}
we see 
\begin{align*}
\int_{\tilde{\gamma }} \frac{d\xi}{\xi}= \int_{\gamma } \frac{(1+ \frac{g}{f})'}{1+\frac{g}{f}}d\xi = Z_{f+g}-Z_f 
\end{align*}
Therefore, once we note  that $\tilde{\gamma }$ as a simple closed contour lies in $\operatorname{Re}(z)>0$ \footnote{This is because  $\abso{g}<\abso{f}$ on $\gamma $.}, and note that $\log(z)$ is well-defined on $\operatorname{Re}(z)>0$ with derivative $\frac{1}{z}$\footnote{by real inverse function theorem.}, we may finally use Fundamental Theorem of Calculus to conclude the \textbf{Rouché's Theorem}: 
\begin{align*}
Z_f=Z_{f+g}
\end{align*}
We close this section by proving the \textbf{Open Mapping Theorem} and the \textbf{Maximum Modulus Principle}. Let $U\subseteq \C$ be some domain, and let $f:U\rightarrow \C$ be holomorphic and non-constant. Because open disks are also domains, and because by Identity Theorem, $f$ can not be constant on any open disks contained by $U$, to prove $f$ is an open map, we only have to prove $f(U)$ is open, without loss of generality.\\

Fix $w_0=f(z_0)$, and define $g:U\rightarrow \C$ by 
\begin{align*}
g(z)\triangleq f(z)- w_0
\end{align*}
Because $g$ is non-constant, by Identity Theorem, there exists some closed disk $K \subseteq U$ that centers $z_0$ and contains no zeros of $g$ other than  $z_0$.   
Defining
\begin{align*}
r\triangleq \min_{\partial K}\abso{g}>0\text{ and }D\triangleq B_r(w_0)
\end{align*}
we may reduce the proof into proving $D \subseteq f(K^{\circ })$. For each $w_1 \neq w_0\in D$, define $h:U\rightarrow \C$ by 
\begin{align*}
h(z)\triangleq f(z)-w_1
\end{align*}
Observing  
\begin{align*}
\abso{g(z)}\geq r> \abso{w_0 - w_1}= \abso{h(z)-g(z)}\text{ for all }z\in \partial K
\end{align*}
we may use Rouché's Theorem  to conclude that $h$ indeed have some zero in $K^{\circ }$. We have shown $f(U)$ is indeed open, which implies $\set{\abso{f(z)}: z \in U}$ is open in $\R$, giving maximum modulus principle as a corollary.  
\section{Riemann Mapping Theorem}
\section{Proof of Cauchy's Integral and Residue Theorem}
\label{CITA}
\chapter{Several Complex Variables}
\section{Untitled}
Given some open subset $U$ of $\C^n$, and some complex-valued function  $f:U\rightarrow \C$, we say $f$ is \textbf{holomorphic} in $z_1$ if for all  $(z_2,\dots, z_n)\in \C^{n-1}$, the function defined by $z_1 \mapsto f(z_1,\dots ,z_n)$ is holomorphic. In the context of theory of several complex variables, we use the notation $\epsilon \triangleq  (\epsilon_1,\dots ,\epsilon _n)$ to denote a tuple of positive real numbers, and we use the notation $B_\epsilon (a)\subseteq \C^n$ to denote \textbf{polydisc} 
\begin{align*}
\set{z\inc^n: \abso{z_i- a_i}<\epsilon_i \text{ for all }i}
\end{align*}
By repeatedly applying single variable version of Cauchy's integral formula, we see that if $f:U\rightarrow \C$ is holomorphic in all of its variables and $\overline{B_\epsilon (a)}\subseteq U$, then for all $z\in B_\epsilon (a)$, we have \textbf{Cauchy's integral formula in several variables}: 
\begin{align*}
f(z)= \frac{1}{(2\pi i)^n} \int_{\abso{\xi_1 - a_1}=\epsilon _1}\cdots \int_{\abso{\xi_n-a_n}= \epsilon_n} \frac{f(\xi_1,\dots ,\xi_n)}{(\xi_1-z_1)\cdots (\xi_n-z_n)}d\xi_n\cdots d\xi_1
\end{align*}
In this note, we use multi-index notation $\alpha = (\alpha _1,\dots ,\alpha _n) $, and adapt the notation 
\begin{align*}
\alpha +1 \triangleq (\alpha_1+1,\dots , \alpha_n+1)
\end{align*}
For each $\xi \in \partial B_\epsilon (a)$, because we have the absolute convergence 
\begin{align*}
  \sum_{\alpha_1 } \cdots \sum_{\alpha _n}  \abso{ \frac{(z-a)^{\alpha }}{(\xi-a)^{\alpha +1}}} \inr
\end{align*}
by Fubini's Theorem, for each sequence $\alpha^k$ that runs through $\N^n$, we have  
\begin{align*}
\sum_{k} \frac{(z-\alpha )^{\alpha ^k}}{(\xi-a)^{\alpha^k +1}}= \sum_{\alpha _1} \cdots \sum_{\alpha _n} \frac{(z-a)^{\alpha }}{(\xi-a)^{\alpha +1}} 
\end{align*}
so it make sense to write the multiple series into a single series 
\begin{align*}
\sum_{\alpha } \frac{(z-a)^{\alpha }}{(\xi-a)^{\alpha+1}}\triangleq \sum_{\alpha_1 } \cdots \sum_{\alpha _n} \frac{(z-a)^{\alpha }}{(\xi-a)^{\alpha +1}} 
\end{align*}
Now, because the convergence of $\sum_{\alpha } \frac{f(\xi)(z-a)^{\alpha }}{(\xi-a)^{\alpha }}$ is uniform on $\partial B_\epsilon (a)$, we have 
\begin{align*}
f(z)= \frac{1}{(2\pi i)^n} \sum_\alpha c_\alpha (z-a)^{\alpha } 
\end{align*}
where 
 \begin{align*}
c_\alpha = \int_{\abso{\xi_1-a_1}=\epsilon_1}\cdots \int_{\abso{\xi_n-a_n}=\epsilon _n} \frac{f(\xi)}{(\xi-a)^{\alpha +1}}d\xi
\end{align*}
\section{Hartog's Theorem on separate holomorphicity}
\section{Bijective holomorphic map is biholomorphic}
\chapter{Some Linear Algebra}
\section{Complexification}
Let $V$ be a finite dimensional real vector space. By \textbf{complexification} $V_\C$ of  $V$, formally, we mean the complex vector space $V\times V$ defined by 
\begin{align*}
  (a+b i)(v_1,v_2)\triangleq (av_1-bv_2, av_2+bv_1)
\end{align*}
To make visual computation easier, we use the notation $v_1+iv_2$  to denote $(v_1,v_2)\in V_\C$. Note that if $V$ has a $\R$-basis  $\set{v_1,\dots ,v_n}$, then clearly $V_\C$ has  $\C$-basis  $\set{v_1,\dots ,v_n}$, so we know $\operatorname{Dim}_\C (V_\C)=\operatorname{Dim}_\R (V)$. By the term \textbf{complex conjugation} on $V_\C$, we mean the $\R$-linear map 
\begin{align*}
\overline{v_1+ iv_2}\triangleq v_1-iv_2
\end{align*}
which is not $\C$-linear. If $W$ is another real vector space, and  $f:V\rightarrow W$ is some $\R$-linear map, then the  \textbf{complexified map} $f_\C:V_\C \rightarrow W_\C$ is the $\C$-linear map defined by 
\begin{align*}
f_\C(v_1+iv_2)\triangleq f(v_1)+ if(v_2)
\end{align*}
Moreover, if $g$ is an inner product on $V$, we may also extend  $g$ onto $V_\C$ by defining 
 \begin{align*}
g_\C (u_1+iu_2, v_1+iv_2)\triangleq g(u_1,v_1)+g(u_2,v_2)+i [g(u_2,v_1)-g(u_1,v_2)]
\end{align*}


 By an \textbf{almost complex} structure on $V$, we mean some $\R$-linear map $J\in \operatorname{End}(V)$ whose squared is the negate of the identity map of $V$. Clearly, if $J$ is an almost complex structure on $V$, then we may make $V$ into a complex vector space by defining 
\begin{align*}
  (a+b i)v\triangleq av+bJ(v) 
\end{align*}
Notably, after this assignment of $\C$-action, we see that $\operatorname{Dim}_{\R}(V)$ must be even, since if $\set{v_1,\dots ,v_n}$ form a $\C$-basis for  $V$, then  $\set{v_1,J(v_1),\dots ,v_n,J(v_n)}$ form a $\R$-basis for  $V$. 

Now, suppose $V$ is a finite dimensional real vector space with an almost complex structure  $J$, and consider the complexified map $J_\C\in \operatorname{End}_{\C}(V_\C)$. Routine computation shows that the squared of $J_\C$ is also the negate of the identity map of  $V_\C$. Let $V^{1,0},V^{0,1}$ be the eigenspace of $V_\C$ with respect to function $J_\C \in \operatorname{End}_\C(V_\C)$ eigenvalues $\pm i$. Computing for all $v\in V_\C$ that 
\begin{align*}
v=\Big( \frac{v- i J_\C(v)}{2} \Big)+ \Big( \frac{v+i J_\C(v)}{2} \Big) \in V^{1,0} \oplus  V^{0,1}
\end{align*}
we see that $V_\C$ is the direct sum of $V^{1,0}$ and $V^{0,1}$. Moreover, because the complex conjugation on $V_\C$ is an  $\R$-isomorphism between  $V^{1,0}$ and $V^{0,1}$\footnote{This is a result of routine computation.}, we see 
\begin{align*}
\operatorname{Dim}_\C (V^{1,0})= \operatorname{Dim}_\C (V^{0,1})=\frac{\operatorname{Dim}_\C (V_\C)}{2}
\end{align*}
This together with the clearly injective $\R$-linear map  $\pi ^{1,0}:V\rightarrow V^{1,0}$ defined by 
\begin{align*}
\pi ^{1,0} (v)\triangleq \frac{v-i J(v)}{2}
\end{align*}
shows\footnote{$\operatorname{Dim}_\R(V)=\operatorname{Dim}_\C (V_\C)=2\operatorname{Dim}_\C (V^{1,0})=\operatorname{Dim}_\R (V^{1,0})$} that $V$ and  $V^{1,0}$ are $\R$-isomorphic.\footnote{Noting that the imaginary unit $i$ forms an almost complex structure on $V^{1,0}$, one may even shows the isomorphism $\pi ^{1,0}(J(v))=i \pi ^{1,0}(v)$.} We adapt the notation 
\begin{align*}
\extp^{p,q}V\triangleq \extp^{p}V^{1,0}\otimes _\C \extp^{q}V^{0,1}
\end{align*}
and refer to elements $\alpha $ of $\extp^{p,q}V$ as \textbf{bidegree} $(p,q)$. 
\begin{theorem}
\textbf{(Decomposition of wedge product of $V_\C$ into bidegrees)} There exists a $\C$-linear isomorphism 
 \begin{align*}
\extp^k V_\C \cong \bigoplus _{p+q=k}\extp^{p,q}V
\end{align*}
\end{theorem}
\begin{proof}
Let $\set{e_1,\dots ,e_n}$ be a $\C$-basis for  $V^{1,0}$. Fix some  $p+q=k$. We know 
\begin{align*}
\bset{e_{i_1}\wedge  \cdots \wedge  e_{i_p}\otimes \overline{e_{j_1}}\wedge  \cdots \wedge  \overline{e_{j_q}}: 1\leq i_1<\cdots <i_p \leq n\text{ and }1\leq j_1<\cdots <j_q \leq n}
\end{align*}
forms a $\C$-basis for $\extp^{p,q}V$. Because $\set{e_1,\dots ,e_n, \overline{e_1},\dots ,\overline{e_n}}$ forms a $\C$-basis for $V_\C$, the $\C$-linear map $\phi_{p,q}:\extp^{p,q}V\rightarrow \extp^{k}V_\C$ determined by 
\begin{align*}
  e_{i_1}\wedge  \cdots \wedge  e_{i_p}\otimes \overline{e_{j_1}}\wedge  \cdots \wedge  \overline{e_{j_q}}\mapsto e_{i_1}\wedge  \cdots \wedge  e_{i_p}\wedge   \overline{e_{j_1}}\wedge  \cdots \wedge  \overline{e_{j_q}}
\end{align*}
is injective. One may now check that the $\C$-linear map
\begin{align*}
\bigoplus_{p+q=k}\phi_{p,q}: \bigoplus_{p+q=k}\extp^{p,q} V \rightarrow \extp^k V_\C
\end{align*}
forms a $\C$-linear isomorphism. 
\end{proof}
\section{Hodge star operator}
Let $(V,g)$ be an oriented $d$-dimensional real inner product space whose orientation is determined by the orthonormal basis $\set{e_1,\dots ,e_d}$, for each exterior power $\extp^k V$, we may well define (?) an inner product $g^k$ on $\extp^k V$ such that for every orthonormal basis $\set{e_1,\dots ,e_d}$ of $V$, the basis  $\set{e_I \in \extp^k V:I=\set{i_1< \cdots <i_k}}$ is orthonormal for $\extp^k V$, and well define (?) the \textbf{Hodge star operator} $\star:\extp^k V \rightarrow \extp^{d-k} V$ so that 
\begin{align*}
\alpha \wedge \star \beta  = g^k(\alpha ,\beta ) \cdot e_1 \wedge  \cdots \wedge  e_d   \text{ for all }\alpha ,\beta  \in \extp^k V
\end{align*}
\begin{mdframed}
Let $(V,g)$ be a finite dimensional real inner product space, and let $J:V\rightarrow V$ be an almost complex structure compatible with $g$.  Let $\omega\in \operatorname{Hom}^2(V\times V,\R)$ be the fundamental form associated with $(V,g,J)$. I read from your note and also the textbook that 
\begin{align}
\label{ggg}
\omega \in \extp^{1,1}V^*
\end{align}
where $\extp^{1,1}W$ is defined by 
\begin{align*}
\extp^{1,1}W \triangleq \extp W^{1,0}\otimes _\C \extp W^{0,1}
\end{align*}
My question is: How does  \myref{statement}{ggg} make sense? Shouldn't $\extp^{1,1}V^*$ be the space of bilinear maps


Thank you for your reply. I just realized that I wrongly identified $\extp^{1,1}V^*$. But I still can't piece things together to show that 
\begin{align}
\extp^{1,1}V^*\text{ is the space of }
\end{align}
 \begin{align*}
g_\C (u_1+iu_2, v_1+iv_2)\triangleq g(u_1,v_1)+g(u_2,v_2)+i [g(u_2,v_1)-g(u_1,v_2)]
\end{align*}
and 
\begin{align*}
\omega_\C (u_1+iu_2, v_1+iv_2)\triangleq \omega(u_1,v_1)+\omega(u_2,v_2)+i [\omega(u_2,v_1)-\omega(u_1,v_2)]
\end{align*}
\begin{align*}
\omega_\C (z_1v_1,z_2v_2)\triangleq z_1 \overline{z_2}\omega (v_1,v_2)
\end{align*}

\begin{align*}
\extp^{1,1}V^* &= \extp^1 (V^{*})^{1,0} \otimes _\C \extp^1 (V^*)^{0,1} \\
&=(V^*)^{1,0}\otimes_\C (V^*)^{0,1} \\
&=\operatorname{Hom}^2([(V^*)^{1,0}]^*\times [(V^*)^{0,1}]^{*}, \C) \\
&=\operatorname{Hom}^2(V^{1,0}\times V^{0,1},\C) 
\end{align*}
\begin{align*}
\operatorname{Hom}^2(\extp (V^*)^{1,0}\times \extp (V^*)^{0,1},\C)
\end{align*}
by definition of tensor product? How cam an element $\omega \in \operatorname{Hom}^2 (V\times V,\R)$ be identified as an element of $\operatorname{Hom}^2(\extp (V^*)^{1,0}\times \extp (V^*)^{0,1},\C)$? 
\end{mdframed}
I realized I had the completely wrong understanding. Please confirm with me if my current understanding is correct. The statement   
\begin{align}
\omega \in \extp^{1,1}V^*
\end{align}
actually means 
\begin{align}
  \omega_\C \in \extp^2 V_\C^* &=\text{ alternating subspace of } V_\C^* \otimes V_\C^* \\
  &=\text{ alternating subspace of }\operatorname{Hom}^2(V_\C\times V_\C ,\C)\\
  &=\extp^{2,0}V^* \oplus \extp^{1,1}V^* \oplus \extp^{0,2}V^*
\end{align}
and that the components of $\omega_\C$ in $\extp^{2,0}V^*$ and $\extp^{0,2}V^*$ are both zero, where we define $\omega_\C$ by 
\begin{align}
&\omega_\C (a_1v_1+ib_1v_2, a_2w_1+ib_2w_2)\\
&\triangleq a_1a_2\omega(v_1,w_1)-b_1b_2\omega(v_2,w_2)+ i[b_1a_2\omega (v_2,w_1)+a_1b_2\omega (v_1,w_2)]
\end{align}


\section{Fundamental form and complex inner product induced by compatible real inner product and almost complex structure}
For each finite dimensional real inner product space $(V,g)$, an almost complex structure $J:V\rightarrow V$ is said to be \textbf{compatible} with $g$ if 
\begin{align*}
g(J(v_1),J(v_2))=g(v_1,v_2)
\end{align*}
Given such compatible $(V,g,J)$, one see that the decomposition $V_\C=V^{1,0}\oplus V^{0,1}$ is orthogonal, and we associate it with its \textbf{fundamental form} $\omega\in \operatorname{Hom}^2(V\times V,\R)$ by 
\begin{align*}
\omega (v,w)\triangleq  g(J(v),w)
\end{align*}
It is easy to see $\omega$ is always alternating, and that $h\triangleq g- i \omega$ forms a complex inner product when $V$ is viewed as a complex vector space. 

\chapter{Complex Manifold}
\section{Definition and Examples}
Given some topological manifold $M$, by a \textbf{holomorphic chart}, we mean an open subset $U\subseteq M$ and a topological embedding  $\phi : U \rightarrow \C^n$. For two complex charts $\phi_i$ and $\phi_j$ to be \textbf{compatible}, the function $\phi_i \circ \phi_j |_{\phi_j (U_i \cap U_j)}$ has to be holomorphic. A collection of complex charts is said to be a \textbf{holomorphic atlas} if the collection covers the whole $M$ and all pairs of the charts in the collection are compatible.  A \textbf{complex structure} on $M$ if exists, is just a maximal holomorphic atlas. By a \textbf{complex manifold}, we mean a topological manifold together with a complex structure. \\

The first example of complex manifold is the \textbf{complex projective space}. Define an equivalence relation on $\C^
{n+1}\setminus \set{0}$ by 
\begin{align*}
 z \sim  w \overset{\triangle}{\iff } z= \ld w \text{ for some }\ld \inc^*
\end{align*}
The complex projective space is defined to be the quotient topological space $\P^n\triangleq (\C^{n+1}\setminus \set{0}) \quotient \sim $. We use the notation $(z_0:\dots :z_n)$ to denote elements of $\P^n$. Clearly, 
 \begin{align*}
U_i \triangleq \set{(z_0: \dots :z_n)\in\P^n : z_i \neq 0}
\end{align*}
forms an open cover of $\P^n$, and the maps 
 \begin{align*}
\phi_i:U_i\rightarrow \C^n,\quad (z_0: \dots:z_n)\mapsto (\frac{z_0}{z_i},\dots , \frac{z_{i-1}}{z_i},\frac{z_{i+1}}{z_i},\dots ,\frac{z_n}{z_i})
\end{align*}
forms a holomorphic atlas. \\

The second example is the \textbf{complex tori}. Define an equivalence relation on $\C^n$ by 
 \begin{align*}
z \sim  w \overset{\triangle }{\iff }\text{ For all  }j\text{ there exists some $a,b \inz$ such that }z_j-w_j= a+ ib
\end{align*}
The complex tori is defined to be the quotient topological space $\C^n \quotient \sim$. Let $\pi $ be the quotient map. Clearly for all $z\inc^n$, if we let $\epsilon = (\frac{1}{2},\dots ,\frac{1}{2})$, then $\pi  (B_\epsilon (z))$ is open in the tori.  
\end{document}
